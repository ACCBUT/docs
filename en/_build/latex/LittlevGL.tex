%% Generated by Sphinx.
\def\sphinxdocclass{report}
\documentclass[letterpaper,10pt,english]{sphinxmanual}
\ifdefined\pdfpxdimen
   \let\sphinxpxdimen\pdfpxdimen\else\newdimen\sphinxpxdimen
\fi \sphinxpxdimen=.75bp\relax

\PassOptionsToPackage{warn}{textcomp}
\usepackage[utf8]{inputenc}
\ifdefined\DeclareUnicodeCharacter
% support both utf8 and utf8x syntaxes
  \ifdefined\DeclareUnicodeCharacterAsOptional
    \def\sphinxDUC#1{\DeclareUnicodeCharacter{"#1}}
  \else
    \let\sphinxDUC\DeclareUnicodeCharacter
  \fi
  \sphinxDUC{00A0}{\nobreakspace}
  \sphinxDUC{2500}{\sphinxunichar{2500}}
  \sphinxDUC{2502}{\sphinxunichar{2502}}
  \sphinxDUC{2514}{\sphinxunichar{2514}}
  \sphinxDUC{251C}{\sphinxunichar{251C}}
  \sphinxDUC{2572}{\textbackslash}
\fi
\usepackage{cmap}
\usepackage[T1]{fontenc}
\usepackage{amsmath,amssymb,amstext}
\usepackage{babel}



\usepackage{times}
\expandafter\ifx\csname T@LGR\endcsname\relax
\else
% LGR was declared as font encoding
  \substitutefont{LGR}{\rmdefault}{cmr}
  \substitutefont{LGR}{\sfdefault}{cmss}
  \substitutefont{LGR}{\ttdefault}{cmtt}
\fi
\expandafter\ifx\csname T@X2\endcsname\relax
  \expandafter\ifx\csname T@T2A\endcsname\relax
  \else
  % T2A was declared as font encoding
    \substitutefont{T2A}{\rmdefault}{cmr}
    \substitutefont{T2A}{\sfdefault}{cmss}
    \substitutefont{T2A}{\ttdefault}{cmtt}
  \fi
\else
% X2 was declared as font encoding
  \substitutefont{X2}{\rmdefault}{cmr}
  \substitutefont{X2}{\sfdefault}{cmss}
  \substitutefont{X2}{\ttdefault}{cmtt}
\fi


\usepackage[Bjarne]{fncychap}
\usepackage{sphinx}

\fvset{fontsize=\small}
\usepackage{geometry}

% Include hyperref last.
\usepackage{hyperref}
% Fix anchor placement for figures with captions.
\usepackage{hypcap}% it must be loaded after hyperref.
% Set up styles of URL: it should be placed after hyperref.
\urlstyle{same}

\usepackage{sphinxmessages}
\setcounter{tocdepth}{2}



\title{LittlevGL Documentation}
\date{Jun 18, 2019}
\release{6.0}
\author{Gabor Kiss-Vamosi}
\newcommand{\sphinxlogo}{\vbox{}}
\renewcommand{\releasename}{Release}
\makeindex
\begin{document}

\pagestyle{empty}
\sphinxmaketitle
\pagestyle{plain}
\sphinxtableofcontents
\pagestyle{normal}
\phantomsection\label{\detokenize{index::doc}}
\sphinxhref{http://docs.littlevgl.com/locale/en}{English} -
\sphinxhref{http://docs.littlevgl.com/locale/pt\_BR}{Portuguese} -
\sphinxhref{http://docs.littlevgl.com/locale/es}{Espanol} -
\sphinxhref{http://docs.littlevgl.com/locale/tr}{Turkish}


\bigskip\hrule\bigskip




\sphinxincludegraphics{{lv_theme_intro}.png}

LittlevGL is a free and open-source graphics library providing everything you need to create embedded GUI with easy-to-use graphical elements, beautiful visual effects and low memory footprint.


\chapter{Table of content}
\label{\detokenize{index:table-of-content}}

\section{Introduction}
\label{\detokenize{intro:introduction}}\label{\detokenize{intro::doc}}
LittlevGL is a free and open-source graphics library providing everything you need to create embedded GUI with easy-to-use graphical elements, beautiful visual effects and low memory footprint.


\subsection{Key features}
\label{\detokenize{intro:key-features}}\begin{itemize}
\item {} 
Powerful building blocks buttons, charts, lists, sliders, images etc

\item {} 
Advanced graphics with animations, anti-aliasing, opacity, smooth scrolling

\item {} 
Various input devices touch pad, mouse, keyboard, encoder etc

\item {} 
Multi language support with UTF-8 encoding

\item {} 
Fully customizable graphical elements

\item {} 
Hardware independent to use with any microcontroller or display

\item {} 
Scalable to operate with little memory (80 kB Flash, 10 kB RAM)

\item {} 
OS, External memory and GPU supported but not required

\item {} 
Single frame buffer operation even with advanced graphical effects

\item {} 
Written in C for maximal compatibility (C++ compatible)

\item {} 
Simulator to start embedded GUI design on PC without embedded hardware

\item {} 
Tutorials, examples, themes for rapid GUI design

\item {} 
Documentation online and offline

\item {} 
Free and open-source under MIT licence

\end{itemize}


\subsection{Requirements}
\label{\detokenize{intro:requirements}}\begin{itemize}
\item {} 
16, 32 or 64 bit microcontroller or processor

\item {} 
16 MHz clock speed

\item {} 
8 kB RAM for static data and \textgreater{}2 KB RAM for dynamic data (graphical objects)

\item {} 
64 kB program memory (flash)

\item {} 
Optionally \textasciitilde{}1/10 screen sized memory for internal buffering (at 240 × 320, 16 bit colors it means 15 kB)

\item {} 
C99 or newer compiler

\end{itemize}


\bigskip\hrule\bigskip


The LittlevGL is designed to be highly portable and to not use any external resources:
\begin{itemize}
\item {} 
No external RAM required (but supported)

\item {} 
No float numbers are used

\item {} 
No GPU needed (but supported)

\item {} 
Only a single frame buffer is required located in:
\begin{itemize}
\item {} 
Internal RAM or

\item {} 
External RAM or

\item {} 
External display controller’s memory

\end{itemize}

\end{itemize}


\bigskip\hrule\bigskip


If you would like to reduce the required hardware resources you can:
\begin{itemize}
\item {} 
Disable the unused object types to save RAM and ROM

\item {} 
Change the size of the graphical buffer to save RAM (see later)

\item {} 
Use more simple styles to reduce the rendering time

\end{itemize}


\section{Get started}
\label{\detokenize{get-started/index:get-started}}\label{\detokenize{get-started/index::doc}}\begin{itemize}
\item {} 
lvgl on GihHub

\item {} 
example projects

\end{itemize}


\subsection{Live demos}
\label{\detokenize{get-started/live-demo:live-demos}}\label{\detokenize{get-started/live-demo::doc}}
See look and feel


\subsection{Micropython}
\label{\detokenize{get-started/micropython:micropython}}\label{\detokenize{get-started/micropython::doc}}
play with it in micropython


\subsection{Simulator on PC}
\label{\detokenize{get-started/pc-simulator:simulator-on-pc}}\label{\detokenize{get-started/pc-simulator::doc}}
You can try out the LittlevGL \sphinxstylestrong{using only your PC} without any development boards. Write a code, run it on the PC and see the result on the monitor. It is cross-platform: Windows, Linux and OSX are supported.The written code is portable, you can simply copy it when using an embedded hardware.

The simulator is also very useful to report bugs because it means common platform for every user. So it’s a good idea the reproduce a bug in simualtor and use the code snippen in the \sphinxhref{https://forum.littelvgl.com}{Forum}.


\subsubsection{Select an IDE}
\label{\detokenize{get-started/pc-simulator:select-an-ide}}
The simulator is ported to valrious IDEs. Choose your favourite IDE, read its README on GitHub, download the project, and load it to the IDE.



In followings the set-up guide of Eclipse CDT is described in more details.


\subsubsection{Set-up Eclipse CDT}
\label{\detokenize{get-started/pc-simulator:set-up-eclipse-cdt}}

\paragraph{Install Eclipse CDT}
\label{\detokenize{get-started/pc-simulator:install-eclipse-cdt}}
Eclipse CDT is C/C++ IDE. You can use other IDEs as well but in this tutorial the configuration for Eclipse CDT is shown.

Eclipse is a Java based software therefore be sure \sphinxstylestrong{Java Runtime Environment} is installed on your system.

On Debian-based distros (e.g. Ubuntu): \sphinxcode{\sphinxupquote{sudo apt-get install default-jre}}

You can download Eclipse’s CDT from: \sphinxhref{http://www.eclipse.org/downloads/}{https://eclipse.org/cdt/}. Start the installer and choose \sphinxstyleemphasis{Eclipse CDT} from the list.


\paragraph{Install SDL 2}
\label{\detokenize{get-started/pc-simulator:install-sdl-2}}
The PC simulator uses the \sphinxhref{https://www.libsdl.org/download-2.0.php}{SDL 2} cross platform library to simulate a TFT display and a touch pad.


\subparagraph{Linux}
\label{\detokenize{get-started/pc-simulator:linux}}
On \sphinxstylestrong{Linux} you can easily install SDL2 using a terminal:
\begin{enumerate}
\def\theenumi{\arabic{enumi}}
\def\labelenumi{\theenumi .}
\makeatletter\def\p@enumii{\p@enumi \theenumi .}\makeatother
\item {} 
Find the current version of SDL2: \sphinxcode{\sphinxupquote{apt-cache search libsdl2 (e.g. libsdl2-2.0-0)}}

\item {} 
Install SDL2: \sphinxcode{\sphinxupquote{sudo apt-get install libsdl2-2.0-0}} (replace with the found version)

\item {} 
Install SDL2 development package: \sphinxcode{\sphinxupquote{sudo apt-get install libsdl2-dev}}

\item {} 
If build essentials are not installed yet: \sphinxcode{\sphinxupquote{sudo apt-get install build-essential}}

\end{enumerate}


\subparagraph{Windows}
\label{\detokenize{get-started/pc-simulator:windows}}
If you are using \sphinxstylestrong{Windows} firstly you need to install MinGW (\sphinxhref{http://mingw-w64.org/doku.php/download}{64 bit version}). After it do the following steps to add SDL2:
\begin{enumerate}
\def\theenumi{\arabic{enumi}}
\def\labelenumi{\theenumi .}
\makeatletter\def\p@enumii{\p@enumi \theenumi .}\makeatother
\item {} 
Download the development libraries of SDL.Go to \sphinxurl{https://www.libsdl.org/download-2.0.php} and download \sphinxstyleemphasis{Development Libraries: SDL2-devel-2.0.5-mingw.tar.gz}

\item {} 
Uncompress the file and go to \sphinxstyleemphasis{x86\_64-w64-mingw32} directory (for 64 bit MinGW) or to \sphinxstyleemphasis{i686-w64-mingw32} (for 32 bit MinGW)

\item {} 
Copy \_…\sphinxstyleemphasis{mingw32/include/SDL2} folder to \sphinxstyleemphasis{C:/MinGW/…/x86\_64-w64-mingw32/include}

\item {} 
Copy \_…\sphinxstyleemphasis{mingw32/lib/} content to \sphinxstyleemphasis{C:/MinGW/…/x86\_64-w64-mingw32/lib}

\item {} 
Copy \_…\sphinxstyleemphasis{mingw32/bin/SDL2.dll} to \sphinxstyleemphasis{\{eclipse\_worksapce\}/pc\_simulator/Debug/}.  Do it later when Eclipse is installed.

\end{enumerate}

Note: If you will use \sphinxstylestrong{Microsoft Visual Studio} instead of Eclipse then you don’t have to install MinGW.


\subparagraph{OSX}
\label{\detokenize{get-started/pc-simulator:osx}}
On \sphinxstylestrong{OSX} you can easily install SDL2 with brew: \sphinxcode{\sphinxupquote{brew install sdl2}}

If something is not working I suggest \sphinxhref{http://lazyfoo.net/tutorials/SDL/01\_hello\_SDL/index.php}{this tutorial} to get started with SDL.


\paragraph{Pre-configured project}
\label{\detokenize{get-started/pc-simulator:pre-configured-project}}
A pre-configured graphics library project (based on the latest release) is always available.
You can find it on \sphinxhref{https://github.com/littlevgl/proj\_pc}{GitHub} or on the \sphinxhref{https://littlevgl.com/download}{Download} page.
(The project is configured for Eclipse CDT.)


\paragraph{Add the pre-configured project to Eclipse CDT}
\label{\detokenize{get-started/pc-simulator:add-the-pre-configured-project-to-eclipse-cdt}}
Run Eclipse CDT. It will show a dialogue about the \sphinxstylestrong{workspace path}. Before accepting it check that path and copy (and unzip) the downloaded pre-configured project there. Now you can accept the workspace path. Of course you can modify this path but in that case copy the project to that location.

Close the start up window and go to \sphinxstylestrong{File-\textgreater{}Import} and choose \sphinxstylestrong{General-\textgreater{}Existing project into Workspace}. \sphinxstylestrong{Browse the root directory} of the project and click \sphinxstylestrong{Finish}

On \sphinxstylestrong{Windows} you have to do two additional things:
\begin{itemize}
\item {} 
Copy the \sphinxstylestrong{SDL2.dll} into the project’s Debug folder

\item {} 
Righ click on the project -\textgreater{} Project properties -\textgreater{} C/C++ Build -\textgreater{} Settings -\textgreater{} Libraries -\textgreater{} Add … and add \sphinxstyleemphasis{mingw32} above SDLmain and SDL. (The order is important: mingw32, SDLmain, SDL)

\end{itemize}


\paragraph{Compile and Run}
\label{\detokenize{get-started/pc-simulator:compile-and-run}}
Now you are ready to run the Littlev Graphics Library on your PC. Click on the Hammer Icon on the top menu bar to Build the project. If you have done everything right you will not get any errors. Note that on some systems additional steps might be required to “see” SDL 2 from Eclipse but in most of cases the configurtions in the downloaded project is enough.

After a success build click on the Play button on the top menu bar to run the project. Now a window should appear in the middle of your screen.

Now everything is ready to use the Littlev Graphics Library in the practice or begin the development on your PC.


\section{Porting}
\label{\detokenize{porting/index:porting}}\label{\detokenize{porting/index::doc}}

\subsection{System overview}
\label{\detokenize{porting/sys:system-overview}}\label{\detokenize{porting/sys::doc}}
\sphinxincludegraphics{{sys}.png}

\sphinxstylestrong{Application}
Your application which creates the GUI and handles the specific tasks.

\sphinxstylestrong{LittlevGL}
The graphics library itself. Your application can communicate with the library to create a GUI. It contains a HAL (Hardware Abstraction Layer) interface to register your display and input device drivers.

\sphinxstylestrong{Driver}
Besides your specific drivers, it contains functions to drive your display, optionally to a GPU and to read the touchpad or buttons.


\bigskip\hrule\bigskip


There are \sphinxstylestrong{two typical hardware set-ups} depending on the MCU has an LCD/TFT driver periphery or not. In both cases, a frame buffer will be required to store the current image of the screen.
\begin{enumerate}
\def\theenumi{\arabic{enumi}}
\def\labelenumi{\theenumi .}
\makeatletter\def\p@enumii{\p@enumi \theenumi .}\makeatother
\item {} 
\sphinxstylestrong{MCU with TFT/LCD driver}
If your MCU has a TFT/LCD driver periphery then you can connect a display directly via RGB interface.
In this case, the frame buffer can be in the internal RAM (if the MCU has enough RAM) or in the external RAM (if the MCU has a memory interface).

\item {} 
\sphinxstylestrong{External display controller}
If the MCU doesn’t have TFT/LCD driver interface then an external display controller (E.g. SSD1963, SSD1306, ILI9341) has to be used.
In this case, the MCU can communicate with the display controller via Parallel port, SPI or sometimes I2C.
The frame buffer is usually located in the display controller which saves a lot of RAM for the MCU.

\end{enumerate}


\subsection{Set-up a project}
\label{\detokenize{porting/project:set-up-a-project}}\label{\detokenize{porting/project::doc}}

\subsubsection{Get the library}
\label{\detokenize{porting/project:get-the-library}}
Littlev Graphics Library is available on GitHub: \sphinxurl{https://github.com/littlevgl/lvgl}.

You can clone it or download the latest version of the library from GitHub or you can use the \sphinxhref{https://littlevgl.com/download}{Download} page as well.

The graphics library is the \sphinxstylestrong{lvgl} directory which should be copied into your project.


\subsubsection{Config file}
\label{\detokenize{porting/project:config-file}}
There is a configuration header file for LittlevGL called \sphinxstylestrong{lv\_conf.h}. It sets the library’s basic behavior, disables unused modules and features, adjusts the size of memory buffers in compile time, etc.

Copy \sphinxstylestrong{lvgl/lv\_conf\_template.h} next to the \sphinxstyleemphasis{lvgl} directory and rename it to \sphinxstyleemphasis{lv\_conf.h}. Open the file and change the \sphinxcode{\sphinxupquote{\#if 0}} at the beginning to \sphinxcode{\sphinxupquote{\#if 1}} to enable its content.

\sphinxstyleemphasis{lv\_conf.h} can be copied other places as well but then you should add \sphinxcode{\sphinxupquote{LV\_CONF\_INCLUDE\_SIMPLE}} define to ou compilers (e.g. \sphinxcode{\sphinxupquote{-DLV\_CONF\_INCLUDE\_SIMPLE}} fo gcc) and set the include path manually.

In the config file comments explain the meaning of the options. Check at least these three config options and modify them according to your hardware:
\begin{enumerate}
\def\theenumi{\arabic{enumi}}
\def\labelenumi{\theenumi .}
\makeatletter\def\p@enumii{\p@enumi \theenumi .}\makeatother
\item {} 
\sphinxstylestrong{LV\_HOR\_RES\_MAX} Your display’s horizontal resolution

\item {} 
\sphinxstylestrong{LV\_VER\_RES\_MAX} Your display’s vertical resolution

\item {} 
\sphinxstylestrong{LV\_COLOR\_DEPTH} 8 for (RG332), 16 for (RGB565) or 32 for (RGB888 and ARGB8888).

\end{enumerate}


\subsubsection{Initialization}
\label{\detokenize{porting/project:initialization}}
In order to use the graphics library you have to initialize it and the other components too. To order of the initialization is:
\begin{enumerate}
\def\theenumi{\arabic{enumi}}
\def\labelenumi{\theenumi .}
\makeatletter\def\p@enumii{\p@enumi \theenumi .}\makeatother
\item {} 
Call \sphinxstyleemphasis{lv\_init()}

\item {} 
Initialize your drivers

\item {} 
Register the display and input devices drivers in LittlevGL.  More about {\hyperref[\detokenize{porting/display::doc}]{\sphinxcrossref{\DUrole{doc,doc,doc}{Display}}}} and {\hyperref[\detokenize{porting/indev::doc}]{\sphinxcrossref{\DUrole{doc,doc,doc}{Input device}}}} registration.

\item {} 
Call \sphinxcode{\sphinxupquote{lv\_tick\_inc(x)}} in every \sphinxcode{\sphinxupquote{x}} milliseconds in an interrupt to tell the elapsed time. {\hyperref[\detokenize{porting/tick::doc}]{\sphinxcrossref{\DUrole{doc,doc,doc}{Learn more}}}}.

\item {} 
Call \sphinxcode{\sphinxupquote{lv\_task\_handler()}} periodically in every few milliseconds to handle LittlevGL related tasks. {\hyperref[\detokenize{porting/task-handler::doc}]{\sphinxcrossref{\DUrole{doc,doc,doc}{Learn more}}}}.

\end{enumerate}


\subsection{Display interface}
\label{\detokenize{porting/display:display-interface}}\label{\detokenize{porting/display::doc}}
To set up a display an \sphinxcode{\sphinxupquote{lv\_disp\_buf\_t}} and an \sphinxcode{\sphinxupquote{lv\_disp\_drv\_t}} variable has to be initialized.
\begin{itemize}
\item {} 
\sphinxstylestrong{lv\_disp\_buf\_t} contains internal graphics buffer(s).

\item {} 
\sphinxstylestrong{lv\_disp\_drv\_t} contains callback functions to iteract with the display and manipulate drawing related things.

\end{itemize}


\subsubsection{Display buffer}
\label{\detokenize{porting/display:display-buffer}}
\sphinxcode{\sphinxupquote{lv\_disp\_buf\_t}} can be initialized like this:

\begin{sphinxVerbatim}[commandchars=\\\{\}]
    \PYG{c+cm}{/*A static or global variable to store the buffers*/}
    \PYG{k}{static} \PYG{n}{lv\PYGZus{}disp\PYGZus{}buf\PYGZus{}t} \PYG{n}{disp\PYGZus{}buf}\PYG{p}{;}
    
    \PYG{c+cm}{/*Static or global buffer(s). The second buffer is optional*/}
    \PYG{k}{static} \PYG{n}{lv\PYGZus{}color\PYGZus{}t} \PYG{n}{buf\PYGZus{}1}\PYG{p}{[}\PYG{n}{MY\PYGZus{}DISP\PYGZus{}HOR\PYGZus{}RES} \PYG{o}{*} \PYG{l+m+mi}{10}\PYG{p}{]}\PYG{p}{;}
    \PYG{k}{static} \PYG{n}{lv\PYGZus{}color\PYGZus{}t} \PYG{n}{buf\PYGZus{}2}\PYG{p}{[}\PYG{n}{MY\PYGZus{}DISP\PYGZus{}HOR\PYGZus{}RES} \PYG{o}{*} \PYG{l+m+mi}{10}\PYG{p}{]}\PYG{p}{;}
    
    \PYG{c+cm}{/*Initalize {}`disp\PYGZus{}buf{}` with the buffer(s) */}
    \PYG{n}{lv\PYGZus{}disp\PYGZus{}buf\PYGZus{}init}\PYG{p}{(}\PYG{o}{\PYGZam{}}\PYG{n}{disp\PYGZus{}buf}\PYG{p}{,} \PYG{n}{buf\PYGZus{}1}\PYG{p}{,} \PYG{n}{buf\PYGZus{}2}\PYG{p}{,} \PYG{n}{MY\PYGZus{}DISP\PYGZus{}HOR\PYGZus{}RES}\PYG{o}{*}\PYG{l+m+mi}{10}\PYG{p}{)}\PYG{p}{;}
\end{sphinxVerbatim}

There are there possible configurations regarding the buffer size:
\begin{enumerate}
\def\theenumi{\arabic{enumi}}
\def\labelenumi{\theenumi .}
\makeatletter\def\p@enumii{\p@enumi \theenumi .}\makeatother
\item {} 
\sphinxstylestrong{One buffer} LittlevGL draws the content of the screen into a buffer and sends it to the display.
The buffer can be smaller than the screen. In this case, the larger areas will be redrawn in multiple parts.
If only small areas changes (e.g. button press) then only those areas will be refreshed.

\item {} 
\sphinxstylestrong{Two non-screen-sized buffers} having two buffers LittlevGL can draw into one buffer while the content of the other buffer is sent to display in the background.
DMA or other hardware should be used to transfer the data to the display to let the CPU draw meanwhile.
This way the rendering and refreshing of the display become parallel.
Similarly to the \sphinxstyleemphasis{One buffer} LittlevGL will draw the display’s content in chunks if the buffer is smaller than the area to refresh.

\item {} 
\sphinxstylestrong{Two screen-sized buffers}.
In contrast to \sphinxstyleemphasis{Two non-screen-sized buffers} LittlevGL will always provide the whole screen’s content not only chunks.
This way the driver can simply change the address of the frame buffer to the buffer received from LittlevGL.
Therefore this method works the best when the MCU has an LCD/TFT interface and the frame buffer is just a location in the RAM.

\end{enumerate}


\subsubsection{Display driver}
\label{\detokenize{porting/display:display-driver}}
Once the buffer initialization is ready the display drivers need to be initialized. In the most simple case only the following two fields of \sphinxcode{\sphinxupquote{lv\_disp\_drv\_t}} needs to be set:
\begin{itemize}
\item {} 
\sphinxstylestrong{buffer} pointer to an initialized \sphinxcode{\sphinxupquote{lv\_disp\_buf\_t}} variable.

\item {} 
\sphinxstylestrong{flush\_cb} a callback function to copy a buffer’s content to a specific area of the display.

\end{itemize}

There are some optional data fields:
\begin{itemize}
\item {} 
\sphinxstylestrong{hor\_res} horizontal resolution of the display. (\sphinxcode{\sphinxupquote{LV\_HOR\_RES\_MAX}} by default from \sphinxstyleemphasis{lv\_conf.h})

\item {} 
\sphinxstylestrong{ver\_res} vertical resolution of the display. (\sphinxcode{\sphinxupquote{LV\_VER\_RES\_MAX}} by default from \sphinxstyleemphasis{lv\_conf.h})

\item {} 
\sphinxstylestrong{color\_chroma\_key} a color which will be drawn as transparent on chrome keyed images. \sphinxcode{\sphinxupquote{LV\_COLOR\_TRANSP}} by default from \sphinxstyleemphasis{lv\_conf.h})

\item {} 
\sphinxstylestrong{user\_data} custom user data for the driver. Its type can be modified in lv\_conf.h.

\item {} 
\sphinxstylestrong{antialiasing} use anti-aliasing (edge smoothing). \sphinxcode{\sphinxupquote{LV\_ANTIALIAS}} by default  from \sphinxstyleemphasis{lv\_conf.h}

\item {} 
\sphinxstylestrong{rotated} if \sphinxcode{\sphinxupquote{1}} swap \sphinxcode{\sphinxupquote{hor\_res}} and \sphinxcode{\sphinxupquote{ver\_res}}. LittlevGL draws in the same direction in both cases (in lines from top to bottom) so the driver also needs to be reconfigured to change the display’s fill direction.

\end{itemize}

To use a GPU the following callbacks can be used:
\begin{itemize}
\item {} 
\sphinxstylestrong{mem\_fill\_cb} fill an area with colors.

\item {} 
\sphinxstylestrong{mem\_blend\_cb} blend two buffers using opacity.

\end{itemize}

Some other optional callbacks to make easier and more optimal to work with monochrome, grayscale or other non-standard FGB displays:
\begin{itemize}
\item {} 
\sphinxstylestrong{rounder\_cb} round the coordinates of areas to redraw. E.g. a 2x2 px can be converted to 2x8.
It can be used if the display controller can refresh only areas with specific height or width (usually 8 px height with monochrome displays).

\item {} 
\sphinxstylestrong{set\_px\_cb} a custom function to write the \sphinxstyleemphasis{display buffer}.
It can be used to store the pixels in a more compact way if the display has a special color format. (e.g. 1 bit monochrome, 2  bit grayscale etc.)
This way the buffers used in \sphinxcode{\sphinxupquote{lv\_disp\_buf\_t}} can be smaller to hold only the required number of bits for the given area size.

\item {} 
\sphinxstylestrong{monitor\_cb} a callback function tell how many pixels were refreshed in how much time.

\end{itemize}

To set the fields of \sphinxstyleemphasis{lv\_disp\_drv\_t} variable it needs to be intialized with \sphinxcode{\sphinxupquote{lv\_disp\_drv\_init(\&disp\_drv)}}.
And finally to register a display for LittlevGL \sphinxcode{\sphinxupquote{lv\_disp\_drv\_register(\&disp\_drv)}} needs to be called.

All together it looks like this:

\begin{sphinxVerbatim}[commandchars=\\\{\}]
    \PYG{n}{lv\PYGZus{}disp\PYGZus{}drv\PYGZus{}t} \PYG{n}{disp\PYGZus{}drv}\PYG{p}{;}                 \PYG{c+cm}{/*A variable to hold the drivers. Can be local varaible*/}
    \PYG{n}{lv\PYGZus{}disp\PYGZus{}drv\PYGZus{}init}\PYG{p}{(}\PYG{o}{\PYGZam{}}\PYG{n}{disp\PYGZus{}drv}\PYG{p}{)}\PYG{p}{;}            \PYG{c+cm}{/*Basic initialization*/}
    \PYG{n}{disp\PYGZus{}drv}\PYG{p}{.}\PYG{n}{buffer} \PYG{o}{=} \PYG{o}{\PYGZam{}}\PYG{n}{disp\PYGZus{}buf}\PYG{p}{;}            \PYG{c+cm}{/*Set an initialized buffer*/}
    \PYG{n}{disp\PYGZus{}drv}\PYG{p}{.}\PYG{n}{flush\PYGZus{}cb} \PYG{o}{=} \PYG{n}{my\PYGZus{}flush\PYGZus{}cb}\PYG{p}{;}        \PYG{c+cm}{/*Set a flush callback to draw to the display*/}
    \PYG{n}{lv\PYGZus{}disp\PYGZus{}t} \PYG{o}{*} \PYG{n}{disp}\PYG{p}{;}
    \PYG{n}{disp} \PYG{o}{=} \PYG{n}{lv\PYGZus{}disp\PYGZus{}drv\PYGZus{}register}\PYG{p}{(}\PYG{o}{\PYGZam{}}\PYG{n}{disp\PYGZus{}drv}\PYG{p}{)}\PYG{p}{;} \PYG{c+cm}{/*Register the dirver and save the created display objects*/}
\end{sphinxVerbatim}

Here some simple examples of the callbacks:

\begin{sphinxVerbatim}[commandchars=\\\{\}]
\PYG{k+kt}{void} \PYG{n+nf}{my\PYGZus{}flush\PYGZus{}cb}\PYG{p}{(}\PYG{n}{lv\PYGZus{}disp\PYGZus{}drv\PYGZus{}t} \PYG{o}{*} \PYG{n}{disp\PYGZus{}drv}\PYG{p}{,} \PYG{k}{const} \PYG{n}{lv\PYGZus{}area\PYGZus{}t} \PYG{o}{*} \PYG{n}{area}\PYG{p}{,} \PYG{n}{lv\PYGZus{}color\PYGZus{}t} \PYG{o}{*} \PYG{n}{color\PYGZus{}p}\PYG{p}{)}
\PYG{p}{\PYGZob{}}
    \PYG{c+cm}{/*The most simple case (but also the slowest) to put all pixels to the screen one\PYGZhy{}by\PYGZhy{}one*/}
    \PYG{k+kt}{int32\PYGZus{}t} \PYG{n}{x}\PYG{p}{,} \PYG{n}{y}\PYG{p}{;}
    \PYG{k}{for}\PYG{p}{(}\PYG{n}{y} \PYG{o}{=} \PYG{n}{area}\PYG{o}{\PYGZhy{}}\PYG{o}{\PYGZgt{}}\PYG{n}{y1}\PYG{p}{;} \PYG{n}{y} \PYG{o}{\PYGZlt{}}\PYG{o}{=} \PYG{n}{area}\PYG{o}{\PYGZhy{}}\PYG{o}{\PYGZgt{}}\PYG{n}{y2}\PYG{p}{;} \PYG{n}{y}\PYG{o}{+}\PYG{o}{+}\PYG{p}{)} \PYG{p}{\PYGZob{}}
        \PYG{k}{for}\PYG{p}{(}\PYG{n}{x} \PYG{o}{=} \PYG{n}{area}\PYG{o}{\PYGZhy{}}\PYG{o}{\PYGZgt{}}\PYG{n}{x1}\PYG{p}{;} \PYG{n}{x} \PYG{o}{\PYGZlt{}}\PYG{o}{=} \PYG{n}{area}\PYG{o}{\PYGZhy{}}\PYG{o}{\PYGZgt{}}\PYG{n}{x2}\PYG{p}{;} \PYG{n}{x}\PYG{o}{+}\PYG{o}{+}\PYG{p}{)} \PYG{p}{\PYGZob{}}
            \PYG{n}{put\PYGZus{}px}\PYG{p}{(}\PYG{n}{x}\PYG{p}{,} \PYG{n}{y}\PYG{p}{,} \PYG{o}{*}\PYG{n}{color\PYGZus{}p}\PYG{p}{)}
            \PYG{n}{color\PYGZus{}p}\PYG{o}{+}\PYG{o}{+}\PYG{p}{;}
        \PYG{p}{\PYGZcb{}}
    \PYG{p}{\PYGZcb{}}

    \PYG{c+cm}{/* IMPORTANT!!!}
\PYG{c+cm}{     * Inform the graphics library that you are ready with the flushing*/}
    \PYG{n}{lv\PYGZus{}disp\PYGZus{}flush\PYGZus{}ready}\PYG{p}{(}\PYG{n}{disp}\PYG{p}{)}\PYG{p}{;}
\PYG{p}{\PYGZcb{}}

\PYG{k+kt}{void} \PYG{n+nf}{my\PYGZus{}mem\PYGZus{}fill\PYGZus{}cb}\PYG{p}{(}\PYG{n}{lv\PYGZus{}disp\PYGZus{}drv\PYGZus{}t} \PYG{o}{*} \PYG{n}{disp\PYGZus{}drv}\PYG{p}{,} \PYG{n}{lv\PYGZus{}color\PYGZus{}t} \PYG{o}{*} \PYG{n}{dest\PYGZus{}buf}\PYG{p}{,} \PYG{k}{const} \PYG{n}{lv\PYGZus{}area\PYGZus{}t} \PYG{o}{*} \PYG{n}{dest\PYGZus{}area}\PYG{p}{,} \PYG{k}{const} \PYG{n}{lv\PYGZus{}area\PYGZus{}t} \PYG{o}{*} \PYG{n}{fill\PYGZus{}area}\PYG{p}{,} \PYG{n}{lv\PYGZus{}color\PYGZus{}t} \PYG{n}{color}\PYG{p}{)}\PYG{p}{;}
\PYG{p}{\PYGZob{}}
    \PYG{c+cm}{/*It\PYGZsq{}s an example code which should be done by your GPU*/}
    \PYG{k+kt}{uint32\PYGZus{}t} \PYG{n}{x}\PYG{p}{,}\PYG{n}{y}\PYG{p}{;}
    \PYG{k}{for}\PYG{p}{(}\PYG{n}{y} \PYG{o}{=} \PYG{l+m+mi}{0}\PYG{p}{;} \PYG{n}{i} \PYG{o}{\PYGZlt{}} \PYG{n}{length}\PYG{p}{;} \PYG{n}{i}\PYG{o}{+}\PYG{o}{+}\PYG{p}{)} \PYG{p}{\PYGZob{}}
        \PYG{n}{dest}\PYG{p}{[}\PYG{n}{i}\PYG{p}{]} \PYG{o}{=} \PYG{n}{color}\PYG{p}{;}
    \PYG{p}{\PYGZcb{}}
\PYG{p}{\PYGZcb{}}

\PYG{k+kt}{void} \PYG{n}{my\PYGZus{}mem\PYGZus{}blend\PYGZus{}cb}\PYG{p}{(}\PYG{n}{lv\PYGZus{}disp\PYGZus{}drv\PYGZus{}t} \PYG{o}{*} \PYG{n}{disp\PYGZus{}drv}\PYG{p}{,} \PYG{n}{lv\PYGZus{}color\PYGZus{}t} \PYG{o}{*} \PYG{n}{dest}\PYG{p}{,} \PYG{k}{const} \PYG{n}{lv\PYGZus{}color\PYGZus{}t} \PYG{o}{*} \PYG{n}{src}\PYG{p}{,} \PYG{k+kt}{uint32\PYGZus{}t} \PYG{n}{length}\PYG{p}{,} \PYG{n}{lv\PYGZus{}opa\PYGZus{}t} \PYG{n}{opa}\PYG{p}{)}
\PYG{p}{\PYGZob{}}
    \PYG{c+cm}{/*It\PYGZsq{}s an example code which should be done by your GPU*/}
    \PYG{k+kt}{uint32\PYGZus{}t} \PYG{n}{i}\PYG{p}{;}
    \PYG{k}{for}\PYG{p}{(}\PYG{n}{i} \PYG{o}{=} \PYG{l+m+mi}{0}\PYG{p}{;} \PYG{n}{i} \PYG{o}{\PYGZlt{}} \PYG{n}{length}\PYG{p}{;} \PYG{n}{i}\PYG{o}{+}\PYG{o}{+}\PYG{p}{)} \PYG{p}{\PYGZob{}}
        \PYG{n}{dest}\PYG{p}{[}\PYG{n}{i}\PYG{p}{]} \PYG{o}{=} \PYG{n}{lv\PYGZus{}color\PYGZus{}mix}\PYG{p}{(}\PYG{n}{dest}\PYG{p}{[}\PYG{n}{i}\PYG{p}{]}\PYG{p}{,} \PYG{n}{src}\PYG{p}{[}\PYG{n}{i}\PYG{p}{]}\PYG{p}{,} \PYG{n}{opa}\PYG{p}{)}\PYG{p}{;}
    \PYG{p}{\PYGZcb{}}
\PYG{p}{\PYGZcb{}}

\PYG{k+kt}{void} \PYG{n}{my\PYGZus{}rounder\PYGZus{}cb}\PYG{p}{(}\PYG{n}{lv\PYGZus{}disp\PYGZus{}drv\PYGZus{}t} \PYG{o}{*} \PYG{n}{disp\PYGZus{}drv}\PYG{p}{,} \PYG{n}{lv\PYGZus{}area\PYGZus{}t} \PYG{o}{*} \PYG{n}{area}\PYG{p}{)}
\PYG{p}{\PYGZob{}}
  \PYG{c+cm}{/* Update the areas as needed. Can be only larger.}
\PYG{c+cm}{   * For example to always have lines 8 px hegiht:*/}
   \PYG{n}{area}\PYG{o}{\PYGZhy{}}\PYG{o}{\PYGZgt{}}\PYG{n}{y1} \PYG{o}{=} \PYG{n}{area}\PYG{o}{\PYGZhy{}}\PYG{o}{\PYGZgt{}}\PYG{n}{y1} \PYG{o}{\PYGZam{}} \PYG{l+m+mh}{0x07}\PYG{p}{;}
   \PYG{n}{area}\PYG{o}{\PYGZhy{}}\PYG{o}{\PYGZgt{}}\PYG{n}{y2} \PYG{o}{=} \PYG{p}{(}\PYG{n}{area}\PYG{o}{\PYGZhy{}}\PYG{o}{\PYGZgt{}}\PYG{n}{y2} \PYG{o}{\PYGZam{}} \PYG{l+m+mh}{0x07}\PYG{p}{)} \PYG{o}{+} \PYG{l+m+mi}{8}\PYG{p}{;} 
\PYG{p}{\PYGZcb{}}

\PYG{k+kt}{void} \PYG{n}{my\PYGZus{}set\PYGZus{}px\PYGZus{}cb}\PYG{p}{(}\PYG{n}{lv\PYGZus{}disp\PYGZus{}drv\PYGZus{}t} \PYG{o}{*} \PYG{n}{disp\PYGZus{}drv}\PYG{p}{,} \PYG{k+kt}{uint8\PYGZus{}t} \PYG{o}{*} \PYG{n}{buf}\PYG{p}{,} \PYG{n}{lv\PYGZus{}coord\PYGZus{}t} \PYG{n}{buf\PYGZus{}w}\PYG{p}{,} \PYG{n}{lv\PYGZus{}coord\PYGZus{}t} \PYG{n}{x}\PYG{p}{,} \PYG{n}{lv\PYGZus{}coord\PYGZus{}t} \PYG{n}{y}\PYG{p}{,} \PYG{n}{lv\PYGZus{}color\PYGZus{}t} \PYG{n}{color}\PYG{p}{,} \PYG{n}{lv\PYGZus{}opa\PYGZus{}t} \PYG{n}{opa}\PYG{p}{)}
\PYG{p}{\PYGZob{}}
    \PYG{c+cm}{/* Write to the buffer as required for the display. }
\PYG{c+cm}{     * Write only 1 bit for monochrome displays mapped vertically:*/}
 \PYG{n}{buf} \PYG{o}{+}\PYG{o}{=} \PYG{n}{buf\PYGZus{}w} \PYG{o}{*} \PYG{p}{(}\PYG{n}{y} \PYG{o}{\PYGZgt{}}\PYG{o}{\PYGZgt{}} \PYG{l+m+mi}{3}\PYG{p}{)} \PYG{o}{+} \PYG{n}{x}\PYG{p}{;}
 \PYG{k}{if}\PYG{p}{(}\PYG{n}{lv\PYGZus{}color\PYGZus{}brightness}\PYG{p}{(}\PYG{n}{color}\PYG{p}{)} \PYG{o}{\PYGZgt{}} \PYG{l+m+mi}{128}\PYG{p}{)} \PYG{p}{(}\PYG{o}{*}\PYG{n}{buf}\PYG{p}{)} \PYG{o}{\textbar{}}\PYG{o}{=} \PYG{p}{(}\PYG{l+m+mi}{1} \PYG{o}{\PYGZlt{}}\PYG{o}{\PYGZlt{}} \PYG{p}{(}\PYG{n}{y} \PYG{o}{\PYGZpc{}} \PYG{l+m+mi}{8}\PYG{p}{)}\PYG{p}{)}\PYG{p}{;}
 \PYG{k}{else} \PYG{p}{(}\PYG{o}{*}\PYG{n}{buf}\PYG{p}{)} \PYG{o}{\PYGZam{}}\PYG{o}{=} \PYG{o}{\PYGZti{}}\PYG{p}{(}\PYG{l+m+mi}{1} \PYG{o}{\PYGZlt{}}\PYG{o}{\PYGZlt{}} \PYG{p}{(}\PYG{n}{y} \PYG{o}{\PYGZpc{}} \PYG{l+m+mi}{8}\PYG{p}{)}\PYG{p}{)}\PYG{p}{;}
\PYG{p}{\PYGZcb{}}

\PYG{k+kt}{void} \PYG{n}{my\PYGZus{}monitor\PYGZus{}cb}\PYG{p}{(}\PYG{n}{lv\PYGZus{}disp\PYGZus{}drv\PYGZus{}t} \PYG{o}{*} \PYG{n}{disp\PYGZus{}drv}\PYG{p}{,} \PYG{k+kt}{uint32\PYGZus{}t} \PYG{n}{time}\PYG{p}{,} \PYG{k+kt}{uint32\PYGZus{}t} \PYG{n}{px}\PYG{p}{)}
\PYG{p}{\PYGZob{}}
  \PYG{n}{printf}\PYG{p}{(}\PYG{l+s}{\PYGZdq{}}\PYG{l+s}{\PYGZpc{}d px refreshed in \PYGZpc{}d ms}\PYG{l+s+se}{\PYGZbs{}n}\PYG{l+s}{\PYGZdq{}}\PYG{p}{,} \PYG{n}{time}\PYG{p}{,} \PYG{n}{ms}\PYG{p}{)}\PYG{p}{;}
\PYG{p}{\PYGZcb{}}
\end{sphinxVerbatim}


\subsubsection{Multi-display support}
\label{\detokenize{porting/display:multi-display-support}}
With LittlevGL multiple displays can be used. Just initialize multiple drivers and buffer and register them. Each display has its own screens and objects on the screens.
To get curently active scrren of a dsplay use \sphinxcode{\sphinxupquote{lv\_disp\_get\_scr\_act(disp)}} (where \sphinxcode{\sphinxupquote{disp}} is the return value of \sphinxcode{\sphinxupquote{lv\_disp\_drv\_register}}).  To set a new screen as active on a display use \sphinxcode{\sphinxupquote{lv\_disp\_set\_scr\_act(screen1)}}.

Or in a shorter form set a default display with \sphinxcode{\sphinxupquote{lv\_disp\_set\_default(disp)}} and get/set the active screen with \sphinxcode{\sphinxupquote{lv\_scr\_act()}} and \sphinxcode{\sphinxupquote{lv\_scr\_load()}}.

Learn more about screens in the \sphinxhref{/overview/objects.html\#screen-the-most-basic-parent}{Screen - the most basic parent} section.


\subsection{Input device interface}
\label{\detokenize{porting/indev:input-device-interface}}\label{\detokenize{porting/indev::doc}}
To set up an input device an \sphinxcode{\sphinxupquote{lv\_indev\_drv\_t}} variable has to be initialized:

\begin{sphinxVerbatim}[commandchars=\\\{\}]
\PYG{n}{lv\PYGZus{}indev\PYGZus{}drv\PYGZus{}t} \PYG{n}{indev\PYGZus{}drv}\PYG{p}{;}
\PYG{n}{lv\PYGZus{}indev\PYGZus{}drv\PYGZus{}init}\PYG{p}{(}\PYG{o}{\PYGZam{}}\PYG{n}{indev\PYGZus{}drv}\PYG{p}{)}\PYG{p}{;}      \PYG{c+cm}{/*Basic initialization*/}
\PYG{n}{indev\PYGZus{}drv}\PYG{p}{.}\PYG{n}{type} \PYG{o}{=}\PYG{p}{.}\PYG{p}{.}\PYG{p}{.}                 \PYG{c+cm}{/*See below.*/}
\PYG{n}{indev\PYGZus{}drv}\PYG{p}{.}\PYG{n}{read\PYGZus{}cb} \PYG{o}{=}\PYG{p}{.}\PYG{p}{.}\PYG{p}{.}              \PYG{c+cm}{/*See below.*/}
\PYG{c+cm}{/*Register the driver in LittlevGL and save the created input devide object*/}
\PYG{n}{lv\PYGZus{}indev\PYGZus{}t} \PYG{o}{*} \PYG{n}{my\PYGZus{}indev} \PYG{o}{=} \PYG{n}{lv\PYGZus{}indev\PYGZus{}drv\PYGZus{}register}\PYG{p}{(}\PYG{o}{\PYGZam{}}\PYG{n}{indev\PYGZus{}drv}\PYG{p}{)}\PYG{p}{;}
\end{sphinxVerbatim}

\sphinxstylestrong{type} can be
\begin{itemize}
\item {} 
\sphinxstylestrong{LV\_INDEV\_TYPE\_POINTER} touchpad or mouse

\item {} 
\sphinxstylestrong{LV\_INDEV\_TYPE\_KEYPAD} keyboard or keypad

\item {} 
\sphinxstylestrong{LV\_INDEV\_TYPE\_ENCODER} oncoder with left, right, push options

\item {} 
\sphinxstylestrong{LV\_INDEV\_TYPE\_BUTTON} external buttons pressing the screen

\end{itemize}

\sphinxstylestrong{read\_cb} is a function pointer which will be called periodically to report the current state of an input device.
It can also buffer data and return \sphinxcode{\sphinxupquote{false}} when no more data to be read or \sphinxcode{\sphinxupquote{true}} when the buffer is not empty.

Visit {\hyperref[\detokenize{overview/indev::doc}]{\sphinxcrossref{\DUrole{doc,doc,doc}{Input devices}}}} to learn more about input devices in general.


\subsubsection{Touchpad, mouse or any pointer}
\label{\detokenize{porting/indev:touchpad-mouse-or-any-pointer}}
Input devices which are able to click points of the screen belong to this category.

\begin{sphinxVerbatim}[commandchars=\\\{\}]
\PYG{n}{indev\PYGZus{}drv}\PYG{p}{.}\PYG{n}{type} \PYG{o}{=} \PYG{n}{LV\PYGZus{}INDEV\PYGZus{}TYPE\PYGZus{}POINTER}\PYG{p}{;}
\PYG{n}{indev\PYGZus{}drv}\PYG{p}{.}\PYG{n}{read\PYGZus{}cb} \PYG{o}{=} \PYG{n}{my\PYGZus{}input\PYGZus{}read}\PYG{p}{;}

\PYG{p}{.}\PYG{p}{.}\PYG{p}{.}

\PYG{k+kt}{bool} \PYG{n}{my\PYGZus{}input\PYGZus{}read}\PYG{p}{(}\PYG{n}{lv\PYGZus{}indev\PYGZus{}drv\PYGZus{}t} \PYG{o}{*} \PYG{n}{drv}\PYG{p}{,} \PYG{n}{lv\PYGZus{}indev\PYGZus{}data\PYGZus{}t}\PYG{o}{*}\PYG{n}{data}\PYG{p}{)}
\PYG{p}{\PYGZob{}}
    \PYG{n}{data}\PYG{o}{\PYGZhy{}}\PYG{o}{\PYGZgt{}}\PYG{n}{point}\PYG{p}{.}\PYG{n}{x} \PYG{o}{=} \PYG{n}{touchpad\PYGZus{}x}\PYG{p}{;} 
    \PYG{n}{data}\PYG{o}{\PYGZhy{}}\PYG{o}{\PYGZgt{}}\PYG{n}{point}\PYG{p}{.}\PYG{n}{y} \PYG{o}{=} \PYG{n}{touchpad\PYGZus{}y}\PYG{p}{;}
    \PYG{n}{data}\PYG{o}{\PYGZhy{}}\PYG{o}{\PYGZgt{}}\PYG{n}{state} \PYG{o}{=} \PYG{n}{LV\PYGZus{}INDEV\PYGZus{}STATE\PYGZus{}PR} \PYG{n}{or} \PYG{n}{LV\PYGZus{}INDEV\PYGZus{}STATE\PYGZus{}REL}\PYG{p}{;}
    \PYG{k}{return} \PYG{n+nb}{false}\PYG{p}{;} \PYG{c+cm}{/*No buffering now so no more data read*/}
\PYG{p}{\PYGZcb{}}
\end{sphinxVerbatim}

\begin{sphinxadmonition}{important}{Important:}
Touchpad drivers must return the last X/Y coordinates even when the state is \sphinxstyleemphasis{LV\_INDEV\_STATE\_REL}.
\end{sphinxadmonition}

To set a mouse cursor use \sphinxcode{\sphinxupquote{lv\_indev\_set\_cursor(my\_indev, \&img\_cursor)}}. (\sphinxcode{\sphinxupquote{my\_indev}} is the return value of \sphinxcode{\sphinxupquote{lv\_indev\_drv\_register}})


\subsubsection{Keypad or keyboard}
\label{\detokenize{porting/indev:keypad-or-keyboard}}
Full keyboards with all the letters or simple keypads with a few navigation buttons belong here.

To use a keyboard/keypad:
\begin{itemize}
\item {} 
Register a \sphinxcode{\sphinxupquote{read\_cb}} function with \sphinxcode{\sphinxupquote{LV\_INDEV\_TYPE\_KEYPAD}} type.

\item {} 
Enable \sphinxcode{\sphinxupquote{LV\_USE\_GROUP}} in \sphinxstyleemphasis{lv\_conf.h}

\item {} 
An object group has to be created: \sphinxcode{\sphinxupquote{lv\_group\_t * g = lv\_group\_create()}}  and objects have to be added to it with \sphinxcode{\sphinxupquote{lv\_group\_add\_obj(g, obj)}}

\item {} 
The created group has to be assigned to an input device: \sphinxcode{\sphinxupquote{lv\_indev\_set\_group(my\_indev, g)}} (\sphinxcode{\sphinxupquote{my\_indev}} is the return value of \sphinxcode{\sphinxupquote{lv\_indev\_drv\_register}})

\item {} 
Use \sphinxcode{\sphinxupquote{LV\_KEY\_...}} to navigate among the objects in the group. See \sphinxcode{\sphinxupquote{lv\_core/lv\_group.h}} for the available keys.

\end{itemize}

\begin{sphinxVerbatim}[commandchars=\\\{\}]
\PYG{n}{indev\PYGZus{}drv}\PYG{p}{.}\PYG{n}{type} \PYG{o}{=} \PYG{n}{LV\PYGZus{}INDEV\PYGZus{}TYPE\PYGZus{}KEYPAD}\PYG{p}{;}
\PYG{n}{indev\PYGZus{}drv}\PYG{p}{.}\PYG{n}{read\PYGZus{}cb} \PYG{o}{=} \PYG{n}{my\PYGZus{}input\PYGZus{}read}\PYG{p}{;}

\PYG{p}{.}\PYG{p}{.}\PYG{p}{.}

\PYG{k+kt}{bool} \PYG{n}{keyboard\PYGZus{}read}\PYG{p}{(}\PYG{n}{lv\PYGZus{}indev\PYGZus{}drv\PYGZus{}t} \PYG{o}{*} \PYG{n}{drv}\PYG{p}{,} \PYG{n}{lv\PYGZus{}indev\PYGZus{}data\PYGZus{}t}\PYG{o}{*}\PYG{n}{data}\PYG{p}{)}\PYG{p}{\PYGZob{}}
  \PYG{n}{data}\PYG{o}{\PYGZhy{}}\PYG{o}{\PYGZgt{}}\PYG{n}{key} \PYG{o}{=} \PYG{n}{last\PYGZus{}key}\PYG{p}{(}\PYG{p}{)}\PYG{p}{;}            \PYG{c+cm}{/*Get the last pressed or released key*/}
  
  \PYG{k}{if}\PYG{p}{(}\PYG{n}{key\PYGZus{}pressed}\PYG{p}{(}\PYG{p}{)}\PYG{p}{)} \PYG{n}{data}\PYG{o}{\PYGZhy{}}\PYG{o}{\PYGZgt{}}\PYG{n}{state} \PYG{o}{=} \PYG{n}{LV\PYGZus{}INDEV\PYGZus{}STATE\PYGZus{}PR}\PYG{p}{;}
  \PYG{k}{else} \PYG{n}{data}\PYG{o}{\PYGZhy{}}\PYG{o}{\PYGZgt{}}\PYG{n}{state} \PYG{o}{=} \PYG{n}{LV\PYGZus{}INDEV\PYGZus{}STATE\PYGZus{}REL}\PYG{p}{;}
  
  \PYG{k}{return} \PYG{n+nb}{false}\PYG{p}{;} \PYG{c+cm}{/*No buffering now so no more data read*/}
\PYG{p}{\PYGZcb{}}
\end{sphinxVerbatim}


\subsubsection{Encoder}
\label{\detokenize{porting/indev:encoder}}
With an encoder you can do 4 things:
\begin{enumerate}
\def\theenumi{\arabic{enumi}}
\def\labelenumi{\theenumi .}
\makeatletter\def\p@enumii{\p@enumi \theenumi .}\makeatother
\item {} 
Press its button

\item {} 
Long press its button

\item {} 
Turn left

\item {} 
Turn right

\end{enumerate}

In short, the Encoder input devices work like this:
\begin{itemize}
\item {} 
By turning the encoder you can focus on the next/previous object.

\item {} 
When you press the encoder on a simple object (like a button), it will be clicked.

\item {} 
If you press the encoder on a complex object (like a list, message box, etc.) the object will go to edit mode where by turning the encoder you can navigate inside the object.

\item {} 
To leave edit mode press long the button.

\end{itemize}

To use an \sphinxstyleemphasis{Encoder} (similarly to the \sphinxstyleemphasis{Keypads}) the objects should be added to groups.

\begin{sphinxVerbatim}[commandchars=\\\{\}]
\PYG{n}{indev\PYGZus{}drv}\PYG{p}{.}\PYG{n}{type} \PYG{o}{=} \PYG{n}{LV\PYGZus{}INDEV\PYGZus{}TYPE\PYGZus{}ENCODER}\PYG{p}{;}
\PYG{n}{indev\PYGZus{}drv}\PYG{p}{.}\PYG{n}{read\PYGZus{}cb} \PYG{o}{=} \PYG{n}{my\PYGZus{}input\PYGZus{}read}\PYG{p}{;}

\PYG{p}{.}\PYG{p}{.}\PYG{p}{.} 
 
\PYG{k+kt}{bool} \PYG{n}{encoder\PYGZus{}read}\PYG{p}{(}\PYG{n}{lv\PYGZus{}indev\PYGZus{}drv\PYGZus{}t} \PYG{o}{*} \PYG{n}{drv}\PYG{p}{,} \PYG{n}{lv\PYGZus{}indev\PYGZus{}data\PYGZus{}t}\PYG{o}{*}\PYG{n}{data}\PYG{p}{)}\PYG{p}{\PYGZob{}}
  \PYG{n}{data}\PYG{o}{\PYGZhy{}}\PYG{o}{\PYGZgt{}}\PYG{n}{enc\PYGZus{}diff} \PYG{o}{=} \PYG{n}{enc\PYGZus{}get\PYGZus{}new\PYGZus{}moves}\PYG{p}{(}\PYG{p}{)}\PYG{p}{;}
  
  \PYG{k}{if}\PYG{p}{(}\PYG{n}{enc\PYGZus{}pressed}\PYG{p}{(}\PYG{p}{)}\PYG{p}{)} \PYG{n}{data}\PYG{o}{\PYGZhy{}}\PYG{o}{\PYGZgt{}}\PYG{n}{state} \PYG{o}{=} \PYG{n}{LV\PYGZus{}INDEV\PYGZus{}STATE\PYGZus{}PR}\PYG{p}{;}
  \PYG{k}{else} \PYG{n}{data}\PYG{o}{\PYGZhy{}}\PYG{o}{\PYGZgt{}}\PYG{n}{state} \PYG{o}{=} \PYG{n}{LV\PYGZus{}INDEV\PYGZus{}STATE\PYGZus{}REL}\PYG{p}{;}
  
  \PYG{k}{return} \PYG{n+nb}{false}\PYG{p}{;} \PYG{c+cm}{/*No buffering now so no more data read*/}
\PYG{p}{\PYGZcb{}}
\end{sphinxVerbatim}


\subsubsection{Button}
\label{\detokenize{porting/indev:button}}
\sphinxstyleemphasis{Buttons} mean external “hardware” buttons next to the screen which are assigned to specific coordinates of the screen.
If a button is pressed it will simulate the pressing on the assigned coordinate. (Similarly to a touchpad)

To assign buttons to coordinates use \sphinxcode{\sphinxupquote{lv\_indev\_set\_button\_points(my\_indev, points\_array)}}.\sphinxcode{\sphinxupquote{points\_array}} should look like \sphinxcode{\sphinxupquote{const lv\_point\_t points\_array{[}{]} =  \{ \{12,30\},\{60,90\}, ...\}}}

\begin{sphinxVerbatim}[commandchars=\\\{\}]
\PYG{n}{indev\PYGZus{}drv}\PYG{p}{.}\PYG{n}{type} \PYG{o}{=} \PYG{n}{LV\PYGZus{}INDEV\PYGZus{}TYPE\PYGZus{}BUTTON}\PYG{p}{;}
\PYG{n}{indev\PYGZus{}drv}\PYG{p}{.}\PYG{n}{read\PYGZus{}cb} \PYG{o}{=} \PYG{n}{my\PYGZus{}input\PYGZus{}read}\PYG{p}{;}

\PYG{p}{.}\PYG{p}{.}\PYG{p}{.}

\PYG{k+kt}{bool} \PYG{n}{button\PYGZus{}read}\PYG{p}{(}\PYG{n}{lv\PYGZus{}indev\PYGZus{}drv\PYGZus{}t} \PYG{o}{*} \PYG{n}{drv}\PYG{p}{,} \PYG{n}{lv\PYGZus{}indev\PYGZus{}data\PYGZus{}t}\PYG{o}{*}\PYG{n}{data}\PYG{p}{)}\PYG{p}{\PYGZob{}}
    \PYG{k}{static} \PYG{k+kt}{uint32\PYGZus{}t} \PYG{n}{last\PYGZus{}btn} \PYG{o}{=} \PYG{l+m+mi}{0}\PYG{p}{;}   \PYG{c+cm}{/*Store the last pressed button*/} 
    \PYG{k+kt}{int} \PYG{n}{btn\PYGZus{}pr} \PYG{o}{=} \PYG{n}{my\PYGZus{}btn\PYGZus{}read}\PYG{p}{(}\PYG{p}{)}\PYG{p}{;}     \PYG{c+cm}{/*Get the ID (0,1,2...) of the pressed button*/}
    \PYG{k}{if}\PYG{p}{(}\PYG{n}{btn\PYGZus{}pr} \PYG{o}{\PYGZgt{}}\PYG{o}{=} \PYG{l+m+mi}{0}\PYG{p}{)} \PYG{p}{\PYGZob{}}               \PYG{c+cm}{/*Is there a button press? (E.g. \PYGZhy{}1 indicated no button was pressed)*/}
       \PYG{n}{last\PYGZus{}btn} \PYG{o}{=} \PYG{n}{btn\PYGZus{}pr}\PYG{p}{;}           \PYG{c+cm}{/*Save the ID of the pressed button*/}
       \PYG{n}{data}\PYG{o}{\PYGZhy{}}\PYG{o}{\PYGZgt{}}\PYG{n}{state} \PYG{o}{=} \PYG{n}{LV\PYGZus{}INDEV\PYGZus{}STATE\PYGZus{}PR}\PYG{p}{;}  \PYG{c+cm}{/*Set the pressed state*/}
    \PYG{p}{\PYGZcb{}} \PYG{k}{else} \PYG{p}{\PYGZob{}}
       \PYG{n}{data}\PYG{o}{\PYGZhy{}}\PYG{o}{\PYGZgt{}}\PYG{n}{state} \PYG{o}{=} \PYG{n}{LV\PYGZus{}INDEV\PYGZus{}STATE\PYGZus{}REL}\PYG{p}{;} \PYG{c+cm}{/*Set the released state*/}
    \PYG{p}{\PYGZcb{}}
  
    \PYG{n}{data}\PYG{o}{\PYGZhy{}}\PYG{o}{\PYGZgt{}}\PYG{n}{btn} \PYG{o}{=} \PYG{n}{last\PYGZus{}btn}\PYG{p}{;}            \PYG{c+cm}{/*Save the last button*/}
   
    \PYG{k}{return} \PYG{n+nb}{false}\PYG{p}{;}                    \PYG{c+cm}{/*No buffering now so no more data read*/}
\PYG{p}{\PYGZcb{}}
\end{sphinxVerbatim}


\subsection{Tick interface}
\label{\detokenize{porting/tick:tick-interface}}\label{\detokenize{porting/tick::doc}}
The LittlevGL needs a system tick to know the elapsed time for animation and other task.

You need to call the \sphinxcode{\sphinxupquote{lv\_tick\_inc(tick\_period)}} function periodically and tell the call period in milliseconds. For example, if called in every millisecond: \sphinxcode{\sphinxupquote{lv\_tick\_inc(1)}}.

\sphinxcode{\sphinxupquote{lv\_tick\_inc}} should be called in a higher priority routin than\sphinxcode{\sphinxupquote{ lv\_task\_handler()}} (e.g. in an interrupt) to precisely know the elapsed milliseconds even if the execution of \sphinxcode{\sphinxupquote{lv\_task\_handler}} takes longer time.

With FreeRTOS \sphinxcode{\sphinxupquote{lv\_tick\_inc}} can be called in \sphinxcode{\sphinxupquote{vApplicationTickHook}}.

On Linux based operation system (e.g. on Raspberry) \sphinxcode{\sphinxupquote{lv\_tick\_inc}} can be called in a thread:

\begin{sphinxVerbatim}[commandchars=\\\{\}]
\PYG{k+kt}{void} \PYG{o}{*} \PYG{n+nf}{tick\PYGZus{}thread} \PYG{p}{(}\PYG{k+kt}{void} \PYG{o}{*}\PYG{n}{args}\PYG{p}{)}
\PYG{p}{\PYGZob{}}
      \PYG{k}{while}\PYG{p}{(}\PYG{l+m+mi}{1}\PYG{p}{)} \PYG{p}{\PYGZob{}}
        \PYG{n}{usleep}\PYG{p}{(}\PYG{l+m+mi}{5}\PYG{o}{*}\PYG{l+m+mi}{1000}\PYG{p}{)}\PYG{p}{;}   \PYG{c+cm}{/*Sleep for 5 millisecond*/}
        \PYG{n}{lv\PYGZus{}tick\PYGZus{}inc}\PYG{p}{(}\PYG{l+m+mi}{5}\PYG{p}{)}\PYG{p}{;}      \PYG{c+cm}{/*Tell LittelvGL that 5 milliseconds were elapsed*/}
    \PYG{p}{\PYGZcb{}} 
\PYG{p}{\PYGZcb{}}
\end{sphinxVerbatim}


\subsection{Task Handler}
\label{\detokenize{porting/task-handler:task-handler}}\label{\detokenize{porting/task-handler::doc}}
To handle the tasks of LittlevGL you need to call \sphinxcode{\sphinxupquote{lv\_task\_handler()}} periodically in one of the followings:
\begin{itemize}
\item {} 
\sphinxstyleemphasis{while(1)} of \sphinxstyleemphasis{main()} function

\item {} 
timer interrupt periodically (low priority then \sphinxcode{\sphinxupquote{lv\_tick\_inc()}})

\item {} 
an OS task periodically

\end{itemize}

The timing is not critical but it should be about 5 milliseconds to keep the system responsive.

Example:

\begin{sphinxVerbatim}[commandchars=\\\{\}]
\PYG{k}{while}\PYG{p}{(}\PYG{l+m+mi}{1}\PYG{p}{)} \PYG{p}{\PYGZob{}}
  \PYG{n}{lv\PYGZus{}task\PYGZus{}handler}\PYG{p}{(}\PYG{p}{)}\PYG{p}{;}
  \PYG{n}{my\PYGZus{}delay\PYGZus{}ms}\PYG{p}{(}\PYG{l+m+mi}{5}\PYG{p}{)}\PYG{p}{;}
\PYG{p}{\PYGZcb{}}
\end{sphinxVerbatim}


\subsection{Sleep management}
\label{\detokenize{porting/sleep:sleep-management}}\label{\detokenize{porting/sleep::doc}}
The MCU can go to sleep when no user input happens. In this case the main \sphinxcode{\sphinxupquote{while(1)}} should look like this:

\begin{sphinxVerbatim}[commandchars=\\\{\}]
\PYG{k}{while}\PYG{p}{(}\PYG{l+m+mi}{1}\PYG{p}{)} \PYG{p}{\PYGZob{}}
  \PYG{c+cm}{/*Normal operation (no sleep) in \PYGZlt{} 1 sec inactivity*/} 
  \PYG{k}{if}\PYG{p}{(}\PYG{n}{lv\PYGZus{}disp\PYGZus{}get\PYGZus{}inactive\PYGZus{}time}\PYG{p}{(}\PYG{n+nb}{NULL}\PYG{p}{)} \PYG{o}{\PYGZlt{}} \PYG{l+m+mi}{1000}\PYG{p}{)} \PYG{p}{\PYGZob{}}
	  \PYG{n}{lv\PYGZus{}task\PYGZus{}handler}\PYG{p}{(}\PYG{p}{)}\PYG{p}{;}
  \PYG{p}{\PYGZcb{}} 
  \PYG{c+cm}{/*Sleep after 1 sec inactivity*/}
  \PYG{k}{else} \PYG{p}{\PYGZob{}}
	  \PYG{n}{timer\PYGZus{}stop}\PYG{p}{(}\PYG{p}{)}\PYG{p}{;}   \PYG{c+cm}{/*Stop the timer where lv\PYGZus{}tick\PYGZus{}inc() is called*/}
	  \PYG{n}{sleep}\PYG{p}{(}\PYG{p}{)}\PYG{p}{;}		    \PYG{c+cm}{/*Sleep the MCU*/}
  \PYG{p}{\PYGZcb{}}
  \PYG{n}{my\PYGZus{}delay\PYGZus{}ms}\PYG{p}{(}\PYG{l+m+mi}{5}\PYG{p}{)}\PYG{p}{;}
\PYG{p}{\PYGZcb{}}
\end{sphinxVerbatim}

You should also add these lines to your input device read function if a press happens:

\begin{sphinxVerbatim}[commandchars=\\\{\}]
\PYG{n}{lv\PYGZus{}tick\PYGZus{}inc}\PYG{p}{(}\PYG{n}{LV\PYGZus{}REFR\PYGZus{}PERIOD}\PYG{p}{)}\PYG{p}{;}  \PYG{c+cm}{/*Force task execution on wake\PYGZhy{}up*/}
\PYG{n}{timer\PYGZus{}start}\PYG{p}{(}\PYG{p}{)}\PYG{p}{;}                \PYG{c+cm}{/*Restart the timer where lv\PYGZus{}tick\PYGZus{}inc() is called*/}
\PYG{n}{lv\PYGZus{}task\PYGZus{}handler}\PYG{p}{(}\PYG{p}{)}\PYG{p}{;}            \PYG{c+cm}{/*Call {}`lv\PYGZus{}task\PYGZus{}handler(){}` manually to process the press event*/}
\end{sphinxVerbatim}

In addition to \sphinxcode{\sphinxupquote{lv\_disp\_get\_inactive\_time()}} you can check \sphinxcode{\sphinxupquote{lv\_anim\_count\_running()}} to see if every animations are finished.


\subsection{Use with an operating system}
\label{\detokenize{porting/os:use-with-an-operating-system}}\label{\detokenize{porting/os::doc}}
LittlevGL is \sphinxstylestrong{not thread-safe} by default. Despite it, it’s quite simple to use LittlevGL inside an operating system.

The \sphinxstylestrong{simple scenario} is to don’t use the operating system’s tasks but use \sphinxcode{\sphinxupquote{lv\_task}}s. An \sphinxstyleemphasis{lv\_task} is a function called periodically in \sphinxcode{\sphinxupquote{lv\_task\_handler}}.
In the \sphinxstyleemphasis{lv\_task} you can get the state of the sensors, buffers, etc and call LittlevGL functions to refresh the GUI.

To create an \sphinxstyleemphasis{lv\_task} use:

\begin{sphinxVerbatim}[commandchars=\\\{\}]
\PYG{n}{lv\PYGZus{}task\PYGZus{}create}\PYG{p}{(}\PYG{n}{my\PYGZus{}func}\PYG{p}{,} \PYG{n}{period\PYGZus{}ms}\PYG{p}{,} \PYG{n}{LV\PYGZus{}TASK\PYGZus{}PRIO\PYGZus{}LOWEST}\PYG{o}{/}\PYG{n}{LOW}\PYG{o}{/}\PYG{n}{MID}\PYG{o}{/}\PYG{n}{HIGH}\PYG{o}{/}\PYG{n}{HIGHEST}\PYG{p}{,} \PYG{n}{custom\PYGZus{}ptr}\PYG{p}{)}
\end{sphinxVerbatim}

If you need to \sphinxstylestrong{use real tasks or threads} you need one mutex which should be taken before the call of \sphinxcode{\sphinxupquote{lv\_task\_handler}} and released after it.
In addition, you have to use to that mutex in other tasks and threads around every LittlevGL (\sphinxcode{\sphinxupquote{lv\_...}}) related function call and code.
This way you can use LittlevGL in a real multitasking environment. Just use a mutex to avoid the concurrent calling of LittlevGL functions.


\section{Overview}
\label{\detokenize{overview/index:overview}}\label{\detokenize{overview/index::doc}}

\subsection{Objects}
\label{\detokenize{overview/objects:objects}}\label{\detokenize{overview/objects::doc}}
In the LittlevGL the \sphinxstylestrong{basic building blocks} of a user interface are the objects.
For example a {\hyperref[\detokenize{object-types/btn::doc}]{\sphinxcrossref{\DUrole{doc,doc,doc}{Button}}}}, {\hyperref[\detokenize{object-types/label::doc}]{\sphinxcrossref{\DUrole{doc,doc,doc}{Label}}}}, {\hyperref[\detokenize{object-types/img::doc}]{\sphinxcrossref{\DUrole{doc,doc,doc}{Image}}}}, {\hyperref[\detokenize{object-types/list::doc}]{\sphinxcrossref{\DUrole{doc,doc,doc}{List}}}}, {\hyperref[\detokenize{object-types/chart::doc}]{\sphinxcrossref{\DUrole{doc,doc,doc}{Chart}}}} or {\hyperref[\detokenize{object-types/ta::doc}]{\sphinxcrossref{\DUrole{doc,doc,doc}{Text area}}}}.

Check all the {\hyperref[\detokenize{object-types/index::doc}]{\sphinxcrossref{\DUrole{doc,doc,doc}{Object types}}}} here.


\subsubsection{Object attributes}
\label{\detokenize{overview/objects:object-attributes}}

\paragraph{Basic attributes}
\label{\detokenize{overview/objects:basic-attributes}}
The objects have basic attributes which are common independently from their type:
\begin{itemize}
\item {} 
Position

\item {} 
Size

\item {} 
Parent

\item {} 
Drag enable

\item {} 
Click enable etc.

\end{itemize}

You can set/get this attributes with \sphinxcode{\sphinxupquote{lv\_obj\_set\_...}} and \sphinxcode{\sphinxupquote{lv\_obj\_get\_...}} functions. For example:

\begin{sphinxVerbatim}[commandchars=\\\{\}]
\PYG{c+cm}{/*Set basic object attributes*/}
\PYG{n}{lv\PYGZus{}obj\PYGZus{}set\PYGZus{}size}\PYG{p}{(}\PYG{n}{btn1}\PYG{p}{,} \PYG{l+m+mi}{100}\PYG{p}{,} \PYG{l+m+mi}{50}\PYG{p}{)}\PYG{p}{;}	 \PYG{c+cm}{/*Button size*/}
\PYG{n}{lv\PYGZus{}obj\PYGZus{}set\PYGZus{}pos}\PYG{p}{(}\PYG{n}{btn1}\PYG{p}{,} \PYG{l+m+mi}{20}\PYG{p}{,}\PYG{l+m+mi}{30}\PYG{p}{)}\PYG{p}{;}      \PYG{c+cm}{/*Button position*/}
\end{sphinxVerbatim}

To see all the available functions visit the Base object’s {\hyperref[\detokenize{object-types/obj::doc}]{\sphinxcrossref{\DUrole{doc,doc,doc}{documentation}}}}.


\paragraph{Specific attributes}
\label{\detokenize{overview/objects:specific-attributes}}
The object types have special attributes too. For example, a slider has
\begin{itemize}
\item {} 
Min. max. values

\item {} 
Current value

\item {} 
Custom styles

\end{itemize}

For these attributes every object type have unique API functions. For example for a slider:

\begin{sphinxVerbatim}[commandchars=\\\{\}]
\PYG{c+cm}{/*Set slider specific attributes*/}
\PYG{n}{lv\PYGZus{}slider\PYGZus{}set\PYGZus{}range}\PYG{p}{(}\PYG{n}{slider1}\PYG{p}{,} \PYG{l+m+mi}{0}\PYG{p}{,} \PYG{l+m+mi}{100}\PYG{p}{)}\PYG{p}{;}	   \PYG{c+cm}{/*Set min. and max. values*/}
\PYG{n}{lv\PYGZus{}slider\PYGZus{}set\PYGZus{}value}\PYG{p}{(}\PYG{n}{slider1}\PYG{p}{,} \PYG{l+m+mi}{40}\PYG{p}{,} \PYG{n}{LV\PYGZus{}ANIM\PYGZus{}ON}\PYG{p}{)}\PYG{p}{;}	\PYG{c+cm}{/*Set the current value (position)*/}
\PYG{n}{lv\PYGZus{}slider\PYGZus{}set\PYGZus{}action}\PYG{p}{(}\PYG{n}{slider1}\PYG{p}{,} \PYG{n}{my\PYGZus{}action}\PYG{p}{)}\PYG{p}{;}     \PYG{c+cm}{/*Set a callback function*/}
\end{sphinxVerbatim}

The API of the of the object types are described in their {\hyperref[\detokenize{object-types/index::doc}]{\sphinxcrossref{\DUrole{doc,doc,doc}{Documentation}}}} but you can also check the respective header files (e.g. \sphinxstyleemphasis{lv\_objx/lv\_slider.h})


\subsubsection{Object’s working mechanisms}
\label{\detokenize{overview/objects:object-s-working-mechanisms}}

\paragraph{Parent-child structure}
\label{\detokenize{overview/objects:parent-child-structure}}
A parent object can be considered as the container of its children. Every object has exactly one parent object (except screens) but a parent can have unlimited number of children.
There is no limitation for the type of the parent but there are typical parent (e.g. button) and typical child (e.g. label) objects.


\paragraph{Moving together}
\label{\detokenize{overview/objects:moving-together}}
If the position of the parent is changed the children will move with the parent.
Therefore all positions are relative to the parent.

The (0;0) coordinates mean the objects will remain in the top left-hand corner of the parent independently from the position of the parent.

\sphinxincludegraphics{{par_child1}.png}

\begin{sphinxVerbatim}[commandchars=\\\{\}]
\PYG{n}{lv\PYGZus{}obj\PYGZus{}t} \PYG{o}{*} \PYG{n}{par} \PYG{o}{=} \PYG{n}{lv\PYGZus{}obj\PYGZus{}create}\PYG{p}{(}\PYG{n}{lv\PYGZus{}scr\PYGZus{}act}\PYG{p}{(}\PYG{p}{)}\PYG{p}{,} \PYG{n+nb}{NULL}\PYG{p}{)}\PYG{p}{;} \PYG{c+cm}{/*Create a parent object on the current screen*/}
\PYG{n}{lv\PYGZus{}obj\PYGZus{}set\PYGZus{}size}\PYG{p}{(}\PYG{n}{par}\PYG{p}{,} \PYG{l+m+mi}{100}\PYG{p}{,} \PYG{l+m+mi}{80}\PYG{p}{)}\PYG{p}{;}	                   \PYG{c+cm}{/*Set the size of the parent*/}

\PYG{n}{lv\PYGZus{}obj\PYGZus{}t} \PYG{o}{*} \PYG{n}{obj1} \PYG{o}{=} \PYG{n}{lv\PYGZus{}obj\PYGZus{}create}\PYG{p}{(}\PYG{n}{par}\PYG{p}{,} \PYG{n+nb}{NULL}\PYG{p}{)}\PYG{p}{;}	         \PYG{c+cm}{/*Create an object on the previously created parent object*/}
\PYG{n}{lv\PYGZus{}obj\PYGZus{}set\PYGZus{}pos}\PYG{p}{(}\PYG{n}{obj1}\PYG{p}{,} \PYG{l+m+mi}{10}\PYG{p}{,} \PYG{l+m+mi}{10}\PYG{p}{)}\PYG{p}{;}	                   \PYG{c+cm}{/*Set the position of the new object*/}
\end{sphinxVerbatim}

Modify the position of the parent:

\sphinxincludegraphics{{par_child2}.png}

\begin{sphinxVerbatim}[commandchars=\\\{\}]
\PYG{n}{lv\PYGZus{}obj\PYGZus{}set\PYGZus{}pos}\PYG{p}{(}\PYG{n}{par}\PYG{p}{,} \PYG{l+m+mi}{50}\PYG{p}{,} \PYG{l+m+mi}{50}\PYG{p}{)}\PYG{p}{;}	\PYG{c+cm}{/*Move the parent. The child will move with it.*/}
\end{sphinxVerbatim}

(For simplicity the adjusting of colors of the objects is not shown in the example.)


\paragraph{Visibility only on the parent}
\label{\detokenize{overview/objects:visibility-only-on-the-parent}}
If a child partially or fully out of its parent then the parts outside will not be visible.

\sphinxincludegraphics{{par_child3}.png}

\begin{sphinxVerbatim}[commandchars=\\\{\}]
\PYG{n}{lv\PYGZus{}obj\PYGZus{}set\PYGZus{}x}\PYG{p}{(}\PYG{n}{obj1}\PYG{p}{,} \PYG{o}{\PYGZhy{}}\PYG{l+m+mi}{30}\PYG{p}{)}\PYG{p}{;}	\PYG{c+cm}{/*Move the child a little bit of the parent*/}
\end{sphinxVerbatim}


\paragraph{Create - delete objects}
\label{\detokenize{overview/objects:create-delete-objects}}
In LittlevGL objects can be created and deleted dynamically in run-time.
It means only the currently created objects consume RAM.
For example, if you need a chart you can create it only when it is required and delete when its already not required.

Every objects type has its own \sphinxstylestrong{create} function with a unified prototype.
It needs two parameters:
\begin{itemize}
\item {} 
a pointer the parent object. To create a screen give \sphinxstyleemphasis{NULL} as parent.

\item {} 
optionally a pointer to an other object with the same type to copy it. Can be \sphinxstyleemphasis{NULL} to not copy an other object.

\end{itemize}

Independently from the object type a common variable type \sphinxcode{\sphinxupquote{lv\_obj\_t}} is used. This pointer can be used later to set or get the attributes of the object.

The create functions look like this:

\begin{sphinxVerbatim}[commandchars=\\\{\}]
\PYG{n}{lv\PYGZus{}obj\PYGZus{}t} \PYG{o}{*} \PYG{n}{lv\PYGZus{}} \PYG{o}{\PYGZlt{}}\PYG{n}{type}\PYG{o}{\PYGZgt{}}\PYG{n}{\PYGZus{}create}\PYG{p}{(}\PYG{n}{lv\PYGZus{}obj\PYGZus{}t} \PYG{o}{*} \PYG{n}{parent}\PYG{p}{,} \PYG{n}{lv\PYGZus{}obj\PYGZus{}t} \PYG{o}{*} \PYG{n}{copy}\PYG{p}{)}\PYG{p}{;}
\end{sphinxVerbatim}

There is a common \sphinxstylestrong{delete} function for all object types. It deletes the object and all of its children.

\begin{sphinxVerbatim}[commandchars=\\\{\}]
\PYG{k+kt}{void} \PYG{n+nf}{lv\PYGZus{}obj\PYGZus{}del}\PYG{p}{(}\PYG{n}{lv\PYGZus{}obj\PYGZus{}t} \PYG{o}{*} \PYG{n}{obj}\PYG{p}{)}\PYG{p}{;}
\end{sphinxVerbatim}

You can delete only the children of an object but leave the object itself “alive”:

\begin{sphinxVerbatim}[commandchars=\\\{\}]
\PYG{k+kt}{void} \PYG{n+nf}{lv\PYGZus{}obj\PYGZus{}clean}\PYG{p}{(}\PYG{n}{lv\PYGZus{}obj\PYGZus{}t} \PYG{o}{*} \PYG{n}{obj}\PYG{p}{)}\PYG{p}{;}
\end{sphinxVerbatim}


\paragraph{Screen \textendash{} the most basic parent}
\label{\detokenize{overview/objects:screen-the-most-basic-parent}}
The screens are special objects which have no parent object. So it is created like:

\begin{sphinxVerbatim}[commandchars=\\\{\}]
\PYG{n}{lv\PYGZus{}obj\PYGZus{}t} \PYG{o}{*} \PYG{n}{scr1} \PYG{o}{=} \PYG{n}{lv\PYGZus{}obj\PYGZus{}create}\PYG{p}{(}\PYG{n+nb}{NULL}\PYG{p}{,} \PYG{n+nb}{NULL}\PYG{p}{)}\PYG{p}{;}
\end{sphinxVerbatim}

Always there is an active screen on display. By default, the library creates and loads one.To get the currently active screen use the \sphinxcode{\sphinxupquote{lv\_scr\_act()}} function to load new one use \sphinxcode{\sphinxupquote{lv\_scr\_load(scr1)}}.

Screens can be created with any object type. For example, a {\hyperref[\detokenize{object-types/obj::doc}]{\sphinxcrossref{\DUrole{doc,doc,doc}{Base object}}}} or an image to make a wallpaper.

Screens are created on the \sphinxstyleemphasis{default display}.
The \sphinxstyleemphasis{deafult screen} is the lastly registered screen with \sphinxcode{\sphinxupquote{lv\_disp\_drv\_register}} (if there is only screen then that one) or you can explicitly selected display with \sphinxcode{\sphinxupquote{lv\_disp\_set\_default(disp)}}.
\sphinxcode{\sphinxupquote{lv\_scr\_act()}} and \sphinxcode{\sphinxupquote{lv\_scr\_load()}} operate on the currently default screen.

Visit {\hyperref[\detokenize{overview/display::doc}]{\sphinxcrossref{\DUrole{doc,doc,doc}{Multi display support}}}} to learn more.


\subsection{Layers}
\label{\detokenize{overview/layers:layers}}\label{\detokenize{overview/layers::doc}}

\subsubsection{Order of creation}
\label{\detokenize{overview/layers:order-of-creation}}
The earlier created object (and its children) will be drawn earlier (nearer to the background). In other words, the lastly created object will be on the top among its siblings.
It is very important, the order is calculated among the objects on the same level (“siblings”).

Layers can be added easily by creating 2 objects (which can be transparent). Firstly ‘A’ and secondly ‘B’. ‘A’ and every object on it will be in the background and can be covered by ‘B’ and its children.

\sphinxincludegraphics{{layers}.png}

\begin{sphinxVerbatim}[commandchars=\\\{\}]
\PYG{c+cm}{/*Create a screen*/}
\PYG{n}{lv\PYGZus{}obj\PYGZus{}t} \PYG{o}{*} \PYG{n}{scr} \PYG{o}{=} \PYG{n}{lv\PYGZus{}obj\PYGZus{}create}\PYG{p}{(}\PYG{n+nb}{NULL}\PYG{p}{,} \PYG{n+nb}{NULL}\PYG{p}{)}\PYG{p}{;}
\PYG{n}{lv\PYGZus{}scr\PYGZus{}load}\PYG{p}{(}\PYG{n}{scr}\PYG{p}{)}\PYG{p}{;}          \PYG{c+cm}{/*Load the screen*/}

\PYG{c+cm}{/*Create 2 buttons*/}
\PYG{n}{lv\PYGZus{}obj\PYGZus{}t} \PYG{o}{*} \PYG{n}{btn1} \PYG{o}{=} \PYG{n}{lv\PYGZus{}btn\PYGZus{}create}\PYG{p}{(}\PYG{n}{scr}\PYG{p}{,} \PYG{n+nb}{NULL}\PYG{p}{)}\PYG{p}{;}         \PYG{c+cm}{/*Create a button on the screen*/}
\PYG{n}{lv\PYGZus{}btn\PYGZus{}set\PYGZus{}fit}\PYG{p}{(}\PYG{n}{btn1}\PYG{p}{,} \PYG{n+nb}{true}\PYG{p}{,} \PYG{n+nb}{true}\PYG{p}{)}\PYG{p}{;}                   \PYG{c+cm}{/*Enable to automatically set the size according to the content*/}
\PYG{n}{lv\PYGZus{}obj\PYGZus{}set\PYGZus{}pos}\PYG{p}{(}\PYG{n}{btn1}\PYG{p}{,} \PYG{l+m+mi}{60}\PYG{p}{,} \PYG{l+m+mi}{40}\PYG{p}{)}\PYG{p}{;}              	   \PYG{c+cm}{/*Set the position of the button*/}

\PYG{n}{lv\PYGZus{}obj\PYGZus{}t} \PYG{o}{*} \PYG{n}{btn2} \PYG{o}{=} \PYG{n}{lv\PYGZus{}btn\PYGZus{}create}\PYG{p}{(}\PYG{n}{scr}\PYG{p}{,} \PYG{n}{btn1}\PYG{p}{)}\PYG{p}{;}         \PYG{c+cm}{/*Copy the first button*/}
\PYG{n}{lv\PYGZus{}obj\PYGZus{}set\PYGZus{}pos}\PYG{p}{(}\PYG{n}{btn2}\PYG{p}{,} \PYG{l+m+mi}{180}\PYG{p}{,} \PYG{l+m+mi}{80}\PYG{p}{)}\PYG{p}{;}                    \PYG{c+cm}{/*Set the position of the button*/}

\PYG{c+cm}{/*Add labels to the buttons*/}
\PYG{n}{lv\PYGZus{}obj\PYGZus{}t} \PYG{o}{*} \PYG{n}{label1} \PYG{o}{=} \PYG{n}{lv\PYGZus{}label\PYGZus{}create}\PYG{p}{(}\PYG{n}{btn1}\PYG{p}{,} \PYG{n+nb}{NULL}\PYG{p}{)}\PYG{p}{;}	\PYG{c+cm}{/*Create a label on the first button*/}
\PYG{n}{lv\PYGZus{}label\PYGZus{}set\PYGZus{}text}\PYG{p}{(}\PYG{n}{label1}\PYG{p}{,} \PYG{l+s}{\PYGZdq{}}\PYG{l+s}{Button 1}\PYG{l+s}{\PYGZdq{}}\PYG{p}{)}\PYG{p}{;}          	\PYG{c+cm}{/*Set the text of the label*/}

\PYG{n}{lv\PYGZus{}obj\PYGZus{}t} \PYG{o}{*} \PYG{n}{label2} \PYG{o}{=} \PYG{n}{lv\PYGZus{}label\PYGZus{}create}\PYG{p}{(}\PYG{n}{btn2}\PYG{p}{,} \PYG{n+nb}{NULL}\PYG{p}{)}\PYG{p}{;}  	\PYG{c+cm}{/*Create a label on the second button*/}
\PYG{n}{lv\PYGZus{}label\PYGZus{}set\PYGZus{}text}\PYG{p}{(}\PYG{n}{label2}\PYG{p}{,} \PYG{l+s}{\PYGZdq{}}\PYG{l+s}{Button 2}\PYG{l+s}{\PYGZdq{}}\PYG{p}{)}\PYG{p}{;}            	\PYG{c+cm}{/*Set the text of the label*/}

\PYG{c+cm}{/*Delete the second label*/}
\PYG{n}{lv\PYGZus{}obj\PYGZus{}del}\PYG{p}{(}\PYG{n}{label2}\PYG{p}{)}\PYG{p}{;}
\end{sphinxVerbatim}


\subsubsection{Bring to the foreground}
\label{\detokenize{overview/layers:bring-to-the-foreground}}
There are several ways to bring an object to the foreground:
\begin{itemize}
\item {} 
Use \sphinxcode{\sphinxupquote{lv\_obj\_set\_top(obj, true)}}. If \sphinxcode{\sphinxupquote{obj}} or any of its children is clicked then LittlevGL will automatically bring the object to the foreground.
It works similarly to the windows on PC. When a window in the background is clicked it will come to the foreground automatically.

\item {} 
Use \sphinxcode{\sphinxupquote{lv\_obj\_move\_foreground(obj)}} and \sphinxcode{\sphinxupquote{lv\_obj\_move\_background(obj)}} to explicitly tell the library to bring an object to the foreground or move to the background.

\item {} 
When \sphinxcode{\sphinxupquote{lv\_obj\_set\_parent(obj, new\_parent)}} is used \sphinxcode{\sphinxupquote{obj}} will be on the foreground on the new parent.

\end{itemize}


\subsubsection{Top and sys layer}
\label{\detokenize{overview/layers:top-and-sys-layer}}
There are two special layers called \sphinxcode{\sphinxupquote{layer\_top}} and \sphinxcode{\sphinxupquote{layer\_sys}}.
Both of them is visible and the same on all screens of a display. \sphinxcode{\sphinxupquote{layer\_top}} is on top of “normal screen” and \sphinxcode{\sphinxupquote{layer\_sys}} is on top of \sphinxcode{\sphinxupquote{layer\_top}} too.

\sphinxcode{\sphinxupquote{layer\_top}} can be used by the user to create some content visible everywhere. For example a menu bar, a pop-up, etc. If the \sphinxcode{\sphinxupquote{click}} attribute is enabled then \sphinxcode{\sphinxupquote{layer\_top}} will absorb all user click and acts as a modal.

\begin{sphinxVerbatim}[commandchars=\\\{\}]
\PYG{n}{lv\PYGZus{}obj\PYGZus{}set\PYGZus{}click}\PYG{p}{(}\PYG{n}{lv\PYGZus{}layer\PYGZus{}top}\PYG{p}{(}\PYG{p}{)}\PYG{p}{,} \PYG{n+nb}{true}\PYG{p}{)}\PYG{p}{;}
\end{sphinxVerbatim}

\sphinxcode{\sphinxupquote{layer\_sys}} is used by LittlevGL. For example, it places the mouse cursor there to be sure it’s always visible.


\subsection{Events}
\label{\detokenize{overview/events:events}}\label{\detokenize{overview/events::doc}}
In LittlevGL events are triggered if something happens which might be interesting to the user. For example an object
\begin{itemize}
\item {} 
is clicked

\item {} 
is dragged

\item {} 
its value has changed, etc.

\end{itemize}

The user can assign a callback function to an object to see these event. In the practice it looks like this:

\begin{sphinxVerbatim}[commandchars=\\\{\}]
\PYG{n}{lv\PYGZus{}obj\PYGZus{}t} \PYG{o}{*} \PYG{n}{btn} \PYG{o}{=} \PYG{n}{lv\PYGZus{}btn\PYGZus{}create}\PYG{p}{(}\PYG{n}{lv\PYGZus{}scr\PYGZus{}act}\PYG{p}{(}\PYG{p}{)}\PYG{p}{,} \PYG{n+nb}{NULL}\PYG{p}{)}\PYG{p}{;}
\PYG{n}{lv\PYGZus{}obj\PYGZus{}set\PYGZus{}event\PYGZus{}cb}\PYG{p}{(}\PYG{n}{btn}\PYG{p}{,} \PYG{n}{my\PYGZus{}event\PYGZus{}cb}\PYG{p}{)}\PYG{p}{;}   \PYG{c+cm}{/*Assign an event callback*/}

\PYG{p}{.}\PYG{p}{.}\PYG{p}{.}

\PYG{k}{static} \PYG{k+kt}{void} \PYG{n}{my\PYGZus{}event\PYGZus{}cb}\PYG{p}{(}\PYG{n}{lv\PYGZus{}obj\PYGZus{}t} \PYG{o}{*} \PYG{n}{obj}\PYG{p}{,} \PYG{n}{lv\PYGZus{}event\PYGZus{}t} \PYG{n}{event}\PYG{p}{)}
\PYG{p}{\PYGZob{}}
    \PYG{k}{switch}\PYG{p}{(}\PYG{n}{event}\PYG{p}{)} \PYG{p}{\PYGZob{}}
        \PYG{k}{case} \PYG{n+nl}{LV\PYGZus{}EVENT\PYGZus{}PRESSED}\PYG{p}{:}
            \PYG{n}{printf}\PYG{p}{(}\PYG{l+s}{\PYGZdq{}}\PYG{l+s}{Pressed}\PYG{l+s+se}{\PYGZbs{}n}\PYG{l+s}{\PYGZdq{}}\PYG{p}{)}\PYG{p}{;}
            \PYG{k}{break}\PYG{p}{;}

        \PYG{k}{case} \PYG{n+nl}{LV\PYGZus{}EVENT\PYGZus{}SHORT\PYGZus{}CLICKED}\PYG{p}{:}
            \PYG{n}{printf}\PYG{p}{(}\PYG{l+s}{\PYGZdq{}}\PYG{l+s}{Short clicked}\PYG{l+s+se}{\PYGZbs{}n}\PYG{l+s}{\PYGZdq{}}\PYG{p}{)}\PYG{p}{;}
            \PYG{k}{break}\PYG{p}{;}

        \PYG{k}{case} \PYG{n+nl}{LV\PYGZus{}EVENT\PYGZus{}CLICKED}\PYG{p}{:}
            \PYG{n}{printf}\PYG{p}{(}\PYG{l+s}{\PYGZdq{}}\PYG{l+s}{Clicked}\PYG{l+s+se}{\PYGZbs{}n}\PYG{l+s}{\PYGZdq{}}\PYG{p}{)}\PYG{p}{;}
            \PYG{k}{break}\PYG{p}{;}

        \PYG{k}{case} \PYG{n+nl}{LV\PYGZus{}EVENT\PYGZus{}LONG\PYGZus{}PRESSED}\PYG{p}{:}
            \PYG{n}{printf}\PYG{p}{(}\PYG{l+s}{\PYGZdq{}}\PYG{l+s}{Long press}\PYG{l+s+se}{\PYGZbs{}n}\PYG{l+s}{\PYGZdq{}}\PYG{p}{)}\PYG{p}{;}
            \PYG{k}{break}\PYG{p}{;}

        \PYG{k}{case} \PYG{n+nl}{LV\PYGZus{}EVENT\PYGZus{}LONG\PYGZus{}PRESSED\PYGZus{}REPEAT}\PYG{p}{:}
            \PYG{n}{printf}\PYG{p}{(}\PYG{l+s}{\PYGZdq{}}\PYG{l+s}{Long press repeat}\PYG{l+s+se}{\PYGZbs{}n}\PYG{l+s}{\PYGZdq{}}\PYG{p}{)}\PYG{p}{;}
            \PYG{k}{break}\PYG{p}{;}

        \PYG{k}{case} \PYG{n+nl}{LV\PYGZus{}EVENT\PYGZus{}RELEASED}\PYG{p}{:}
            \PYG{n}{printf}\PYG{p}{(}\PYG{l+s}{\PYGZdq{}}\PYG{l+s}{Released}\PYG{l+s+se}{\PYGZbs{}n}\PYG{l+s}{\PYGZdq{}}\PYG{p}{)}\PYG{p}{;}
            \PYG{k}{break}\PYG{p}{;}

       \PYG{c+cm}{/*Etc.*/}
\PYG{p}{\PYGZcb{}}
\end{sphinxVerbatim}

More objects can use the same \sphinxstyleemphasis{event callback}.


\subsubsection{Event types}
\label{\detokenize{overview/events:event-types}}
The following event types exist:


\paragraph{Generic events}
\label{\detokenize{overview/events:generic-events}}
Any object can receive these events independently from their type. I.e. these events are sent to Buttons, Labels, Sliders, etc.


\subparagraph{Input device related}
\label{\detokenize{overview/events:input-device-related}}
Sent when an object is pressed, released, etc by the user. They are used for \sphinxstyleemphasis{Keypad}, \sphinxstyleemphasis{Encoder} and \sphinxstyleemphasis{Button} input devices as well not only for \sphinxstyleemphasis{Pointers}. Visit the {\hyperref[\detokenize{overview/indev::doc}]{\sphinxcrossref{\DUrole{doc,doc,doc}{Overview of input devices}}}} section to learn more about them.
\begin{itemize}
\item {} 
\sphinxstylestrong{LV\_EVENT\_PRESSED} The object has been pressed

\item {} 
\sphinxstylestrong{LV\_EVENT\_PRESSING} The object is being pressed (sent continuously while pressing)

\item {} 
\sphinxstylestrong{LV\_EVENT\_PRESS\_LOST} Still pressing but slid from the objects

\item {} 
\sphinxstylestrong{LV\_EVENT\_SHORT\_CLICKED} Released before l\sphinxcode{\sphinxupquote{LV\_INDEV\_LONG\_PRESS\_TIME}}. Not called if dragged.

\item {} 
\sphinxstylestrong{LV\_EVENT\_LONG\_PRESSED}  Pressing for \sphinxcode{\sphinxupquote{LV\_INDEV\_LONG\_PRESS\_TIME}} time.  Not called if dragged.

\item {} 
\sphinxstylestrong{LV\_EVENT\_LONG\_PRESSED\_REPEAT} Called after \sphinxcode{\sphinxupquote{LV\_INDEV\_LONG\_PRESS\_TIME}} in every \sphinxcode{\sphinxupquote{LV\_INDEV\_LONG\_PRESS\_REP\_TIME}} ms.  Not called if dragged.

\item {} 
\sphinxstylestrong{LV\_EVENT\_CLICKED} Called on release if not dragged (regardless to long press)

\item {} 
\sphinxstylestrong{LV\_EVENT\_RELEASED}  Called in every case when the object has been released even if it was dragged. Not called if slid from the object while pressing and released outside of the object. In this case, \sphinxcode{\sphinxupquote{LV\_EVENT\_PRESS\_LOST}} is sent.

\end{itemize}


\subparagraph{Pointer related}
\label{\detokenize{overview/events:pointer-related}}
These events are sent only by pointer-like input devices (E.g. mouse or touchpad)
\begin{itemize}
\item {} 
\sphinxstylestrong{LV\_EVENT\_DRAG\_BEGIN} Dragging of the object has started

\item {} 
\sphinxstylestrong{LV\_EVENT\_DRAG\_END} Dragging finished (including drag throw)

\item {} 
\sphinxstylestrong{LV\_EVENT\_DRAG\_THROW\_BEGIN} Drag throw started (released after drag with “momentum”)

\end{itemize}


\subparagraph{Keypad and encoder related}
\label{\detokenize{overview/events:keypad-and-encoder-related}}
These events are sent by keypad and encoder input devices. Learn more about \sphinxstyleemphasis{Groups} in {[}overview/indev{]}(Input devices) section.
\begin{itemize}
\item {} 
\sphinxstylestrong{LV\_EVENT\_KEY} A \sphinxstyleemphasis{Key} is sent to the object. Typically when it was pressed or repeated after a long press

\item {} 
\sphinxstylestrong{LV\_EVENT\_FOCUSED} The object is focused in its group

\item {} 
\sphinxstylestrong{LV\_EVENT\_DEFOCUSED} The object is defocused in its group

\end{itemize}


\subparagraph{General events}
\label{\detokenize{overview/events:general-events}}
Other general events sent by the library.
\begin{itemize}
\item {} 
\sphinxstylestrong{LV\_EVENT\_DELETE} The object is being deleted. Free the related user-allocated data.

\end{itemize}


\paragraph{Special events}
\label{\detokenize{overview/events:special-events}}
These events are specific to a partiual object type.
\begin{itemize}
\item {} 
\sphinxstylestrong{LV\_EVENT\_VALUE\_CHANGED} The object value has changed (e.g. for a {\hyperref[\detokenize{object-types/slider::doc}]{\sphinxcrossref{\DUrole{doc,doc,doc}{Slider}}}})

\item {} 
\sphinxstylestrong{LV\_EVENT\_INSERT} Something is inserted to the object. (Typically to a {\hyperref[\detokenize{object-types/ta::doc}]{\sphinxcrossref{\DUrole{doc,doc,doc}{Text area}}}})

\item {} 
\sphinxstylestrong{LV\_EVENT\_APPLY}  “Ok”, “Apply” or similar specific button has clicked. (Typically from a {\hyperref[\detokenize{object-types/kb::doc}]{\sphinxcrossref{\DUrole{doc,doc,doc}{Keyboard}}}} object)

\item {} 
\sphinxstylestrong{LV\_EVENT\_CANCEL} “Close”, “Cancel” or similar specific button has clicked. (Typically from a {\hyperref[\detokenize{object-types/kb::doc}]{\sphinxcrossref{\DUrole{doc,doc,doc}{Keyboard}}}} object)

\item {} 
\sphinxstylestrong{LV\_EVENT\_REFRESH} Query to refresh the object. Never sent by the library but can be sent by the user.

\end{itemize}

To see exactly which events are used by an object type see the particular {\hyperref[\detokenize{object-types/index::doc}]{\sphinxcrossref{\DUrole{doc,doc,doc}{Object type’s documentation}}}}.


\subsubsection{Custom data}
\label{\detokenize{overview/events:custom-data}}
Some events might contain custom data. For example \sphinxcode{\sphinxupquote{LV\_EVENT\_VALUE\_CHANGED}} in some cases tells the new value. For more info see the particular {\hyperref[\detokenize{object-types/index::doc}]{\sphinxcrossref{\DUrole{doc,doc,doc}{Object type’s documentation}}}}.
The get the custom data in the event callback use \sphinxcode{\sphinxupquote{lv\_event\_get\_data()}}.


\subsubsection{Send events manually}
\label{\detokenize{overview/events:send-events-manually}}
To manually send events to an object use \sphinxcode{\sphinxupquote{lv\_event\_send(obj, LV\_EVENT\_..., \&custom\_data)}}.

It can be used for example to manually close a message box by simualting a button press:

\begin{sphinxVerbatim}[commandchars=\\\{\}]
\PYG{c+cm}{/*Simulate the press of the first button (indexes start from zero)*/}
\PYG{k+kt}{uint32\PYGZus{}t} \PYG{n}{btn\PYGZus{}id} \PYG{o}{=} \PYG{l+m+mi}{0}\PYG{p}{;}
\PYG{n}{lv\PYGZus{}event\PYGZus{}send}\PYG{p}{(}\PYG{n}{mbox}\PYG{p}{,} \PYG{n}{LV\PYGZus{}EVENT\PYGZus{}VALUE\PYGZus{}CHANGED}\PYG{p}{,} \PYG{o}{\PYGZam{}}\PYG{n}{btn\PYGZus{}id}\PYG{p}{)}\PYG{p}{;}
\end{sphinxVerbatim}

Or to ask refresh in a generic way.

\begin{sphinxVerbatim}[commandchars=\\\{\}]
\PYG{n}{lv\PYGZus{}event\PYGZus{}send}\PYG{p}{(}\PYG{n}{label}\PYG{p}{,} \PYG{n}{LV\PYGZus{}EVENT\PYGZus{}REFRESH}\PYG{p}{,} \PYG{n+nb}{NULL}\PYG{p}{)}\PYG{p}{;}
\end{sphinxVerbatim}


\subsection{Styles}
\label{\detokenize{overview/styles:styles}}\label{\detokenize{overview/styles::doc}}
\sphinxstyleemphasis{Styles} are used to set the appearance of the objects. A style is a structure variable with attributes like colors, paddings, opacity, etc.

There is common style type called \sphinxstylestrong{lv\_style\_t} for every obejct type.

Styles are assigned to the objects and by setting the fields of the \sphinxcode{\sphinxupquote{lv\_style\_t}} variables you can influence the appearance of the objects using that style.

\begin{sphinxadmonition}{important}{Important:}
The objects store only a pointer to a style so the style cannot be a local variable which is destroyed after the function exists. \sphinxstylestrong{You should use static, global or dynamically allocated variables.}
\end{sphinxadmonition}

\begin{sphinxVerbatim}[commandchars=\\\{\}]
\PYG{n}{lv\PYGZus{}style\PYGZus{}t} \PYG{n}{style\PYGZus{}1}\PYG{p}{;}             \PYG{c+cm}{/*OK! Global variables for styles are fine*/}
\PYG{k}{static} \PYG{n}{lv\PYGZus{}style\PYGZus{}t} \PYG{n}{style\PYGZus{}2}\PYG{p}{;}      \PYG{c+cm}{/*OK! Static variables outside the functions are fine*/}
\PYG{k+kt}{void} \PYG{n+nf}{my\PYGZus{}screen\PYGZus{}create}\PYG{p}{(}\PYG{k+kt}{void}\PYG{p}{)} 
\PYG{p}{\PYGZob{}}
  \PYG{k}{static} \PYG{n}{lv\PYGZus{}style\PYGZus{}t} \PYG{n}{style\PYGZus{}3}\PYG{p}{;}    \PYG{c+cm}{/*OK! Static variables in the functions are fine*/}
  \PYG{n}{lv\PYGZus{}style\PYGZus{}t} \PYG{n}{style\PYGZus{}4}\PYG{p}{;}           \PYG{c+cm}{/*WRONG! Styles can\PYGZsq{}t be local variables*/}

  \PYG{p}{.}\PYG{p}{.}\PYG{p}{.}
\PYG{p}{\PYGZcb{}}
\end{sphinxVerbatim}


\subsubsection{Use the styles}
\label{\detokenize{overview/styles:use-the-styles}}
The objects have a \sphinxstyleemphasis{Main style} which determines the appearance of their background or main part. However, object types can have additional styles too.

Some object has only one style. E.g.
\begin{itemize}
\item {} 
Label

\item {} 
Image

\item {} 
Line, etc

\end{itemize}

For example, a slider has 3 styles:
\begin{itemize}
\item {} 
Background (main style)

\item {} 
Indicator

\item {} 
Know

\end{itemize}

Every object type has its own style set/get functions. For example

\begin{sphinxVerbatim}[commandchars=\\\{\}]
\PYG{k}{const} \PYG{n}{lv\PYGZus{}style\PYGZus{}t} \PYG{o}{*} \PYG{n}{btn\PYGZus{}style} \PYG{o}{=} \PYG{n}{lv\PYGZus{}btn\PYGZus{}get\PYGZus{}style}\PYG{p}{(}\PYG{n}{btn}\PYG{p}{,} \PYG{n}{LV\PYGZus{}BTN\PYGZus{}STYLE\PYGZus{}REL}\PYG{p}{)}\PYG{p}{;}
\PYG{n}{lv\PYGZus{}btn\PYGZus{}set\PYGZus{}style}\PYG{p}{(}\PYG{n}{btn}\PYG{p}{,} \PYG{n}{LV\PYGZus{}BTN\PYGZus{}STYLE\PYGZus{}REL}\PYG{p}{,} \PYG{o}{\PYGZam{}}\PYG{n}{new\PYGZus{}style}\PYG{p}{)}\PYG{p}{;}
\end{sphinxVerbatim}

The styles supported by an object type  (\sphinxstyleemphasis{LV\_\textless{}OBJ\_TYPE\textgreater{}\sphinxstyleemphasis{STYLE}\textless{}STYLE\_TYPE\textgreater{}}) see the documentation of the particular {\hyperref[\detokenize{object-types/index::doc}]{\sphinxcrossref{\DUrole{doc,doc,doc}{Object type}}}}.

If you \sphinxstylestrong{modify a style which is already used} by one or more objects then the objects have to be notified about the style is changed. You have two options to do that:

\begin{sphinxVerbatim}[commandchars=\\\{\}]
\PYG{c+cm}{/*Notify an object about its style is modified*/}
\PYG{k+kt}{void} \PYG{n+nf}{lv\PYGZus{}obj\PYGZus{}refresh\PYGZus{}style}\PYG{p}{(}\PYG{n}{lv\PYGZus{}obj\PYGZus{}t} \PYG{o}{*} \PYG{n}{obj}\PYG{p}{)}\PYG{p}{;} 

 \PYG{c+cm}{/*Notify all objects with a given style. (NULL to notify all objects)*/}
\PYG{k+kt}{void} \PYG{n+nf}{lv\PYGZus{}obj\PYGZus{}report\PYGZus{}style\PYGZus{}mod}\PYG{p}{(}\PYG{k+kt}{void} \PYG{o}{*} \PYG{n}{style}\PYG{p}{)}\PYG{p}{;}
\end{sphinxVerbatim}

\sphinxcode{\sphinxupquote{lv\_obj\_report\_style\_mod}} can refresh only the \sphinxstyleemphasis{Main styles}.


\subsubsection{Inherit styles}
\label{\detokenize{overview/styles:inherit-styles}}
If the \sphinxstyleemphasis{Main style} of an object is \sphinxcode{\sphinxupquote{NULL}} then its style will be inherited from its parent’s style. It makes easier to create a consistent design. Don’t forget a style describes a lot of properties at the same time. So for example, if you set a button’s style and create a label on it with \sphinxcode{\sphinxupquote{NULL}} style then the label will be rendered according to the button’s style. In other words, the button makes sure its children will look well on it.

Setting the \sphinxcode{\sphinxupquote{glass}} style property will prevent inheriting that style. You should use it if the style is transparent so that its children use colors and others from its grandparent.


\subsubsection{Style properties}
\label{\detokenize{overview/styles:style-properties}}
A style has 5 main parts: common, body, text, image and line. An object will use those fields which are relevant to it.
For example, \sphinxstyleemphasis{Lines} don’t care about the \sphinxstyleemphasis{letter\_space}.
To see which fields are used by an object type see their {\hyperref[\detokenize{object-types/index::doc}]{\sphinxcrossref{\DUrole{doc,doc,doc}{Documentation}}}}.

The fields of a style structure are the followings:


\paragraph{Common properties}
\label{\detokenize{overview/styles:common-properties}}\begin{itemize}
\item {} 
\sphinxstylestrong{glass} 1: Do not inherit this style

\end{itemize}


\paragraph{Body style properties}
\label{\detokenize{overview/styles:body-style-properties}}
Used by the rectangle-like objects
\begin{itemize}
\item {} 
\sphinxstylestrong{body.main\_color} Main color (top color)

\item {} 
\sphinxstylestrong{body.grad\_color} Gradient color (bottom color)

\item {} 
\sphinxstylestrong{body.radius} Corner radius. (set to \sphinxcode{\sphinxupquote{LV\_RADIUS\_CIRCLE}} to draw circle)

\item {} 
\sphinxstylestrong{body.opa} Opacity (0..255 or \sphinxstyleemphasis{LV\_OPA\_TRANSP}, \sphinxstyleemphasis{LV\_OPA\_10}, \sphinxstyleemphasis{LV\_OPA\_20} … \sphinxstyleemphasis{LV\_OPA\_COVER})

\item {} 
\sphinxstylestrong{body.border.color} Border color

\item {} 
\sphinxstylestrong{body.border.width} Border width

\item {} 
\sphinxstylestrong{body.border.part} Border parts (\sphinxstyleemphasis{LV\_BORDER\_LEFT/RIGHT/TOP/BOTTOM/FULL} or ‘OR’ed values)

\item {} 
\sphinxstylestrong{body.border.opa} Border opacity (0..255 or \sphinxstyleemphasis{LV\_OPA\_TRANSP}, \sphinxstyleemphasis{LV\_OPA\_10}, \sphinxstyleemphasis{LV\_OPA\_20} … \sphinxstyleemphasis{LV\_OPA\_COVER})

\item {} 
\sphinxstylestrong{body.shadow.color} Shadow color

\item {} 
\sphinxstylestrong{body.shadow.width} Shadow width

\item {} 
\sphinxstylestrong{body.shadow.type} Shadow type (\sphinxstyleemphasis{LV\_SHADOW\_BOTTOM/FULL})

\item {} 
\sphinxstylestrong{body.padding.top} Top padding

\item {} 
\sphinxstylestrong{body.padding.bottom} Bottom padding

\item {} 
\sphinxstylestrong{body.padding.left} Left padding

\item {} 
\sphinxstylestrong{body.padding.right} Right padding

\item {} 
\sphinxstylestrong{body.padding.inner} Inner padding (between content elements or children)

\end{itemize}


\paragraph{Text style properties}
\label{\detokenize{overview/styles:text-style-properties}}
Used by the objects which show texts
\begin{itemize}
\item {} 
\sphinxstylestrong{text.color} Text color

\item {} 
\sphinxstylestrong{text.sel\_color} Selected text color

\item {} 
\sphinxstylestrong{text.font} Pointer to a font

\item {} 
\sphinxstylestrong{text.opa} Text opacity (0..255 or LV\_OPA\_TRANSP, LV\_OPA\_10, LV\_OPA\_20 … LV\_OPA\_COVER*)

\item {} 
\sphinxstylestrong{text.letter\_space} Letter space

\item {} 
\sphinxstylestrong{text.line\_space} Line space

\end{itemize}


\paragraph{Image style properties}
\label{\detokenize{overview/styles:image-style-properties}}
Used by image-like objects or icons on objects
\begin{itemize}
\item {} 
\sphinxstylestrong{image.color} Color for image re-coloring based on the pixels brightness

\item {} 
\sphinxstylestrong{image.intense} Re-color intensity (0..255 or \sphinxstyleemphasis{LV\_OPA\_TRANSP}, \sphinxstyleemphasis{LV\_OPA\_10}, \sphinxstyleemphasis{LV\_OPA\_20} … \sphinxstyleemphasis{LV\_OPA\_COVER})

\item {} 
\sphinxstylestrong{image.opa} Image opacity (0..255 or \sphinxstyleemphasis{LV\_OPA\_TRANSP}, \sphinxstyleemphasis{LV\_OPA\_10}, \sphinxstyleemphasis{LV\_OPA\_20} … \sphinxstyleemphasis{LV\_OPA\_COVER})

\end{itemize}


\paragraph{Line style properties}
\label{\detokenize{overview/styles:line-style-properties}}
Used by objects containing lines or line-like elements
\begin{itemize}
\item {} 
\sphinxstylestrong{line.color} Line color

\item {} 
\sphinxstylestrong{line.width} Line width

\item {} 
\sphinxstylestrong{line.opa} Line opacity (0..255 or \sphinxstyleemphasis{LV\_OPA\_TRANSP}, \sphinxstyleemphasis{LV\_OPA\_10}, \sphinxstyleemphasis{LV\_OPA\_20} … \sphinxstyleemphasis{LV\_OPA\_COVER})

\end{itemize}


\subsubsection{Built-in styles}
\label{\detokenize{overview/styles:built-in-styles}}
There are several built-in styles in the library:

\sphinxincludegraphics{{style-built-in}.png}

As you can see there is a style for screens, for buttons, plain and pretty styles and transparent styles as well.

The \sphinxcode{\sphinxupquote{lv\_style\_transp}}, \sphinxcode{\sphinxupquote{lv\_style\_transp\_fit}} and \sphinxcode{\sphinxupquote{lv\_style\_transp\_tight}} differ only in paddings: for \sphinxcode{\sphinxupquote{lv\_style\_transp\_tight}} all padings are zero, for \sphinxcode{\sphinxupquote{lv\_style\_transp\_fit}} only hor and ver paddings are zero but has inner padding.

\begin{sphinxadmonition}{important}{Important:}
Transparent built-in styles have \sphinxstyleemphasis{glass = 1} by default which means these styles (e.g. their colors) won’t be inherited by children.
\end{sphinxadmonition}

The built in styles are global \sphinxcode{\sphinxupquote{lv\_style\_t}} variables. You can use them like:

\begin{sphinxVerbatim}[commandchars=\\\{\}]
\PYG{n}{lv\PYGZus{}btn\PYGZus{}set\PYGZus{}style}\PYG{p}{(}\PYG{n}{obj}\PYG{p}{,} \PYG{n}{LV\PYGZus{}BTN\PYGZus{}STYLE\PYGZus{}REL}\PYG{p}{,} \PYG{o}{\PYGZam{}}\PYG{n}{lv\PYGZus{}style\PYGZus{}btn\PYGZus{}rel}\PYG{p}{)}
\end{sphinxVerbatim}

You can modify the built-in styles or you can create new styles. When creating new styles it is recommended to first copy a built-in style to be sure all fields are initialized with a proper value. The \sphinxcode{\sphinxupquote{lv\_style\_copy(\&dest\_style, \&src\_style)}} can be used to copy styles.


\subsubsection{Style animations}
\label{\detokenize{overview/styles:style-animations}}
You change the styles with animations using \sphinxcode{\sphinxupquote{lv\_style\_anim\_...()}} function. Two styles are required to represent the \sphinxstyleemphasis{start} and \sphinxstyleemphasis{end} state, and a third style which will be animated. Here is an example to show how it works.

\begin{sphinxVerbatim}[commandchars=\\\{\}]
\PYG{n}{lv\PYGZus{}anim\PYGZus{}t} \PYG{n}{a}\PYG{p}{;}
\PYG{n}{lv\PYGZus{}style\PYGZus{}anim\PYGZus{}init}\PYG{p}{(}\PYG{o}{\PYGZam{}}\PYG{n}{a}\PYG{p}{)}\PYG{p}{;}                                                 \PYG{c+cm}{/*A basic initialization*/}
\PYG{n}{lv\PYGZus{}style\PYGZus{}anim\PYGZus{}set\PYGZus{}styles}\PYG{p}{(}\PYG{o}{\PYGZam{}}\PYG{n}{a}\PYG{p}{,} \PYG{o}{\PYGZam{}}\PYG{n}{style\PYGZus{}to\PYGZus{}anim}\PYG{p}{,} \PYG{o}{\PYGZam{}}\PYG{n}{style\PYGZus{}start}\PYG{p}{,} \PYG{o}{\PYGZam{}}\PYG{n}{style\PYGZus{}end}\PYG{p}{)}\PYG{p}{;} \PYG{c+cm}{/*Set the styles to use*/}
\PYG{n}{lv\PYGZus{}style\PYGZus{}anim\PYGZus{}set\PYGZus{}time}\PYG{p}{(}\PYG{o}{\PYGZam{}}\PYG{n}{a}\PYG{p}{,} \PYG{n}{duration}\PYG{p}{,} \PYG{n}{delay}\PYG{p}{)}\PYG{p}{;}                           \PYG{c+cm}{/*Set the duration and delay*/}
\PYG{n}{lv\PYGZus{}style\PYGZus{}anim\PYGZus{}create}\PYG{p}{(}\PYG{o}{\PYGZam{}}\PYG{n}{a}\PYG{p}{)}\PYG{p}{;}                                               \PYG{c+cm}{/*Create the animation*/}
\end{sphinxVerbatim}

To see the whole API of style animations see \sphinxcode{\sphinxupquote{lv\_core/lv\_style.h}}.

Here you can learn more about the {\hyperref[\detokenize{overview/animations::doc}]{\sphinxcrossref{\DUrole{doc,doc,doc}{Animations}}}}.


\subsubsection{Style example}
\label{\detokenize{overview/styles:style-example}}
The example below demonstrates the usage of styles.

\sphinxincludegraphics{{style-example}.png}Styles usage example in LittlevGL Embedded Graphics Library

\begin{sphinxVerbatim}[commandchars=\\\{\}]
\PYG{c+cm}{/*Create a style*/}
\PYG{k}{static} \PYG{n}{lv\PYGZus{}style\PYGZus{}t} \PYG{n}{style1}\PYG{p}{;}
\PYG{n}{lv\PYGZus{}style\PYGZus{}copy}\PYG{p}{(}\PYG{o}{\PYGZam{}}\PYG{n}{style1}\PYG{p}{,} \PYG{o}{\PYGZam{}}\PYG{n}{lv\PYGZus{}style\PYGZus{}plain}\PYG{p}{)}\PYG{p}{;}    \PYG{c+cm}{/*Copy a built\PYGZhy{}in style to initialize the new style*/}
\PYG{n}{style1}\PYG{p}{.}\PYG{n}{body}\PYG{p}{.}\PYG{n}{main\PYGZus{}color} \PYG{o}{=} \PYG{n}{LV\PYGZus{}COLOR\PYGZus{}WHITE}\PYG{p}{;}
\PYG{n}{style1}\PYG{p}{.}\PYG{n}{body}\PYG{p}{.}\PYG{n}{grad\PYGZus{}color} \PYG{o}{=} \PYG{n}{LV\PYGZus{}COLOR\PYGZus{}BLUE}\PYG{p}{;}
\PYG{n}{style1}\PYG{p}{.}\PYG{n}{body}\PYG{p}{.}\PYG{n}{radius} \PYG{o}{=} \PYG{l+m+mi}{10}\PYG{p}{;}
\PYG{n}{style1}\PYG{p}{.}\PYG{n}{body}\PYG{p}{.}\PYG{n}{border}\PYG{p}{.}\PYG{n}{color} \PYG{o}{=} \PYG{n}{LV\PYGZus{}COLOR\PYGZus{}GRAY}\PYG{p}{;}
\PYG{n}{style1}\PYG{p}{.}\PYG{n}{body}\PYG{p}{.}\PYG{n}{border}\PYG{p}{.}\PYG{n}{width} \PYG{o}{=} \PYG{l+m+mi}{2}\PYG{p}{;}
\PYG{n}{style1}\PYG{p}{.}\PYG{n}{body}\PYG{p}{.}\PYG{n}{border}\PYG{p}{.}\PYG{n}{opa} \PYG{o}{=} \PYG{n}{LV\PYGZus{}OPA\PYGZus{}50}\PYG{p}{;}
\PYG{n}{style1}\PYG{p}{.}\PYG{n}{body}\PYG{p}{.}\PYG{n}{padding}\PYG{p}{.}\PYG{n}{left} \PYG{o}{=} \PYG{l+m+mi}{5}\PYG{p}{;}            \PYG{c+cm}{/*Horizontal padding, used by the bar indicator below*/}
\PYG{n}{style1}\PYG{p}{.}\PYG{n}{body}\PYG{p}{.}\PYG{n}{padding}\PYG{p}{.}\PYG{n}{right} \PYG{o}{=} \PYG{l+m+mi}{5}\PYG{p}{;}
\PYG{n}{style1}\PYG{p}{.}\PYG{n}{body}\PYG{p}{.}\PYG{n}{padding}\PYG{p}{.}\PYG{n}{top} \PYG{o}{=} \PYG{l+m+mi}{5}\PYG{p}{;}            \PYG{c+cm}{/*Vertical padding, used by the bar indicator below*/}
\PYG{n}{style1}\PYG{p}{.}\PYG{n}{body}\PYG{p}{.}\PYG{n}{padding}\PYG{p}{.}\PYG{n}{bottom} \PYG{o}{=} \PYG{l+m+mi}{5}\PYG{p}{;}
\PYG{n}{style1}\PYG{p}{.}\PYG{n}{text}\PYG{p}{.}\PYG{n}{color} \PYG{o}{=} \PYG{n}{LV\PYGZus{}COLOR\PYGZus{}RED}\PYG{p}{;}

\PYG{c+cm}{/*Create a simple object*/}
\PYG{n}{lv\PYGZus{}obj\PYGZus{}t} \PYG{o}{*}\PYG{n}{obj1} \PYG{o}{=} \PYG{n}{lv\PYGZus{}obj\PYGZus{}create}\PYG{p}{(}\PYG{n}{lv\PYGZus{}scr\PYGZus{}act}\PYG{p}{(}\PYG{p}{)}\PYG{p}{,} \PYG{n+nb}{NULL}\PYG{p}{)}\PYG{p}{;}
\PYG{n}{lv\PYGZus{}obj\PYGZus{}set\PYGZus{}style}\PYG{p}{(}\PYG{n}{obj1}\PYG{p}{,} \PYG{o}{\PYGZam{}}\PYG{n}{style1}\PYG{p}{)}\PYG{p}{;}                        \PYG{c+cm}{/*Apply the created style*/}
\PYG{n}{lv\PYGZus{}obj\PYGZus{}set\PYGZus{}pos}\PYG{p}{(}\PYG{n}{obj1}\PYG{p}{,} \PYG{l+m+mi}{20}\PYG{p}{,} \PYG{l+m+mi}{20}\PYG{p}{)}\PYG{p}{;}                           \PYG{c+cm}{/*Set the position*/}

\PYG{c+cm}{/*Create a label on the object. The label\PYGZsq{}s style is NULL by default*/}
\PYG{n}{lv\PYGZus{}obj\PYGZus{}t} \PYG{o}{*}\PYG{n}{label} \PYG{o}{=} \PYG{n}{lv\PYGZus{}label\PYGZus{}create}\PYG{p}{(}\PYG{n}{obj1}\PYG{p}{,} \PYG{n+nb}{NULL}\PYG{p}{)}\PYG{p}{;}
\PYG{n}{lv\PYGZus{}obj\PYGZus{}align}\PYG{p}{(}\PYG{n}{label}\PYG{p}{,} \PYG{n+nb}{NULL}\PYG{p}{,} \PYG{n}{LV\PYGZus{}ALIGN\PYGZus{}CENTER}\PYG{p}{,} \PYG{l+m+mi}{0}\PYG{p}{,} \PYG{l+m+mi}{0}\PYG{p}{)}\PYG{p}{;}       \PYG{c+cm}{/*Align the label to the middle*/}

\PYG{c+cm}{/*Create a bar*/}
\PYG{n}{lv\PYGZus{}obj\PYGZus{}t} \PYG{o}{*}\PYG{n}{bar1} \PYG{o}{=} \PYG{n}{lv\PYGZus{}bar\PYGZus{}create}\PYG{p}{(}\PYG{n}{lv\PYGZus{}scr\PYGZus{}act}\PYG{p}{(}\PYG{p}{)}\PYG{p}{,} \PYG{n+nb}{NULL}\PYG{p}{)}\PYG{p}{;}
\PYG{n}{lv\PYGZus{}bar\PYGZus{}set\PYGZus{}style}\PYG{p}{(}\PYG{n}{bar1}\PYG{p}{,} \PYG{n}{LV\PYGZus{}BAR\PYGZus{}STYLE\PYGZus{}INDIC}\PYG{p}{,} \PYG{o}{\PYGZam{}}\PYG{n}{style1}\PYG{p}{)}\PYG{p}{;}    \PYG{c+cm}{/*Modify the indicator\PYGZsq{}s style*/}
\PYG{n}{lv\PYGZus{}bar\PYGZus{}set\PYGZus{}value}\PYG{p}{(}\PYG{n}{bar1}\PYG{p}{,} \PYG{l+m+mi}{70}\PYG{p}{)}\PYG{p}{;}                             \PYG{c+cm}{/*Set the bar\PYGZsq{}s value*/}
\end{sphinxVerbatim}


\subsubsection{Themes}
\label{\detokenize{overview/styles:themes}}
To create styles for your GUI is challenging because you need a deeper understanding of the library and you need to have some design skills. In addition, it takes a lot of time to create so many styles.

To speed up the design part themes are introduced. A theme is a style collection which contains the required styles for every object type. For example 5 styles for buttons to describe their 5 possible states.
Check the \sphinxhref{https://littlevgl.com/themes}{Existing themes} or try some in the \sphinxhref{https://littlevgl.com/live-demo}{Live demo} section.

To be more specific a theme is a structure variable which contains a lot of lv\_style\_t * fields. For buttons:

\begin{sphinxVerbatim}[commandchars=\\\{\}]
\PYG{n}{theme}\PYG{p}{.}\PYG{n}{btn}\PYG{p}{.}\PYG{n}{rel}       \PYG{c+cm}{/*Released button style*/}
\PYG{n}{theme}\PYG{p}{.}\PYG{n}{btn}\PYG{p}{.}\PYG{n}{pr}        \PYG{c+cm}{/*Pressed button style*/}
\PYG{n}{theme}\PYG{p}{.}\PYG{n}{btn}\PYG{p}{.}\PYG{n}{tgl\PYGZus{}rel}   \PYG{c+cm}{/*Toggled released button style*/}
\PYG{n}{theme}\PYG{p}{.}\PYG{n}{btn}\PYG{p}{.}\PYG{n}{tgl\PYGZus{}pr}    \PYG{c+cm}{/*Toggled pressed button style*/}
\PYG{n}{theme}\PYG{p}{.}\PYG{n}{btn}\PYG{p}{.}\PYG{n}{ina}       \PYG{c+cm}{/*Inactive button style*/}
\end{sphinxVerbatim}

A theme can initialized by: \sphinxcode{\sphinxupquote{lv\_theme\_\textless{}name\textgreater{}\_init(hue, font)}}. Where \sphinxcode{\sphinxupquote{hue}} is a Hue value from \sphinxhref{https://en.wikipedia.org/wiki/Hue}{HSV color space} (0..360) and \sphinxcode{\sphinxupquote{font}} is the font applied in the theme (\sphinxcode{\sphinxupquote{NULL}} to use the \sphinxcode{\sphinxupquote{LV\_FONT\_DEFAULT}})

When a theme is initialized its styles can be used like this:

\sphinxincludegraphics{{theme-example}.png}

\begin{sphinxVerbatim}[commandchars=\\\{\}]
\PYG{c+cm}{/*Create a default slider*/}
\PYG{n}{lv\PYGZus{}obj\PYGZus{}t} \PYG{o}{*}\PYG{n}{slider} \PYG{o}{=} \PYG{n}{lv\PYGZus{}slider\PYGZus{}create}\PYG{p}{(}\PYG{n}{lv\PYGZus{}scr\PYGZus{}act}\PYG{p}{(}\PYG{p}{)}\PYG{p}{,} \PYG{n+nb}{NULL}\PYG{p}{)}\PYG{p}{;}
\PYG{n}{lv\PYGZus{}slider\PYGZus{}set\PYGZus{}value}\PYG{p}{(}\PYG{n}{slider}\PYG{p}{,} \PYG{l+m+mi}{70}\PYG{p}{)}\PYG{p}{;}
\PYG{n}{lv\PYGZus{}obj\PYGZus{}set\PYGZus{}pos}\PYG{p}{(}\PYG{n}{slider}\PYG{p}{,} \PYG{l+m+mi}{10}\PYG{p}{,} \PYG{l+m+mi}{10}\PYG{p}{)}\PYG{p}{;}

\PYG{c+cm}{/*Initialize the alien theme with a redish hue*/}
\PYG{n}{lv\PYGZus{}theme\PYGZus{}t} \PYG{o}{*}\PYG{n}{th} \PYG{o}{=} \PYG{n}{lv\PYGZus{}theme\PYGZus{}alien\PYGZus{}init}\PYG{p}{(}\PYG{l+m+mi}{10}\PYG{p}{,} \PYG{n+nb}{NULL}\PYG{p}{)}\PYG{p}{;}

\PYG{c+cm}{/*Create a new slider and apply the themes styles*/}
\PYG{n}{slider} \PYG{o}{=} \PYG{n}{lv\PYGZus{}slider\PYGZus{}create}\PYG{p}{(}\PYG{n}{lv\PYGZus{}scr\PYGZus{}act}\PYG{p}{(}\PYG{p}{)}\PYG{p}{,} \PYG{n+nb}{NULL}\PYG{p}{)}\PYG{p}{;}
\PYG{n}{lv\PYGZus{}slider\PYGZus{}set\PYGZus{}value}\PYG{p}{(}\PYG{n}{slider}\PYG{p}{,} \PYG{l+m+mi}{70}\PYG{p}{)}\PYG{p}{;}
\PYG{n}{lv\PYGZus{}obj\PYGZus{}set\PYGZus{}pos}\PYG{p}{(}\PYG{n}{slider}\PYG{p}{,} \PYG{l+m+mi}{10}\PYG{p}{,} \PYG{l+m+mi}{50}\PYG{p}{)}\PYG{p}{;}
\PYG{n}{lv\PYGZus{}slider\PYGZus{}set\PYGZus{}style}\PYG{p}{(}\PYG{n}{slider}\PYG{p}{,} \PYG{n}{LV\PYGZus{}SLIDER\PYGZus{}STYLE\PYGZus{}BG}\PYG{p}{,} \PYG{n}{th}\PYG{o}{\PYGZhy{}}\PYG{o}{\PYGZgt{}}\PYG{n}{slider}\PYG{p}{.}\PYG{n}{bg}\PYG{p}{)}\PYG{p}{;}
\PYG{n}{lv\PYGZus{}slider\PYGZus{}set\PYGZus{}style}\PYG{p}{(}\PYG{n}{slider}\PYG{p}{,} \PYG{n}{LV\PYGZus{}SLIDER\PYGZus{}STYLE\PYGZus{}INDIC}\PYG{p}{,} \PYG{n}{th}\PYG{o}{\PYGZhy{}}\PYG{o}{\PYGZgt{}}\PYG{n}{slider}\PYG{p}{.}\PYG{n}{indic}\PYG{p}{)}\PYG{p}{;}
\PYG{n}{lv\PYGZus{}slider\PYGZus{}set\PYGZus{}style}\PYG{p}{(}\PYG{n}{slider}\PYG{p}{,} \PYG{n}{LV\PYGZus{}SLIDER\PYGZus{}STYLE\PYGZus{}KNOB}\PYG{p}{,} \PYG{n}{th}\PYG{o}{\PYGZhy{}}\PYG{o}{\PYGZgt{}}\PYG{n}{slider}\PYG{p}{.}\PYG{n}{knob}\PYG{p}{)}\PYG{p}{;}
\end{sphinxVerbatim}

You can ask the library to automatically apply the styles from a theme when you create new objects. To do this use \sphinxcode{\sphinxupquote{lv\_theme\_set\_current(th)}};

\begin{sphinxVerbatim}[commandchars=\\\{\}]
\PYG{c+cm}{/*Initialize the alien theme with a redish hue*/}
\PYG{n}{lv\PYGZus{}theme\PYGZus{}t} \PYG{o}{*}\PYG{n}{th} \PYG{o}{=} \PYG{n}{lv\PYGZus{}theme\PYGZus{}alien\PYGZus{}init}\PYG{p}{(}\PYG{l+m+mi}{10}\PYG{p}{,} \PYG{n+nb}{NULL}\PYG{p}{)}\PYG{p}{;}
\PYG{n}{lv\PYGZus{}theme\PYGZus{}set\PYGZus{}current}\PYG{p}{(}\PYG{n}{th}\PYG{p}{)}\PYG{p}{;}

\PYG{c+cm}{/*Craete a slider. It will use the style from teh current theme.*/}
\PYG{n}{slider} \PYG{o}{=} \PYG{n}{lv\PYGZus{}slider\PYGZus{}create}\PYG{p}{(}\PYG{n}{lv\PYGZus{}scr\PYGZus{}act}\PYG{p}{(}\PYG{p}{)}\PYG{p}{,} \PYG{n+nb}{NULL}\PYG{p}{)}\PYG{p}{;}
\end{sphinxVerbatim}

Themes can be enabled or disabled one by on in \sphinxcode{\sphinxupquote{lv\_conf.h}}.


\paragraph{Live update}
\label{\detokenize{overview/styles:live-update}}
By defult if \sphinxcode{\sphinxupquote{lv\_theme\_set\_current(th)}} is called again it won’t refresh the styles of the exisitin objects. To enable live update of themes enable \sphinxcode{\sphinxupquote{LV\_THEME\_LIVE\_UPDATE}} in \sphinxcode{\sphinxupquote{lv\_conf.h}}.

Live update will update only those objects whose style are from the theme, i.e. created after the first call of \sphinxcode{\sphinxupquote{lv\_theme\_set\_current(th)}} or the styles were set manually


\subsection{Input devices}
\label{\detokenize{overview/indev:input-devices}}\label{\detokenize{overview/indev::doc}}
Assume porting is already read


\subsubsection{Run time config}
\label{\detokenize{overview/indev:run-time-config}}

\subsubsection{Pointers}
\label{\detokenize{overview/indev:pointers}}
cursor

API


\subsubsection{Keypad and encoder}
\label{\detokenize{overview/indev:keypad-and-encoder}}

\paragraph{Groups}
\label{\detokenize{overview/indev:groups}}

\paragraph{Keys}
\label{\detokenize{overview/indev:keys}}
ENTER special


\paragraph{Keypads}
\label{\detokenize{overview/indev:keypads}}

\paragraph{Encoders}
\label{\detokenize{overview/indev:encoders}}
Edit and navigation mode

API


\subsection{Displays}
\label{\detokenize{overview/display:displays}}\label{\detokenize{overview/display::doc}}
Assume porting is already read


\subsubsection{Multi-display support}
\label{\detokenize{overview/display:multi-display-support}}
How why?

API


\subsubsection{Run time config}
\label{\detokenize{overview/display:run-time-config}}

\subsection{Fonts}
\label{\detokenize{overview/fonts:fonts}}\label{\detokenize{overview/fonts::doc}}
In LittlevGL fonts are collections of bitmaps and other informations required to render the images of the letters (glyph). A font is stored in a \sphinxstylestrong{lv\_font\_t} variable and can be set it in style’s \sphinxstyleemphasis{text.font} field. For example:

\begin{sphinxVerbatim}[commandchars=\\\{\}]
\PYG{n}{my\PYGZus{}style}\PYG{p}{.}\PYG{n}{text}\PYG{p}{.}\PYG{n}{font} \PYG{o}{=} \PYG{o}{\PYGZam{}}\PYG{n}{lv\PYGZus{}font\PYGZus{}roboto\PYGZus{}28}\PYG{p}{;}  \PYG{c+cm}{/*Set a larger font*/}
\end{sphinxVerbatim}

The fonts have a \sphinxstylestrong{bpp (Bit-Per-Pixel)} property. It shows how many bits are used to describe a pixel in the font. The value stored for a pixel determines the pixel’s opacity.
This way with higher \sphinxstyleemphasis{bpp} the edges of the letter can be smoother. The possible \sphinxstyleemphasis{bpp} values are 1, 2, 4 and 8 (higher value means better quality).

The \sphinxstyleemphasis{bpp} also affects the required memory size to store the font. E.g. \sphinxstyleemphasis{bpp = 4} makes the font \textasciitilde{}4 times greater compared to \sphinxstyleemphasis{bpp = 1}.


\subsubsection{Unicode support}
\label{\detokenize{overview/fonts:unicode-support}}
LittlevGL supports \sphinxstylestrong{UTF-8} encoded Unicode characters.
You need to configure your editor to save your code/text as UTF-8 (usually this the default) and be sure \sphinxcode{\sphinxupquote{LV\_TXT\_ENC}} is set to \sphinxcode{\sphinxupquote{LV\_TXT\_ENC\_UTF8}} in \sphinxstyleemphasis{lv\_conf.h}. (This is the default value)

To test it try

\begin{sphinxVerbatim}[commandchars=\\\{\}]
\PYG{n}{lv\PYGZus{}obj\PYGZus{}t} \PYG{o}{*} \PYG{n}{label1} \PYG{o}{=} \PYG{n}{lv\PYGZus{}label\PYGZus{}create}\PYG{p}{(}\PYG{n}{lv\PYGZus{}scr\PYGZus{}act}\PYG{p}{(}\PYG{p}{)}\PYG{p}{,} \PYG{n+nb}{NULL}\PYG{p}{)}\PYG{p}{;}
\PYG{n}{lv\PYGZus{}label\PYGZus{}set\PYGZus{}text}\PYG{p}{(}\PYG{n}{label1}\PYG{p}{,} \PYG{n}{LV\PYGZus{}SYMBOL\PYGZus{}OK}\PYG{p}{)}\PYG{p}{;}
\end{sphinxVerbatim}

If all works well a \(\checkmark\) character should be displayed.


\subsubsection{Built-in fonts}
\label{\detokenize{overview/fonts:built-in-fonts}}
There are several built-in fonts in different sizes which can be enabled in lv\_conf.h by \sphinxstyleemphasis{LV\_FONT\_…} defines:
\begin{itemize}
\item {} 
\sphinxcode{\sphinxupquote{LV\_FONT\_ROBOTO\_12}} 12 px

\item {} 
\sphinxcode{\sphinxupquote{LV\_FONT\_ROBOTO\_16}} 16 px

\item {} 
\sphinxcode{\sphinxupquote{LV\_FONT\_ROBOTO\_22}} 22 px

\item {} 
\sphinxcode{\sphinxupquote{LV\_FONT\_ROBOTO\_28}} 28 px

\end{itemize}

The built-in fonts are \sphinxstylestrong{global variables} with names like \sphinxcode{\sphinxupquote{lv\_font\_roboto\_16}} for 16 px hight font. To use them in a style just add a pointer to a font variable like shown above.

The built-in fonts have \sphinxstyleemphasis{bpp = 4}, contains the ASCII characters and uses the \sphinxhref{https://fonts.google.com/specimen/Roboto}{Roboto} font.

In addition to the ASCII rangle, the following symbols are also added to the built-in fonts from the \sphinxhref{https://fontawesome.com/}{FontAwesome} font.

\sphinxincludegraphics{{symbols}.png}

The symbols can be used as:

\begin{sphinxVerbatim}[commandchars=\\\{\}]
\PYG{n}{lv\PYGZus{}label\PYGZus{}set\PYGZus{}text}\PYG{p}{(}\PYG{n}{my\PYGZus{}label}\PYG{p}{,} \PYG{n}{LV\PYGZus{}SYMBOL\PYGZus{}OK}\PYG{p}{)}\PYG{p}{;}
\end{sphinxVerbatim}

Or with together with strings:

\begin{sphinxVerbatim}[commandchars=\\\{\}]
\PYG{n}{lv\PYGZus{}label\PYGZus{}set\PYGZus{}text}\PYG{p}{(}\PYG{n}{my\PYGZus{}label}\PYG{p}{,} \PYG{n}{LV\PYGZus{}SYMBOL\PYGZus{}OK} \PYG{l+s}{\PYGZdq{}}\PYG{l+s}{Apply}\PYG{l+s}{\PYGZdq{}}\PYG{p}{)}\PYG{p}{;}
\end{sphinxVerbatim}

Or more symbols together:

\begin{sphinxVerbatim}[commandchars=\\\{\}]
\PYG{n}{lv\PYGZus{}label\PYGZus{}set\PYGZus{}text}\PYG{p}{(}\PYG{n}{my\PYGZus{}label}\PYG{p}{,} \PYG{n}{LV\PYGZus{}SYMBOL\PYGZus{}OK} \PYG{n}{LV\PYGZus{}SYMBOL\PYGZus{}WIFI} \PYG{n}{LV\PYGZus{}SYMBOL\PYGZus{}PLAY}\PYG{p}{)}\PYG{p}{;}
\end{sphinxVerbatim}


\subsubsection{Add new font}
\label{\detokenize{overview/fonts:add-new-font}}
There are several ways to add a new font to your project:
\begin{enumerate}
\def\theenumi{\arabic{enumi}}
\def\labelenumi{\theenumi .}
\makeatletter\def\p@enumii{\p@enumi \theenumi .}\makeatother
\item {} 
The most simple way is to use the \sphinxhref{hhttps://littlevgl.com/font\_conv\_new}{Online font converter}. Just set the parameters, click the \sphinxstyleemphasis{Convert} button, copy the font to your project and use it.

\item {} 
Use the \sphinxhref{https://github.com/littlevgl/lv\_font\_conv}{Offline font converter}. (Requires Node.js to be installed)

\item {} 
If you want to create something like the built-in fonts (Roboto font and symbols) but in different size and/or ranges you can use the \sphinxcode{\sphinxupquote{built\_in\_font\_gen.py}} script in \sphinxcode{\sphinxupquote{lvgl/scripts/built\_in\_font}} folder.
(It requires Python and \sphinxcode{\sphinxupquote{lv\_font\_conv}} to be installed)

\end{enumerate}

To declare the font in a file use \sphinxcode{\sphinxupquote{LV\_FONT\_DECLARE(my\_font\_name)}}.

To make to font globally available add them to \sphinxcode{\sphinxupquote{LV\_FONT\_CUSTOM\_DECLARE}} in \sphinxstyleemphasis{lv\_conf.h}.


\subsubsection{Add new symbols}
\label{\detokenize{overview/fonts:add-new-symbols}}
The built-in symbols are created from \sphinxhref{https://fontawesome.com/}{FontAwesome} font. To add new symbols from the FontAwesome font do the following steps:
\begin{enumerate}
\def\theenumi{\arabic{enumi}}
\def\labelenumi{\theenumi .}
\makeatletter\def\p@enumii{\p@enumi \theenumi .}\makeatother
\item {} 
Search symbol on \sphinxurl{https://fontawesome.com}. For example the \sphinxhref{https://fontawesome.com}{USB symbol}

\item {} 
Open the \sphinxhref{https://littlevgl.com/font\_conv\_new}{Online font converter} add \sphinxhref{https://littlevgl.com/tools/FontAwesome.ttf}{FontAwesome.ttf} and add the Unicode ID of the symbol to the range field. E.g.\sphinxcode{\sphinxupquote{ 0xf287}} for the USB symbol.
More symbols can be enumerated with \sphinxcode{\sphinxupquote{,}}.

\item {} 
Convert the font and copy it to your project.

\item {} 
Convert the Unicode value to UTF8. You can do it e.g.on \sphinxhref{http://www.ltg.ed.ac.uk/\%7Erichard/utf-8.cgi?input=f287\&mode=hex}{this site}. For \sphinxcode{\sphinxupquote{0xf287}} the \sphinxstyleemphasis{Hex UTF-8 bytes} are \sphinxcode{\sphinxupquote{EF 8A 87}}.

\item {} 
Create a \sphinxcode{\sphinxupquote{define}} from the UTF8 values: \sphinxcode{\sphinxupquote{\#define MY\_USB\_SYMBOL "\textbackslash{}xEF\textbackslash{}x8A\textbackslash{}x87"}}

\item {} 
Use the symbol as the built-in symbols. \sphinxcode{\sphinxupquote{lv\_label\_set\_text(label, MY\_USB\_SYMBOL)}}

\end{enumerate}


\subsubsection{Add a new font engine}
\label{\detokenize{overview/fonts:add-a-new-font-engine}}
LittlevGL’s font interface is designed to be very flexible.
You don’t need to use LittlevGL’s internal font engine but you can add your own.
For example use \sphinxhref{https://www.freetype.org/}{FreeType} to real-time render glyphs from TTF fonts or use an external flash to store the font’s bitmap and read them when the library need them.

To do this a custom \sphinxcode{\sphinxupquote{lv\_font\_t}} variable needs to be created:

\begin{sphinxVerbatim}[commandchars=\\\{\}]
\PYG{c+cm}{/*Describe the properties of a font*/}
\PYG{n}{lv\PYGZus{}font\PYGZus{}t} \PYG{n}{my\PYGZus{}font}\PYG{p}{;}
\PYG{n}{my\PYGZus{}font}\PYG{p}{.}\PYG{n}{get\PYGZus{}glyph\PYGZus{}dsc} \PYG{o}{=} \PYG{n}{my\PYGZus{}get\PYGZus{}glyph\PYGZus{}dsc\PYGZus{}cb}\PYG{p}{;}        \PYG{c+cm}{/*Set a callback to get info about gylphs*/}
\PYG{n}{my\PYGZus{}font}\PYG{p}{.}\PYG{n}{get\PYGZus{}glyph\PYGZus{}bitmap} \PYG{o}{=} \PYG{n}{my\PYGZus{}get\PYGZus{}glyph\PYGZus{}bitmap\PYGZus{}cb}\PYG{p}{;}  \PYG{c+cm}{/*Set a callback to get bitmap of a glyp*/}
\PYG{n}{my\PYGZus{}font}\PYG{p}{.}\PYG{n}{line\PYGZus{}height} \PYG{o}{=} \PYG{n}{height}\PYG{p}{;}                       \PYG{c+cm}{/*The real line height where any text fits*/}
\PYG{n}{my\PYGZus{}font}\PYG{p}{.}\PYG{n}{base\PYGZus{}line} \PYG{o}{=} \PYG{n}{base\PYGZus{}line}\PYG{p}{;}                      \PYG{c+cm}{/*Base line measured from the top of line\PYGZus{}height*/}
\PYG{n}{my\PYGZus{}font}\PYG{p}{.}\PYG{n}{dsc} \PYG{o}{=} \PYG{n}{something\PYGZus{}required}\PYG{p}{;}                   \PYG{c+cm}{/*Store any implementation specific data here*/}
\PYG{n}{my\PYGZus{}font}\PYG{p}{.}\PYG{n}{user\PYGZus{}data} \PYG{o}{=} \PYG{n}{user\PYGZus{}data}\PYG{p}{;}                      \PYG{c+cm}{/*Optionally some extra user data*/}

\PYG{p}{.}\PYG{p}{.}\PYG{p}{.}

\PYG{c+cm}{/* Get info about glyph of {}`unicode\PYGZus{}letter{}` in {}`font{}` font. }
\PYG{c+cm}{ * Store the result in {}`dsc\PYGZus{}out{}`.}
\PYG{c+cm}{ * The next letter ({}`unicode\PYGZus{}letter\PYGZus{}next{}`) might be used to calculate the width required by this glyph (kerning)}
\PYG{c+cm}{ */}
\PYG{k+kt}{bool} \PYG{n}{my\PYGZus{}get\PYGZus{}glyph\PYGZus{}dsc\PYGZus{}cb}\PYG{p}{(}\PYG{k}{const} \PYG{n}{lv\PYGZus{}font\PYGZus{}t} \PYG{o}{*} \PYG{n}{font}\PYG{p}{,} \PYG{n}{lv\PYGZus{}font\PYGZus{}glyph\PYGZus{}dsc\PYGZus{}t} \PYG{o}{*} \PYG{n}{dsc\PYGZus{}out}\PYG{p}{,} \PYG{k+kt}{uint32\PYGZus{}t} \PYG{n}{unicode\PYGZus{}letter}\PYG{p}{,} \PYG{k+kt}{uint32\PYGZus{}t} \PYG{n}{unicode\PYGZus{}letter\PYGZus{}next}\PYG{p}{)}
\PYG{p}{\PYGZob{}}
    \PYG{c+cm}{/*Your code here*/}
  
    \PYG{c+cm}{/* Sotore the result.}
\PYG{c+cm}{     * For example ... }
\PYG{c+cm}{     */}
    \PYG{n}{dsc\PYGZus{}out}\PYG{o}{\PYGZhy{}}\PYG{o}{\PYGZgt{}}\PYG{n}{adv\PYGZus{}w} \PYG{o}{=} \PYG{l+m+mi}{12}\PYG{p}{;}        \PYG{c+cm}{/*Horizontal space required by the glyph in [px]*/}
    \PYG{n}{dsc\PYGZus{}out}\PYG{o}{\PYGZhy{}}\PYG{o}{\PYGZgt{}}\PYG{n}{box\PYGZus{}h} \PYG{o}{=} \PYG{l+m+mi}{8}\PYG{p}{;}         \PYG{c+cm}{/*Height of the bitmap in [px]*/}
    \PYG{n}{dsc\PYGZus{}out}\PYG{o}{\PYGZhy{}}\PYG{o}{\PYGZgt{}}\PYG{n}{box\PYGZus{}w} \PYG{o}{=} \PYG{l+m+mi}{6}\PYG{p}{;}         \PYG{c+cm}{/*Width of the bitmap in [px]*/}
    \PYG{n}{dsc\PYGZus{}out}\PYG{o}{\PYGZhy{}}\PYG{o}{\PYGZgt{}}\PYG{n}{ofs\PYGZus{}x} \PYG{o}{=} \PYG{l+m+mi}{0}\PYG{p}{;}         \PYG{c+cm}{/*X offset of the bitmap in [pf]*/}
    \PYG{n}{dsc\PYGZus{}out}\PYG{o}{\PYGZhy{}}\PYG{o}{\PYGZgt{}}\PYG{n}{ofs\PYGZus{}y} \PYG{o}{=} \PYG{l+m+mi}{3}\PYG{p}{;}         \PYG{c+cm}{/*Y ofset of the bitmap measored from the as line*/}
    \PYG{n}{dsc\PYGZus{}out}\PYG{o}{\PYGZhy{}}\PYG{o}{\PYGZgt{}}\PYG{n}{bpp}   \PYG{o}{=} \PYG{l+m+mi}{2}\PYG{p}{;}         \PYG{c+cm}{/*Bit per pixel: 1/2/4/8*/}
    
    \PYG{k}{return} \PYG{n+nb}{true}\PYG{p}{;}                \PYG{c+cm}{/*true: glyph found; false: glyph was not found*/}
\PYG{p}{\PYGZcb{}}


\PYG{c+cm}{/* Get the btmap of {}`unicode\PYGZus{}letter{}` from {}`font{}`. */}
\PYG{k}{const} \PYG{k+kt}{uint8\PYGZus{}t} \PYG{o}{*} \PYG{n}{my\PYGZus{}get\PYGZus{}glyph\PYGZus{}bitmap\PYGZus{}cb}\PYG{p}{(}\PYG{k}{const} \PYG{n}{lv\PYGZus{}font\PYGZus{}t} \PYG{o}{*} \PYG{n}{font}\PYG{p}{,} \PYG{k+kt}{uint32\PYGZus{}t} \PYG{n}{unicode\PYGZus{}letter}\PYG{p}{)}
\PYG{p}{\PYGZob{}}
    \PYG{c+cm}{/* Your code here */}
    
    \PYG{c+cm}{/* The bitmap should be a continuous bitstream where }
\PYG{c+cm}{     * each pixel is represented by {}`bpp{}` bits */} 
    
    \PYG{k}{return} \PYG{n}{bitmap}\PYG{p}{;}    \PYG{c+cm}{/*Or NULL if not found*/}
\PYG{p}{\PYGZcb{}}
\end{sphinxVerbatim}


\subsection{Images}
\label{\detokenize{overview/images:images}}\label{\detokenize{overview/images::doc}}

\subsection{File system}
\label{\detokenize{overview/file-system:file-system}}\label{\detokenize{overview/file-system::doc}}

\subsection{Animations}
\label{\detokenize{overview/animations:animations}}\label{\detokenize{overview/animations::doc}}
You can automatically change the value of a variable between a start and an end value using animations.
The animation will happen by the periodical call of an “animator” function with the corresponding value parameter.

The \sphinxstyleemphasis{animator} functions has the following prototype:

\begin{sphinxVerbatim}[commandchars=\\\{\}]
\PYG{k+kt}{void} \PYG{n+nf}{func}\PYG{p}{(}\PYG{k+kt}{void} \PYG{o}{*} \PYG{n}{var}\PYG{p}{,} \PYG{n}{lv\PYGZus{}anim\PYGZus{}var\PYGZus{}t} \PYG{n}{value}\PYG{p}{)}\PYG{p}{;}
\end{sphinxVerbatim}

This prototype is compatible with the majority of the \sphinxstyleemphasis{set} function of LittlevGL. For example \sphinxcode{\sphinxupquote{lv\_obj\_set\_x(obj, value)}} or \sphinxcode{\sphinxupquote{lv\_obj\_set\_width(obj, value)}}


\subsubsection{Create an animation}
\label{\detokenize{overview/animations:create-an-animation}}
To create an animation an \sphinxcode{\sphinxupquote{lv\_anim\_t}} variable has to be initialized and configured with \sphinxcode{\sphinxupquote{lv\_anim\_set\_...()}} functions.

\begin{sphinxVerbatim}[commandchars=\\\{\}]
\PYG{n}{lv\PYGZus{}anim\PYGZus{}t} \PYG{n}{a}\PYG{p}{;}
\PYG{n}{lv\PYGZus{}anim\PYGZus{}set\PYGZus{}exec\PYGZus{}cb}\PYG{p}{(}\PYG{o}{\PYGZam{}}\PYG{n}{a}\PYG{p}{,} \PYG{n}{btn1}\PYG{p}{,} \PYG{n}{lv\PYGZus{}obj\PYGZus{}set\PYGZus{}x}\PYG{p}{)}\PYG{p}{;}    \PYG{c+cm}{/*Set the animator function and variable to animate*/} 
\PYG{n}{lv\PYGZus{}anim\PYGZus{}set\PYGZus{}time}\PYG{p}{(}\PYG{o}{\PYGZam{}}\PYG{n}{a}\PYG{p}{,} \PYG{n}{duration}\PYG{p}{,} \PYG{n}{delay}\PYG{p}{)}\PYG{p}{;}
\PYG{n}{lv\PYGZus{}anim\PYGZus{}set\PYGZus{}values}\PYG{p}{(}\PYG{o}{\PYGZam{}}\PYG{n}{a}\PYG{p}{,} \PYG{n}{start}\PYG{p}{,} \PYG{n}{end}\PYG{p}{)}\PYG{p}{;}             \PYG{c+cm}{/*Set start and end values. E.g. 0, 150*/}
\PYG{n}{lv\PYGZus{}anim\PYGZus{}set\PYGZus{}path\PYGZus{}cb}\PYG{p}{(}\PYG{o}{\PYGZam{}}\PYG{n}{a}\PYG{p}{,} \PYG{n}{lv\PYGZus{}anim\PYGZus{}path\PYGZus{}linear}\PYG{p}{)}\PYG{p}{;}   \PYG{c+cm}{/*Set path from {}`lv\PYGZus{}anim\PYGZus{}path\PYGZus{}...{}` functions or a custom one.*/}
\PYG{n}{lv\PYGZus{}anim\PYGZus{}set\PYGZus{}ready\PYGZus{}cb}\PYG{p}{(}\PYG{o}{\PYGZam{}}\PYG{n}{a}\PYG{p}{,} \PYG{n}{ready\PYGZus{}cb}\PYG{p}{)}\PYG{p}{;}             \PYG{c+cm}{/*Set a callback to call then anaimtion is ready. (Optional)*/}
\PYG{n}{lv\PYGZus{}anim\PYGZus{}set\PYGZus{}playback}\PYG{p}{(}\PYG{o}{\PYGZam{}}\PYG{n}{a}\PYG{p}{,} \PYG{n}{wait\PYGZus{}time}\PYG{p}{)}\PYG{p}{;}            \PYG{c+cm}{/*Enable playback of teh animation with {}`wait\PYGZus{}time{}` delay*/}
\PYG{n}{lv\PYGZus{}anim\PYGZus{}set\PYGZus{}repeat}\PYG{p}{(}\PYG{o}{\PYGZam{}}\PYG{n}{a}\PYG{p}{,} \PYG{n}{wait\PYGZus{}time}\PYG{p}{)}\PYG{p}{;}              \PYG{c+cm}{/*Enable repeate of teh animation with {}`wait\PYGZus{}time{}` delay. Can be compined with playback*/}

\PYG{n}{lv\PYGZus{}anim\PYGZus{}create}\PYG{p}{(}\PYG{o}{\PYGZam{}}\PYG{n}{a}\PYG{p}{)}\PYG{p}{;}                             \PYG{c+cm}{/*Start the animation*/}
\end{sphinxVerbatim}

You can apply \sphinxstylestrong{multiple different animations} on the same variable at the same time.
For example animate the x and y coordinates with \sphinxcode{\sphinxupquote{lv\_obj\_set\_x}} and \sphinxcode{\sphinxupquote{lv\_obj\_set\_y}}. However, only one animation can exist with a given variable and function pair.
Therefore \sphinxcode{\sphinxupquote{lv\_anim\_create()}} will delete the already existing variable-function animations.


\subsubsection{Animation path}
\label{\detokenize{overview/animations:animation-path}}
You can determinate the \sphinxstylestrong{path of animation}. In the most simple case, it is linear which means the current value between \sphinxstyleemphasis{start} and \sphinxstyleemphasis{end}  is changed linearly.
A \sphinxstyleemphasis{path} is a function which calculates the next value to set based on the current state of the animation. Currently, there are the following built-in paths:
\begin{itemize}
\item {} 
\sphinxstylestrong{lv\_anim\_path\_linear} linear animation

\item {} 
\sphinxstylestrong{lv\_anim\_path\_step} change in one step at the end

\item {} 
\sphinxstylestrong{lv\_anim\_path\_ease\_in} slow at the beginning

\item {} 
\sphinxstylestrong{lv\_anim\_path\_ease\_out} slow at the end

\item {} 
\sphinxstylestrong{lv\_anim\_path\_ease\_in\_out} slow at the beginning and end too

\item {} 
\sphinxstylestrong{lv\_anim\_path\_overshoot} overshoot the end value

\item {} 
\sphinxstylestrong{lv\_anim\_path\_bounce} bonce back a little from the end value (like hitting a wall)

\end{itemize}


\subsubsection{Speed vs time}
\label{\detokenize{overview/animations:speed-vs-time}}
By default, you can set the animation time. But in some cases, the \sphinxstylestrong{animation speed} is more practical.

The \sphinxcode{\sphinxupquote{lv\_anim\_speed\_to\_time(speed, start, end)}} function calculates the required time in milliseconds to reach the end value from a start value with the given speed.
The speed is interpreted in \sphinxstyleemphasis{unit/sec} dimension. For example \sphinxcode{\sphinxupquote{lv\_anim\_speed\_to\_time(20,0,100)}} will give 5000 milliseconds.For example in case of \sphinxcode{\sphinxupquote{lv\_obj\_set\_x}} \sphinxstyleemphasis{unit} is pixels so \sphinxstyleemphasis{20} means \sphinxstyleemphasis{20 px/sec} speed.


\subsubsection{Delete animations}
\label{\detokenize{overview/animations:delete-animations}}
You can \sphinxstylestrong{delete an animation} by \sphinxcode{\sphinxupquote{lv\_anim\_del(var, func)}} by providing the animated variable and its animator function.


\subsection{Drawing}
\label{\detokenize{overview/drawing:drawing}}\label{\detokenize{overview/drawing::doc}}
With LittlevGL you don’t need to draw anything manually. Just create objects (like buttons and labels), move and change them and LittlevGL will refresh and redraw what is required.

However, it might be useful to have a basic understanding of how drawing happens in LittlevGL.

The basic concept is to not draw directly to screen but draw to an internal buffer first and then copy that buffer to screen when the rendering is ready. It has two main advantages:
\begin{enumerate}
\def\theenumi{\arabic{enumi}}
\def\labelenumi{\theenumi .}
\makeatletter\def\p@enumii{\p@enumi \theenumi .}\makeatother
\item {} 
\sphinxstylestrong{Avoids flickering} while layers of the UI are drawn. E.g. when drawing a \sphinxstyleemphasis{background + button + text} each “stage” would be visible for a short time.

\item {} 
\sphinxstylestrong{It’s faster} because when pixels are redrawn multiple times (e.g. background + button + text) it’s faster to modify a buffer in RAM and finally write one pixel once
than read/write a display directly on each pixel access. (e.g. via a display controller with SPI interface).

\end{enumerate}


\subsubsection{Buffering types}
\label{\detokenize{overview/drawing:buffering-types}}
As you already might learn in the {\hyperref[\detokenize{porting/display::doc}]{\sphinxcrossref{\DUrole{doc,doc,doc}{Porting}}}} section there are 3 types of buffering:
\begin{enumerate}
\def\theenumi{\arabic{enumi}}
\def\labelenumi{\theenumi .}
\makeatletter\def\p@enumii{\p@enumi \theenumi .}\makeatother
\item {} 
\sphinxstylestrong{One buffer} LittlevGL draws the content of the screen into a buffer and sends it to the display. The buffer can be smaller than the screen. In this case, the larger areas will be redrawn in multiple parts. If only small areas changes (e.g. button press) then only those areas will be refreshed.

\item {} 
\sphinxstylestrong{Two non-screen-sized buffers} having two buffers LittlevGL can draw into one buffer while the content of the other buffer is sent to display in the background.
DMA or other hardware should be used to transfer the data to the display to let the CPU draw meanwhile.
This way the rendering and refreshing of the display become parallel. Similarly to the \sphinxstyleemphasis{One buffer} LittlevGL will draw the display’s content in chunks if the buffer is smaller than the area to refresh.

\item {} 
\sphinxstylestrong{Two screen-sized buffers}.
In contrast to \sphinxstyleemphasis{Two non-screen-sized buffers} LittlevGL will always provide the whole screen’s content not only chunks. This way the driver can simply change the address of the frame buffer to the buffer received from LittlevGL.
Therefore this method works the best when the MCU has an LCD/TFT interface and the frame buffer is just a location in the RAM.

\end{enumerate}


\subsubsection{Mechanism of screen refreshing}
\label{\detokenize{overview/drawing:mechanism-of-screen-refreshing}}\begin{enumerate}
\def\theenumi{\arabic{enumi}}
\def\labelenumi{\theenumi .}
\makeatletter\def\p@enumii{\p@enumi \theenumi .}\makeatother
\item {} 
Something happens on the GUI which requires redrawing. E.g. a button has been pressed, a chart has been changed or an animation happened, etc.

\item {} 
LittlevGL saves the changed object’s old and new area into a buffer, called \sphinxstyleemphasis{Invalid area buffer}. For optimization in some cases objects are not added to the buffer:
\begin{itemize}
\item {} 
Hidden objects are not added

\item {} 
Objects completely out of their parent are not added

\item {} 
Areas out of the parent are cropped to the parent’s area

\item {} 
The object on other screens are not added

\end{itemize}

\item {} 
In every \sphinxcode{\sphinxupquote{LV\_DISP\_DEF\_REFR\_PERIOD}} (set in \sphinxstyleemphasis{lv\_conf.h}):
\begin{itemize}
\item {} 
LittlevGL checks the invalid areas and joins the adjacent or intersecting areas

\item {} 
Takes the first joined area if it’s smaller the \sphinxstyleemphasis{display buffer} then simply draws the areas content to the \sphinxstyleemphasis{display buffer}. If the area doesn’t fit into the buffer draw as many lines as possible to the \sphinxstyleemphasis{display buffer}.

\item {} 
When the area is drawn call \sphinxcode{\sphinxupquote{flush\_cb}} from the display driver to refresh the display

\item {} 
If the area was larger than the buffer redraw the remaining parts too.

\item {} 
Do the same with all the joined areas.

\end{itemize}

\end{enumerate}

While an area is redrawn the library searches the most top object which covers the area to redraw and starts to draw from that object.
For example, if a button’s label has changed the library will see that it’s enough to draw the button under the text and it’s not required to draw the background too.

The difference between buffer types regarding the drawing mechanism is the following:
\begin{enumerate}
\def\theenumi{\arabic{enumi}}
\def\labelenumi{\theenumi .}
\makeatletter\def\p@enumii{\p@enumi \theenumi .}\makeatother
\item {} 
\sphinxstylestrong{One buffer} LittlevGL needs to wait for \sphinxcode{\sphinxupquote{lv\_disp\_flush\_ready()}} (called at the end of \sphinxcode{\sphinxupquote{flush\_cb}}) before starting to redraw the next part.

\item {} 
\sphinxstylestrong{Two non-screen-sized buffers} LittlevGL can immediately draw to the second buffer when the first is sent to \sphinxcode{\sphinxupquote{flush\_cb}} because the flushing should be done by DMA (or similar hardware) in the background.

\item {} 
\sphinxstylestrong{Two screen-sized buffers} After calling \sphinxcode{\sphinxupquote{flush\_cb}} the first buffer if being displayed as frame buffer. Its content is copied to the second buffer and all the changes are drawn on top of it.

\end{enumerate}


\section{Object types}
\label{\detokenize{object-types/index:object-types}}\label{\detokenize{object-types/index::doc}}

\subsection{Base object (lv\_obj)}
\label{\detokenize{object-types/obj:base-object-lv-obj}}\label{\detokenize{object-types/obj::doc}}

\subsection{Arc (lv\_arc)}
\label{\detokenize{object-types/arc:arc-lv-arc}}\label{\detokenize{object-types/arc::doc}}

\subsubsection{Overview}
\label{\detokenize{object-types/arc:overview}}
The \sphinxstyleemphasis{Arc} object \sphinxstylestrong{draws an arc} within \sphinxstylestrong{start and end angles} and with a given \sphinxstylestrong{thickness}.

To set the angles use the \sphinxcode{\sphinxupquote{lv\_arc\_set\_angles(arc, start\_angle, end\_angle)}} function. The zero degree is at the bottom of the object and the degrees are increasing in a counter-clockwise direction.
The angles should be in {[}0;360{]} range.

The \sphinxstylestrong{width and height} of the \sphinxstyleemphasis{Arc} should be the \sphinxstylestrong{same}.

Currently the \sphinxstyleemphasis{Arc} object \sphinxstylestrong{does not support anti-aliasing}.


\subsubsection{Styles}
\label{\detokenize{object-types/arc:styles}}
To set the style of an \sphinxstyleemphasis{Arc} object use \sphinxcode{\sphinxupquote{lv\_arc\_set\_style(arc, LV\_ARC\_STYLE\_MAIN, \&style)}}
\begin{itemize}
\item {} 
\sphinxstylestrong{line.rounded} make the endpoints rounded (opacity won’t work properly if set to 1)

\item {} 
\sphinxstylestrong{line.width} the thickness of the arc

\item {} 
\sphinxstylestrong{line.color} the color of the arc.

\end{itemize}


\subsubsection{Events}
\label{\detokenize{object-types/arc:events}}
Only the \sphinxhref{/overview/events.html\#generic-events}{Genreric events} are sent by the object type.

Learn more about {\hyperref[\detokenize{overview/events::doc}]{\sphinxcrossref{\DUrole{doc,doc,doc}{Events}}}}.


\subsubsection{Keys}
\label{\detokenize{object-types/arc:keys}}
No \sphinxstyleemphasis{Keys} are processed by the object type.

Learn more about {\hyperref[\detokenize{overview/indev::doc}]{\sphinxcrossref{\DUrole{doc,doc,doc}{Keys}}}}.


\subsubsection{Example}
\label{\detokenize{object-types/arc:example}}

\paragraph{C}
\label{\detokenize{object-types/arc:c}}
\sphinxincludegraphics{{arc_1}.png}Arc image

code

\begin{sphinxVerbatim}[commandchars=\\\{\}]
\PYG{c+cm}{/*Create style for the Arcs*/}
\PYG{n}{lv\PYGZus{}style\PYGZus{}t} \PYG{n}{style}\PYG{p}{;}
\PYG{n}{lv\PYGZus{}style\PYGZus{}copy}\PYG{p}{(}\PYG{o}{\PYGZam{}}\PYG{n}{style}\PYG{p}{,} \PYG{o}{\PYGZam{}}\PYG{n}{lv\PYGZus{}style\PYGZus{}plain}\PYG{p}{)}\PYG{p}{;}
\PYG{n}{style}\PYG{p}{.}\PYG{n}{line}\PYG{p}{.}\PYG{n}{color} \PYG{o}{=} \PYG{n}{LV\PYGZus{}COLOR\PYGZus{}BLUE}\PYG{p}{;}           \PYG{c+cm}{/*Arc color*/}
\PYG{n}{style}\PYG{p}{.}\PYG{n}{line}\PYG{p}{.}\PYG{n}{width} \PYG{o}{=} \PYG{l+m+mi}{8}\PYG{p}{;}                       \PYG{c+cm}{/*Arc width*/}

\PYG{c+cm}{/*Create an Arc*/}
\PYG{n}{lv\PYGZus{}obj\PYGZus{}t} \PYG{o}{*} \PYG{n}{arc} \PYG{o}{=} \PYG{n}{lv\PYGZus{}arc\PYGZus{}create}\PYG{p}{(}\PYG{n}{lv\PYGZus{}scr\PYGZus{}act}\PYG{p}{(}\PYG{p}{)}\PYG{p}{,} \PYG{n+nb}{NULL}\PYG{p}{)}\PYG{p}{;}
\PYG{n}{lv\PYGZus{}arc\PYGZus{}set\PYGZus{}style}\PYG{p}{(}\PYG{n}{arc}\PYG{p}{,} \PYG{n}{LV\PYGZus{}ARC\PYGZus{}STYLE\PYGZus{}MAIN}\PYG{p}{,} \PYG{o}{\PYGZam{}}\PYG{n}{style}\PYG{p}{)}\PYG{p}{;}          \PYG{c+cm}{/*Use the new style*/}
\PYG{n}{lv\PYGZus{}arc\PYGZus{}set\PYGZus{}angles}\PYG{p}{(}\PYG{n}{arc}\PYG{p}{,} \PYG{l+m+mi}{90}\PYG{p}{,} \PYG{l+m+mi}{60}\PYG{p}{)}\PYG{p}{;}
\PYG{n}{lv\PYGZus{}obj\PYGZus{}set\PYGZus{}size}\PYG{p}{(}\PYG{n}{arc}\PYG{p}{,} \PYG{l+m+mi}{150}\PYG{p}{,} \PYG{l+m+mi}{150}\PYG{p}{)}\PYG{p}{;}
\PYG{n}{lv\PYGZus{}obj\PYGZus{}align}\PYG{p}{(}\PYG{n}{arc}\PYG{p}{,} \PYG{n+nb}{NULL}\PYG{p}{,} \PYG{n}{LV\PYGZus{}ALIGN\PYGZus{}CENTER}\PYG{p}{,} \PYG{l+m+mi}{0}\PYG{p}{,} \PYG{l+m+mi}{0}\PYG{p}{)}\PYG{p}{;}

\PYG{c+cm}{/*Copy the previous Arc and set different angles and size*/}
\PYG{n}{arc} \PYG{o}{=} \PYG{n}{lv\PYGZus{}arc\PYGZus{}create}\PYG{p}{(}\PYG{n}{lv\PYGZus{}scr\PYGZus{}act}\PYG{p}{(}\PYG{p}{)}\PYG{p}{,} \PYG{n}{arc}\PYG{p}{)}\PYG{p}{;}
\PYG{n}{lv\PYGZus{}arc\PYGZus{}set\PYGZus{}angles}\PYG{p}{(}\PYG{n}{arc}\PYG{p}{,} \PYG{l+m+mi}{90}\PYG{p}{,} \PYG{l+m+mi}{20}\PYG{p}{)}\PYG{p}{;}
\PYG{n}{lv\PYGZus{}obj\PYGZus{}set\PYGZus{}size}\PYG{p}{(}\PYG{n}{arc}\PYG{p}{,} \PYG{l+m+mi}{125}\PYG{p}{,} \PYG{l+m+mi}{125}\PYG{p}{)}\PYG{p}{;}
\PYG{n}{lv\PYGZus{}obj\PYGZus{}align}\PYG{p}{(}\PYG{n}{arc}\PYG{p}{,} \PYG{n+nb}{NULL}\PYG{p}{,} \PYG{n}{LV\PYGZus{}ALIGN\PYGZus{}CENTER}\PYG{p}{,} \PYG{l+m+mi}{0}\PYG{p}{,} \PYG{l+m+mi}{0}\PYG{p}{)}\PYG{p}{;}

\PYG{c+cm}{/*Copy the previous Arc and set different angles and size*/}
\PYG{n}{arc} \PYG{o}{=} \PYG{n}{lv\PYGZus{}arc\PYGZus{}create}\PYG{p}{(}\PYG{n}{lv\PYGZus{}scr\PYGZus{}act}\PYG{p}{(}\PYG{p}{)}\PYG{p}{,} \PYG{n}{arc}\PYG{p}{)}\PYG{p}{;}
\PYG{n}{lv\PYGZus{}arc\PYGZus{}set\PYGZus{}angles}\PYG{p}{(}\PYG{n}{arc}\PYG{p}{,} \PYG{l+m+mi}{90}\PYG{p}{,} \PYG{l+m+mi}{310}\PYG{p}{)}\PYG{p}{;}
\PYG{n}{lv\PYGZus{}obj\PYGZus{}set\PYGZus{}size}\PYG{p}{(}\PYG{n}{arc}\PYG{p}{,} \PYG{l+m+mi}{100}\PYG{p}{,} \PYG{l+m+mi}{100}\PYG{p}{)}\PYG{p}{;}
\PYG{n}{lv\PYGZus{}obj\PYGZus{}align}\PYG{p}{(}\PYG{n}{arc}\PYG{p}{,} \PYG{n+nb}{NULL}\PYG{p}{,} \PYG{n}{LV\PYGZus{}ALIGN\PYGZus{}CENTER}\PYG{p}{,} \PYG{l+m+mi}{0}\PYG{p}{,} \PYG{l+m+mi}{0}\PYG{p}{)}\PYG{p}{;}
\end{sphinxVerbatim}


\paragraph{MicroPython}
\label{\detokenize{object-types/arc:micropython}}
No examples yet.


\subsubsection{API}
\label{\detokenize{object-types/arc:api}}

\subsection{Bar (lv\_bar)}
\label{\detokenize{object-types/bar:bar-lv-bar}}\label{\detokenize{object-types/bar::doc}}

\subsubsection{Overview}
\label{\detokenize{object-types/bar:overview}}
The Bar objects have got two main parts:
\begin{enumerate}
\def\theenumi{\arabic{enumi}}
\def\labelenumi{\theenumi .}
\makeatletter\def\p@enumii{\p@enumi \theenumi .}\makeatother
\item {} 
a \sphinxstylestrong{background} which is the object itself

\item {} 
an \sphinxstylestrong{indicator} which shape is similar to the background but its width/height can be adjusted.

\end{enumerate}

The orientation of the bar can be \sphinxstylestrong{vertical or horizontal} according to the width/height ratio. Logically on horizontal bars the indicator width, on vertical bars the indicator height can be changed.

A \sphinxstylestrong{new value} can be set by: \sphinxcode{\sphinxupquote{lv\_bar\_set\_value(bar, new\_value, LV\_ANIM\_ON/OFF)}}.
The value is interpreted in \sphinxstylestrong{range} (minimum and maximum values) which can be modified with \sphinxcode{\sphinxupquote{lv\_bar\_set\_range(bar, min, max)}}.
The default range is: 1..100.

The new value in \sphinxcode{\sphinxupquote{lv\_bar\_set\_value}} can be set with our without an \sphinxstylestrong{animation} depending on the last paramter (\sphinxcode{\sphinxupquote{LV\_ANIM\_ON/OFF}}).
The the of teh animation can be adjusted by \sphinxcode{\sphinxupquote{lv\_bar\_set\_anim\_time(bar, 100)}}. The time is is  millisecinds unit.

The bar can be drawn \sphinxstylestrong{symmetrical} to zero (drawn from zero left ot right) if it’s enabled with \sphinxcode{\sphinxupquote{lv\_bar\_set\_sym(bar, true)}}


\subsubsection{Styles}
\label{\detokenize{object-types/bar:styles}}
To set the style of an \sphinxstyleemphasis{Bar} object use \sphinxcode{\sphinxupquote{lv\_bar\_set\_style(arc, LV\_BAR\_STYLE\_MAIN, \&style)}}
\begin{itemize}
\item {} 
\sphinxstylestrong{LV\_BAR\_STYLE\_BG} is an {\hyperref[\detokenize{object-types/obj::doc}]{\sphinxcrossref{\DUrole{doc,doc,doc}{Base object}}}} therefore it uses its style elements. Its default style is: \sphinxcode{\sphinxupquote{lv\_style\_pretty}}.

\item {} 
\sphinxstylestrong{LV\_BAR\_STYLE\_INDIC} is similar to the background. It uses the \sphinxstyleemphasis{left}, \sphinxstyleemphasis{right}, \sphinxstyleemphasis{top} and \sphinxstyleemphasis{bottom} paddings to keeps some space form the edges of the background. Its default style is: \sphinxcode{\sphinxupquote{lv\_style\_pretty\_color}}.

\end{itemize}


\subsubsection{Events}
\label{\detokenize{object-types/bar:events}}
Only the \sphinxhref{/overview/events.html\#generic-events}{Genreric events} are sent by the object type.

Learn more about {\hyperref[\detokenize{overview/events::doc}]{\sphinxcrossref{\DUrole{doc,doc,doc}{Events}}}}.


\subsubsection{Keys}
\label{\detokenize{object-types/bar:keys}}
No \sphinxstyleemphasis{Keys} are processed by the object type.

Learn more about {\hyperref[\detokenize{overview/indev::doc}]{\sphinxcrossref{\DUrole{doc,doc,doc}{Keys}}}}.


\subsubsection{Example}
\label{\detokenize{object-types/bar:example}}

\paragraph{C}
\label{\detokenize{object-types/bar:c}}
\sphinxincludegraphics{{bar_1}.png}Bar image

code

\begin{sphinxVerbatim}[commandchars=\\\{\}]
\PYG{c+cm}{/*Create a default bar*/}
\PYG{n}{lv\PYGZus{}obj\PYGZus{}t} \PYG{o}{*} \PYG{n}{bar1} \PYG{o}{=} \PYG{n}{lv\PYGZus{}bar\PYGZus{}create}\PYG{p}{(}\PYG{n}{lv\PYGZus{}scr\PYGZus{}act}\PYG{p}{(}\PYG{p}{)}\PYG{p}{,} \PYG{n+nb}{NULL}\PYG{p}{)}\PYG{p}{;}
\PYG{n}{lv\PYGZus{}obj\PYGZus{}set\PYGZus{}size}\PYG{p}{(}\PYG{n}{bar1}\PYG{p}{,} \PYG{l+m+mi}{200}\PYG{p}{,} \PYG{l+m+mi}{30}\PYG{p}{)}\PYG{p}{;}
\PYG{n}{lv\PYGZus{}obj\PYGZus{}align}\PYG{p}{(}\PYG{n}{bar1}\PYG{p}{,} \PYG{n+nb}{NULL}\PYG{p}{,} \PYG{n}{LV\PYGZus{}ALIGN\PYGZus{}IN\PYGZus{}TOP\PYGZus{}RIGHT}\PYG{p}{,} \PYG{o}{\PYGZhy{}}\PYG{l+m+mi}{20}\PYG{p}{,} \PYG{l+m+mi}{30}\PYG{p}{)}\PYG{p}{;}
\PYG{n}{lv\PYGZus{}bar\PYGZus{}set\PYGZus{}value}\PYG{p}{(}\PYG{n}{bar1}\PYG{p}{,} \PYG{l+m+mi}{70}\PYG{p}{)}\PYG{p}{;}

\PYG{c+cm}{/*Create a label right to the bar*/}
\PYG{n}{lv\PYGZus{}obj\PYGZus{}t} \PYG{o}{*} \PYG{n}{bar1\PYGZus{}label} \PYG{o}{=} \PYG{n}{lv\PYGZus{}label\PYGZus{}create}\PYG{p}{(}\PYG{n}{lv\PYGZus{}scr\PYGZus{}act}\PYG{p}{(}\PYG{p}{)}\PYG{p}{,} \PYG{n+nb}{NULL}\PYG{p}{)}\PYG{p}{;}
\PYG{n}{lv\PYGZus{}label\PYGZus{}set\PYGZus{}text}\PYG{p}{(}\PYG{n}{bar1\PYGZus{}label}\PYG{p}{,} \PYG{l+s}{\PYGZdq{}}\PYG{l+s}{Default}\PYG{l+s}{\PYGZdq{}}\PYG{p}{)}\PYG{p}{;}
\PYG{n}{lv\PYGZus{}obj\PYGZus{}align}\PYG{p}{(}\PYG{n}{bar1\PYGZus{}label}\PYG{p}{,} \PYG{n}{bar1}\PYG{p}{,} \PYG{n}{LV\PYGZus{}ALIGN\PYGZus{}OUT\PYGZus{}LEFT\PYGZus{}MID}\PYG{p}{,} \PYG{o}{\PYGZhy{}}\PYG{l+m+mi}{10}\PYG{p}{,} \PYG{l+m+mi}{0}\PYG{p}{)}\PYG{p}{;}

\PYG{c+cm}{/*Create a bar and an indicator style*/}
\PYG{k}{static} \PYG{n}{lv\PYGZus{}style\PYGZus{}t} \PYG{n}{style\PYGZus{}bar}\PYG{p}{;}
\PYG{k}{static} \PYG{n}{lv\PYGZus{}style\PYGZus{}t} \PYG{n}{style\PYGZus{}indic}\PYG{p}{;}

\PYG{n}{lv\PYGZus{}style\PYGZus{}copy}\PYG{p}{(}\PYG{o}{\PYGZam{}}\PYG{n}{style\PYGZus{}bar}\PYG{p}{,} \PYG{o}{\PYGZam{}}\PYG{n}{lv\PYGZus{}style\PYGZus{}pretty}\PYG{p}{)}\PYG{p}{;}
\PYG{n}{style\PYGZus{}bar}\PYG{p}{.}\PYG{n}{body}\PYG{p}{.}\PYG{n}{main\PYGZus{}color} \PYG{o}{=} \PYG{n}{LV\PYGZus{}COLOR\PYGZus{}BLACK}\PYG{p}{;}
\PYG{n}{style\PYGZus{}bar}\PYG{p}{.}\PYG{n}{body}\PYG{p}{.}\PYG{n}{grad\PYGZus{}color} \PYG{o}{=} \PYG{n}{LV\PYGZus{}COLOR\PYGZus{}GRAY}\PYG{p}{;}
\PYG{n}{style\PYGZus{}bar}\PYG{p}{.}\PYG{n}{body}\PYG{p}{.}\PYG{n}{radius} \PYG{o}{=} \PYG{n}{LV\PYGZus{}RADIUS\PYGZus{}CIRCLE}\PYG{p}{;}
\PYG{n}{style\PYGZus{}bar}\PYG{p}{.}\PYG{n}{body}\PYG{p}{.}\PYG{n}{border}\PYG{p}{.}\PYG{n}{color} \PYG{o}{=} \PYG{n}{LV\PYGZus{}COLOR\PYGZus{}WHITE}\PYG{p}{;}

\PYG{n}{lv\PYGZus{}style\PYGZus{}copy}\PYG{p}{(}\PYG{o}{\PYGZam{}}\PYG{n}{style\PYGZus{}indic}\PYG{p}{,} \PYG{o}{\PYGZam{}}\PYG{n}{lv\PYGZus{}style\PYGZus{}pretty}\PYG{p}{)}\PYG{p}{;}
\PYG{n}{style\PYGZus{}indic}\PYG{p}{.}\PYG{n}{body}\PYG{p}{.}\PYG{n}{grad\PYGZus{}color} \PYG{o}{=}  \PYG{n}{LV\PYGZus{}COLOR\PYGZus{}GREEN}\PYG{p}{;}
\PYG{n}{style\PYGZus{}indic}\PYG{p}{.}\PYG{n}{body}\PYG{p}{.}\PYG{n}{main\PYGZus{}color}\PYG{o}{=}  \PYG{n}{LV\PYGZus{}COLOR\PYGZus{}LIME}\PYG{p}{;}
\PYG{n}{style\PYGZus{}indic}\PYG{p}{.}\PYG{n}{body}\PYG{p}{.}\PYG{n}{radius} \PYG{o}{=} \PYG{n}{LV\PYGZus{}RADIUS\PYGZus{}CIRCLE}\PYG{p}{;}
\PYG{n}{style\PYGZus{}indic}\PYG{p}{.}\PYG{n}{body}\PYG{p}{.}\PYG{n}{shadow}\PYG{p}{.}\PYG{n}{width} \PYG{o}{=} \PYG{l+m+mi}{10}\PYG{p}{;}
\PYG{n}{style\PYGZus{}indic}\PYG{p}{.}\PYG{n}{body}\PYG{p}{.}\PYG{n}{shadow}\PYG{p}{.}\PYG{n}{color} \PYG{o}{=} \PYG{n}{LV\PYGZus{}COLOR\PYGZus{}LIME}\PYG{p}{;}
\PYG{n}{style\PYGZus{}indic}\PYG{p}{.}\PYG{n}{body}\PYG{p}{.}\PYG{n}{padding}\PYG{p}{.}\PYG{n}{hor} \PYG{o}{=} \PYG{l+m+mi}{3}\PYG{p}{;}           \PYG{c+cm}{/*Make the indicator a little bit smaller*/}
\PYG{n}{style\PYGZus{}indic}\PYG{p}{.}\PYG{n}{body}\PYG{p}{.}\PYG{n}{padding}\PYG{p}{.}\PYG{n}{ver} \PYG{o}{=} \PYG{l+m+mi}{3}\PYG{p}{;}

\PYG{c+cm}{/*Create a second bar*/}
\PYG{n}{lv\PYGZus{}obj\PYGZus{}t} \PYG{o}{*} \PYG{n}{bar2} \PYG{o}{=} \PYG{n}{lv\PYGZus{}bar\PYGZus{}create}\PYG{p}{(}\PYG{n}{lv\PYGZus{}scr\PYGZus{}act}\PYG{p}{(}\PYG{p}{)}\PYG{p}{,} \PYG{n}{bar1}\PYG{p}{)}\PYG{p}{;}
\PYG{n}{lv\PYGZus{}bar\PYGZus{}set\PYGZus{}style}\PYG{p}{(}\PYG{n}{bar2}\PYG{p}{,} \PYG{n}{LV\PYGZus{}BAR\PYGZus{}STYLE\PYGZus{}BG}\PYG{p}{,} \PYG{o}{\PYGZam{}}\PYG{n}{style\PYGZus{}bar}\PYG{p}{)}\PYG{p}{;}
\PYG{n}{lv\PYGZus{}bar\PYGZus{}set\PYGZus{}style}\PYG{p}{(}\PYG{n}{bar2}\PYG{p}{,} \PYG{n}{LV\PYGZus{}BAR\PYGZus{}STYLE\PYGZus{}INDIC}\PYG{p}{,} \PYG{o}{\PYGZam{}}\PYG{n}{style\PYGZus{}indic}\PYG{p}{)}\PYG{p}{;}
\PYG{n}{lv\PYGZus{}obj\PYGZus{}align}\PYG{p}{(}\PYG{n}{bar2}\PYG{p}{,} \PYG{n}{bar1}\PYG{p}{,} \PYG{n}{LV\PYGZus{}ALIGN\PYGZus{}OUT\PYGZus{}BOTTOM\PYGZus{}MID}\PYG{p}{,} \PYG{l+m+mi}{0}\PYG{p}{,} \PYG{l+m+mi}{30}\PYG{p}{)}\PYG{p}{;} \PYG{c+cm}{/*Align below \PYGZsq{}bar1\PYGZsq{}*/}

\PYG{c+cm}{/*Create a second label*/}
\PYG{n}{lv\PYGZus{}obj\PYGZus{}t} \PYG{o}{*} \PYG{n}{bar2\PYGZus{}label} \PYG{o}{=} \PYG{n}{lv\PYGZus{}label\PYGZus{}create}\PYG{p}{(}\PYG{n}{lv\PYGZus{}scr\PYGZus{}act}\PYG{p}{(}\PYG{p}{)}\PYG{p}{,} \PYG{n}{bar1\PYGZus{}label}\PYG{p}{)}\PYG{p}{;}
\PYG{n}{lv\PYGZus{}label\PYGZus{}set\PYGZus{}text}\PYG{p}{(}\PYG{n}{bar2\PYGZus{}label}\PYG{p}{,} \PYG{l+s}{\PYGZdq{}}\PYG{l+s}{Modified}\PYG{l+s}{\PYGZdq{}}\PYG{p}{)}\PYG{p}{;}
\PYG{n}{lv\PYGZus{}obj\PYGZus{}align}\PYG{p}{(}\PYG{n}{bar2\PYGZus{}label}\PYG{p}{,} \PYG{n}{bar2}\PYG{p}{,} \PYG{n}{LV\PYGZus{}ALIGN\PYGZus{}OUT\PYGZus{}LEFT\PYGZus{}MID}\PYG{p}{,} \PYG{o}{\PYGZhy{}}\PYG{l+m+mi}{10}\PYG{p}{,} \PYG{l+m+mi}{0}\PYG{p}{)}\PYG{p}{;}
\end{sphinxVerbatim}


\paragraph{MicroPython}
\label{\detokenize{object-types/bar:micropython}}
No examples yet.


\subsubsection{API}
\label{\detokenize{object-types/bar:api}}

\subsection{Button (lv\_btn)}
\label{\detokenize{object-types/btn:button-lv-btn}}\label{\detokenize{object-types/btn::doc}}

\subsubsection{Overview}
\label{\detokenize{object-types/btn:overview}}
Buttons are simple rectagnle-like objects but they change their style and state when they are pressed or released.

Buttons can be in one of the \sphinxstylestrong{five possible states}:
\begin{itemize}
\item {} 
\sphinxstylestrong{LV\_BTN\_STATE\_REL} Released state

\item {} 
\sphinxstylestrong{LV\_BTN\_STATE\_PR} Pressed state

\item {} 
\sphinxstylestrong{LV\_BTN\_STATE\_TGL\_REL} Toggled released state

\item {} 
\sphinxstylestrong{LV\_BTN\_STATE\_TGL\_PR} Toggled pressed state

\item {} 
\sphinxstylestrong{LV\_BTN\_STATE\_INA} Inactive state

\end{itemize}

The buttons can be configured as \sphinxstylestrong{toggle button} with \sphinxcode{\sphinxupquote{lv\_btn\_set\_toggle(btn, true)}}. In this case on release, the button goes to toggled released state.

You can set the button’s state manually with \sphinxcode{\sphinxupquote{lv\_btn\_set\_state(btn, LV\_BTN\_STATE\_TGL\_REL)}}.

Similarly to {\hyperref[\detokenize{object-types/cont::doc}]{\sphinxcrossref{\DUrole{doc,doc,doc}{Containers}}}} buttons also have \sphinxstylestrong{layout} and \sphinxstylestrong{fit} attributes.
\begin{itemize}
\item {} 
\sphinxcode{\sphinxupquote{lv\_btn\_set\_layout(btn, LV\_LAYOUT\_...) }}set a layout. The default is \sphinxcode{\sphinxupquote{LV\_LAYOUT\_CENTER}}.
So if you add a label then it will be automatically aligned to the middle and can’t be moved with \sphinxcode{\sphinxupquote{lv\_obj\_set\_pos()}}.
You can disable the layout with \sphinxcode{\sphinxupquote{lv\_btn\_set\_layout(btn, LV\_LAYOUT\_OFF)}}

\item {} 
\sphinxcode{\sphinxupquote{lv\_btn\_set\_fit/fit2/fit4(btn, LV\_FIT\_..)}} enables to set the button width and/or height automatically according to the children, parent and fit type.

\end{itemize}


\subsubsection{Styles}
\label{\detokenize{object-types/btn:styles}}
A button can have 5 independent styles for the 5 state. You can set them via: \sphinxcode{\sphinxupquote{lv\_btn\_set\_style(btn, LV\_BTN\_STYLE\_..., \&style)}}. The styles use the \sphinxcode{\sphinxupquote{style.body}} properties.
\begin{itemize}
\item {} 
\sphinxstylestrong{LV\_BTN\_STYLE\_REL} style of the released state. Default: \sphinxcode{\sphinxupquote{lv\_style\_btn\_rel}}

\item {} 
\sphinxstylestrong{LV\_BTN\_STYLE\_PR} style of the pressed state. Default: \sphinxcode{\sphinxupquote{lv\_style\_btn\_pr}}

\item {} 
\sphinxstylestrong{LV\_BTN\_STYLE\_TGL\_REL} style of the toggled released state. Default: \sphinxcode{\sphinxupquote{lv\_style\_btn\_tgl\_rel}}

\item {} 
\sphinxstylestrong{LV\_BTN\_STYLE\_TGL\_PR} style of the toggled pressed state. Default: \sphinxcode{\sphinxupquote{lv\_style\_btn\_tgl\_pr}}

\item {} 
\sphinxstylestrong{LV\_BTN\_STYLE\_INA} style of the inactive state. Default: \sphinxcode{\sphinxupquote{lv\_style\_btn\_ina}}

\end{itemize}

When labels are created on a button, it’s a good practive to set the button’s \sphinxcode{\sphinxupquote{style.text}} properties too. Because labels have \sphinxcode{\sphinxupquote{style = NULL}} by default they inherit the parent’s (button) style.
Hence you don’t need to craete a new style for the label.


\subsubsection{Events}
\label{\detokenize{object-types/btn:events}}
Besided the \sphinxhref{/overview/events.html\#generic-events}{Genreric events} the following \sphinxhref{/overview/events.html\#special-events}{Special events} are sent by the buttons:
\begin{itemize}
\item {} 
\sphinxstylestrong{LV\_EVENT\_VALUE\_CHANGED} sent when the button is toggled.

\end{itemize}

Note that the generic input device related events (like \sphinxcode{\sphinxupquote{LV\_EVENT\_PRESSED}}) are sent in inactive state too. You need to check the state with \sphinxcode{\sphinxupquote{lv\_btn\_get\_state(btn)}} to ignore the events from inactive buttons.

Learn more about {\hyperref[\detokenize{overview/events::doc}]{\sphinxcrossref{\DUrole{doc,doc,doc}{Events}}}}.


\subsubsection{Keys}
\label{\detokenize{object-types/btn:keys}}
The following \sphinxstyleemphasis{Keys} are processed by the Buttons:
\begin{itemize}
\item {} 
\sphinxstylestrong{LV\_KEY\_RIGHT/UP} Go to toggled state if toggling is enabled

\item {} 
\sphinxstylestrong{LV\_KEY\_LEFT/DOWN} Go to non-toggled state if toggling is  enabled

\end{itemize}

Note that, as usual, the state of \sphinxcode{\sphinxupquote{LV\_KEY\_ENTER}} is translated to \sphinxcode{\sphinxupquote{LV\_EVENT\_PRESSED/PRESSING/RELEASED}} etc.

Learn more about {\hyperref[\detokenize{overview/indev::doc}]{\sphinxcrossref{\DUrole{doc,doc,doc}{Keys}}}}.


\subsubsection{Example}
\label{\detokenize{object-types/btn:example}}

\paragraph{C}
\label{\detokenize{object-types/btn:c}}
\sphinxincludegraphics{{btn_1}.png}Button image

code

\begin{sphinxVerbatim}[commandchars=\\\{\}]
\PYG{k}{static} \PYG{n}{lv\PYGZus{}res\PYGZus{}t} \PYG{n+nf}{btn\PYGZus{}click\PYGZus{}action}\PYG{p}{(}\PYG{n}{lv\PYGZus{}obj\PYGZus{}t} \PYG{o}{*} \PYG{n}{btn}\PYG{p}{)}
\PYG{p}{\PYGZob{}}
    \PYG{k+kt}{uint8\PYGZus{}t} \PYG{n}{id} \PYG{o}{=} \PYG{n}{lv\PYGZus{}obj\PYGZus{}get\PYGZus{}free\PYGZus{}num}\PYG{p}{(}\PYG{n}{btn}\PYG{p}{)}\PYG{p}{;}

    \PYG{n}{printf}\PYG{p}{(}\PYG{l+s}{\PYGZdq{}}\PYG{l+s}{Button \PYGZpc{}d is released}\PYG{l+s+se}{\PYGZbs{}n}\PYG{l+s}{\PYGZdq{}}\PYG{p}{,} \PYG{n}{id}\PYG{p}{)}\PYG{p}{;}

    \PYG{c+cm}{/* The button is released.}
\PYG{c+cm}{     * Make something here */}

    \PYG{k}{return} \PYG{n}{LV\PYGZus{}RES\PYGZus{}OK}\PYG{p}{;} \PYG{c+cm}{/*Return OK if the button is not deleted*/}
\PYG{p}{\PYGZcb{}}

\PYG{p}{.}
\PYG{p}{.}
\PYG{p}{.}

\PYG{c+cm}{/*Create a title label*/}
\PYG{n}{lv\PYGZus{}obj\PYGZus{}t} \PYG{o}{*} \PYG{n}{label} \PYG{o}{=} \PYG{n}{lv\PYGZus{}label\PYGZus{}create}\PYG{p}{(}\PYG{n}{lv\PYGZus{}scr\PYGZus{}act}\PYG{p}{(}\PYG{p}{)}\PYG{p}{,} \PYG{n+nb}{NULL}\PYG{p}{)}\PYG{p}{;}
\PYG{n}{lv\PYGZus{}label\PYGZus{}set\PYGZus{}text}\PYG{p}{(}\PYG{n}{label}\PYG{p}{,} \PYG{l+s}{\PYGZdq{}}\PYG{l+s}{Default buttons}\PYG{l+s}{\PYGZdq{}}\PYG{p}{)}\PYG{p}{;}
\PYG{n}{lv\PYGZus{}obj\PYGZus{}align}\PYG{p}{(}\PYG{n}{label}\PYG{p}{,} \PYG{n+nb}{NULL}\PYG{p}{,} \PYG{n}{LV\PYGZus{}ALIGN\PYGZus{}IN\PYGZus{}TOP\PYGZus{}MID}\PYG{p}{,} \PYG{l+m+mi}{0}\PYG{p}{,} \PYG{l+m+mi}{5}\PYG{p}{)}\PYG{p}{;}

\PYG{c+cm}{/*Create a normal button*/}
\PYG{n}{lv\PYGZus{}obj\PYGZus{}t} \PYG{o}{*} \PYG{n}{btn1} \PYG{o}{=} \PYG{n}{lv\PYGZus{}btn\PYGZus{}create}\PYG{p}{(}\PYG{n}{lv\PYGZus{}scr\PYGZus{}act}\PYG{p}{(}\PYG{p}{)}\PYG{p}{,} \PYG{n+nb}{NULL}\PYG{p}{)}\PYG{p}{;}
\PYG{n}{lv\PYGZus{}cont\PYGZus{}set\PYGZus{}fit}\PYG{p}{(}\PYG{n}{btn1}\PYG{p}{,} \PYG{n+nb}{true}\PYG{p}{,} \PYG{n+nb}{true}\PYG{p}{)}\PYG{p}{;} \PYG{c+cm}{/*Enable resizing horizontally and vertically*/}
\PYG{n}{lv\PYGZus{}obj\PYGZus{}align}\PYG{p}{(}\PYG{n}{btn1}\PYG{p}{,} \PYG{n}{label}\PYG{p}{,} \PYG{n}{LV\PYGZus{}ALIGN\PYGZus{}OUT\PYGZus{}BOTTOM\PYGZus{}MID}\PYG{p}{,} \PYG{l+m+mi}{0}\PYG{p}{,} \PYG{l+m+mi}{10}\PYG{p}{)}\PYG{p}{;}
\PYG{n}{lv\PYGZus{}obj\PYGZus{}set\PYGZus{}free\PYGZus{}num}\PYG{p}{(}\PYG{n}{btn1}\PYG{p}{,} \PYG{l+m+mi}{1}\PYG{p}{)}\PYG{p}{;}   \PYG{c+cm}{/*Set a unique number for the button*/}
\PYG{n}{lv\PYGZus{}btn\PYGZus{}set\PYGZus{}action}\PYG{p}{(}\PYG{n}{btn1}\PYG{p}{,} \PYG{n}{LV\PYGZus{}BTN\PYGZus{}ACTION\PYGZus{}CLICK}\PYG{p}{,} \PYG{n}{btn\PYGZus{}click\PYGZus{}action}\PYG{p}{)}\PYG{p}{;}

\PYG{c+cm}{/*Add a label to the button*/}
\PYG{n}{label} \PYG{o}{=} \PYG{n}{lv\PYGZus{}label\PYGZus{}create}\PYG{p}{(}\PYG{n}{btn1}\PYG{p}{,} \PYG{n+nb}{NULL}\PYG{p}{)}\PYG{p}{;}
\PYG{n}{lv\PYGZus{}label\PYGZus{}set\PYGZus{}text}\PYG{p}{(}\PYG{n}{label}\PYG{p}{,} \PYG{l+s}{\PYGZdq{}}\PYG{l+s}{Normal}\PYG{l+s}{\PYGZdq{}}\PYG{p}{)}\PYG{p}{;}

\PYG{c+cm}{/*Copy the button and set toggled state. (The release action is copied too)*/}
\PYG{n}{lv\PYGZus{}obj\PYGZus{}t} \PYG{o}{*} \PYG{n}{btn2} \PYG{o}{=} \PYG{n}{lv\PYGZus{}btn\PYGZus{}create}\PYG{p}{(}\PYG{n}{lv\PYGZus{}scr\PYGZus{}act}\PYG{p}{(}\PYG{p}{)}\PYG{p}{,} \PYG{n}{btn1}\PYG{p}{)}\PYG{p}{;}
\PYG{n}{lv\PYGZus{}obj\PYGZus{}align}\PYG{p}{(}\PYG{n}{btn2}\PYG{p}{,} \PYG{n}{btn1}\PYG{p}{,} \PYG{n}{LV\PYGZus{}ALIGN\PYGZus{}OUT\PYGZus{}BOTTOM\PYGZus{}MID}\PYG{p}{,} \PYG{l+m+mi}{0}\PYG{p}{,} \PYG{l+m+mi}{10}\PYG{p}{)}\PYG{p}{;}
\PYG{n}{lv\PYGZus{}btn\PYGZus{}set\PYGZus{}state}\PYG{p}{(}\PYG{n}{btn2}\PYG{p}{,} \PYG{n}{LV\PYGZus{}BTN\PYGZus{}STATE\PYGZus{}TGL\PYGZus{}REL}\PYG{p}{)}\PYG{p}{;}  \PYG{c+cm}{/*Set toggled state*/}
\PYG{n}{lv\PYGZus{}obj\PYGZus{}set\PYGZus{}free\PYGZus{}num}\PYG{p}{(}\PYG{n}{btn2}\PYG{p}{,} \PYG{l+m+mi}{2}\PYG{p}{)}\PYG{p}{;}               \PYG{c+cm}{/*Set a unique number for the button*/}

\PYG{c+cm}{/*Add a label to the toggled button*/}
\PYG{n}{label} \PYG{o}{=} \PYG{n}{lv\PYGZus{}label\PYGZus{}create}\PYG{p}{(}\PYG{n}{btn2}\PYG{p}{,} \PYG{n+nb}{NULL}\PYG{p}{)}\PYG{p}{;}
\PYG{n}{lv\PYGZus{}label\PYGZus{}set\PYGZus{}text}\PYG{p}{(}\PYG{n}{label}\PYG{p}{,} \PYG{l+s}{\PYGZdq{}}\PYG{l+s}{Toggled}\PYG{l+s}{\PYGZdq{}}\PYG{p}{)}\PYG{p}{;}

\PYG{c+cm}{/*Copy the button and set inactive state.*/}
\PYG{n}{lv\PYGZus{}obj\PYGZus{}t} \PYG{o}{*} \PYG{n}{btn3} \PYG{o}{=} \PYG{n}{lv\PYGZus{}btn\PYGZus{}create}\PYG{p}{(}\PYG{n}{lv\PYGZus{}scr\PYGZus{}act}\PYG{p}{(}\PYG{p}{)}\PYG{p}{,} \PYG{n}{btn1}\PYG{p}{)}\PYG{p}{;}
\PYG{n}{lv\PYGZus{}obj\PYGZus{}align}\PYG{p}{(}\PYG{n}{btn3}\PYG{p}{,} \PYG{n}{btn2}\PYG{p}{,} \PYG{n}{LV\PYGZus{}ALIGN\PYGZus{}OUT\PYGZus{}BOTTOM\PYGZus{}MID}\PYG{p}{,} \PYG{l+m+mi}{0}\PYG{p}{,} \PYG{l+m+mi}{10}\PYG{p}{)}\PYG{p}{;}
\PYG{n}{lv\PYGZus{}btn\PYGZus{}set\PYGZus{}state}\PYG{p}{(}\PYG{n}{btn3}\PYG{p}{,} \PYG{n}{LV\PYGZus{}BTN\PYGZus{}STATE\PYGZus{}INA}\PYG{p}{)}\PYG{p}{;}   \PYG{c+cm}{/*Set inactive state*/}
\PYG{n}{lv\PYGZus{}obj\PYGZus{}set\PYGZus{}free\PYGZus{}num}\PYG{p}{(}\PYG{n}{btn3}\PYG{p}{,} \PYG{l+m+mi}{3}\PYG{p}{)}\PYG{p}{;}               \PYG{c+cm}{/*Set a unique number for the button*/}

\PYG{c+cm}{/*Add a label to the inactive button*/}
\PYG{n}{label} \PYG{o}{=} \PYG{n}{lv\PYGZus{}label\PYGZus{}create}\PYG{p}{(}\PYG{n}{btn3}\PYG{p}{,} \PYG{n+nb}{NULL}\PYG{p}{)}\PYG{p}{;}
\PYG{n}{lv\PYGZus{}label\PYGZus{}set\PYGZus{}text}\PYG{p}{(}\PYG{n}{label}\PYG{p}{,} \PYG{l+s}{\PYGZdq{}}\PYG{l+s}{Inactive}\PYG{l+s}{\PYGZdq{}}\PYG{p}{)}\PYG{p}{;}
\end{sphinxVerbatim}


\paragraph{MicroPython}
\label{\detokenize{object-types/btn:micropython}}
No examples yet.


\subsubsection{API}
\label{\detokenize{object-types/btn:api}}

\subsection{Button matrix (lv\_btnm)}
\label{\detokenize{object-types/btnm:button-matrix-lv-btnm}}\label{\detokenize{object-types/btnm::doc}}

\subsubsection{Overview}
\label{\detokenize{object-types/btnm:overview}}
The Button Matrix objects can display \sphinxstylestrong{multiple buttons} in rows and culomns.

The Button matrix object is very light weighted becasue the buttons are not really created just drawn on the fly.
This way 1 button uses only 8 extra byte instead of the \textasciitilde{}100-150 byte size of a normal {\hyperref[\detokenize{object-types/btn::doc}]{\sphinxcrossref{\DUrole{doc,doc,doc}{Button}}}} object.

The buttons have texts on them which can be specified a descriptor string array, called \sphinxstyleemphasis{map}. The map can be set with \sphinxcode{\sphinxupquote{lv\_btnm\_set\_map(btnm, my\_map)}}.

The \sphinxstylestrong{declaration of a map} looks like \sphinxcode{\sphinxupquote{const char * map{[}{]} = \{"btn1", "btn2", "btn3", ""\}}}. Note that \sphinxstylestrong{the last element has to be an empty string}!

Use \sphinxcode{\sphinxupquote{"\textbackslash{}n"}} in the map  to make \sphinxstylestrong{line break}. E.g. \sphinxcode{\sphinxupquote{\{"btn1", "btn2", "\textbackslash{}n", "btn3", ""\}}}. The button’s width is recalculated in every line.

The buttons width can be set relative to the other button in the same line with \sphinxcode{\sphinxupquote{lv\_btnm\_set\_btn\_width(btnm, btn\_id, width)}}
E.g. in a line with two buttons: \sphinxstyleemphasis{btn 1 width = 1} and \sphinxstyleemphasis{btn 2 width = 2}, \sphinxstyleemphasis{btn 1} will have 33 \% width adnd \sphinxstyleemphasis{btn 2} will have 66 \% width.

In addition to width each button can be customized with following paramters:
\begin{itemize}
\item {} 
\sphinxstylestrong{LV\_BTNM\_CTRL\_HIDDEN} make a button hidden

\item {} 
\sphinxstylestrong{LV\_BTNM\_CTRL\_NO\_REPEAT} disable repating when the button is long pressed

\item {} 
\sphinxstylestrong{LV\_BTNM\_CTRL\_INACTIVE} make a button inactive

\item {} 
\sphinxstylestrong{LV\_BTNM\_CTRL\_TGL\_ENABLE} enable toggling of a button

\item {} 
\sphinxstylestrong{LV\_BTNM\_CTRL\_TGL\_STATE} set the toggle state

\item {} 
\sphinxstylestrong{LV\_BTNM\_CTRL\_CLICK\_TRIG} if 0 the button will react on press, if 1 will ract on release

\end{itemize}

The set or clear a button’s control attribute use \sphinxcode{\sphinxupquote{lv\_btnm\_set\_btn\_ctrl(btnm, btn\_id, LV\_BTNM\_CTRL\_...)}} and \sphinxcode{\sphinxupquote{lv\_btnm\_clear\_btn\_ctrl(btnm, btn\_id, LV\_BTNM\_CTRL\_...)}} respectively. More \sphinxcode{\sphinxupquote{LV\_BTNM\_CTRL\_...}} values can be \sphinxstyleemphasis{Or}ed

The set/clear the same control attribute for all buttons of a button matrix use \sphinxcode{\sphinxupquote{lv\_btnm\_set\_btn\_ctrl\_all(btnm, btn\_id, LV\_BTNM\_CTRL\_...)}} and \sphinxcode{\sphinxupquote{lv\_btnm\_clear\_btn\_ctrl\_all(btnm, btn\_id, LV\_BTNM\_CTRL\_...)}}.

The \sphinxstyleemphasis{“One toggle”} feature can be enable with \sphinxcode{\sphinxupquote{lv\_btnm\_set\_one\_toggle(btnm, true)}} to allow only one toggled button at once.

The set a control map for a butto nmatrix (similarly to the map for the text) use \sphinxcode{\sphinxupquote{lv\_btnm\_set\_ctrl\_map(btnm, ctrl\_map)}}.
An element of \sphinxcode{\sphinxupquote{ctrl\_map}} should look like \sphinxcode{\sphinxupquote{ctrl\_map{[}0{]} = width \textbar{} LV\_BTNM\_CTRL\_NO\_REPEAT \textbar{}  LV\_BTNM\_CTRL\_TGL\_ENABLE}}. The number of elemnts should be equal to the number of buttons (excluiding new lines).

The \sphinxstylestrong{texts} on the button can be \sphinxstylestrong{recolored} similarly to the recolor feature for {\hyperref[\detokenize{object-types/label::doc}]{\sphinxcrossref{\DUrole{doc,doc,doc}{Label}}}} object. To enabel it use \sphinxcode{\sphinxupquote{lv\_btnm\_set\_recolor(btnm, true)}}. After that a button with \sphinxcode{\sphinxupquote{\#FF0000 Red\#}} text will be red.


\subsubsection{Styles}
\label{\detokenize{object-types/btnm:styles}}
The Button matrix works with 6 styles: a background and 5 button styles for each states. You can set the styles with \sphinxcode{\sphinxupquote{lv\_btnm\_set\_style(btn, LV\_BTNM\_STYLE\_..., \&style)}}.
The background and the buttons use the \sphinxcode{\sphinxupquote{style.body}} properties. The labels use the \sphinxcode{\sphinxupquote{style.text}} properties of the button styles.
\begin{itemize}
\item {} 
\sphinxstylestrong{LV\_BTNM\_STYLE\_BG} Background style. Uses all \sphinxstyleemphasis{style.body} properties including \sphinxstyleemphasis{padding} Default: \sphinxstyleemphasis{lv\_style\_pretty}

\item {} 
\sphinxstylestrong{LV\_BTNM\_STYLE\_BTN\_REL} style of the released  buttons. Default: \sphinxstyleemphasis{lv\_style\_btn\_rel}

\item {} 
\sphinxstylestrong{LV\_BTNM\_STYLE\_BTN\_PR} style of the pressed buttons. Default: \sphinxstyleemphasis{lv\_style\_btn\_pr}

\item {} 
\sphinxstylestrong{LV\_BTNM\_STYLE\_BTN\_TGL\_REL} style of the toggled released  buttons. Default: \sphinxstyleemphasis{lv\_style\_btn\_tgl\_rel}

\item {} 
\sphinxstylestrong{LV\_BTNM\_STYLE\_BTN\_TGL\_PR} style of the toggled pressed  buttons. Default: \sphinxstyleemphasis{lv\_style\_btn\_tgl\_pr}

\item {} 
\sphinxstylestrong{LV\_BTNM\_STYLE\_BTN\_INA} style of the inactive  buttons. Default: \sphinxstyleemphasis{lv\_style\_btn\_ina}

\end{itemize}


\subsubsection{Events}
\label{\detokenize{object-types/btnm:events}}
Besided the \sphinxhref{/overview/events.html\#generic-events}{Genreric events} the following \sphinxhref{/overview/events.html\#special-events}{Special events} are sent by the button matrices:
\begin{itemize}
\item {} 
\sphinxstylestrong{LV\_EVENT\_VALUE\_CHANGED} sent when the button is pressed/released or repeated after long press. The event data is set to ID of the pressed/released button.

\end{itemize}

Learn more about {\hyperref[\detokenize{overview/events::doc}]{\sphinxcrossref{\DUrole{doc,doc,doc}{Events}}}}.

\#\#Keys

The following \sphinxstyleemphasis{Keys} are processed by the Buttons:
\begin{itemize}
\item {} 
\sphinxstylestrong{LV\_KEY\_RIGHT/UP/LEFT/RIGHT} To navigate among the buttons and elect one

\item {} 
\sphinxstylestrong{LV\_KEY\_ENTER} To press/release the selected button

\end{itemize}

Learn more about {\hyperref[\detokenize{overview/indev::doc}]{\sphinxcrossref{\DUrole{doc,doc,doc}{Keys}}}}.


\subsubsection{Example}
\label{\detokenize{object-types/btnm:example}}

\paragraph{C}
\label{\detokenize{object-types/btnm:c}}
\sphinxincludegraphics{{btnm_1}.png}Button matrix image

code

\begin{sphinxVerbatim}[commandchars=\\\{\}]

\PYG{c+cm}{/*Called when a button is released ot long pressed*/}
\PYG{k}{static} \PYG{n}{lv\PYGZus{}res\PYGZus{}t} \PYG{n+nf}{btnm\PYGZus{}action}\PYG{p}{(}\PYG{n}{lv\PYGZus{}obj\PYGZus{}t} \PYG{o}{*} \PYG{n}{btnm}\PYG{p}{,} \PYG{k}{const} \PYG{k+kt}{char} \PYG{o}{*}\PYG{n}{txt}\PYG{p}{)}
\PYG{p}{\PYGZob{}}
    \PYG{n}{printf}\PYG{p}{(}\PYG{l+s}{\PYGZdq{}}\PYG{l+s}{Button: \PYGZpc{}s released}\PYG{l+s+se}{\PYGZbs{}n}\PYG{l+s}{\PYGZdq{}}\PYG{p}{,} \PYG{n}{txt}\PYG{p}{)}\PYG{p}{;}

    \PYG{k}{return} \PYG{n}{LV\PYGZus{}RES\PYGZus{}OK}\PYG{p}{;} \PYG{c+cm}{/*Return OK because the button matrix is not deleted*/}
\PYG{p}{\PYGZcb{}}


\PYG{p}{.}
\PYG{p}{.}
\PYG{p}{.}

\PYG{c+cm}{/*Create a button descriptor string array*/}
\PYG{k}{static} \PYG{k}{const} \PYG{k+kt}{char} \PYG{o}{*} \PYG{n}{btnm\PYGZus{}map}\PYG{p}{[}\PYG{p}{]} \PYG{o}{=} \PYG{p}{\PYGZob{}}\PYG{l+s}{\PYGZdq{}}\PYG{l+s}{1}\PYG{l+s}{\PYGZdq{}}\PYG{p}{,} \PYG{l+s}{\PYGZdq{}}\PYG{l+s}{2}\PYG{l+s}{\PYGZdq{}}\PYG{p}{,} \PYG{l+s}{\PYGZdq{}}\PYG{l+s}{3}\PYG{l+s}{\PYGZdq{}}\PYG{p}{,} \PYG{l+s}{\PYGZdq{}}\PYG{l+s}{4}\PYG{l+s}{\PYGZdq{}}\PYG{p}{,} \PYG{l+s}{\PYGZdq{}}\PYG{l+s}{5}\PYG{l+s}{\PYGZdq{}}\PYG{p}{,} \PYG{l+s}{\PYGZdq{}}\PYG{l+s+se}{\PYGZbs{}n}\PYG{l+s}{\PYGZdq{}}\PYG{p}{,}
                           \PYG{l+s}{\PYGZdq{}}\PYG{l+s}{6}\PYG{l+s}{\PYGZdq{}}\PYG{p}{,} \PYG{l+s}{\PYGZdq{}}\PYG{l+s}{7}\PYG{l+s}{\PYGZdq{}}\PYG{p}{,} \PYG{l+s}{\PYGZdq{}}\PYG{l+s}{8}\PYG{l+s}{\PYGZdq{}}\PYG{p}{,} \PYG{l+s}{\PYGZdq{}}\PYG{l+s}{9}\PYG{l+s}{\PYGZdq{}}\PYG{p}{,} \PYG{l+s}{\PYGZdq{}}\PYG{l+s}{0}\PYG{l+s}{\PYGZdq{}}\PYG{p}{,} \PYG{l+s}{\PYGZdq{}}\PYG{l+s+se}{\PYGZbs{}n}\PYG{l+s}{\PYGZdq{}}\PYG{p}{,}
                           \PYG{l+s}{\PYGZdq{}}\PYG{l+s+se}{\PYGZbs{}202}\PYG{l+s}{Action1}\PYG{l+s}{\PYGZdq{}}\PYG{p}{,} \PYG{l+s}{\PYGZdq{}}\PYG{l+s}{Action2}\PYG{l+s}{\PYGZdq{}}\PYG{p}{,} \PYG{l+s}{\PYGZdq{}}\PYG{l+s}{\PYGZdq{}}\PYG{p}{\PYGZcb{}}\PYG{p}{;}




\PYG{c+cm}{/*Create a default button matrix*/}
\PYG{n}{lv\PYGZus{}obj\PYGZus{}t} \PYG{o}{*} \PYG{n}{btnm1} \PYG{o}{=} \PYG{n}{lv\PYGZus{}btnm\PYGZus{}create}\PYG{p}{(}\PYG{n}{lv\PYGZus{}scr\PYGZus{}act}\PYG{p}{(}\PYG{p}{)}\PYG{p}{,} \PYG{n+nb}{NULL}\PYG{p}{)}\PYG{p}{;}
\PYG{n}{lv\PYGZus{}btnm\PYGZus{}set\PYGZus{}map}\PYG{p}{(}\PYG{n}{btnm1}\PYG{p}{,} \PYG{n}{btnm\PYGZus{}map}\PYG{p}{)}\PYG{p}{;}
\PYG{n}{lv\PYGZus{}btnm\PYGZus{}set\PYGZus{}action}\PYG{p}{(}\PYG{n}{btnm1}\PYG{p}{,} \PYG{n}{btnm\PYGZus{}action}\PYG{p}{)}\PYG{p}{;}
\PYG{n}{lv\PYGZus{}obj\PYGZus{}set\PYGZus{}size}\PYG{p}{(}\PYG{n}{btnm1}\PYG{p}{,} \PYG{n}{LV\PYGZus{}HOR\PYGZus{}RES}\PYG{p}{,} \PYG{n}{LV\PYGZus{}VER\PYGZus{}RES} \PYG{o}{/} \PYG{l+m+mi}{2}\PYG{p}{)}\PYG{p}{;}

\PYG{c+cm}{/*Create a new style for the button matrix back ground*/}
\PYG{k}{static} \PYG{n}{lv\PYGZus{}style\PYGZus{}t} \PYG{n}{style\PYGZus{}bg}\PYG{p}{;}
\PYG{n}{lv\PYGZus{}style\PYGZus{}copy}\PYG{p}{(}\PYG{o}{\PYGZam{}}\PYG{n}{style\PYGZus{}bg}\PYG{p}{,} \PYG{o}{\PYGZam{}}\PYG{n}{lv\PYGZus{}style\PYGZus{}plain}\PYG{p}{)}\PYG{p}{;}
\PYG{n}{style\PYGZus{}bg}\PYG{p}{.}\PYG{n}{body}\PYG{p}{.}\PYG{n}{main\PYGZus{}color} \PYG{o}{=} \PYG{n}{LV\PYGZus{}COLOR\PYGZus{}SILVER}\PYG{p}{;}
\PYG{n}{style\PYGZus{}bg}\PYG{p}{.}\PYG{n}{body}\PYG{p}{.}\PYG{n}{grad\PYGZus{}color} \PYG{o}{=} \PYG{n}{LV\PYGZus{}COLOR\PYGZus{}SILVER}\PYG{p}{;}
\PYG{n}{style\PYGZus{}bg}\PYG{p}{.}\PYG{n}{body}\PYG{p}{.}\PYG{n}{padding}\PYG{p}{.}\PYG{n}{hor} \PYG{o}{=} \PYG{l+m+mi}{0}\PYG{p}{;}
\PYG{n}{style\PYGZus{}bg}\PYG{p}{.}\PYG{n}{body}\PYG{p}{.}\PYG{n}{padding}\PYG{p}{.}\PYG{n}{ver} \PYG{o}{=} \PYG{l+m+mi}{0}\PYG{p}{;}
\PYG{n}{style\PYGZus{}bg}\PYG{p}{.}\PYG{n}{body}\PYG{p}{.}\PYG{n}{padding}\PYG{p}{.}\PYG{n}{inner} \PYG{o}{=} \PYG{l+m+mi}{0}\PYG{p}{;}

\PYG{c+cm}{/*Create 2 button styles*/}
\PYG{k}{static} \PYG{n}{lv\PYGZus{}style\PYGZus{}t} \PYG{n}{style\PYGZus{}btn\PYGZus{}rel}\PYG{p}{;}
\PYG{k}{static} \PYG{n}{lv\PYGZus{}style\PYGZus{}t} \PYG{n}{style\PYGZus{}btn\PYGZus{}pr}\PYG{p}{;}
\PYG{n}{lv\PYGZus{}style\PYGZus{}copy}\PYG{p}{(}\PYG{o}{\PYGZam{}}\PYG{n}{style\PYGZus{}btn\PYGZus{}rel}\PYG{p}{,} \PYG{o}{\PYGZam{}}\PYG{n}{lv\PYGZus{}style\PYGZus{}btn\PYGZus{}rel}\PYG{p}{)}\PYG{p}{;}
\PYG{n}{style\PYGZus{}btn\PYGZus{}rel}\PYG{p}{.}\PYG{n}{body}\PYG{p}{.}\PYG{n}{main\PYGZus{}color} \PYG{o}{=} \PYG{n}{LV\PYGZus{}COLOR\PYGZus{}MAKE}\PYG{p}{(}\PYG{l+m+mh}{0x30}\PYG{p}{,} \PYG{l+m+mh}{0x30}\PYG{p}{,} \PYG{l+m+mh}{0x30}\PYG{p}{)}\PYG{p}{;}
\PYG{n}{style\PYGZus{}btn\PYGZus{}rel}\PYG{p}{.}\PYG{n}{body}\PYG{p}{.}\PYG{n}{grad\PYGZus{}color} \PYG{o}{=} \PYG{n}{LV\PYGZus{}COLOR\PYGZus{}BLACK}\PYG{p}{;}
\PYG{n}{style\PYGZus{}btn\PYGZus{}rel}\PYG{p}{.}\PYG{n}{body}\PYG{p}{.}\PYG{n}{border}\PYG{p}{.}\PYG{n}{color} \PYG{o}{=} \PYG{n}{LV\PYGZus{}COLOR\PYGZus{}SILVER}\PYG{p}{;}
\PYG{n}{style\PYGZus{}btn\PYGZus{}rel}\PYG{p}{.}\PYG{n}{body}\PYG{p}{.}\PYG{n}{border}\PYG{p}{.}\PYG{n}{width} \PYG{o}{=} \PYG{l+m+mi}{1}\PYG{p}{;}
\PYG{n}{style\PYGZus{}btn\PYGZus{}rel}\PYG{p}{.}\PYG{n}{body}\PYG{p}{.}\PYG{n}{border}\PYG{p}{.}\PYG{n}{opa} \PYG{o}{=} \PYG{n}{LV\PYGZus{}OPA\PYGZus{}50}\PYG{p}{;}
\PYG{n}{style\PYGZus{}btn\PYGZus{}rel}\PYG{p}{.}\PYG{n}{body}\PYG{p}{.}\PYG{n}{radius} \PYG{o}{=} \PYG{l+m+mi}{0}\PYG{p}{;}

\PYG{n}{lv\PYGZus{}style\PYGZus{}copy}\PYG{p}{(}\PYG{o}{\PYGZam{}}\PYG{n}{style\PYGZus{}btn\PYGZus{}pr}\PYG{p}{,} \PYG{o}{\PYGZam{}}\PYG{n}{style\PYGZus{}btn\PYGZus{}rel}\PYG{p}{)}\PYG{p}{;}
\PYG{n}{style\PYGZus{}btn\PYGZus{}pr}\PYG{p}{.}\PYG{n}{body}\PYG{p}{.}\PYG{n}{main\PYGZus{}color} \PYG{o}{=} \PYG{n}{LV\PYGZus{}COLOR\PYGZus{}MAKE}\PYG{p}{(}\PYG{l+m+mh}{0x55}\PYG{p}{,} \PYG{l+m+mh}{0x96}\PYG{p}{,} \PYG{l+m+mh}{0xd8}\PYG{p}{)}\PYG{p}{;}
\PYG{n}{style\PYGZus{}btn\PYGZus{}pr}\PYG{p}{.}\PYG{n}{body}\PYG{p}{.}\PYG{n}{grad\PYGZus{}color} \PYG{o}{=} \PYG{n}{LV\PYGZus{}COLOR\PYGZus{}MAKE}\PYG{p}{(}\PYG{l+m+mh}{0x37}\PYG{p}{,} \PYG{l+m+mh}{0x62}\PYG{p}{,} \PYG{l+m+mh}{0x90}\PYG{p}{)}\PYG{p}{;}
\PYG{n}{style\PYGZus{}btn\PYGZus{}pr}\PYG{p}{.}\PYG{n}{text}\PYG{p}{.}\PYG{n}{color} \PYG{o}{=} \PYG{n}{LV\PYGZus{}COLOR\PYGZus{}MAKE}\PYG{p}{(}\PYG{l+m+mh}{0xbb}\PYG{p}{,} \PYG{l+m+mh}{0xd5}\PYG{p}{,} \PYG{l+m+mh}{0xf1}\PYG{p}{)}\PYG{p}{;}

\PYG{c+cm}{/*Create a second button matrix with the new styles*/}
\PYG{n}{lv\PYGZus{}obj\PYGZus{}t} \PYG{o}{*} \PYG{n}{btnm2} \PYG{o}{=} \PYG{n}{lv\PYGZus{}btnm\PYGZus{}create}\PYG{p}{(}\PYG{n}{lv\PYGZus{}scr\PYGZus{}act}\PYG{p}{(}\PYG{p}{)}\PYG{p}{,} \PYG{n}{btnm1}\PYG{p}{)}\PYG{p}{;}
\PYG{n}{lv\PYGZus{}btnm\PYGZus{}set\PYGZus{}style}\PYG{p}{(}\PYG{n}{btnm2}\PYG{p}{,} \PYG{n}{LV\PYGZus{}BTNM\PYGZus{}STYLE\PYGZus{}BG}\PYG{p}{,} \PYG{o}{\PYGZam{}}\PYG{n}{style\PYGZus{}bg}\PYG{p}{)}\PYG{p}{;}
\PYG{n}{lv\PYGZus{}btnm\PYGZus{}set\PYGZus{}style}\PYG{p}{(}\PYG{n}{btnm2}\PYG{p}{,} \PYG{n}{LV\PYGZus{}BTNM\PYGZus{}STYLE\PYGZus{}BTN\PYGZus{}REL}\PYG{p}{,} \PYG{o}{\PYGZam{}}\PYG{n}{style\PYGZus{}btn\PYGZus{}rel}\PYG{p}{)}\PYG{p}{;}
\PYG{n}{lv\PYGZus{}btnm\PYGZus{}set\PYGZus{}style}\PYG{p}{(}\PYG{n}{btnm2}\PYG{p}{,} \PYG{n}{LV\PYGZus{}BTNM\PYGZus{}STYLE\PYGZus{}BTN\PYGZus{}PR}\PYG{p}{,} \PYG{o}{\PYGZam{}}\PYG{n}{style\PYGZus{}btn\PYGZus{}pr}\PYG{p}{)}\PYG{p}{;}
\PYG{n}{lv\PYGZus{}obj\PYGZus{}align}\PYG{p}{(}\PYG{n}{btnm2}\PYG{p}{,} \PYG{n}{btnm1}\PYG{p}{,} \PYG{n}{LV\PYGZus{}ALIGN\PYGZus{}OUT\PYGZus{}BOTTOM\PYGZus{}MID}\PYG{p}{,} \PYG{l+m+mi}{0}\PYG{p}{,} \PYG{l+m+mi}{0}\PYG{p}{)}\PYG{p}{;}
\end{sphinxVerbatim}


\paragraph{MicroPython}
\label{\detokenize{object-types/btnm:micropython}}
No examples yet.


\subsubsection{API}
\label{\detokenize{object-types/btnm:api}}

\subsection{Calendar (lv\_calendar)}
\label{\detokenize{object-types/calendar:calendar-lv-calendar}}\label{\detokenize{object-types/calendar::doc}}

\subsubsection{Overview}
\label{\detokenize{object-types/calendar:overview}}
The Calendar object is a classic calendar which can:
\begin{itemize}
\item {} 
highlight the current day and week

\item {} 
highlight any user-defined dates

\item {} 
display the name of the days

\item {} 
go the next/previous month by button click

\item {} 
highlight the clicked day

\end{itemize}

The set and get dates in the calendar the \sphinxcode{\sphinxupquote{lv\_calendar\_date\_t}} type is used which is a structure with \sphinxcode{\sphinxupquote{year}}, \sphinxcode{\sphinxupquote{month}} and \sphinxcode{\sphinxupquote{day}} fields.

To set the \sphinxstylestrong{current date} use the \sphinxcode{\sphinxupquote{lv\_calendar\_set\_today\_date(calendar, \&today\_date)}} function.

To set the \sphinxstylestrong{shown date} use \sphinxcode{\sphinxupquote{lv\_calendar\_set\_shown\_date(calendar, \&shown\_date)}};

The list of \sphinxstylestrong{highlighted dates} should be stored in a \sphinxcode{\sphinxupquote{lv\_calendar\_date\_t}} array a loaded by \sphinxcode{\sphinxupquote{lv\_calendar\_set\_highlighted\_dates(calendar, \&highlighted\_dates)}}.Only the arrays pointer will be saved so the array should be a static or global variable.

The \sphinxstylestrong{name of the days} can be adjusted with \sphinxcode{\sphinxupquote{lv\_calendar\_set\_day\_names(calendar, day\_names)}} where \sphinxcode{\sphinxupquote{day\_names}} looks like \sphinxcode{\sphinxupquote{const char * day\_names{[}7{]} = \{"Su", "Mo", ...\};}}


\subsubsection{Styles}
\label{\detokenize{object-types/calendar:styles}}
You can set the styles with \sphinxcode{\sphinxupquote{lv\_calendar\_set\_style(btn, LV\_CALENDAR\_STYLE\_..., \&style)}}.
\begin{itemize}
\item {} 
\sphinxstylestrong{LV\_CALENDAR\_STYLE\_BG} Style of the background using the \sphinxcode{\sphinxupquote{body}} properties and the style of the date numbers using the \sphinxcode{\sphinxupquote{text}} properties. \sphinxcode{\sphinxupquote{body.padding.left/rigth/bottom}} paddig will be added on the edges. around the date numbers.

\item {} 
\sphinxstylestrong{LV\_CALENDAR\_STYLE\_HEADER} Style of the header where the current year and month is displayed. \sphinxcode{\sphinxupquote{body}} and \sphinxcode{\sphinxupquote{text}} properties are used.

\item {} 
\sphinxstylestrong{LV\_CALENDAR\_STYLE\_HEADER\_PR} Pressed header style, used when the next/prev. month button is being pressed. \sphinxcode{\sphinxupquote{text}} properties are used by the arrows.

\item {} 
\sphinxstylestrong{LV\_CALENDAR\_STYLE\_DAY\_NAMES} Style of the day names. \sphinxcode{\sphinxupquote{text}} properties are used by the day texts and \sphinxcode{\sphinxupquote{body.padding.top}} determines the space above the day names.

\item {} 
\sphinxstylestrong{LV\_CALENDAR\_STYLE\_HIGHLIGHTED\_DAYS} \sphinxcode{\sphinxupquote{text}} properties are used to adjust the style of the highlights days

\item {} 
\sphinxstylestrong{LV\_CALENDAR\_STYLE\_INACTIVE\_DAYS} \sphinxcode{\sphinxupquote{text}} properties are used to adjust the style of the visible days of previous/next month.

\item {} 
\sphinxstylestrong{LV\_CALENDAR\_STYLE\_WEEK\_BOX} \sphinxcode{\sphinxupquote{body}} properties are used to set the style of the week box

\item {} 
\sphinxstylestrong{LV\_CALENDAR\_STYLE\_TODAY\_BOX} \sphinxcode{\sphinxupquote{body}} and \sphinxcode{\sphinxupquote{text}} properties are used to set the style of the today box

\end{itemize}


\subsubsection{Events}
\label{\detokenize{object-types/calendar:events}}
Besided the \sphinxhref{/overview/events.html\#generic-events}{Genreric events} the following \sphinxhref{/overview/events.html\#special-events}{Special events} are sent by the calendars:
\sphinxstylestrong{LV\_EVENT\_VALUE\_CHANGED} is sent when  the current month has changed.

In Input device realted events \sphinxcode{\sphinxupquote{lv\_calendar\_get\_pressed\_date(caledar)}} tells which day is currently being pressed or return \sphinxcode{\sphinxupquote{NULL}} if no date is pressed.


\subsubsection{Keys}
\label{\detokenize{object-types/calendar:keys}}
No \sphinxstyleemphasis{Keys} are processed by the object type.

Learn more about {\hyperref[\detokenize{overview/indev::doc}]{\sphinxcrossref{\DUrole{doc,doc,doc}{Keys}}}}.


\subsubsection{Example}
\label{\detokenize{object-types/calendar:example}}

\paragraph{C}
\label{\detokenize{object-types/calendar:c}}
\sphinxincludegraphics{{calendar_1}.png}Calendar image

code

\begin{sphinxVerbatim}[commandchars=\\\{\}]
\PYG{k}{static} \PYG{n}{lv\PYGZus{}style\PYGZus{}t} \PYG{n}{style\PYGZus{}week\PYGZus{}box}\PYG{p}{;}
\PYG{n}{lv\PYGZus{}style\PYGZus{}copy}\PYG{p}{(}\PYG{o}{\PYGZam{}}\PYG{n}{style\PYGZus{}week\PYGZus{}box}\PYG{p}{,} \PYG{o}{\PYGZam{}}\PYG{n}{lv\PYGZus{}style\PYGZus{}plain}\PYG{p}{)}\PYG{p}{;}
\PYG{n}{style\PYGZus{}week\PYGZus{}box}\PYG{p}{.}\PYG{n}{body}\PYG{p}{.}\PYG{n}{border}\PYG{p}{.}\PYG{n}{width} \PYG{o}{=} \PYG{l+m+mi}{1}\PYG{p}{;}
\PYG{n}{style\PYGZus{}week\PYGZus{}box}\PYG{p}{.}\PYG{n}{body}\PYG{p}{.}\PYG{n}{border}\PYG{p}{.}\PYG{n}{color} \PYG{o}{=} \PYG{n}{LV\PYGZus{}COLOR\PYGZus{}HEX3}\PYG{p}{(}\PYG{l+m+mh}{0x333}\PYG{p}{)}\PYG{p}{;}
\PYG{n}{style\PYGZus{}week\PYGZus{}box}\PYG{p}{.}\PYG{n}{body}\PYG{p}{.}\PYG{n}{empty} \PYG{o}{=} \PYG{l+m+mi}{1}\PYG{p}{;}
\PYG{n}{style\PYGZus{}week\PYGZus{}box}\PYG{p}{.}\PYG{n}{body}\PYG{p}{.}\PYG{n}{radius} \PYG{o}{=} \PYG{n}{LV\PYGZus{}RADIUS\PYGZus{}CIRCLE}\PYG{p}{;}
\PYG{n}{style\PYGZus{}week\PYGZus{}box}\PYG{p}{.}\PYG{n}{body}\PYG{p}{.}\PYG{n}{padding}\PYG{p}{.}\PYG{n}{ver} \PYG{o}{=} \PYG{l+m+mi}{3}\PYG{p}{;}
\PYG{n}{style\PYGZus{}week\PYGZus{}box}\PYG{p}{.}\PYG{n}{body}\PYG{p}{.}\PYG{n}{padding}\PYG{p}{.}\PYG{n}{hor} \PYG{o}{=} \PYG{l+m+mi}{3}\PYG{p}{;}

\PYG{c+cm}{/*Create a style for today*/}
\PYG{k}{static} \PYG{n}{lv\PYGZus{}style\PYGZus{}t} \PYG{n}{style\PYGZus{}today\PYGZus{}box}\PYG{p}{;}
\PYG{n}{lv\PYGZus{}style\PYGZus{}copy}\PYG{p}{(}\PYG{o}{\PYGZam{}}\PYG{n}{style\PYGZus{}today\PYGZus{}box}\PYG{p}{,} \PYG{o}{\PYGZam{}}\PYG{n}{lv\PYGZus{}style\PYGZus{}plain}\PYG{p}{)}\PYG{p}{;}
\PYG{n}{style\PYGZus{}today\PYGZus{}box}\PYG{p}{.}\PYG{n}{body}\PYG{p}{.}\PYG{n}{border}\PYG{p}{.}\PYG{n}{width} \PYG{o}{=} \PYG{l+m+mi}{2}\PYG{p}{;}
\PYG{n}{style\PYGZus{}today\PYGZus{}box}\PYG{p}{.}\PYG{n}{body}\PYG{p}{.}\PYG{n}{border}\PYG{p}{.}\PYG{n}{color} \PYG{o}{=} \PYG{n}{LV\PYGZus{}COLOR\PYGZus{}NAVY}\PYG{p}{;}
\PYG{n}{style\PYGZus{}today\PYGZus{}box}\PYG{p}{.}\PYG{n}{body}\PYG{p}{.}\PYG{n}{empty} \PYG{o}{=} \PYG{l+m+mi}{1}\PYG{p}{;}
\PYG{n}{style\PYGZus{}today\PYGZus{}box}\PYG{p}{.}\PYG{n}{body}\PYG{p}{.}\PYG{n}{radius} \PYG{o}{=} \PYG{n}{LV\PYGZus{}RADIUS\PYGZus{}CIRCLE}\PYG{p}{;}
\PYG{n}{style\PYGZus{}today\PYGZus{}box}\PYG{p}{.}\PYG{n}{body}\PYG{p}{.}\PYG{n}{padding}\PYG{p}{.}\PYG{n}{ver} \PYG{o}{=} \PYG{l+m+mi}{3}\PYG{p}{;}
\PYG{n}{style\PYGZus{}today\PYGZus{}box}\PYG{p}{.}\PYG{n}{body}\PYG{p}{.}\PYG{n}{padding}\PYG{p}{.}\PYG{n}{hor} \PYG{o}{=} \PYG{l+m+mi}{3}\PYG{p}{;}
\PYG{n}{style\PYGZus{}today\PYGZus{}box}\PYG{p}{.}\PYG{n}{text}\PYG{p}{.}\PYG{n}{color}\PYG{o}{=} \PYG{n}{LV\PYGZus{}COLOR\PYGZus{}BLUE}\PYG{p}{;}

\PYG{c+cm}{/*Create a style for the highlighted days*/}
\PYG{k}{static} \PYG{n}{lv\PYGZus{}style\PYGZus{}t} \PYG{n}{style\PYGZus{}highlighted\PYGZus{}day}\PYG{p}{;}
\PYG{n}{lv\PYGZus{}style\PYGZus{}copy}\PYG{p}{(}\PYG{o}{\PYGZam{}}\PYG{n}{style\PYGZus{}highlighted\PYGZus{}day}\PYG{p}{,} \PYG{o}{\PYGZam{}}\PYG{n}{lv\PYGZus{}style\PYGZus{}plain}\PYG{p}{)}\PYG{p}{;}
\PYG{n}{style\PYGZus{}highlighted\PYGZus{}day}\PYG{p}{.}\PYG{n}{body}\PYG{p}{.}\PYG{n}{border}\PYG{p}{.}\PYG{n}{width} \PYG{o}{=} \PYG{l+m+mi}{2}\PYG{p}{;}
\PYG{n}{style\PYGZus{}highlighted\PYGZus{}day}\PYG{p}{.}\PYG{n}{body}\PYG{p}{.}\PYG{n}{border}\PYG{p}{.}\PYG{n}{color} \PYG{o}{=} \PYG{n}{LV\PYGZus{}COLOR\PYGZus{}NAVY}\PYG{p}{;}
\PYG{n}{style\PYGZus{}highlighted\PYGZus{}day}\PYG{p}{.}\PYG{n}{body}\PYG{p}{.}\PYG{n}{empty} \PYG{o}{=} \PYG{l+m+mi}{1}\PYG{p}{;}
\PYG{n}{style\PYGZus{}highlighted\PYGZus{}day}\PYG{p}{.}\PYG{n}{body}\PYG{p}{.}\PYG{n}{radius} \PYG{o}{=} \PYG{n}{LV\PYGZus{}RADIUS\PYGZus{}CIRCLE}\PYG{p}{;}
\PYG{n}{style\PYGZus{}highlighted\PYGZus{}day}\PYG{p}{.}\PYG{n}{body}\PYG{p}{.}\PYG{n}{padding}\PYG{p}{.}\PYG{n}{ver} \PYG{o}{=} \PYG{l+m+mi}{3}\PYG{p}{;}
\PYG{n}{style\PYGZus{}highlighted\PYGZus{}day}\PYG{p}{.}\PYG{n}{body}\PYG{p}{.}\PYG{n}{padding}\PYG{p}{.}\PYG{n}{hor} \PYG{o}{=} \PYG{l+m+mi}{3}\PYG{p}{;}
\PYG{n}{style\PYGZus{}highlighted\PYGZus{}day}\PYG{p}{.}\PYG{n}{text}\PYG{p}{.}\PYG{n}{color}\PYG{o}{=} \PYG{n}{LV\PYGZus{}COLOR\PYGZus{}BLUE}\PYG{p}{;}

\PYG{c+cm}{/*Apply the styles*/}
\PYG{n}{lv\PYGZus{}calendar\PYGZus{}set\PYGZus{}style}\PYG{p}{(}\PYG{n}{calendar}\PYG{p}{,} \PYG{n}{LV\PYGZus{}CALENDAR\PYGZus{}STYLE\PYGZus{}WEEK\PYGZus{}BOX}\PYG{p}{,} \PYG{o}{\PYGZam{}}\PYG{n}{style\PYGZus{}week\PYGZus{}box}\PYG{p}{)}\PYG{p}{;}
\PYG{n}{lv\PYGZus{}calendar\PYGZus{}set\PYGZus{}style}\PYG{p}{(}\PYG{n}{calendar}\PYG{p}{,} \PYG{n}{LV\PYGZus{}CALENDAR\PYGZus{}STYLE\PYGZus{}TODAY\PYGZus{}BOX}\PYG{p}{,} \PYG{o}{\PYGZam{}}\PYG{n}{style\PYGZus{}today\PYGZus{}box}\PYG{p}{)}\PYG{p}{;}
\PYG{n}{lv\PYGZus{}calendar\PYGZus{}set\PYGZus{}style}\PYG{p}{(}\PYG{n}{calendar}\PYG{p}{,} \PYG{n}{LV\PYGZus{}CALENDAR\PYGZus{}STYLE\PYGZus{}HIGHLIGHTED\PYGZus{}DAYS}\PYG{p}{,} \PYG{o}{\PYGZam{}}\PYG{n}{style\PYGZus{}highlighted\PYGZus{}day}\PYG{p}{)}\PYG{p}{;}


\PYG{c+cm}{/*Set the today*/}
\PYG{n}{lv\PYGZus{}calendar\PYGZus{}date\PYGZus{}t} \PYG{n}{today}\PYG{p}{;}
\PYG{n}{today}\PYG{p}{.}\PYG{n}{year} \PYG{o}{=} \PYG{l+m+mi}{2018}\PYG{p}{;}
\PYG{n}{today}\PYG{p}{.}\PYG{n}{month} \PYG{o}{=} \PYG{l+m+mi}{10}\PYG{p}{;}
\PYG{n}{today}\PYG{p}{.}\PYG{n}{day} \PYG{o}{=} \PYG{l+m+mi}{23}\PYG{p}{;}

\PYG{n}{lv\PYGZus{}calendar\PYGZus{}set\PYGZus{}today\PYGZus{}date}\PYG{p}{(}\PYG{n}{calendar}\PYG{p}{,} \PYG{o}{\PYGZam{}}\PYG{n}{today}\PYG{p}{)}\PYG{p}{;}
\PYG{n}{lv\PYGZus{}calendar\PYGZus{}set\PYGZus{}showed\PYGZus{}date}\PYG{p}{(}\PYG{n}{calendar}\PYG{p}{,} \PYG{o}{\PYGZam{}}\PYG{n}{today}\PYG{p}{)}\PYG{p}{;}

\PYG{c+cm}{/*Highlight some days*/}
\PYG{k}{static} \PYG{n}{lv\PYGZus{}calendar\PYGZus{}date\PYGZus{}t} \PYG{n}{highlihted\PYGZus{}days}\PYG{p}{[}\PYG{l+m+mi}{3}\PYG{p}{]}\PYG{p}{;}       \PYG{c+cm}{/*Only it\PYGZsq{}s pointer will be saved so should be static*/}
\PYG{n}{highlihted\PYGZus{}days}\PYG{p}{[}\PYG{l+m+mi}{0}\PYG{p}{]}\PYG{p}{.}\PYG{n}{year} \PYG{o}{=} \PYG{l+m+mi}{2018}\PYG{p}{;}
\PYG{n}{highlihted\PYGZus{}days}\PYG{p}{[}\PYG{l+m+mi}{0}\PYG{p}{]}\PYG{p}{.}\PYG{n}{month} \PYG{o}{=} \PYG{l+m+mi}{10}\PYG{p}{;}
\PYG{n}{highlihted\PYGZus{}days}\PYG{p}{[}\PYG{l+m+mi}{0}\PYG{p}{]}\PYG{p}{.}\PYG{n}{day} \PYG{o}{=} \PYG{l+m+mi}{6}\PYG{p}{;}

\PYG{n}{highlihted\PYGZus{}days}\PYG{p}{[}\PYG{l+m+mi}{1}\PYG{p}{]}\PYG{p}{.}\PYG{n}{year} \PYG{o}{=} \PYG{l+m+mi}{2018}\PYG{p}{;}
\PYG{n}{highlihted\PYGZus{}days}\PYG{p}{[}\PYG{l+m+mi}{1}\PYG{p}{]}\PYG{p}{.}\PYG{n}{month} \PYG{o}{=} \PYG{l+m+mi}{10}\PYG{p}{;}
\PYG{n}{highlihted\PYGZus{}days}\PYG{p}{[}\PYG{l+m+mi}{1}\PYG{p}{]}\PYG{p}{.}\PYG{n}{day} \PYG{o}{=} \PYG{l+m+mi}{11}\PYG{p}{;}

\PYG{n}{highlihted\PYGZus{}days}\PYG{p}{[}\PYG{l+m+mi}{2}\PYG{p}{]}\PYG{p}{.}\PYG{n}{year} \PYG{o}{=} \PYG{l+m+mi}{2018}\PYG{p}{;}
\PYG{n}{highlihted\PYGZus{}days}\PYG{p}{[}\PYG{l+m+mi}{2}\PYG{p}{]}\PYG{p}{.}\PYG{n}{month} \PYG{o}{=} \PYG{l+m+mi}{11}\PYG{p}{;}
\PYG{n}{highlihted\PYGZus{}days}\PYG{p}{[}\PYG{l+m+mi}{2}\PYG{p}{]}\PYG{p}{.}\PYG{n}{day} \PYG{o}{=} \PYG{l+m+mi}{22}\PYG{p}{;}

\PYG{n}{lv\PYGZus{}calendar\PYGZus{}set\PYGZus{}highlighted\PYGZus{}dates}\PYG{p}{(}\PYG{n}{calendar}\PYG{p}{,} \PYG{n}{highlihted\PYGZus{}days}\PYG{p}{,} \PYG{l+m+mi}{3}\PYG{p}{)}\PYG{p}{;}
\end{sphinxVerbatim}


\paragraph{MicroPython}
\label{\detokenize{object-types/calendar:micropython}}
No examples yet.


\subsubsection{API}
\label{\detokenize{object-types/calendar:api}}

\subsection{Canvas (lv\_canvas)}
\label{\detokenize{object-types/canvas:canvas-lv-canvas}}\label{\detokenize{object-types/canvas::doc}}

\subsubsection{Overview}
\label{\detokenize{object-types/canvas:overview}}
A canvas is like an image with a buffer where user can draw anything.
To assign a buffer to a canvas use \sphinxcode{\sphinxupquote{lv\_canvas\_set\_buffer(canvas, buffer, width, height, LV\_IMG\_CF\_TRUE\_COLOR\_ALPHA)}}. \sphinxcode{\sphinxupquote{buffer}} is static buffer (not just a local variable) to hold the image of the canvas. For example
\sphinxcode{\sphinxupquote{static lv\_color\_t buffer{[}LV\_CANVAS\_BUF\_SIZE\_TRUE\_COLOR(width, height){]}}}. \sphinxcode{\sphinxupquote{LV\_CANVAS\_BUF\_SIZE\_...}} macros help to determine the size of the buffer with different color formats.

The set a pixel on the canvas use \sphinxcode{\sphinxupquote{lv\_canvas\_set\_px(canvas, x, y, LV\_COLOR\_RED)}}. With \sphinxcode{\sphinxupquote{LV\_IMG\_CF\_INDEXED\_...}} or \sphinxcode{\sphinxupquote{LV\_IMG\_CF\_ALPHA\_...}} the index of the color or the alpha value needs to be passed as color. E.g. \sphinxcode{\sphinxupquote{lv\_color\_t c; c.full = 3;}}

For \sphinxcode{\sphinxupquote{LV\_IMG\_CF\_INDEXED\_...}} color formats the palette needs to set with  \sphinxcode{\sphinxupquote{lv\_canvas\_set\_palette(canvas, 3, LV\_COLOR\_RED)}}. It sets pixels with \sphinxstyleemphasis{index=3} to red.

\sphinxcode{\sphinxupquote{lv\_canvas\_fill\_bg(canvas, LV\_COLOR\_BLUE)}} fills teh whole canvas to blue.

An array of pixel can be copied to the canvas with \sphinxcode{\sphinxupquote{lv\_canvas\_copy\_buf(canvas, buffer\_to\_copy, x, y, width, height)}}. The color format of the buffer and the canvas need match.

To draw something to the canvas use
\begin{itemize}
\item {} 
\sphinxcode{\sphinxupquote{lv\_canvas\_draw\_rect(canvas, x, y, width, heigth, \&style)}}

\item {} 
\sphinxcode{\sphinxupquote{lv\_canvas\_draw\_text(canvas, x, y, max\_width, \&style, txt, LV\_LABEL\_ALIGN\_LEFT/CENTER/RIGTH)}}

\item {} 
\sphinxcode{\sphinxupquote{lv\_canvas\_draw\_img(canvas, x, y, \&img\_src, \&style)}}

\item {} 
\sphinxcode{\sphinxupquote{lv\_canvas\_draw\_line(canvas, point\_array, point\_cnt, \&style)}}

\item {} 
\sphinxcode{\sphinxupquote{lv\_canvas\_draw\_polygon(canvas, points\_array, point\_cnt, \&style)}}

\item {} 
\sphinxcode{\sphinxupquote{lv\_canvas\_draw\_arc(canvas, x, y, radius, start\_angle, end\_angle, \&style)}}

\end{itemize}

An rotated image can be added to canvas with \sphinxcode{\sphinxupquote{lv\_canvas\_rotate(canvas, \&imd\_dsc, angle, x, y, pivot\_x, pivot\_y)}}. It will rotate the image shown by \sphinxcode{\sphinxupquote{img\_dsc}} around the given pivot and stores it on the \sphinxcode{\sphinxupquote{x}}, \sphinxcode{\sphinxupquote{y}} coordinates of \sphinxcode{\sphinxupquote{canvas}}.
Instead of \sphinxcode{\sphinxupquote{img\_dsc}} and the buffer of an other canvas also can be used by \sphinxcode{\sphinxupquote{lv\_canvas\_get\_img(canvas)}}.

Note that a canvas can’t be roteted on itself but a source and destination (the canvas).


\subsubsection{Styles}
\label{\detokenize{object-types/canvas:styles}}
You can set the styles with \sphinxcode{\sphinxupquote{lv\_canvas\_set\_style(btn, LV\_CANVAS\_STYLE\_MAIN, \&style)}}.
\sphinxcode{\sphinxupquote{style.image.color}} is used to tell the base color with \sphinxcode{\sphinxupquote{LV\_IMG\_CF\_ALPHA\_...}} color format.


\subsubsection{Events}
\label{\detokenize{object-types/canvas:events}}
Only the \sphinxhref{/overview/events.html\#generic-events}{Genreric events} are sent by the object type.

Learn more about {\hyperref[\detokenize{overview/events::doc}]{\sphinxcrossref{\DUrole{doc,doc,doc}{Events}}}}.


\subsubsection{Keys}
\label{\detokenize{object-types/canvas:keys}}
No \sphinxstyleemphasis{Keys} are processed by the object type.

Learn more about {\hyperref[\detokenize{overview/indev::doc}]{\sphinxcrossref{\DUrole{doc,doc,doc}{Keys}}}}.


\subsubsection{Example}
\label{\detokenize{object-types/canvas:example}}

\paragraph{C}
\label{\detokenize{object-types/canvas:c}}
code

\begin{sphinxVerbatim}[commandchars=\\\{\}]

\end{sphinxVerbatim}


\paragraph{MicroPython}
\label{\detokenize{object-types/canvas:micropython}}
No examples yet.


\subsubsection{API}
\label{\detokenize{object-types/canvas:api}}

\subsection{Check box (lv\_cb)}
\label{\detokenize{object-types/cb:check-box-lv-cb}}\label{\detokenize{object-types/cb::doc}}

\subsubsection{Overview}
\label{\detokenize{object-types/cb:overview}}
The Check Box objects are built from a Button \sphinxstylestrong{background} which contains an also Button \sphinxstylestrong{bullet} and a \sphinxstylestrong{label} to realize a classical check box.
The \sphinxstylestrong{text} can be modified by the \sphinxcode{\sphinxupquote{lv\_cb\_set\_text(cb, "New text")}} function. It will dynamically alloacte the text.
To set a static text use \sphinxcode{\sphinxupquote{lv\_cb\_set\_static\_text(cb, txt\_buf)}}. This way only a pointer will be stored to \sphinxcode{\sphinxupquote{txt\_buf}} so it needs shouldn’t deallocated while the checkbox exists.

You can manually \sphinxstylestrong{check / un-check} the Check box  via \sphinxcode{\sphinxupquote{lv\_cb\_set\_checked(cb, true/false)}}.

To make the chackbox inactive use \sphinxcode{\sphinxupquote{lv\_cb\_set\_inactive(cb, true)}}.


\subsubsection{Styles}
\label{\detokenize{object-types/cb:styles}}
The Check box styles can be modified with \sphinxcode{\sphinxupquote{lv\_cb\_set\_style(cb, LV\_CB\_STYLE\_..., \&style)}}.
\begin{itemize}
\item {} 
\sphinxstylestrong{LV\_CB\_STYLE\_BG} Background style. Uses all \sphinxcode{\sphinxupquote{style.body}} properties. The label’s style comes from  \sphinxcode{\sphinxupquote{style.text}}. Default: \sphinxcode{\sphinxupquote{lv\_style\_transp}}

\item {} 
\sphinxstylestrong{LV\_CB\_STYLE\_BOX\_REL} Style of the released box. Uses the \sphinxcode{\sphinxupquote{style.body}} properties. Default: \sphinxcode{\sphinxupquote{lv\_style\_btn\_rel}}

\item {} 
\sphinxstylestrong{LV\_CB\_STYLE\_BOX\_PR} Style of the pressed box. Uses the \sphinxcode{\sphinxupquote{style.body}} properties. Default: \sphinxcode{\sphinxupquote{lv\_style\_btn\_pr}}

\item {} 
\sphinxstylestrong{LV\_CB\_STYLE\_BOX\_TGL\_REL} Style of the checked released box. Uses the \sphinxcode{\sphinxupquote{style.body}} properties. Default: \sphinxcode{\sphinxupquote{lv\_style\_btn\_tgl\_rel}}

\item {} 
\sphinxstylestrong{LV\_CB\_STYLE\_BOX\_TGL\_PR} Style of the checked released box. Uses the \sphinxcode{\sphinxupquote{style.body}} properties. Default: \sphinxcode{\sphinxupquote{lv\_style\_btn\_tgl\_pr}}

\item {} 
\sphinxstylestrong{LV\_CB\_STYLE\_BOX\_INA} Style of the inactive box. Uses the \sphinxcode{\sphinxupquote{style.body}} properties. Default: \sphinxcode{\sphinxupquote{lv\_style\_btn\_ina}}

\end{itemize}


\subsubsection{Events}
\label{\detokenize{object-types/cb:events}}
Besided the \sphinxhref{/overview/events.html\#generic-events}{Genreric events} the following \sphinxhref{/overview/events.html\#special-events}{Special events} are sent by the Check boxes:
\begin{itemize}
\item {} 
\sphinxstylestrong{LV\_EVENT\_VALUE\_CHANGED} sent when the Check box is toggled.

\end{itemize}

Note that the generic input device related events (like \sphinxcode{\sphinxupquote{LV\_EVENT\_PRESSED}}) are sent in inactive state too. You need to check the state with \sphinxcode{\sphinxupquote{lv\_cb\_is\_inactive(cb)}} to ignore the events from inactive Chek boxes.

Learn more about {\hyperref[\detokenize{overview/events::doc}]{\sphinxcrossref{\DUrole{doc,doc,doc}{Events}}}}.


\subsubsection{Keys}
\label{\detokenize{object-types/cb:keys}}
The following \sphinxstyleemphasis{Keys} are processed by the Buttons:
\begin{itemize}
\item {} 
\sphinxstylestrong{LV\_KEY\_RIGHT/UP} Go to toggled state if toggling is enabled

\item {} 
\sphinxstylestrong{LV\_KEY\_LEFT/DOWN} Go to non-toggled state if toggling is  enabled

\end{itemize}

Note that, as usual, the state of \sphinxcode{\sphinxupquote{LV\_KEY\_ENTER}} is translated to \sphinxcode{\sphinxupquote{LV\_EVENT\_PRESSED/PRESSING/RELEASED}} etc.

Learn more about {\hyperref[\detokenize{overview/indev::doc}]{\sphinxcrossref{\DUrole{doc,doc,doc}{Keys}}}}.


\subsubsection{Example}
\label{\detokenize{object-types/cb:example}}

\paragraph{C}
\label{\detokenize{object-types/cb:c}}
\sphinxincludegraphics{{cb_1}.png}Checkbox image

code

\begin{sphinxVerbatim}[commandchars=\\\{\}]
\PYG{k}{static} \PYG{n}{lv\PYGZus{}res\PYGZus{}t} \PYG{n+nf}{cb\PYGZus{}release\PYGZus{}action}\PYG{p}{(}\PYG{n}{lv\PYGZus{}obj\PYGZus{}t} \PYG{o}{*} \PYG{n}{cb}\PYG{p}{)}
\PYG{p}{\PYGZob{}}
    \PYG{c+cm}{/*A check box is clicked*/}
    \PYG{n}{printf}\PYG{p}{(}\PYG{l+s}{\PYGZdq{}}\PYG{l+s}{\PYGZpc{}s state: \PYGZpc{}d}\PYG{l+s+se}{\PYGZbs{}n}\PYG{l+s}{\PYGZdq{}}\PYG{p}{,} \PYG{n}{lv\PYGZus{}cb\PYGZus{}get\PYGZus{}text}\PYG{p}{(}\PYG{n}{cb}\PYG{p}{)}\PYG{p}{,} \PYG{n}{lv\PYGZus{}cb\PYGZus{}is\PYGZus{}checked}\PYG{p}{(}\PYG{n}{cb}\PYG{p}{)}\PYG{p}{)}\PYG{p}{;}

    \PYG{k}{return} \PYG{n}{LV\PYGZus{}RES\PYGZus{}OK}\PYG{p}{;}
\PYG{p}{\PYGZcb{}}

\PYG{p}{.}
\PYG{p}{.}
\PYG{p}{.}


\PYG{c+cm}{/********************************************}
\PYG{c+cm}{ * Create a container for the check boxes}
\PYG{c+cm}{ ********************************************/}

\PYG{c+cm}{/*Create  border style*/}
\PYG{k}{static} \PYG{n}{lv\PYGZus{}style\PYGZus{}t} \PYG{n}{style\PYGZus{}border}\PYG{p}{;}
\PYG{n}{lv\PYGZus{}style\PYGZus{}copy}\PYG{p}{(}\PYG{o}{\PYGZam{}}\PYG{n}{style\PYGZus{}border}\PYG{p}{,} \PYG{o}{\PYGZam{}}\PYG{n}{lv\PYGZus{}style\PYGZus{}pretty\PYGZus{}color}\PYG{p}{)}\PYG{p}{;}
\PYG{n}{style\PYGZus{}border}\PYG{p}{.}\PYG{n}{glass} \PYG{o}{=} \PYG{l+m+mi}{1}\PYG{p}{;}
\PYG{n}{style\PYGZus{}border}\PYG{p}{.}\PYG{n}{body}\PYG{p}{.}\PYG{n}{empty} \PYG{o}{=} \PYG{l+m+mi}{1}\PYG{p}{;}

\PYG{c+cm}{/*Create a container*/}
\PYG{n}{lv\PYGZus{}obj\PYGZus{}t} \PYG{o}{*} \PYG{n}{cont}\PYG{p}{;}
\PYG{n}{cont} \PYG{o}{=} \PYG{n}{lv\PYGZus{}cont\PYGZus{}create}\PYG{p}{(}\PYG{n}{lv\PYGZus{}scr\PYGZus{}act}\PYG{p}{(}\PYG{p}{)}\PYG{p}{,} \PYG{n+nb}{NULL}\PYG{p}{)}\PYG{p}{;}
\PYG{n}{lv\PYGZus{}cont\PYGZus{}set\PYGZus{}layout}\PYG{p}{(}\PYG{n}{cont}\PYG{p}{,} \PYG{n}{LV\PYGZus{}LAYOUT\PYGZus{}COL\PYGZus{}L}\PYG{p}{)}\PYG{p}{;}      \PYG{c+cm}{/*Arrange the children in a column*/}
\PYG{n}{lv\PYGZus{}cont\PYGZus{}set\PYGZus{}fit}\PYG{p}{(}\PYG{n}{cont}\PYG{p}{,} \PYG{n+nb}{true}\PYG{p}{,} \PYG{n+nb}{true}\PYG{p}{)}\PYG{p}{;}              \PYG{c+cm}{/*Fit the size to the content*/}
\PYG{n}{lv\PYGZus{}obj\PYGZus{}set\PYGZus{}style}\PYG{p}{(}\PYG{n}{cont}\PYG{p}{,} \PYG{o}{\PYGZam{}}\PYG{n}{style\PYGZus{}border}\PYG{p}{)}\PYG{p}{;}

\PYG{c+cm}{/**************************}
\PYG{c+cm}{ * Create check boxes}
\PYG{c+cm}{ *************************/}

\PYG{c+cm}{/*Create check box*/}
\PYG{n}{lv\PYGZus{}obj\PYGZus{}t} \PYG{o}{*} \PYG{n}{cb}\PYG{p}{;}
\PYG{n}{cb} \PYG{o}{=} \PYG{n}{lv\PYGZus{}cb\PYGZus{}create}\PYG{p}{(}\PYG{n}{cont}\PYG{p}{,} \PYG{n+nb}{NULL}\PYG{p}{)}\PYG{p}{;}
\PYG{n}{lv\PYGZus{}cb\PYGZus{}set\PYGZus{}text}\PYG{p}{(}\PYG{n}{cb}\PYG{p}{,} \PYG{l+s}{\PYGZdq{}}\PYG{l+s}{Potato}\PYG{l+s}{\PYGZdq{}}\PYG{p}{)}\PYG{p}{;}
\PYG{n}{lv\PYGZus{}cb\PYGZus{}set\PYGZus{}action}\PYG{p}{(}\PYG{n}{cb}\PYG{p}{,} \PYG{n}{cb\PYGZus{}release\PYGZus{}action}\PYG{p}{)}\PYG{p}{;}

\PYG{c+cm}{/*Copy the previous check box*/}
\PYG{n}{cb} \PYG{o}{=} \PYG{n}{lv\PYGZus{}cb\PYGZus{}create}\PYG{p}{(}\PYG{n}{cont}\PYG{p}{,} \PYG{n}{cb}\PYG{p}{)}\PYG{p}{;}
\PYG{n}{lv\PYGZus{}cb\PYGZus{}set\PYGZus{}text}\PYG{p}{(}\PYG{n}{cb}\PYG{p}{,} \PYG{l+s}{\PYGZdq{}}\PYG{l+s}{Onion}\PYG{l+s}{\PYGZdq{}}\PYG{p}{)}\PYG{p}{;}

\PYG{c+cm}{/*Copy the previous check box*/}
\PYG{n}{cb} \PYG{o}{=} \PYG{n}{lv\PYGZus{}cb\PYGZus{}create}\PYG{p}{(}\PYG{n}{cont}\PYG{p}{,} \PYG{n}{cb}\PYG{p}{)}\PYG{p}{;}
\PYG{n}{lv\PYGZus{}cb\PYGZus{}set\PYGZus{}text}\PYG{p}{(}\PYG{n}{cb}\PYG{p}{,} \PYG{l+s}{\PYGZdq{}}\PYG{l+s}{Carrot}\PYG{l+s}{\PYGZdq{}}\PYG{p}{)}\PYG{p}{;}

\PYG{c+cm}{/*Copy the previous check box*/}
\PYG{n}{cb} \PYG{o}{=} \PYG{n}{lv\PYGZus{}cb\PYGZus{}create}\PYG{p}{(}\PYG{n}{cont}\PYG{p}{,} \PYG{n}{cb}\PYG{p}{)}\PYG{p}{;}
\PYG{n}{lv\PYGZus{}cb\PYGZus{}set\PYGZus{}text}\PYG{p}{(}\PYG{n}{cb}\PYG{p}{,} \PYG{l+s}{\PYGZdq{}}\PYG{l+s}{Salad}\PYG{l+s}{\PYGZdq{}}\PYG{p}{)}\PYG{p}{;}

\PYG{c+cm}{/*Align the container to the middle*/}
\PYG{n}{lv\PYGZus{}obj\PYGZus{}align}\PYG{p}{(}\PYG{n}{cont}\PYG{p}{,} \PYG{n+nb}{NULL}\PYG{p}{,} \PYG{n}{LV\PYGZus{}ALIGN\PYGZus{}CENTER}\PYG{p}{,} \PYG{l+m+mi}{0}\PYG{p}{,} \PYG{l+m+mi}{0}\PYG{p}{)}\PYG{p}{;}
\end{sphinxVerbatim}


\paragraph{MicroPython}
\label{\detokenize{object-types/cb:micropython}}
No examples yet.


\subsubsection{API}
\label{\detokenize{object-types/cb:api}}

\subsection{Chart (lv\_chart)}
\label{\detokenize{object-types/chart:chart-lv-chart}}\label{\detokenize{object-types/chart::doc}}

\subsubsection{Overview}
\label{\detokenize{object-types/chart:overview}}
Charts have a rectangle-like background with horizontal and vertical division lines. You can add any number of \sphinxstylestrong{series} to the charts by \sphinxcode{\sphinxupquote{lv\_chart\_add\_series(chart, color)}}.
It allocates data for a \sphinxcode{\sphinxupquote{lv\_chart\_series\_t}} structure which contains the chosen \sphinxcode{\sphinxupquote{color}} and an array for the data points.

You have several options to set the data of series:
\begin{enumerate}
\def\theenumi{\arabic{enumi}}
\def\labelenumi{\theenumi .}
\makeatletter\def\p@enumii{\p@enumi \theenumi .}\makeatother
\item {} 
Set the values manually in the array like \sphinxcode{\sphinxupquote{ser1-\textgreater{}points{[}3{]} = 7}} and refresh the chart with \sphinxcode{\sphinxupquote{lv\_chart\_refresh(chart)}}.

\item {} 
Use the \sphinxcode{\sphinxupquote{lv\_chart\_set\_next(chart, ser, value)}}

\item {} 
Initialize all points to a given value with: \sphinxcode{\sphinxupquote{lv\_chart\_init\_points(chart, ser, value)}}.

\item {} 
Set all points from an array with: \sphinxcode{\sphinxupquote{lv\_chart\_set\_points(chart, ser, value\_array)}}.

\end{enumerate}

\sphinxcode{\sphinxupquote{lv\_chart\_set\_next}} can behave in two way depeding on \sphinxstyleemphasis{update mode}:
\begin{itemize}
\item {} 
\sphinxstylestrong{LV\_CHART\_UPDATE\_MODE\_SHIFT} Shift old data to the left and add the new one o the right

\item {} 
\sphinxstylestrong{LV\_CHART\_UPDATE\_MODE\_CIRCULAR} Add the new data in a cirdclar way. (Like an ECG diagram)

\end{itemize}

To update mode can be changed with \sphinxcode{\sphinxupquote{lv\_chart\_set\_update\_mode(chart, LV\_CHART\_UPDATE\_MODE\_...)}}.

The following \sphinxstylestrong{data display types} exists:
\begin{itemize}
\item {} 
\sphinxstylestrong{LV\_CHART\_TYPE\_NONE} do not display any data. It can be used to hide a serie.

\item {} 
\sphinxstylestrong{LV\_CHART\_TYPE\_LINE} draw lines between the points

\item {} 
\sphinxstylestrong{LV\_CHART\_TYPE\_COL} Draw columns

\item {} 
\sphinxstylestrong{LV\_CHART\_TYPE\_POINT} Draw points

\item {} 
\sphinxstylestrong{LV\_CHART\_TYPE\_AREA} Draw areas (fill the area below the lines)

\item {} 
\sphinxstylestrong{LV\_CHART\_TYPE\_VERTICAL\_LINE} Draw only vertical lines to connect the points. Useful if the chart width is equal to the numnber of points.

\end{itemize}

You can specify the display type with \sphinxcode{\sphinxupquote{lv\_chart\_set\_type(chart, LV\_CHART\_TYPE\_...)}}. The types can be ORed (like \sphinxcode{\sphinxupquote{LV\_CHART\_TYPE\_LINE \textbar{} LV\_CHART\_TYPE\_POINT}}).

You can specify a the \sphinxstylestrong{min. and max. values in y} directions with \sphinxcode{\sphinxupquote{lv\_chart\_set\_range(chart, y\_min, y\_max)}}. The value of the points will be scaled proportionally. The default range is: 0..100.

The \sphinxstylestrong{number of points} in the data lines can be modified by \sphinxcode{\sphinxupquote{lv\_chart\_set\_point\_count(chart, point\_num)}}. The default value is 10.

The \sphinxstylestrong{number of horizontal and vertical division lines}  can be modified by \sphinxcode{\sphinxupquote{lv\_chart\_set\_div\_line\_count(chart, hdiv\_num, vdiv\_num)}}. The default settings are 3 horizontal and 5 vertical division lines.

To set the \sphinxstylestrong{line width} and \sphinxstylestrong{point radius} of the series use the \sphinxcode{\sphinxupquote{lv\_chart\_set\_series\_width(chart, size)}} function. The default value is: 2.

The *\sphinxstylestrong{opacity of the data lines} can be specified by \sphinxcode{\sphinxupquote{lv\_chart\_set\_series\_opa(chart, opa)}}. The default value is: OPA\_COVER.

You can apply a \sphinxstylestrong{dark color fade} on the bottom of columns and points by \sphinxcode{\sphinxupquote{lv\_chart\_set\_series\_darking(chart, effect)}} function. The default dark level is OPA\_50.

\sphinxstylestrong{Ticks} and \sphinxstylestrong{texts to ticks} can be added with

\begin{sphinxVerbatim}[commandchars=\\\{\}]
\PYG{n}{lv\PYGZus{}chart\PYGZus{}set\PYGZus{}x\PYGZus{}ticks}\PYG{p}{(}\PYG{n}{chart}\PYG{p}{,} \PYG{n}{list\PYGZus{}of\PYGZus{}values}\PYG{p}{,} \PYG{n}{num\PYGZus{}tick\PYGZus{}marks}\PYG{p}{,} 
                     \PYG{n}{major\PYGZus{}tick\PYGZus{}len}\PYG{p}{,} \PYG{n}{minor\PYGZus{}tick\PYGZus{}len}\PYG{p}{,} 
                     \PYG{n}{LV\PYGZus{}CHART\PYGZus{}AXIS\PYGZus{}DRAW\PYGZus{}LAST\PYGZus{}TICK}\PYG{p}{)}\PYG{p}{;}
\end{sphinxVerbatim}

\sphinxcode{\sphinxupquote{list\_of\_values}} is an array with \sphinxcode{\sphinxupquote{num\_tick\_marks}} \sphinxcode{\sphinxupquote{'\textbackslash{}n'}} termianted text (expect the last) with text for the ticks. E.g. \sphinxcode{\sphinxupquote{const char * list\_of\_values = "first\textbackslash{}nseco\textbackslash{}nthird"}}.
\sphinxcode{\sphinxupquote{major\_tick\_len}} and \sphinxcode{\sphinxupquote{min\_tick\_len}} is the length of the tick marks when the tick is on the division line or when it isn respectively. \sphinxcode{\sphinxupquote{LV\_CHART\_AXIS\_DRAW\_LAST\_TICK}} is the only supported mode now.

For y axis you can use \sphinxcode{\sphinxupquote{lv\_chart\_set\_y\_ticks}}.

\sphinxcode{\sphinxupquote{lv\_chart\_set\_margin(chart, 20)}} needs to used to add some extra space arounf the chart for the ticks and texts.


\subsubsection{Styles}
\label{\detokenize{object-types/chart:styles}}
You can set the styles with \sphinxcode{\sphinxupquote{lv\_chart\_set\_style(btn, LV\_CHART\_STYLE\_MAIN, \&style)}}.
\begin{itemize}
\item {} 
\sphinxstylestrong{style.body} properties set the background’s appearance

\item {} 
\sphinxstylestrong{style.line} properties set the division lines’ appearance

\item {} 
\sphinxstylestrong{style.text} properties set the axis labels’ appearance

\end{itemize}


\subsubsection{Events}
\label{\detokenize{object-types/chart:events}}
Only the \sphinxhref{/overview/events.html\#generic-events}{Genreric events} are sent by the object type.

Learn more about {\hyperref[\detokenize{overview/events::doc}]{\sphinxcrossref{\DUrole{doc,doc,doc}{Events}}}}.


\subsubsection{Keys}
\label{\detokenize{object-types/chart:keys}}
No \sphinxstyleemphasis{Keys} are processed by the object type.

Learn more about {\hyperref[\detokenize{overview/indev::doc}]{\sphinxcrossref{\DUrole{doc,doc,doc}{Keys}}}}.


\subsubsection{Example}
\label{\detokenize{object-types/chart:example}}

\paragraph{C}
\label{\detokenize{object-types/chart:c}}
\sphinxincludegraphics{{chart_1}.png}Chart image

code

\begin{sphinxVerbatim}[commandchars=\\\{\}]
\PYG{c+cm}{/*Create a style for the chart*/}
\PYG{k}{static} \PYG{n}{lv\PYGZus{}style\PYGZus{}t} \PYG{n}{style}\PYG{p}{;}
\PYG{n}{lv\PYGZus{}style\PYGZus{}copy}\PYG{p}{(}\PYG{o}{\PYGZam{}}\PYG{n}{style}\PYG{p}{,} \PYG{o}{\PYGZam{}}\PYG{n}{lv\PYGZus{}style\PYGZus{}pretty}\PYG{p}{)}\PYG{p}{;}
\PYG{n}{style}\PYG{p}{.}\PYG{n}{body}\PYG{p}{.}\PYG{n}{shadow}\PYG{p}{.}\PYG{n}{width} \PYG{o}{=} \PYG{l+m+mi}{6}\PYG{p}{;}
\PYG{n}{style}\PYG{p}{.}\PYG{n}{body}\PYG{p}{.}\PYG{n}{shadow}\PYG{p}{.}\PYG{n}{color} \PYG{o}{=} \PYG{n}{LV\PYGZus{}COLOR\PYGZus{}GRAY}\PYG{p}{;}
\PYG{n}{style}\PYG{p}{.}\PYG{n}{line}\PYG{p}{.}\PYG{n}{color} \PYG{o}{=} \PYG{n}{LV\PYGZus{}COLOR\PYGZus{}GRAY}\PYG{p}{;}

\PYG{c+cm}{/*Create a chart*/}
\PYG{n}{lv\PYGZus{}obj\PYGZus{}t} \PYG{o}{*} \PYG{n}{chart}\PYG{p}{;}
\PYG{n}{chart} \PYG{o}{=} \PYG{n}{lv\PYGZus{}chart\PYGZus{}create}\PYG{p}{(}\PYG{n}{lv\PYGZus{}scr\PYGZus{}act}\PYG{p}{(}\PYG{p}{)}\PYG{p}{,} \PYG{n+nb}{NULL}\PYG{p}{)}\PYG{p}{;}
\PYG{n}{lv\PYGZus{}obj\PYGZus{}set\PYGZus{}size}\PYG{p}{(}\PYG{n}{chart}\PYG{p}{,} \PYG{l+m+mi}{200}\PYG{p}{,} \PYG{l+m+mi}{150}\PYG{p}{)}\PYG{p}{;}
\PYG{n}{lv\PYGZus{}obj\PYGZus{}set\PYGZus{}style}\PYG{p}{(}\PYG{n}{chart}\PYG{p}{,} \PYG{o}{\PYGZam{}}\PYG{n}{style}\PYG{p}{)}\PYG{p}{;}
\PYG{n}{lv\PYGZus{}obj\PYGZus{}align}\PYG{p}{(}\PYG{n}{chart}\PYG{p}{,} \PYG{n+nb}{NULL}\PYG{p}{,} \PYG{n}{LV\PYGZus{}ALIGN\PYGZus{}CENTER}\PYG{p}{,} \PYG{l+m+mi}{0}\PYG{p}{,} \PYG{l+m+mi}{0}\PYG{p}{)}\PYG{p}{;}
\PYG{n}{lv\PYGZus{}chart\PYGZus{}set\PYGZus{}type}\PYG{p}{(}\PYG{n}{chart}\PYG{p}{,} \PYG{n}{LV\PYGZus{}CHART\PYGZus{}TYPE\PYGZus{}POINT} \PYG{o}{\textbar{}} \PYG{n}{LV\PYGZus{}CHART\PYGZus{}TYPE\PYGZus{}LINE}\PYG{p}{)}\PYG{p}{;}   \PYG{c+cm}{/*Show lines and points too*/}
\PYG{n}{lv\PYGZus{}chart\PYGZus{}set\PYGZus{}series\PYGZus{}opa}\PYG{p}{(}\PYG{n}{chart}\PYG{p}{,} \PYG{n}{LV\PYGZus{}OPA\PYGZus{}70}\PYG{p}{)}\PYG{p}{;}                            \PYG{c+cm}{/*Opacity of the data series*/}
\PYG{n}{lv\PYGZus{}chart\PYGZus{}set\PYGZus{}series\PYGZus{}width}\PYG{p}{(}\PYG{n}{chart}\PYG{p}{,} \PYG{l+m+mi}{4}\PYG{p}{)}\PYG{p}{;}                                  \PYG{c+cm}{/*Line width and point radious*/}

\PYG{n}{lv\PYGZus{}chart\PYGZus{}set\PYGZus{}range}\PYG{p}{(}\PYG{n}{chart}\PYG{p}{,} \PYG{l+m+mi}{0}\PYG{p}{,} \PYG{l+m+mi}{100}\PYG{p}{)}\PYG{p}{;}

\PYG{c+cm}{/*Add two data series*/}
\PYG{n}{lv\PYGZus{}chart\PYGZus{}series\PYGZus{}t} \PYG{o}{*} \PYG{n}{ser1} \PYG{o}{=} \PYG{n}{lv\PYGZus{}chart\PYGZus{}add\PYGZus{}series}\PYG{p}{(}\PYG{n}{chart}\PYG{p}{,} \PYG{n}{LV\PYGZus{}COLOR\PYGZus{}RED}\PYG{p}{)}\PYG{p}{;}
\PYG{n}{lv\PYGZus{}chart\PYGZus{}series\PYGZus{}t} \PYG{o}{*} \PYG{n}{ser2} \PYG{o}{=} \PYG{n}{lv\PYGZus{}chart\PYGZus{}add\PYGZus{}series}\PYG{p}{(}\PYG{n}{chart}\PYG{p}{,} \PYG{n}{LV\PYGZus{}COLOR\PYGZus{}GREEN}\PYG{p}{)}\PYG{p}{;}

\PYG{c+cm}{/*Set the next points on \PYGZsq{}dl1\PYGZsq{}*/}
\PYG{n}{lv\PYGZus{}chart\PYGZus{}set\PYGZus{}next}\PYG{p}{(}\PYG{n}{chart}\PYG{p}{,} \PYG{n}{ser1}\PYG{p}{,} \PYG{l+m+mi}{10}\PYG{p}{)}\PYG{p}{;}
\PYG{n}{lv\PYGZus{}chart\PYGZus{}set\PYGZus{}next}\PYG{p}{(}\PYG{n}{chart}\PYG{p}{,} \PYG{n}{ser1}\PYG{p}{,} \PYG{l+m+mi}{50}\PYG{p}{)}\PYG{p}{;}
\PYG{n}{lv\PYGZus{}chart\PYGZus{}set\PYGZus{}next}\PYG{p}{(}\PYG{n}{chart}\PYG{p}{,} \PYG{n}{ser1}\PYG{p}{,} \PYG{l+m+mi}{70}\PYG{p}{)}\PYG{p}{;}
\PYG{n}{lv\PYGZus{}chart\PYGZus{}set\PYGZus{}next}\PYG{p}{(}\PYG{n}{chart}\PYG{p}{,} \PYG{n}{ser1}\PYG{p}{,} \PYG{l+m+mi}{90}\PYG{p}{)}\PYG{p}{;}

\PYG{c+cm}{/*Directly set points on \PYGZsq{}dl2\PYGZsq{}*/}
\PYG{n}{ser2}\PYG{o}{\PYGZhy{}}\PYG{o}{\PYGZgt{}}\PYG{n}{points}\PYG{p}{[}\PYG{l+m+mi}{0}\PYG{p}{]} \PYG{o}{=} \PYG{l+m+mi}{90}\PYG{p}{;}
\PYG{n}{ser2}\PYG{o}{\PYGZhy{}}\PYG{o}{\PYGZgt{}}\PYG{n}{points}\PYG{p}{[}\PYG{l+m+mi}{1}\PYG{p}{]} \PYG{o}{=} \PYG{l+m+mi}{70}\PYG{p}{;}
\PYG{n}{ser2}\PYG{o}{\PYGZhy{}}\PYG{o}{\PYGZgt{}}\PYG{n}{points}\PYG{p}{[}\PYG{l+m+mi}{2}\PYG{p}{]} \PYG{o}{=} \PYG{l+m+mi}{65}\PYG{p}{;}
\PYG{n}{ser2}\PYG{o}{\PYGZhy{}}\PYG{o}{\PYGZgt{}}\PYG{n}{points}\PYG{p}{[}\PYG{l+m+mi}{3}\PYG{p}{]} \PYG{o}{=} \PYG{l+m+mi}{65}\PYG{p}{;}
\PYG{n}{ser2}\PYG{o}{\PYGZhy{}}\PYG{o}{\PYGZgt{}}\PYG{n}{points}\PYG{p}{[}\PYG{l+m+mi}{4}\PYG{p}{]} \PYG{o}{=} \PYG{l+m+mi}{65}\PYG{p}{;}
\PYG{n}{ser2}\PYG{o}{\PYGZhy{}}\PYG{o}{\PYGZgt{}}\PYG{n}{points}\PYG{p}{[}\PYG{l+m+mi}{5}\PYG{p}{]} \PYG{o}{=} \PYG{l+m+mi}{65}\PYG{p}{;}

\PYG{n}{lv\PYGZus{}chart\PYGZus{}refresh}\PYG{p}{(}\PYG{n}{chart}\PYG{p}{)}\PYG{p}{;} \PYG{c+cm}{/*Required after direct set*/}
\end{sphinxVerbatim}


\paragraph{MicroPython}
\label{\detokenize{object-types/chart:micropython}}
No examples yet.


\subsubsection{API}
\label{\detokenize{object-types/chart:api}}

\subsection{Container}
\label{\detokenize{object-types/cont:container}}\label{\detokenize{object-types/cont::doc}}

\subsubsection{Overview}
\label{\detokenize{object-types/cont:overview}}
The containers are \sphinxstylestrong{rectangle-like object} with some special features.

You can apply a \sphinxstylestrong{layout} on the containers to automatically order their children. The layout spacing comes from \sphinxcode{\sphinxupquote{style.body.padding. ...}} properties. The possible layout options:
\begin{itemize}
\item {} 
\sphinxstylestrong{LV\_LAYOUT\_OFF} Do not align the children

\item {} 
\sphinxstylestrong{LV\_LAYOUT\_CENTER} Align children to the center in column and keep \sphinxcode{\sphinxupquote{padding.inner}} space between them

\item {} 
\sphinxstylestrong{LV\_LAYOUT\_COL\_}: Align children in a left justified column. Keep \sphinxcode{\sphinxupquote{padding.left}} space on the left, \sphinxcode{\sphinxupquote{pad.top}} space on the top and \sphinxcode{\sphinxupquote{padding.inner}} space between the children.

\item {} 
\sphinxstylestrong{LV\_LAYOUT\_COL\_M} Align children in centered column. Keep \sphinxcode{\sphinxupquote{padding.top}} space on the top and \sphinxcode{\sphinxupquote{padding.inner}} space between the children.

\item {} 
\sphinxstylestrong{LV\_LAYOUT\_COL\_R} Align children in a right justified column. Keep \sphinxcode{\sphinxupquote{padding.right}} space on the right, \sphinxcode{\sphinxupquote{padding.top}} space on the top and \sphinxcode{\sphinxupquote{padding.inner}} space between the children.

\item {} 
\sphinxstylestrong{LV\_LAYOUT\_ROW\_T} Align children in a top justified row. Keep \sphinxcode{\sphinxupquote{padding.left}} space on the left, \sphinxcode{\sphinxupquote{padding.top}} space on the top and \sphinxcode{\sphinxupquote{padding.inner}} space between the children.

\item {} 
\sphinxstylestrong{LV\_LAYOUT\_ROW\_M} Align children in centered row. Keep \sphinxcode{\sphinxupquote{padding.left}} space on the left and \sphinxcode{\sphinxupquote{padding.inner}} space between the children.

\item {} 
\sphinxstylestrong{LV\_LAYOUT\_ROW\_B} Align children in a bottom justified row. Keep \sphinxcode{\sphinxupquote{padding.left}} space on the left, \sphinxcode{\sphinxupquote{padding.bottom}} space on the bottom and \sphinxcode{\sphinxupquote{padding.inner}} space between the children.

\item {} 
\sphinxstylestrong{LV\_LAYOUT\_PRETTY} Put as may objects as possible in a row (with at least \sphinxcode{\sphinxupquote{padding.inner}} space and \sphinxcode{\sphinxupquote{padding.left/right}} space on the sides). Divide the space in each line equally between the children.
Keep \sphinxcode{\sphinxupquote{padding.top}} space on the top and \sphinxcode{\sphinxupquote{pad.inner}} space between the lines.

\item {} 
\sphinxstylestrong{LV\_LAYOUT\_GRID} Similar to \sphinxcode{\sphinxupquote{LV\_LAYOUT\_PRETTY}} but not divide horizontal space equally just let \sphinxcode{\sphinxupquote{padding.left/right}} on the edges and \sphinxcode{\sphinxupquote{padding.inner}} space betweenthe elemnts.

\end{itemize}

Container have an \sphinxstylestrong{auto fit} features whcih can automaticall change the size of the Container accoring to its children and/or parent. The following optionas are exist:
\begin{itemize}
\item {} 
\sphinxstylestrong{LV\_FIT\_NONE} Do not change the size automatically

\item {} 
\sphinxstylestrong{LV\_FIT\_TIGHT} Set the size to involve all children by keeping \sphinxcode{\sphinxupquote{padding.top/bottom/left/right}} space on the edges.

\item {} 
\sphinxstylestrong{LV\_FIT\_FLOOD} Set the size to the parents size by keeping \sphinxcode{\sphinxupquote{padding.top/bottom/left/right}} (from the parent’s style) space.

\item {} 
\sphinxstylestrong{LV\_FINT\_FILL} Use \sphinxcode{\sphinxupquote{LV\_FIT\_FLOOD}} while smaller then the parent and \sphinxcode{\sphinxupquote{LV\_FIT\_TIGHT}} when larger.

\end{itemize}

To set the auto fit use \sphinxcode{\sphinxupquote{lv\_cont\_set\_fit(cont, LV\_FIT\_...)}}. It will set the same auto fit in every directions.
To use different auto fit horizontally and vertically use \sphinxcode{\sphinxupquote{lv\_cont\_set\_fit2(cont, hor\_fit\_type, ver\_fit\_type)}}.
To use different auto fit in all 4 directions use \sphinxcode{\sphinxupquote{lv\_cont\_set\_fit4(cont, left\_fit\_type, right\_fit\_type, top\_fit\_type, bottom\_fit\_type)}}.


\subsubsection{Styles}
\label{\detokenize{object-types/cont:styles}}
You can set the styles with \sphinxcode{\sphinxupquote{lv\_cont\_set\_style(btn, LV\_CONT\_STYLE\_MAIN, \&style)}}.
\begin{itemize}
\item {} 
\sphinxstylestrong{style.body} properties are used.

\end{itemize}


\subsubsection{Events}
\label{\detokenize{object-types/cont:events}}
Only the \sphinxhref{/overview/events.html\#generic-events}{Genreric events} are sent by the object type.

Learn more about {\hyperref[\detokenize{overview/events::doc}]{\sphinxcrossref{\DUrole{doc,doc,doc}{Events}}}}.


\subsubsection{Keys}
\label{\detokenize{object-types/cont:keys}}
No \sphinxstyleemphasis{Keys} are processed by the object type.

Learn more about {\hyperref[\detokenize{overview/indev::doc}]{\sphinxcrossref{\DUrole{doc,doc,doc}{Keys}}}}.


\subsubsection{Keys}
\label{\detokenize{object-types/cont:id1}}
The following \sphinxstyleemphasis{Keys} are processed by the Buttons:
\begin{itemize}
\item {} 
\sphinxstylestrong{LV\_KEY\_RIGHT/UP} Go to toggled state if toggling is enabled

\item {} 
\sphinxstylestrong{LV\_KEY\_LEFT/DOWN} Go to non-toggled state if toggling is  enabled

\end{itemize}

Note that, as usual, the state of \sphinxcode{\sphinxupquote{LV\_KEY\_ENTER}} is translated to \sphinxcode{\sphinxupquote{LV\_EVENT\_PRESSED/PRESSING/RELEASED}} etc.


\subsubsection{Example}
\label{\detokenize{object-types/cont:example}}

\paragraph{C}
\label{\detokenize{object-types/cont:c}}
\sphinxincludegraphics{{cont_1}.png}Container image

code

\begin{sphinxVerbatim}[commandchars=\\\{\}]
\PYG{n}{lv\PYGZus{}obj\PYGZus{}t} \PYG{o}{*} \PYG{n}{box1}\PYG{p}{;}
\PYG{n}{box1} \PYG{o}{=} \PYG{n}{lv\PYGZus{}cont\PYGZus{}create}\PYG{p}{(}\PYG{n}{lv\PYGZus{}scr\PYGZus{}act}\PYG{p}{(}\PYG{p}{)}\PYG{p}{,} \PYG{n+nb}{NULL}\PYG{p}{)}\PYG{p}{;}
\PYG{n}{lv\PYGZus{}obj\PYGZus{}set\PYGZus{}style}\PYG{p}{(}\PYG{n}{box1}\PYG{p}{,} \PYG{o}{\PYGZam{}}\PYG{n}{lv\PYGZus{}style\PYGZus{}pretty}\PYG{p}{)}\PYG{p}{;}
\PYG{n}{lv\PYGZus{}cont\PYGZus{}set\PYGZus{}fit}\PYG{p}{(}\PYG{n}{box1}\PYG{p}{,} \PYG{n+nb}{true}\PYG{p}{,} \PYG{n+nb}{true}\PYG{p}{)}\PYG{p}{;}

\PYG{c+cm}{/*Add a text to the container*/}
\PYG{n}{lv\PYGZus{}obj\PYGZus{}t} \PYG{o}{*} \PYG{n}{txt} \PYG{o}{=} \PYG{n}{lv\PYGZus{}label\PYGZus{}create}\PYG{p}{(}\PYG{n}{box1}\PYG{p}{,} \PYG{n+nb}{NULL}\PYG{p}{)}\PYG{p}{;}
\PYG{n}{lv\PYGZus{}label\PYGZus{}set\PYGZus{}text}\PYG{p}{(}\PYG{n}{txt}\PYG{p}{,} \PYG{l+s}{\PYGZdq{}}\PYG{l+s}{Lorem ipsum dolor}\PYG{l+s+se}{\PYGZbs{}n}\PYG{l+s}{\PYGZdq{}}
                       \PYG{l+s}{\PYGZdq{}}\PYG{l+s}{sit amet, consectetur}\PYG{l+s+se}{\PYGZbs{}n}\PYG{l+s}{\PYGZdq{}}
                       \PYG{l+s}{\PYGZdq{}}\PYG{l+s}{adipiscing elit, sed do}\PYG{l+s+se}{\PYGZbs{}n}\PYG{l+s}{\PYGZdq{}}
                       \PYG{l+s}{\PYGZdq{}}\PYG{l+s}{eiusmod tempor incididunt}\PYG{l+s+se}{\PYGZbs{}n}\PYG{l+s}{\PYGZdq{}}
                       \PYG{l+s}{\PYGZdq{}}\PYG{l+s}{ut labore et dolore}\PYG{l+s+se}{\PYGZbs{}n}\PYG{l+s}{\PYGZdq{}}
                       \PYG{l+s}{\PYGZdq{}}\PYG{l+s}{magna aliqua.}\PYG{l+s}{\PYGZdq{}}\PYG{p}{)}\PYG{p}{;}

\PYG{n}{lv\PYGZus{}obj\PYGZus{}align}\PYG{p}{(}\PYG{n}{box1}\PYG{p}{,} \PYG{n+nb}{NULL}\PYG{p}{,} \PYG{n}{LV\PYGZus{}ALIGN\PYGZus{}IN\PYGZus{}TOP\PYGZus{}LEFT}\PYG{p}{,} \PYG{l+m+mi}{10}\PYG{p}{,} \PYG{l+m+mi}{10}\PYG{p}{)}\PYG{p}{;}      \PYG{c+cm}{/*Align the container*/}

\PYG{c+cm}{/*Create a style*/}
\PYG{k}{static} \PYG{n}{lv\PYGZus{}style\PYGZus{}t} \PYG{n}{style}\PYG{p}{;}
\PYG{n}{lv\PYGZus{}style\PYGZus{}copy}\PYG{p}{(}\PYG{o}{\PYGZam{}}\PYG{n}{style}\PYG{p}{,} \PYG{o}{\PYGZam{}}\PYG{n}{lv\PYGZus{}style\PYGZus{}pretty\PYGZus{}color}\PYG{p}{)}\PYG{p}{;}
\PYG{n}{style}\PYG{p}{.}\PYG{n}{body}\PYG{p}{.}\PYG{n}{shadow}\PYG{p}{.}\PYG{n}{width} \PYG{o}{=} \PYG{l+m+mi}{6}\PYG{p}{;}
\PYG{n}{style}\PYG{p}{.}\PYG{n}{body}\PYG{p}{.}\PYG{n}{padding}\PYG{p}{.}\PYG{n}{hor} \PYG{o}{=} \PYG{l+m+mi}{5}\PYG{p}{;}                                 \PYG{c+cm}{/*Set a great horizontal padding*/}

\PYG{c+cm}{/*Create an other container*/}
\PYG{n}{lv\PYGZus{}obj\PYGZus{}t} \PYG{o}{*} \PYG{n}{box2}\PYG{p}{;}
\PYG{n}{box2} \PYG{o}{=} \PYG{n}{lv\PYGZus{}cont\PYGZus{}create}\PYG{p}{(}\PYG{n}{lv\PYGZus{}scr\PYGZus{}act}\PYG{p}{(}\PYG{p}{)}\PYG{p}{,} \PYG{n+nb}{NULL}\PYG{p}{)}\PYG{p}{;}
\PYG{n}{lv\PYGZus{}obj\PYGZus{}set\PYGZus{}style}\PYG{p}{(}\PYG{n}{box2}\PYG{p}{,} \PYG{o}{\PYGZam{}}\PYG{n}{style}\PYG{p}{)}\PYG{p}{;}     \PYG{c+cm}{/*Set the new style*/}
\PYG{n}{lv\PYGZus{}cont\PYGZus{}set\PYGZus{}fit}\PYG{p}{(}\PYG{n}{box2}\PYG{p}{,} \PYG{n+nb}{true}\PYG{p}{,} \PYG{n+nb}{false}\PYG{p}{)}\PYG{p}{;} \PYG{c+cm}{/*Do not enable the vertical fit */}
\PYG{n}{lv\PYGZus{}obj\PYGZus{}set\PYGZus{}height}\PYG{p}{(}\PYG{n}{box2}\PYG{p}{,} \PYG{l+m+mi}{55}\PYG{p}{)}\PYG{p}{;}        \PYG{c+cm}{/*Set a fix height*/}

\PYG{c+cm}{/*Add a text to the new container*/}
\PYG{n}{lv\PYGZus{}obj\PYGZus{}t} \PYG{o}{*} \PYG{n}{txt2} \PYG{o}{=} \PYG{n}{lv\PYGZus{}label\PYGZus{}create}\PYG{p}{(}\PYG{n}{box2}\PYG{p}{,} \PYG{n+nb}{NULL}\PYG{p}{)}\PYG{p}{;}
\PYG{n}{lv\PYGZus{}label\PYGZus{}set\PYGZus{}text}\PYG{p}{(}\PYG{n}{txt2}\PYG{p}{,} \PYG{l+s}{\PYGZdq{}}\PYG{l+s}{No vertical fit 1...}\PYG{l+s+se}{\PYGZbs{}n}\PYG{l+s}{\PYGZdq{}}
                        \PYG{l+s}{\PYGZdq{}}\PYG{l+s}{No vertical fit 2...}\PYG{l+s+se}{\PYGZbs{}n}\PYG{l+s}{\PYGZdq{}}
                        \PYG{l+s}{\PYGZdq{}}\PYG{l+s}{No vertical fit 3...}\PYG{l+s+se}{\PYGZbs{}n}\PYG{l+s}{\PYGZdq{}}
                        \PYG{l+s}{\PYGZdq{}}\PYG{l+s}{No vertical fit 4...}\PYG{l+s}{\PYGZdq{}}\PYG{p}{)}\PYG{p}{;}

\PYG{c+cm}{/*Align the container to the bottom of the previous*/}
\PYG{n}{lv\PYGZus{}obj\PYGZus{}align}\PYG{p}{(}\PYG{n}{box2}\PYG{p}{,} \PYG{n}{box1}\PYG{p}{,} \PYG{n}{LV\PYGZus{}ALIGN\PYGZus{}OUT\PYGZus{}BOTTOM\PYGZus{}MID}\PYG{p}{,} \PYG{l+m+mi}{30}\PYG{p}{,} \PYG{o}{\PYGZhy{}}\PYG{l+m+mi}{30}\PYG{p}{)}\PYG{p}{;}
\end{sphinxVerbatim}


\paragraph{MicroPython}
\label{\detokenize{object-types/cont:micropython}}
No examples yet.


\subsubsection{API}
\label{\detokenize{object-types/cont:api}}

\subsection{Drop down list (lv\_ddlist)}
\label{\detokenize{object-types/ddlist:drop-down-list-lv-ddlist}}\label{\detokenize{object-types/ddlist::doc}}

\subsubsection{Overview}
\label{\detokenize{object-types/ddlist:overview}}
Drop Down Lists allow you to simply \sphinxstylestrong{select one option from more}. The Drop Down List is closed by default an show the currently selected text. If you click on it the this list opens and all the options are shown.

The \sphinxstylestrong{options} are passed to the Drop Down List as a \sphinxstylestrong{string} with \sphinxcode{\sphinxupquote{lv\_ddlist\_set\_options(ddlist, options)}}. The options should be separated by \sphinxcode{\sphinxupquote{\textbackslash{}n}}. For example: \sphinxcode{\sphinxupquote{"First\textbackslash{}nSecond\textbackslash{}nThird"}}.

You can \sphinxstylestrong{select an option manually} with \sphinxcode{\sphinxupquote{lv\_ddlist\_set\_selected(ddlist, id)}}, where \sphinxstyleemphasis{id} is the index of an option.

By default the list’s \sphinxstylestrong{height} is adjusted automatically to show all options. The \sphinxcode{\sphinxupquote{lv\_ddlist\_set\_fix\_height(ddlist, height)}} sets a fixed height for the opened list. \sphinxcode{\sphinxupquote{0}} means to use auto height.

The \sphinxstylestrong{width} is also adjusted automatically. To prevent this apply \sphinxcode{\sphinxupquote{lv\_ddlist\_set\_fix\_width(ddlist, width)}}. \sphinxcode{\sphinxupquote{0}} means to use auto width.

Similarly to {\hyperref[\detokenize{object-types/page::doc}]{\sphinxcrossref{\DUrole{doc,doc,doc}{Page}}}} with fix height the Drop Down List supports various \sphinxstylestrong{scrollbar display modes}. It can be set by \sphinxcode{\sphinxupquote{lv\_ddlist\_set\_sb\_mode(ddlist, LV\_SB\_MODE\_...)}}.

The Drop Down List open/close animation time is adjusted by \sphinxcode{\sphinxupquote{lv\_ddlist\_set\_anim\_time(ddlist, anim\_time)}}. Zero animation time means no animation.

A \sphinxstylestrong{down arrow} can be added to the left sid of the drop down list with \sphinxcode{\sphinxupquote{lv\_ddlist\_set\_draw\_arrow(ddlist, true)}}.

To align the label horizontally use \sphinxcode{\sphinxupquote{lv\_ddlist\_set\_align(ddlist, LV\_LABEL\_ALIGN\_LEFT/CENTER/RIGHT)}}.

You can force the Drop down list to \sphinxstylestrong{stay opened} when an option is selected with \sphinxcode{\sphinxupquote{lv\_ddlist\_set\_stay\_open(ddlist, true)}}.


\subsubsection{Styles}
\label{\detokenize{object-types/ddlist:styles}}
The \sphinxcode{\sphinxupquote{lv\_ddlist\_set\_style(ddlist, LV\_DDLIST\_STYLE\_..., \&style)}} set the styles of a Drop Down List.
\begin{itemize}
\item {} 
\sphinxstylestrong{LV\_DDLIST\_STYLE\_BG} Style of the background. All \sphinxstyleemphasis{style.body} properties are used. It is used for the label’s style from \sphinxstyleemphasis{style.text}. Default: \sphinxstyleemphasis{lv\_style\_pretty}

\item {} 
\sphinxstylestrong{LV\_DDLIST\_STYLE\_SEL} Style of the selected option.  The \sphinxstyleemphasis{style.body} properties are used. The selected option will be recolored with \sphinxstyleemphasis{text.color}. Default: \sphinxstyleemphasis{lv\_style\_plain\_color}

\item {} 
\sphinxstylestrong{LV\_DDLIST\_STYLE\_SB} Style of the scrollbar. The \sphinxstyleemphasis{style.body} properties are used. Default: \sphinxstyleemphasis{lv\_style\_plain\_color}

\end{itemize}


\subsubsection{Events}
\label{\detokenize{object-types/ddlist:events}}
Besided the \sphinxhref{/overview/events.html\#generic-events}{Genreric events} the following \sphinxhref{/overview/events.html\#special-events}{Special events} are sent by the Drop down lists:
\begin{itemize}
\item {} 
\sphinxstylestrong{LV\_EVENT\_VALUE\_CHANGED} sent when the a new option is selected

\end{itemize}

Learn more about {\hyperref[\detokenize{overview/events::doc}]{\sphinxcrossref{\DUrole{doc,doc,doc}{Events}}}}.


\subsubsection{Keys}
\label{\detokenize{object-types/ddlist:keys}}
The following \sphinxstyleemphasis{Keys} are processed by the Buttons:
\begin{itemize}
\item {} 
\sphinxstylestrong{LV\_KEY\_RIGHT/DOWN} Select the next option

\item {} 
\sphinxstylestrong{LV\_KEY\_LEFT/UP} Select the previous option

\item {} 
\sphinxstylestrong{LY\_KEY\_ENTER} Apply the selected option (Send \sphinxcode{\sphinxupquote{LV\_EVENT\_VALUE\_CHANGED}} event and close the Drop down list)

\end{itemize}


\subsubsection{Example}
\label{\detokenize{object-types/ddlist:example}}

\paragraph{C}
\label{\detokenize{object-types/ddlist:c}}
\sphinxincludegraphics{{ddlist_1}.png}Drop down list image

code

\begin{sphinxVerbatim}[commandchars=\\\{\}]
\PYG{k}{static} \PYG{n}{lv\PYGZus{}res\PYGZus{}t} \PYG{n+nf}{ddlist\PYGZus{}action}\PYG{p}{(}\PYG{n}{lv\PYGZus{}obj\PYGZus{}t} \PYG{o}{*} \PYG{n}{ddlist}\PYG{p}{)}
\PYG{p}{\PYGZob{}}
    \PYG{k+kt}{uint8\PYGZus{}t} \PYG{n}{id} \PYG{o}{=} \PYG{n}{lv\PYGZus{}obj\PYGZus{}get\PYGZus{}free\PYGZus{}num}\PYG{p}{(}\PYG{n}{ddlist}\PYG{p}{)}\PYG{p}{;}

    \PYG{k+kt}{char} \PYG{n}{sel\PYGZus{}str}\PYG{p}{[}\PYG{l+m+mi}{32}\PYG{p}{]}\PYG{p}{;}
    \PYG{n}{lv\PYGZus{}ddlist\PYGZus{}get\PYGZus{}selected\PYGZus{}str}\PYG{p}{(}\PYG{n}{ddlist}\PYG{p}{,} \PYG{n}{sel\PYGZus{}str}\PYG{p}{)}\PYG{p}{;}
    \PYG{n}{printf}\PYG{p}{(}\PYG{l+s}{\PYGZdq{}}\PYG{l+s}{Ddlist \PYGZpc{}d new option: \PYGZpc{}s }\PYG{l+s+se}{\PYGZbs{}n}\PYG{l+s}{\PYGZdq{}}\PYG{p}{,} \PYG{n}{id}\PYG{p}{,} \PYG{n}{sel\PYGZus{}str}\PYG{p}{)}\PYG{p}{;}

    \PYG{k}{return} \PYG{n}{LV\PYGZus{}RES\PYGZus{}OK}\PYG{p}{;} \PYG{c+cm}{/*Return OK if the drop down list is not deleted*/}
\PYG{p}{\PYGZcb{}}


\PYG{p}{.}
\PYG{p}{.}
\PYG{p}{.}

\PYG{c+cm}{/*Create a drop down list*/}
\PYG{n}{lv\PYGZus{}obj\PYGZus{}t} \PYG{o}{*} \PYG{n}{ddl1} \PYG{o}{=} \PYG{n}{lv\PYGZus{}ddlist\PYGZus{}create}\PYG{p}{(}\PYG{n}{lv\PYGZus{}scr\PYGZus{}act}\PYG{p}{(}\PYG{p}{)}\PYG{p}{,} \PYG{n+nb}{NULL}\PYG{p}{)}\PYG{p}{;}
\PYG{n}{lv\PYGZus{}ddlist\PYGZus{}set\PYGZus{}options}\PYG{p}{(}\PYG{n}{ddl1}\PYG{p}{,} \PYG{l+s}{\PYGZdq{}}\PYG{l+s}{Apple}\PYG{l+s+se}{\PYGZbs{}n}\PYG{l+s}{\PYGZdq{}}
                            \PYG{l+s}{\PYGZdq{}}\PYG{l+s}{Banana}\PYG{l+s+se}{\PYGZbs{}n}\PYG{l+s}{\PYGZdq{}}
                            \PYG{l+s}{\PYGZdq{}}\PYG{l+s}{Orange}\PYG{l+s+se}{\PYGZbs{}n}\PYG{l+s}{\PYGZdq{}}
                            \PYG{l+s}{\PYGZdq{}}\PYG{l+s}{Melon}\PYG{l+s+se}{\PYGZbs{}n}\PYG{l+s}{\PYGZdq{}}
                            \PYG{l+s}{\PYGZdq{}}\PYG{l+s}{Grape}\PYG{l+s+se}{\PYGZbs{}n}\PYG{l+s}{\PYGZdq{}}
                            \PYG{l+s}{\PYGZdq{}}\PYG{l+s}{Raspberry}\PYG{l+s}{\PYGZdq{}}\PYG{p}{)}\PYG{p}{;}
\PYG{n}{lv\PYGZus{}obj\PYGZus{}align}\PYG{p}{(}\PYG{n}{ddl1}\PYG{p}{,} \PYG{n+nb}{NULL}\PYG{p}{,} \PYG{n}{LV\PYGZus{}ALIGN\PYGZus{}IN\PYGZus{}TOP\PYGZus{}LEFT}\PYG{p}{,} \PYG{l+m+mi}{30}\PYG{p}{,} \PYG{l+m+mi}{10}\PYG{p}{)}\PYG{p}{;}
\PYG{n}{lv\PYGZus{}obj\PYGZus{}set\PYGZus{}free\PYGZus{}num}\PYG{p}{(}\PYG{n}{ddl1}\PYG{p}{,} \PYG{l+m+mi}{1}\PYG{p}{)}\PYG{p}{;}               \PYG{c+cm}{/*Set a unique ID*/}
\PYG{n}{lv\PYGZus{}ddlist\PYGZus{}set\PYGZus{}action}\PYG{p}{(}\PYG{n}{ddl1}\PYG{p}{,} \PYG{n}{ddlist\PYGZus{}action}\PYG{p}{)}\PYG{p}{;}  \PYG{c+cm}{/*Set a function to call when anew option is chosen*/}

\PYG{c+cm}{/*Create a style*/}
\PYG{k}{static} \PYG{n}{lv\PYGZus{}style\PYGZus{}t} \PYG{n}{style\PYGZus{}bg}\PYG{p}{;}
\PYG{n}{lv\PYGZus{}style\PYGZus{}copy}\PYG{p}{(}\PYG{o}{\PYGZam{}}\PYG{n}{style\PYGZus{}bg}\PYG{p}{,} \PYG{o}{\PYGZam{}}\PYG{n}{lv\PYGZus{}style\PYGZus{}pretty}\PYG{p}{)}\PYG{p}{;}
\PYG{n}{style\PYGZus{}bg}\PYG{p}{.}\PYG{n}{body}\PYG{p}{.}\PYG{n}{shadow}\PYG{p}{.}\PYG{n}{width} \PYG{o}{=} \PYG{l+m+mi}{4}\PYG{p}{;} \PYG{c+cm}{/*Enable the shadow*/}
\PYG{n}{style\PYGZus{}bg}\PYG{p}{.}\PYG{n}{text}\PYG{p}{.}\PYG{n}{color} \PYG{o}{=} \PYG{n}{LV\PYGZus{}COLOR\PYGZus{}MAKE}\PYG{p}{(}\PYG{l+m+mh}{0x10}\PYG{p}{,} \PYG{l+m+mh}{0x20}\PYG{p}{,} \PYG{l+m+mh}{0x50}\PYG{p}{)}\PYG{p}{;}

\PYG{c+cm}{/*Copy the drop down list and set the new style\PYGZus{}bg*/}
\PYG{n}{lv\PYGZus{}obj\PYGZus{}t} \PYG{o}{*} \PYG{n}{ddl2} \PYG{o}{=} \PYG{n}{lv\PYGZus{}ddlist\PYGZus{}create}\PYG{p}{(}\PYG{n}{lv\PYGZus{}scr\PYGZus{}act}\PYG{p}{(}\PYG{p}{)}\PYG{p}{,} \PYG{n}{ddl1}\PYG{p}{)}\PYG{p}{;}
\PYG{n}{lv\PYGZus{}obj\PYGZus{}align}\PYG{p}{(}\PYG{n}{ddl2}\PYG{p}{,} \PYG{n+nb}{NULL}\PYG{p}{,} \PYG{n}{LV\PYGZus{}ALIGN\PYGZus{}IN\PYGZus{}TOP\PYGZus{}RIGHT}\PYG{p}{,} \PYG{o}{\PYGZhy{}}\PYG{l+m+mi}{30}\PYG{p}{,} \PYG{l+m+mi}{10}\PYG{p}{)}\PYG{p}{;}
\PYG{n}{lv\PYGZus{}obj\PYGZus{}set\PYGZus{}free\PYGZus{}num}\PYG{p}{(}\PYG{n}{ddl2}\PYG{p}{,} \PYG{l+m+mi}{2}\PYG{p}{)}\PYG{p}{;}       \PYG{c+cm}{/*Set a unique ID*/}
\PYG{n}{lv\PYGZus{}obj\PYGZus{}set\PYGZus{}style}\PYG{p}{(}\PYG{n}{ddl2}\PYG{p}{,} \PYG{o}{\PYGZam{}}\PYG{n}{style\PYGZus{}bg}\PYG{p}{)}\PYG{p}{;}
\end{sphinxVerbatim}


\paragraph{MicroPython}
\label{\detokenize{object-types/ddlist:micropython}}
No examples yet.


\subsubsection{API}
\label{\detokenize{object-types/ddlist:api}}

\subsection{Gauge (lv\_gauge)}
\label{\detokenize{object-types/gauge:gauge-lv-gauge}}\label{\detokenize{object-types/gauge::doc}}

\subsubsection{Overview}
\label{\detokenize{object-types/gauge:overview}}
The gauge is a meter with \sphinxstylestrong{scale labels} and \sphinxstylestrong{needles}.
You can use the \sphinxcode{\sphinxupquote{lv\_gauge\_set\_scale(gauge, angle, line\_num, label\_cnt)}} function to adjust the scale angle and the number of the scale lines and labels.
The default settings are:  220 degrees, 6 scale labels and 21 lines.

The gauge can show \sphinxstylestrong{more then one needles}.
Use the \sphinxcode{\sphinxupquote{lv\_gauge\_set\_needle\_count(gauge, needle\_num, color\_array)}} function to set the number of needles and an array with colors for each needle. The array must be static or global variable becasue only its pointer is stored.

You can use \sphinxcode{\sphinxupquote{lv\_gauge\_set\_value(gauge, needle\_id, value)}} to \sphinxstylestrong{set the value of a needle}.

To set a \sphinxstylestrong{critical value} use \sphinxcode{\sphinxupquote{lv\_gauge\_set\_critical\_value(gauge, value)}}. The scale color will be changed to \sphinxcode{\sphinxupquote{line.color}} after this value. (default: 80)

The \sphinxstylestrong{range} of the gauge can be specified by \sphinxcode{\sphinxupquote{lv\_gauge\_set\_range(gauge, min, max)}}. The deafult range is 0..100.


\subsubsection{Styles}
\label{\detokenize{object-types/gauge:styles}}
The gauge uses one style which can be set by \sphinxcode{\sphinxupquote{lv\_gauge\_set\_style(gauge, LV\_GAUGE\_STYLE\_MAIN, \&style)}}. The gauge’s properties are derived from the following style attributes:
\begin{itemize}
\item {} 
\sphinxstylestrong{body.main\_color} line’s color at the beginning of the scale

\item {} 
\sphinxstylestrong{body.grad\_color} line’s color at the end of the scale (gradient with main color)

\item {} 
\sphinxstylestrong{body.padding.hor} line length

\item {} 
\sphinxstylestrong{body.padding.inner} label distance from the scale lines

\item {} 
\sphinxstylestrong{body.radius} radius of needle origin circle

\item {} 
\sphinxstylestrong{line.width} line width

\item {} 
\sphinxstylestrong{line.color} line’s color after the critical value

\item {} 
\sphinxstylestrong{text.font/color/letter\_space} label attributes

\end{itemize}


\subsubsection{Events}
\label{\detokenize{object-types/gauge:events}}
Only the \sphinxhref{/overview/events.html\#generic-events}{Genreric events} are sent by the object type.

Learn more about {\hyperref[\detokenize{overview/events::doc}]{\sphinxcrossref{\DUrole{doc,doc,doc}{Events}}}}.


\subsubsection{Keys}
\label{\detokenize{object-types/gauge:keys}}
No \sphinxstyleemphasis{Keys} are processed by the object type.

Learn more about {\hyperref[\detokenize{overview/indev::doc}]{\sphinxcrossref{\DUrole{doc,doc,doc}{Keys}}}}.


\subsubsection{Example}
\label{\detokenize{object-types/gauge:example}}

\paragraph{C}
\label{\detokenize{object-types/gauge:c}}
\sphinxincludegraphics{{gauge_1}.png}Gauge image

code

\begin{sphinxVerbatim}[commandchars=\\\{\}]
\PYG{c+cm}{/*Create a style*/}
\PYG{k}{static} \PYG{n}{lv\PYGZus{}style\PYGZus{}t} \PYG{n}{style}\PYG{p}{;}
\PYG{n}{lv\PYGZus{}style\PYGZus{}copy}\PYG{p}{(}\PYG{o}{\PYGZam{}}\PYG{n}{style}\PYG{p}{,} \PYG{o}{\PYGZam{}}\PYG{n}{lv\PYGZus{}style\PYGZus{}pretty\PYGZus{}color}\PYG{p}{)}\PYG{p}{;}
\PYG{n}{style}\PYG{p}{.}\PYG{n}{body}\PYG{p}{.}\PYG{n}{main\PYGZus{}color} \PYG{o}{=} \PYG{n}{LV\PYGZus{}COLOR\PYGZus{}HEX3}\PYG{p}{(}\PYG{l+m+mh}{0x666}\PYG{p}{)}\PYG{p}{;}     \PYG{c+cm}{/*Line color at the beginning*/}
\PYG{n}{style}\PYG{p}{.}\PYG{n}{body}\PYG{p}{.}\PYG{n}{grad\PYGZus{}color} \PYG{o}{=}  \PYG{n}{LV\PYGZus{}COLOR\PYGZus{}HEX3}\PYG{p}{(}\PYG{l+m+mh}{0x666}\PYG{p}{)}\PYG{p}{;}    \PYG{c+cm}{/*Line color at the end*/}
\PYG{n}{style}\PYG{p}{.}\PYG{n}{body}\PYG{p}{.}\PYG{n}{padding}\PYG{p}{.}\PYG{n}{hor} \PYG{o}{=} \PYG{l+m+mi}{10}\PYG{p}{;}                      \PYG{c+cm}{/*Scale line length*/}
\PYG{n}{style}\PYG{p}{.}\PYG{n}{body}\PYG{p}{.}\PYG{n}{padding}\PYG{p}{.}\PYG{n}{inner} \PYG{o}{=} \PYG{l+m+mi}{8} \PYG{p}{;}                    \PYG{c+cm}{/*Scale label padding*/}
\PYG{n}{style}\PYG{p}{.}\PYG{n}{body}\PYG{p}{.}\PYG{n}{border}\PYG{p}{.}\PYG{n}{color} \PYG{o}{=} \PYG{n}{LV\PYGZus{}COLOR\PYGZus{}HEX3}\PYG{p}{(}\PYG{l+m+mh}{0x333}\PYG{p}{)}\PYG{p}{;}   \PYG{c+cm}{/*Needle middle circle color*/}
\PYG{n}{style}\PYG{p}{.}\PYG{n}{line}\PYG{p}{.}\PYG{n}{width} \PYG{o}{=} \PYG{l+m+mi}{3}\PYG{p}{;}
\PYG{n}{style}\PYG{p}{.}\PYG{n}{text}\PYG{p}{.}\PYG{n}{color} \PYG{o}{=} \PYG{n}{LV\PYGZus{}COLOR\PYGZus{}HEX3}\PYG{p}{(}\PYG{l+m+mh}{0x333}\PYG{p}{)}\PYG{p}{;}
\PYG{n}{style}\PYG{p}{.}\PYG{n}{line}\PYG{p}{.}\PYG{n}{color} \PYG{o}{=} \PYG{n}{LV\PYGZus{}COLOR\PYGZus{}RED}\PYG{p}{;}                  \PYG{c+cm}{/*Line color after the critical value*/}


\PYG{c+cm}{/*Describe the color for the needles*/}
\PYG{k}{static} \PYG{n}{lv\PYGZus{}color\PYGZus{}t} \PYG{n}{needle\PYGZus{}colors}\PYG{p}{[}\PYG{p}{]} \PYG{o}{=} \PYG{p}{\PYGZob{}}\PYG{n}{LV\PYGZus{}COLOR\PYGZus{}BLUE}\PYG{p}{,} \PYG{n}{LV\PYGZus{}COLOR\PYGZus{}ORANGE}\PYG{p}{,} \PYG{n}{LV\PYGZus{}COLOR\PYGZus{}PURPLE}\PYG{p}{\PYGZcb{}}\PYG{p}{;}

\PYG{c+cm}{/*Create a gauge*/}
\PYG{n}{lv\PYGZus{}obj\PYGZus{}t} \PYG{o}{*} \PYG{n}{gauge1} \PYG{o}{=} \PYG{n}{lv\PYGZus{}gauge\PYGZus{}create}\PYG{p}{(}\PYG{n}{lv\PYGZus{}scr\PYGZus{}act}\PYG{p}{(}\PYG{p}{)}\PYG{p}{,} \PYG{n+nb}{NULL}\PYG{p}{)}\PYG{p}{;}
\PYG{n}{lv\PYGZus{}gauge\PYGZus{}set\PYGZus{}style}\PYG{p}{(}\PYG{n}{gauge1}\PYG{p}{,} \PYG{o}{\PYGZam{}}\PYG{n}{style}\PYG{p}{)}\PYG{p}{;}
\PYG{n}{lv\PYGZus{}gauge\PYGZus{}set\PYGZus{}needle\PYGZus{}count}\PYG{p}{(}\PYG{n}{gauge1}\PYG{p}{,} \PYG{l+m+mi}{3}\PYG{p}{,} \PYG{n}{needle\PYGZus{}colors}\PYG{p}{)}\PYG{p}{;}
\PYG{n}{lv\PYGZus{}obj\PYGZus{}align}\PYG{p}{(}\PYG{n}{gauge1}\PYG{p}{,} \PYG{n+nb}{NULL}\PYG{p}{,} \PYG{n}{LV\PYGZus{}ALIGN\PYGZus{}CENTER}\PYG{p}{,} \PYG{l+m+mi}{0}\PYG{p}{,} \PYG{l+m+mi}{20}\PYG{p}{)}\PYG{p}{;}

\PYG{c+cm}{/*Set the values*/}
\PYG{n}{lv\PYGZus{}gauge\PYGZus{}set\PYGZus{}value}\PYG{p}{(}\PYG{n}{gauge1}\PYG{p}{,} \PYG{l+m+mi}{0}\PYG{p}{,} \PYG{l+m+mi}{10}\PYG{p}{)}\PYG{p}{;}
\PYG{n}{lv\PYGZus{}gauge\PYGZus{}set\PYGZus{}value}\PYG{p}{(}\PYG{n}{gauge1}\PYG{p}{,} \PYG{l+m+mi}{1}\PYG{p}{,} \PYG{l+m+mi}{20}\PYG{p}{)}\PYG{p}{;}
\PYG{n}{lv\PYGZus{}gauge\PYGZus{}set\PYGZus{}value}\PYG{p}{(}\PYG{n}{gauge1}\PYG{p}{,} \PYG{l+m+mi}{2}\PYG{p}{,} \PYG{l+m+mi}{30}\PYG{p}{)}\PYG{p}{;}
\end{sphinxVerbatim}


\paragraph{MicroPython}
\label{\detokenize{object-types/gauge:micropython}}
No examples yet.


\subsubsection{API}
\label{\detokenize{object-types/gauge:api}}

\subsection{Image (lv\_img)}
\label{\detokenize{object-types/img:image-lv-img}}\label{\detokenize{object-types/img::doc}}

\subsubsection{Overview}
\label{\detokenize{object-types/img:overview}}
The Images are the basic object to \sphinxstylestrong{display images}. To provide maximum flexibility the \sphinxstylestrong{source of the image} can be:
\begin{itemize}
\item {} 
a variable in the code (a C array with the pixels)

\item {} 
a file stored externally (like on an SD card)

\item {} 
a text with {\hyperref[\detokenize{overview/fonts::doc}]{\sphinxcrossref{\DUrole{doc,doc,doc}{Symbols}}}}

\end{itemize}

To set the source of an image use \sphinxcode{\sphinxupquote{lv\_img\_set\_src(img, src)}}

To generate a pixel array \sphinxstylestrong{from a PNG, JPG or BMP} image use the \sphinxhref{https://littlevgl.com/image-to-c-array}{Online image converter tool} and set the converted image with its pointer: \sphinxcode{\sphinxupquote{lv\_img\_set\_src(img1, \&converted\_img\_var);}}
To make the variable visible in the C file you need to declare it with \sphinxcode{\sphinxupquote{LV\_IMG\_DECLARE(converted\_img\_var)}}

To use \sphinxstylestrong{external files} you also need to convert the image files using the online converter tool but now you should select the binary Output format.
You also need to use LittlevGL’s file system module and regsiter a driver with some functions for the basic file operation. Got to the {\hyperref[\detokenize{overview/file-system::doc}]{\sphinxcrossref{\DUrole{doc,doc,doc}{File system}}}} to learn more.
To set an image source form a file use \sphinxcode{\sphinxupquote{lv\_img\_set\_src(img, "S:folder1/my\_img.bin")}}

You can set a \sphinxstylestrong{symbol}  similarly to {\hyperref[\detokenize{object-types/label::doc}]{\sphinxcrossref{\DUrole{doc,doc,doc}{Labels}}}}. In this case, the image will be rendered as text according to the \sphinxstyleemphasis{font} specified in the style.  It enables to use light weighted mono-color
“letters” instead of real images. You can set symbol like \sphinxcode{\sphinxupquote{lv\_img\_set\_src(img1, LV\_SYMBOL\_OK)}}

Images and labels are sometimes for the same thing. E.g.to describe what a button does. Therefore Images and Labels are somewhat interchangable.
To handle this images can even display \sphinxstylestrong{texts} by using \sphinxcode{\sphinxupquote{LV\_SYMBOL\_DUMMY}} as prefix of the text. For example \sphinxcode{\sphinxupquote{lv\_img\_set\_src(img, LV\_SYMBOL\_DUMMY "Some text")}}

The internal (variable) and external images support 2 \sphinxstylestrong{transparency handling} methods:
\begin{itemize}
\item {} 
\sphinxstylestrong{Chrome keying} pixels with \sphinxcode{\sphinxupquote{LV\_COLOR\_TRANSP}} (lv\_conf.h) color will be transparent

\item {} 
\sphinxstylestrong{Alpha byte} An alpha byte is added to every pixel

\end{itemize}

Besides \sphinxstyleemphasis{True color} color format the following format are also supported:
\begin{itemize}
\item {} 
\sphinxstylestrong{Indexed} image has a palette

\item {} 
\sphinxstylestrong{Alpha indexed} only alpha values are stored

\end{itemize}

These options can be selected in the font converter. To learn more about the color formats read the {\hyperref[\detokenize{overview/images::doc}]{\sphinxcrossref{\DUrole{doc,doc,doc}{Images}}}} section.

The images can be \sphinxstylestrong{re-colored in run-time} to any color according to the brightness of the pixels.
It is very useful to show different states (selected, inactive, pressed etc) of an image without storing more versions of the same image.
This feature can be enabled in the style by setting \sphinxcode{\sphinxupquote{img.intense}} between \sphinxcode{\sphinxupquote{LV\_OPA\_TRANSP}} (no recolor, value: 0) and \sphinxcode{\sphinxupquote{LV\_OPA\_COVER}} (full recolor, value: 255).
The default value is \sphinxcode{\sphinxupquote{LV\_OPA\_TRANSP}} so this feature is disabled.

It is possible to \sphinxstylestrong{automatically set the size} of the image object to the image source’s width and height if enabled by the \sphinxcode{\sphinxupquote{lv\_img\_set\_auto\_size(image, true)}} function.
If \sphinxstyleemphasis{auto size} is enabled then when a new file is set the object size is automatically changed. Later you can modify the size manually. The \sphinxstyleemphasis{auto size} is enabled by default if the image is not a screen

If the object size is greater then the image size in any directions then the image will be repeated like a mosaic.
It’s a very useful a feature to create a large image from only a very narrow source.
For example you can have a \sphinxstyleemphasis{300 x 1} image with a special gradient and set it as a wallpaper using the mosaic feature.

The images’ default style is \sphinxstyleemphasis{NULL} so they \sphinxstylestrong{inherit the parent’s style}.


\subsubsection{Styles}
\label{\detokenize{object-types/img:styles}}
The images uses one style which can be set by \sphinxcode{\sphinxupquote{lv\_img\_set\_style(lmeter, LV\_IMG\_STYLE\_MAIN, \&style)}}. All the \sphinxcode{\sphinxupquote{style.image}} propertis are used:
\begin{itemize}
\item {} 
\sphinxstylestrong{image.inense} intensity of recoloring (0..255 or \sphinxstyleemphasis{LV\_OPA\_…})

\item {} 
\sphinxstylestrong{image.color} color for recoloring or color of the alpha indexed images

\item {} 
\sphinxstylestrong{image.opa} overall opacitiy of image

\end{itemize}

When the Image obejct idplays a text then \sphinxcode{\sphinxupquote{style.text}} properties are used. See {\hyperref[\detokenize{object-types/label::doc}]{\sphinxcrossref{\DUrole{doc,doc,doc}{Label}}}} for more information.


\subsubsection{Events}
\label{\detokenize{object-types/img:events}}
Only the \sphinxhref{/overview/events.html\#generic-events}{Genreric events} are sent by the object type.

Learn more about {\hyperref[\detokenize{overview/events::doc}]{\sphinxcrossref{\DUrole{doc,doc,doc}{Events}}}}.


\subsubsection{Keys}
\label{\detokenize{object-types/img:keys}}
No \sphinxstyleemphasis{Keys} are processed by the object type.

Learn more about {\hyperref[\detokenize{overview/indev::doc}]{\sphinxcrossref{\DUrole{doc,doc,doc}{Keys}}}}.


\subsubsection{Example}
\label{\detokenize{object-types/img:example}}

\paragraph{C}
\label{\detokenize{object-types/img:c}}
\sphinxincludegraphics{{img_1}.png}Example of Image in LittlevGL Graphics Library 

code

\begin{sphinxVerbatim}[commandchars=\\\{\}]

\PYG{c+cm}{/*Create the first image without re\PYGZhy{}color*/}
\PYG{n}{lv\PYGZus{}obj\PYGZus{}t} \PYG{o}{*} \PYG{n}{img1} \PYG{o}{=} \PYG{n}{lv\PYGZus{}img\PYGZus{}create}\PYG{p}{(}\PYG{n}{lv\PYGZus{}scr\PYGZus{}act}\PYG{p}{(}\PYG{p}{)}\PYG{p}{,} \PYG{n+nb}{NULL}\PYG{p}{)}\PYG{p}{;}
\PYG{n}{lv\PYGZus{}img\PYGZus{}set\PYGZus{}src}\PYG{p}{(}\PYG{n}{img1}\PYG{p}{,} \PYG{o}{\PYGZam{}}\PYG{n}{img\PYGZus{}cw}\PYG{p}{)}\PYG{p}{;}
\PYG{n}{lv\PYGZus{}obj\PYGZus{}align}\PYG{p}{(}\PYG{n}{img1}\PYG{p}{,} \PYG{n+nb}{NULL}\PYG{p}{,} \PYG{n}{LV\PYGZus{}ALIGN\PYGZus{}IN\PYGZus{}TOP\PYGZus{}LEFT}\PYG{p}{,} \PYG{l+m+mi}{20}\PYG{p}{,} \PYG{l+m+mi}{40}\PYG{p}{)}\PYG{p}{;}

\PYG{c+cm}{/*Create style to re\PYGZhy{}color with light blue*/}
\PYG{k}{static} \PYG{n}{lv\PYGZus{}style\PYGZus{}t} \PYG{n}{style\PYGZus{}img2}\PYG{p}{;}
\PYG{n}{lv\PYGZus{}style\PYGZus{}copy}\PYG{p}{(} \PYG{o}{\PYGZam{}}\PYG{n}{style\PYGZus{}img2}\PYG{p}{,} \PYG{o}{\PYGZam{}}\PYG{n}{lv\PYGZus{}style\PYGZus{}plain}\PYG{p}{)}\PYG{p}{;}
\PYG{n}{style\PYGZus{}img2}\PYG{p}{.}\PYG{n}{image}\PYG{p}{.}\PYG{n}{color} \PYG{o}{=} \PYG{n}{LV\PYGZus{}COLOR\PYGZus{}HEX}\PYG{p}{(}\PYG{l+m+mh}{0x003b75}\PYG{p}{)}\PYG{p}{;}
\PYG{n}{style\PYGZus{}img2}\PYG{p}{.}\PYG{n}{image}\PYG{p}{.}\PYG{n}{intense} \PYG{o}{=} \PYG{n}{LV\PYGZus{}OPA\PYGZus{}50}\PYG{p}{;}

\PYG{c+cm}{/*Create an image with the light blue style*/}
\PYG{n}{lv\PYGZus{}obj\PYGZus{}t} \PYG{o}{*} \PYG{n}{img2} \PYG{o}{=} \PYG{n}{lv\PYGZus{}img\PYGZus{}create}\PYG{p}{(}\PYG{n}{lv\PYGZus{}scr\PYGZus{}act}\PYG{p}{(}\PYG{p}{)}\PYG{p}{,} \PYG{n}{img1}\PYG{p}{)}\PYG{p}{;}
\PYG{n}{lv\PYGZus{}obj\PYGZus{}set\PYGZus{}style}\PYG{p}{(}\PYG{n}{img2}\PYG{p}{,} \PYG{o}{\PYGZam{}}\PYG{n}{style\PYGZus{}img2}\PYG{p}{)}\PYG{p}{;}
\PYG{n}{lv\PYGZus{}obj\PYGZus{}align}\PYG{p}{(}\PYG{n}{img2}\PYG{p}{,} \PYG{n+nb}{NULL}\PYG{p}{,} \PYG{n}{LV\PYGZus{}ALIGN\PYGZus{}IN\PYGZus{}TOP\PYGZus{}MID}\PYG{p}{,} \PYG{l+m+mi}{0}\PYG{p}{,} \PYG{l+m+mi}{40}\PYG{p}{)}\PYG{p}{;}

\PYG{c+cm}{/*Create style to re\PYGZhy{}color with dark blue*/}
\PYG{k}{static} \PYG{n}{lv\PYGZus{}style\PYGZus{}t} \PYG{n}{style\PYGZus{}img3}\PYG{p}{;}
\PYG{n}{lv\PYGZus{}style\PYGZus{}copy}\PYG{p}{(}\PYG{o}{\PYGZam{}}\PYG{n}{style\PYGZus{}img3}\PYG{p}{,} \PYG{o}{\PYGZam{}}\PYG{n}{lv\PYGZus{}style\PYGZus{}plain}\PYG{p}{)}\PYG{p}{;}
\PYG{n}{style\PYGZus{}img3}\PYG{p}{.}\PYG{n}{image}\PYG{p}{.}\PYG{n}{color} \PYG{o}{=} \PYG{n}{LV\PYGZus{}COLOR\PYGZus{}HEX}\PYG{p}{(}\PYG{l+m+mh}{0x003b75}\PYG{p}{)}\PYG{p}{;}
\PYG{n}{style\PYGZus{}img3}\PYG{p}{.}\PYG{n}{image}\PYG{p}{.}\PYG{n}{intense} \PYG{o}{=} \PYG{n}{LV\PYGZus{}OPA\PYGZus{}90}\PYG{p}{;}

\PYG{c+cm}{/*Create an image with the dark blue style*/}
\PYG{n}{lv\PYGZus{}obj\PYGZus{}t} \PYG{o}{*} \PYG{n}{img3} \PYG{o}{=} \PYG{n}{lv\PYGZus{}img\PYGZus{}create}\PYG{p}{(}\PYG{n}{lv\PYGZus{}scr\PYGZus{}act}\PYG{p}{(}\PYG{p}{)}\PYG{p}{,} \PYG{n}{img2}\PYG{p}{)}\PYG{p}{;}
\PYG{n}{lv\PYGZus{}obj\PYGZus{}set\PYGZus{}style}\PYG{p}{(}\PYG{n}{img3}\PYG{p}{,} \PYG{o}{\PYGZam{}}\PYG{n}{style\PYGZus{}img3}\PYG{p}{)}\PYG{p}{;}
\PYG{n}{lv\PYGZus{}obj\PYGZus{}align}\PYG{p}{(}\PYG{n}{img3}\PYG{p}{,} \PYG{n+nb}{NULL}\PYG{p}{,} \PYG{n}{LV\PYGZus{}ALIGN\PYGZus{}IN\PYGZus{}TOP\PYGZus{}RIGHT}\PYG{p}{,} \PYG{o}{\PYGZhy{}}\PYG{l+m+mi}{20}\PYG{p}{,} \PYG{l+m+mi}{40}\PYG{p}{)}\PYG{p}{;}


\PYG{c+cm}{/**************************************}
\PYG{c+cm}{ * Create an image with symbols}
\PYG{c+cm}{ **************************************/}

\PYG{c+cm}{/*Create a string from symbols*/}
\PYG{k+kt}{char} \PYG{n}{buf}\PYG{p}{[}\PYG{l+m+mi}{32}\PYG{p}{]}\PYG{p}{;}
\PYG{n}{sprintf}\PYG{p}{(}\PYG{n}{buf}\PYG{p}{,} \PYG{l+s}{\PYGZdq{}}\PYG{l+s}{\PYGZpc{}s\PYGZpc{}s\PYGZpc{}s\PYGZpc{}s\PYGZpc{}s\PYGZpc{}s\PYGZpc{}s}\PYG{l+s}{\PYGZdq{}}\PYG{p}{,}
             \PYG{n}{SYMBOL\PYGZus{}DRIVE}\PYG{p}{,} \PYG{n}{SYMBOL\PYGZus{}FILE}\PYG{p}{,} \PYG{n}{SYMBOL\PYGZus{}DIRECTORY}\PYG{p}{,} \PYG{n}{SYMBOL\PYGZus{}SETTINGS}\PYG{p}{,}
             \PYG{n}{SYMBOL\PYGZus{}POWER}\PYG{p}{,} \PYG{n}{SYMBOL\PYGZus{}GPS}\PYG{p}{,} \PYG{n}{SYMBOL\PYGZus{}BLUETOOTH}\PYG{p}{)}\PYG{p}{;}

\PYG{c+cm}{/*Create style with a symbol font*/}
\PYG{k}{static} \PYG{n}{lv\PYGZus{}style\PYGZus{}t} \PYG{n}{style\PYGZus{}sym}\PYG{p}{;}
\PYG{n}{lv\PYGZus{}style\PYGZus{}copy}\PYG{p}{(}\PYG{o}{\PYGZam{}}\PYG{n}{style\PYGZus{}sym}\PYG{p}{,} \PYG{o}{\PYGZam{}}\PYG{n}{lv\PYGZus{}style\PYGZus{}plain}\PYG{p}{)}\PYG{p}{;}
\PYG{c+c1}{// The built\PYGZhy{}in fonts are extended with symbols}
\PYG{n}{style\PYGZus{}sym}\PYG{p}{.}\PYG{n}{text}\PYG{p}{.}\PYG{n}{font} \PYG{o}{=} \PYG{o}{\PYGZam{}}\PYG{n}{lv\PYGZus{}font\PYGZus{}dejavu\PYGZus{}60}\PYG{p}{;}
\PYG{n}{style\PYGZus{}sym}\PYG{p}{.}\PYG{n}{text}\PYG{p}{.}\PYG{n}{letter\PYGZus{}space} \PYG{o}{=} \PYG{l+m+mi}{10}\PYG{p}{;}

\PYG{c+cm}{/*Create an image and use the string as source*/}
\PYG{n}{lv\PYGZus{}obj\PYGZus{}t} \PYG{o}{*} \PYG{n}{img\PYGZus{}sym} \PYG{o}{=} \PYG{n}{lv\PYGZus{}img\PYGZus{}create}\PYG{p}{(}\PYG{n}{lv\PYGZus{}scr\PYGZus{}act}\PYG{p}{(}\PYG{p}{)}\PYG{p}{,} \PYG{n+nb}{NULL}\PYG{p}{)}\PYG{p}{;}
\PYG{n}{lv\PYGZus{}img\PYGZus{}set\PYGZus{}src}\PYG{p}{(}\PYG{n}{img\PYGZus{}sym}\PYG{p}{,} \PYG{n}{buf}\PYG{p}{)}\PYG{p}{;}
\PYG{n}{lv\PYGZus{}img\PYGZus{}set\PYGZus{}style}\PYG{p}{(}\PYG{n}{img\PYGZus{}sym}\PYG{p}{,} \PYG{o}{\PYGZam{}}\PYG{n}{style\PYGZus{}sym}\PYG{p}{)}\PYG{p}{;}
\PYG{n}{lv\PYGZus{}obj\PYGZus{}align}\PYG{p}{(}\PYG{n}{img\PYGZus{}sym}\PYG{p}{,} \PYG{n+nb}{NULL}\PYG{p}{,} \PYG{n}{LV\PYGZus{}ALIGN\PYGZus{}IN\PYGZus{}BOTTOM\PYGZus{}MID}\PYG{p}{,} \PYG{l+m+mi}{0}\PYG{p}{,} \PYG{o}{\PYGZhy{}}\PYG{l+m+mi}{30}\PYG{p}{)}\PYG{p}{;}


\PYG{c+cm}{/*Create description labels*/}
\PYG{n}{lv\PYGZus{}obj\PYGZus{}t} \PYG{o}{*} \PYG{n}{label} \PYG{o}{=} \PYG{n}{lv\PYGZus{}label\PYGZus{}create}\PYG{p}{(}\PYG{n}{lv\PYGZus{}scr\PYGZus{}act}\PYG{p}{(}\PYG{p}{)}\PYG{p}{,} \PYG{n+nb}{NULL}\PYG{p}{)}\PYG{p}{;}
\PYG{n}{lv\PYGZus{}label\PYGZus{}set\PYGZus{}text}\PYG{p}{(}\PYG{n}{label}\PYG{p}{,} \PYG{l+s}{\PYGZdq{}}\PYG{l+s}{Re\PYGZhy{}color the images in run time}\PYG{l+s}{\PYGZdq{}}\PYG{p}{)}\PYG{p}{;}
\PYG{n}{lv\PYGZus{}obj\PYGZus{}align}\PYG{p}{(}\PYG{n}{label}\PYG{p}{,} \PYG{n+nb}{NULL}\PYG{p}{,} \PYG{n}{LV\PYGZus{}ALIGN\PYGZus{}IN\PYGZus{}TOP\PYGZus{}MID}\PYG{p}{,} \PYG{l+m+mi}{0}\PYG{p}{,} \PYG{l+m+mi}{15}\PYG{p}{)}\PYG{p}{;}

\PYG{n}{label} \PYG{o}{=} \PYG{n}{lv\PYGZus{}label\PYGZus{}create}\PYG{p}{(}\PYG{n}{lv\PYGZus{}scr\PYGZus{}act}\PYG{p}{(}\PYG{p}{)}\PYG{p}{,} \PYG{n+nb}{NULL}\PYG{p}{)}\PYG{p}{;}
\PYG{n}{lv\PYGZus{}label\PYGZus{}set\PYGZus{}text}\PYG{p}{(}\PYG{n}{label}\PYG{p}{,} \PYG{l+s}{\PYGZdq{}}\PYG{l+s}{Use symbols from fonts as images}\PYG{l+s}{\PYGZdq{}}\PYG{p}{)}\PYG{p}{;}
\PYG{n}{lv\PYGZus{}obj\PYGZus{}align}\PYG{p}{(}\PYG{n}{label}\PYG{p}{,} \PYG{n+nb}{NULL}\PYG{p}{,} \PYG{n}{LV\PYGZus{}ALIGN\PYGZus{}IN\PYGZus{}BOTTOM\PYGZus{}MID}\PYG{p}{,} \PYG{l+m+mi}{0}\PYG{p}{,} \PYG{o}{\PYGZhy{}}\PYG{l+m+mi}{80}\PYG{p}{)}\PYG{p}{;}
\end{sphinxVerbatim}


\paragraph{MicroPython}
\label{\detokenize{object-types/img:micropython}}
No examples yet.


\subsubsection{API}
\label{\detokenize{object-types/img:api}}

\subsection{Image button (lv\_imgbtn)}
\label{\detokenize{object-types/imgbtn:image-button-lv-imgbtn}}\label{\detokenize{object-types/imgbtn::doc}}

\subsubsection{Overview}
\label{\detokenize{object-types/imgbtn:overview}}
The Image button is very similar to the simple Button object. The only difference is it displays user-defined images in each state instead of drawing a button.
Before reading this please read the {\hyperref[\detokenize{object-types/btn::doc}]{\sphinxcrossref{\DUrole{doc,doc,doc}{Button}}}} section too.

To set the image in a state the \sphinxcode{\sphinxupquote{lv\_imgbtn\_set\_src(imgbtn, LV\_BTN\_STATE\_..., \&img\_src)}} The image sources works the same as described in the {\hyperref[\detokenize{object-types/img::doc}]{\sphinxcrossref{\DUrole{doc,doc,doc}{Image object}}}}.

If \sphinxcode{\sphinxupquote{LV\_IMGBTN\_TILED}} is enabled in \sphinxstyleemphasis{lv\_conf.h} three source can be set for state:
\begin{itemize}
\item {} 
left

\item {} 
ceter

\item {} 
right

\end{itemize}

The \sphinxstyleemphasis{center} image will repeated to fill the width of object. Therefore with \sphinxcode{\sphinxupquote{LV\_IMGBTN\_TILED}} you can set the width of the Image button while without it the width will be always the same as the image source’s width.

The \sphinxstylestrong{states} also work like with Button object. It can be set with \sphinxcode{\sphinxupquote{lv\_imgbtn\_set\_state(imgbtn, LV\_BTN\_STATE\_...)}}.

The \sphinxstylestrong{toggle} feature can be enabled with \sphinxcode{\sphinxupquote{lv\_imgbtn\_set\_toggle(imgbtn, true)}}


\subsubsection{Style usage}
\label{\detokenize{object-types/imgbtn:style-usage}}
Similarly to normal Buttons, Image buttons also have 5 independent styles for the 5 state. You can set them via: \sphinxcode{\sphinxupquote{lv\_imgbtn\_set\_style(btn, LV\_IMGBTN\_STYLE\_..., \&style)}}. The styles use the \sphinxcode{\sphinxupquote{style.image}} properties.
\begin{itemize}
\item {} 
\sphinxstylestrong{LV\_IMGBTN\_STYLE\_REL} style of the released state. Default: \sphinxcode{\sphinxupquote{lv\_style\_btn\_rel}}

\item {} 
\sphinxstylestrong{LV\_IMGBTN\_STYLE\_PR} style of the pressed state. Default: \sphinxcode{\sphinxupquote{lv\_style\_btn\_pr}}

\item {} 
\sphinxstylestrong{LV\_IMGBTN\_STYLE\_TGL\_REL} style of the toggled released state. Default: \sphinxcode{\sphinxupquote{lv\_style\_btn\_tgl\_rel}}

\item {} 
\sphinxstylestrong{LV\_IMGBTN\_STYLE\_TGL\_PR} style of the toggled pressed state. Default: \sphinxcode{\sphinxupquote{lv\_style\_btn\_tgl\_pr}}

\item {} 
\sphinxstylestrong{LV\_IMGBTN\_STYLE\_INA} style of the inactive state. Default: \sphinxcode{\sphinxupquote{lv\_style\_btn\_ina}}

\end{itemize}

When labels are created on a button, it’s a good practive to set the image button’s \sphinxcode{\sphinxupquote{style.text}} properties too. Because labels have \sphinxcode{\sphinxupquote{style = NULL}} by default they inherit the parent’s (image button) style.
Hence you don’t need to craete a new style for the label.


\subsubsection{Events}
\label{\detokenize{object-types/imgbtn:events}}
Besided the \sphinxhref{/overview/events.html\#generic-events}{Genreric events} the following \sphinxhref{/overview/events.html\#special-events}{Special events} are sent by the buttons:
\begin{itemize}
\item {} 
\sphinxstylestrong{LV\_EVENT\_VALUE\_CHANGED} sent when the button is toggled.

\end{itemize}

Note that the generic input device related events (like \sphinxcode{\sphinxupquote{LV\_EVENT\_PRESSED}}) are sent in inactive state too. You need to check the state with \sphinxcode{\sphinxupquote{lv\_btn\_get\_state(btn)}} to ignore the events from inactive buttons.

Learn more about {\hyperref[\detokenize{overview/events::doc}]{\sphinxcrossref{\DUrole{doc,doc,doc}{Events}}}}.


\subsubsection{Keys}
\label{\detokenize{object-types/imgbtn:keys}}
The following \sphinxstyleemphasis{Keys} are processed by the Buttons:
\begin{itemize}
\item {} 
\sphinxstylestrong{LV\_KEY\_RIGHT/UP} Go to toggled state if toggling is enabled

\item {} 
\sphinxstylestrong{LV\_KEY\_LEFT/DOWN} Go to non-toggled state if toggling is  enabled

\end{itemize}

Note that, as usual, the state of \sphinxcode{\sphinxupquote{LV\_KEY\_ENTER}} is translated to \sphinxcode{\sphinxupquote{LV\_EVENT\_PRESSED/PRESSING/RELEASED}} etc.

Learn more about {\hyperref[\detokenize{overview/indev::doc}]{\sphinxcrossref{\DUrole{doc,doc,doc}{Keys}}}}.


\subsubsection{Example}
\label{\detokenize{object-types/imgbtn:example}}

\paragraph{C}
\label{\detokenize{object-types/imgbtn:c}}
\sphinxincludegraphics{{imgbtn_1}.png}Image button image

code

\begin{sphinxVerbatim}[commandchars=\\\{\}]
\PYG{c+cm}{/*Create style to make the button darker when pressed*/}
\PYG{n}{lv\PYGZus{}style\PYGZus{}t} \PYG{n}{style\PYGZus{}pr}\PYG{p}{;}
\PYG{n}{lv\PYGZus{}style\PYGZus{}copy}\PYG{p}{(}\PYG{o}{\PYGZam{}}\PYG{n}{style\PYGZus{}pr}\PYG{p}{,} \PYG{o}{\PYGZam{}}\PYG{n}{lv\PYGZus{}style\PYGZus{}plain}\PYG{p}{)}\PYG{p}{;}
\PYG{n}{style\PYGZus{}pr}\PYG{p}{.}\PYG{n}{image}\PYG{p}{.}\PYG{n}{color} \PYG{o}{=} \PYG{n}{LV\PYGZus{}COLOR\PYGZus{}BLACK}\PYG{p}{;}
\PYG{n}{style\PYGZus{}pr}\PYG{p}{.}\PYG{n}{image}\PYG{p}{.}\PYG{n}{intense} \PYG{o}{=} \PYG{n}{LV\PYGZus{}OPA\PYGZus{}50}\PYG{p}{;}
\PYG{n}{style\PYGZus{}pr}\PYG{p}{.}\PYG{n}{text}\PYG{p}{.}\PYG{n}{color} \PYG{o}{=} \PYG{n}{LV\PYGZus{}COLOR\PYGZus{}HEX3}\PYG{p}{(}\PYG{l+m+mh}{0xaaa}\PYG{p}{)}\PYG{p}{;}

\PYG{n}{LV\PYGZus{}IMG\PYGZus{}DECLARE}\PYG{p}{(}\PYG{n}{imgbtn\PYGZus{}green}\PYG{p}{)}\PYG{p}{;}
\PYG{n}{LV\PYGZus{}IMG\PYGZus{}DECLARE}\PYG{p}{(}\PYG{n}{imgbtn\PYGZus{}blue}\PYG{p}{)}\PYG{p}{;}

\PYG{c+cm}{/*Create an Image button*/}
\PYG{n}{lv\PYGZus{}obj\PYGZus{}t} \PYG{o}{*} \PYG{n}{imgbtn1} \PYG{o}{=} \PYG{n}{lv\PYGZus{}imgbtn\PYGZus{}create}\PYG{p}{(}\PYG{n}{lv\PYGZus{}scr\PYGZus{}act}\PYG{p}{(}\PYG{p}{)}\PYG{p}{,} \PYG{n+nb}{NULL}\PYG{p}{)}\PYG{p}{;}
\PYG{n}{lv\PYGZus{}imgbtn\PYGZus{}set\PYGZus{}src}\PYG{p}{(}\PYG{n}{imgbtn1}\PYG{p}{,} \PYG{n}{LV\PYGZus{}BTN\PYGZus{}STATE\PYGZus{}REL}\PYG{p}{,} \PYG{o}{\PYGZam{}}\PYG{n}{imgbtn\PYGZus{}green}\PYG{p}{)}\PYG{p}{;}
\PYG{n}{lv\PYGZus{}imgbtn\PYGZus{}set\PYGZus{}src}\PYG{p}{(}\PYG{n}{imgbtn1}\PYG{p}{,} \PYG{n}{LV\PYGZus{}BTN\PYGZus{}STATE\PYGZus{}PR}\PYG{p}{,} \PYG{o}{\PYGZam{}}\PYG{n}{imgbtn\PYGZus{}green}\PYG{p}{)}\PYG{p}{;}
\PYG{n}{lv\PYGZus{}imgbtn\PYGZus{}set\PYGZus{}src}\PYG{p}{(}\PYG{n}{imgbtn1}\PYG{p}{,} \PYG{n}{LV\PYGZus{}BTN\PYGZus{}STATE\PYGZus{}TGL\PYGZus{}REL}\PYG{p}{,} \PYG{o}{\PYGZam{}}\PYG{n}{imgbtn\PYGZus{}blue}\PYG{p}{)}\PYG{p}{;}
\PYG{n}{lv\PYGZus{}imgbtn\PYGZus{}set\PYGZus{}src}\PYG{p}{(}\PYG{n}{imgbtn1}\PYG{p}{,} \PYG{n}{LV\PYGZus{}BTN\PYGZus{}STATE\PYGZus{}TGL\PYGZus{}PR}\PYG{p}{,} \PYG{o}{\PYGZam{}}\PYG{n}{imgbtn\PYGZus{}blue}\PYG{p}{)}\PYG{p}{;}
\PYG{n}{lv\PYGZus{}imgbtn\PYGZus{}set\PYGZus{}style}\PYG{p}{(}\PYG{n}{imgbtn1}\PYG{p}{,} \PYG{n}{LV\PYGZus{}BTN\PYGZus{}STATE\PYGZus{}PR}\PYG{p}{,} \PYG{o}{\PYGZam{}}\PYG{n}{style\PYGZus{}pr}\PYG{p}{)}\PYG{p}{;}        \PYG{c+cm}{/*Use the darker style in the pressed state*/}
\PYG{n}{lv\PYGZus{}imgbtn\PYGZus{}set\PYGZus{}style}\PYG{p}{(}\PYG{n}{imgbtn1}\PYG{p}{,} \PYG{n}{LV\PYGZus{}BTN\PYGZus{}STATE\PYGZus{}TGL\PYGZus{}PR}\PYG{p}{,} \PYG{o}{\PYGZam{}}\PYG{n}{style\PYGZus{}pr}\PYG{p}{)}\PYG{p}{;}
\PYG{n}{lv\PYGZus{}imgbtn\PYGZus{}set\PYGZus{}toggle}\PYG{p}{(}\PYG{n}{imgbtn1}\PYG{p}{,} \PYG{n+nb}{true}\PYG{p}{)}\PYG{p}{;}
\PYG{n}{lv\PYGZus{}obj\PYGZus{}align}\PYG{p}{(}\PYG{n}{imgbtn1}\PYG{p}{,} \PYG{n+nb}{NULL}\PYG{p}{,} \PYG{n}{LV\PYGZus{}ALIGN\PYGZus{}CENTER}\PYG{p}{,} \PYG{l+m+mi}{0}\PYG{p}{,} \PYG{o}{\PYGZhy{}}\PYG{l+m+mi}{40}\PYG{p}{)}\PYG{p}{;}

\PYG{c+cm}{/*Create a label on the Image button*/}
\PYG{n}{lv\PYGZus{}obj\PYGZus{}t} \PYG{o}{*} \PYG{n}{label} \PYG{o}{=} \PYG{n}{lv\PYGZus{}label\PYGZus{}create}\PYG{p}{(}\PYG{n}{imgbtn1}\PYG{p}{,} \PYG{n+nb}{NULL}\PYG{p}{)}\PYG{p}{;}
\PYG{n}{lv\PYGZus{}label\PYGZus{}set\PYGZus{}text}\PYG{p}{(}\PYG{n}{label}\PYG{p}{,} \PYG{l+s}{\PYGZdq{}}\PYG{l+s}{Button}\PYG{l+s}{\PYGZdq{}}\PYG{p}{)}\PYG{p}{;}

\PYG{c+cm}{/*Copy the fist image button and set Toggled state*/}
\PYG{n}{lv\PYGZus{}obj\PYGZus{}t} \PYG{o}{*} \PYG{n}{imgbtn2} \PYG{o}{=} \PYG{n}{lv\PYGZus{}imgbtn\PYGZus{}create}\PYG{p}{(}\PYG{n}{lv\PYGZus{}scr\PYGZus{}act}\PYG{p}{(}\PYG{p}{)}\PYG{p}{,} \PYG{n}{imgbtn1}\PYG{p}{)}\PYG{p}{;}
\PYG{n}{lv\PYGZus{}btn\PYGZus{}set\PYGZus{}state}\PYG{p}{(}\PYG{n}{imgbtn2}\PYG{p}{,} \PYG{n}{LV\PYGZus{}BTN\PYGZus{}STATE\PYGZus{}TGL\PYGZus{}REL}\PYG{p}{)}\PYG{p}{;}
\PYG{n}{lv\PYGZus{}obj\PYGZus{}align}\PYG{p}{(}\PYG{n}{imgbtn2}\PYG{p}{,} \PYG{n}{imgbtn1}\PYG{p}{,} \PYG{n}{LV\PYGZus{}ALIGN\PYGZus{}OUT\PYGZus{}BOTTOM\PYGZus{}MID}\PYG{p}{,} \PYG{l+m+mi}{0}\PYG{p}{,} \PYG{l+m+mi}{20}\PYG{p}{)}\PYG{p}{;}

\PYG{c+cm}{/*Create a label on the Image button*/}
\PYG{n}{label} \PYG{o}{=} \PYG{n}{lv\PYGZus{}label\PYGZus{}create}\PYG{p}{(}\PYG{n}{imgbtn2}\PYG{p}{,} \PYG{n+nb}{NULL}\PYG{p}{)}\PYG{p}{;}
\PYG{n}{lv\PYGZus{}label\PYGZus{}set\PYGZus{}text}\PYG{p}{(}\PYG{n}{label}\PYG{p}{,} \PYG{l+s}{\PYGZdq{}}\PYG{l+s}{Button}\PYG{l+s}{\PYGZdq{}}\PYG{p}{)}\PYG{p}{;}

\end{sphinxVerbatim}


\paragraph{MicroPython}
\label{\detokenize{object-types/imgbtn:micropython}}
No examples yet.


\subsubsection{API}
\label{\detokenize{object-types/imgbtn:api}}

\subsection{Keyboard (lv\_kb)}
\label{\detokenize{object-types/kb:keyboard-lv-kb}}\label{\detokenize{object-types/kb::doc}}

\subsubsection{Overview}
\label{\detokenize{object-types/kb:overview}}
The Keybord object is a special {\hyperref[\detokenize{object-types/btnm::doc}]{\sphinxcrossref{\DUrole{doc,doc,doc}{Button matrix}}}} with predefined key maps and other features to realize a virtual keyboard to write text.

The Keyboards have two \sphinxstylestrong{modes}:
\begin{itemize}
\item {} 
\sphinxstylestrong{LV\_KB\_MODE\_TEXT} display letters, number and special characters

\item {} 
\sphinxstylestrong{LV\_KB\_MODE\_NUM} display numbers, +/- sign and decimal dot

\end{itemize}

To set the mode use \sphinxcode{\sphinxupquote{lv\_kb\_set\_mode(kb, mode)}}. The default is  \sphinxstyleemphasis{LV\_KB\_MODE\_TEXT}

You can assign a {\hyperref[\detokenize{object-types/ta::doc}]{\sphinxcrossref{\DUrole{doc,doc,doc}{Text area}}}} to the Keyboard to automatically put the clicked characters there.
To assign the Text area use \sphinxcode{\sphinxupquote{lv\_kb\_set\_ta(kb, ta)}}.

The assigned Text area’s \sphinxstylestrong{cursor} can be \sphinxstylestrong{managed} by the keyboard: when the keyboard is assigned the previous Text area’s cursor will be hidden an the new’s will be shown.
When the keyboard is closed by the \sphinxstyleemphasis{Ok} or \sphinxstyleemphasis{Cloase} buttons the cursor also wil be hidden. The cursor manager feature is enabled by \sphinxcode{\sphinxupquote{lv\_kb\_set\_cursor\_manage(kb, true)}}. The default is not manage.

You can specify a \sphinxstylestrong{new map} (layout) for the keyboard with \sphinxcode{\sphinxupquote{lv\_kb\_set\_map(kb, map)}}. and \sphinxcode{\sphinxupquote{lv\_kb\_set\_ctrl\_map(kb, ctrl\_map)}}.
Learn more about in the {\hyperref[\detokenize{object-types/btnm::doc}]{\sphinxcrossref{\DUrole{doc,doc,doc}{Button matrix}}}} object.
Keep in mind using following keywords will have the same effect as with the original map:
\begin{itemize}
\item {} 
\sphinxstyleemphasis{LV\_SYMBOL\_OK} Apply

\item {} 
\sphinxstyleemphasis{SYMBOL\_CLOSE} Close

\item {} 
\sphinxstyleemphasis{LV\_SYMBOL\_LEFT} Move the cursor left

\item {} 
\sphinxstyleemphasis{LV\_SYMBOL\_RIGHT} Move the cursor right

\item {} 
\sphinxstyleemphasis{“ABC”} load the uppersace map

\item {} 
\sphinxstyleemphasis{“abc”} load the lower case map

\item {} 
\sphinxstyleemphasis{“Enter”} new line

\item {} 
\sphinxstyleemphasis{“Bkps”} Delete on the left

\end{itemize}

The keyboard has a \sphinxstylestrong{default event handler} callback called \sphinxcode{\sphinxupquote{lv\_kb\_def\_event\_cb}}.
It handles the button pressing, map changeing, the assigned Text area, etc.
You can completly replace it with your custom event handler but you can call \sphinxcode{\sphinxupquote{lv\_kb\_def\_event\_cb}} at he beginning of your event handler to handle the same things as before.


\subsubsection{Styles}
\label{\detokenize{object-types/kb:styles}}
The Keyboards work with 6 styles: a background and 5 button styles for each states.
You can set the styles with \sphinxcode{\sphinxupquote{lv\_kb\_set\_style(btn, LV\_KB\_STYLE\_..., \&style)}}.
The background and the buttons use the \sphinxcode{\sphinxupquote{style.body}} properties.
The labels use the \sphinxcode{\sphinxupquote{style.text}} properties of the buttons’ styles.
\begin{itemize}
\item {} 
\sphinxstylestrong{LV\_KB\_STYLE\_BG} Background style. Uses all \sphinxcode{\sphinxupquote{style.body}} properties including \sphinxcode{\sphinxupquote{padding}} Default: \sphinxcode{\sphinxupquote{lv\_style\_pretty}}

\item {} 
\sphinxstylestrong{LV\_KB\_STYLE\_BTN\_REL} style of the released  buttons. Default: \sphinxcode{\sphinxupquote{lv\_style\_btn\_rel}}

\item {} 
\sphinxstylestrong{LV\_KB\_STYLE\_BTN\_PR} style of the pressed buttons. Default: \sphinxcode{\sphinxupquote{lv\_style\_btn\_pr}}

\item {} 
\sphinxstylestrong{LV\_KB\_STYLE\_BTN\_TGL\_REL} style of the toggled released  buttons. Default: \sphinxcode{\sphinxupquote{lv\_style\_btn\_tgl\_rel}}

\item {} 
\sphinxstylestrong{LV\_KB\_STYLE\_BTN\_TGL\_PR} style of the toggled pressed  buttons. Default: \sphinxcode{\sphinxupquote{lv\_style\_btn\_tgl\_pr}}

\item {} 
\sphinxstylestrong{LV\_KB\_STYLE\_BTN\_INA} style of the inactive  buttons. Default: \sphinxcode{\sphinxupquote{lv\_style\_btn\_ina}}

\end{itemize}


\subsubsection{Events}
\label{\detokenize{object-types/kb:events}}
Besided the \sphinxhref{/overview/events.html\#generic-events}{Genreric events} the following \sphinxhref{/overview/events.html\#special-events}{Special events} are sent by the keyboards:
\begin{itemize}
\item {} 
\sphinxstylestrong{LV\_EVENT\_VALUE\_CHANGED} sent when the button is pressed/released or repeated after long press. The event data is set to ID of the pressed/released button.

\item {} 
\sphinxstylestrong{LV\_EVENT\_APPLY} the \sphinxstyleemphasis{Ok} button is clicked

\item {} 
\sphinxstylestrong{LV\_EVENT\_CANCEL} the \sphinxstyleemphasis{Close} button is clicked

\end{itemize}

Learn more about {\hyperref[\detokenize{overview/events::doc}]{\sphinxcrossref{\DUrole{doc,doc,doc}{Events}}}}.


\subsubsection{Keys}
\label{\detokenize{object-types/kb:keys}}
The following \sphinxstyleemphasis{Keys} are processed by the Buttons:
\begin{itemize}
\item {} 
\sphinxstylestrong{LV\_KEY\_RIGHT/UP/LEFT/RIGHT} To navigate among the buttons and elect one

\item {} 
\sphinxstylestrong{LV\_KEY\_ENTER} To press/release the selected button

\end{itemize}

Learn more about {\hyperref[\detokenize{overview/indev::doc}]{\sphinxcrossref{\DUrole{doc,doc,doc}{Keys}}}}.


\subsubsection{Examples}
\label{\detokenize{object-types/kb:examples}}

\paragraph{C}
\label{\detokenize{object-types/kb:c}}
\sphinxincludegraphics{{kb_1}.png}Keyboard image

code

\begin{sphinxVerbatim}[commandchars=\\\{\}]
\PYG{c+cm}{/*Create styles for the keyboard*/}
\PYG{k}{static} \PYG{n}{lv\PYGZus{}style\PYGZus{}t} \PYG{n}{rel\PYGZus{}style}\PYG{p}{,} \PYG{n}{pr\PYGZus{}style}\PYG{p}{;}

\PYG{n}{lv\PYGZus{}style\PYGZus{}copy}\PYG{p}{(}\PYG{o}{\PYGZam{}}\PYG{n}{rel\PYGZus{}style}\PYG{p}{,} \PYG{o}{\PYGZam{}}\PYG{n}{lv\PYGZus{}style\PYGZus{}btn\PYGZus{}rel}\PYG{p}{)}\PYG{p}{;}
\PYG{n}{rel\PYGZus{}style}\PYG{p}{.}\PYG{n}{body}\PYG{p}{.}\PYG{n}{radius} \PYG{o}{=} \PYG{l+m+mi}{0}\PYG{p}{;}

\PYG{n}{lv\PYGZus{}style\PYGZus{}copy}\PYG{p}{(}\PYG{o}{\PYGZam{}}\PYG{n}{pr\PYGZus{}style}\PYG{p}{,} \PYG{o}{\PYGZam{}}\PYG{n}{lv\PYGZus{}style\PYGZus{}btn\PYGZus{}pr}\PYG{p}{)}\PYG{p}{;}
\PYG{n}{pr\PYGZus{}style}\PYG{p}{.}\PYG{n}{body}\PYG{p}{.}\PYG{n}{radius} \PYG{o}{=} \PYG{l+m+mi}{0}\PYG{p}{;}

\PYG{c+cm}{/*Create a keyboard and apply the styles*/}
\PYG{n}{lv\PYGZus{}obj\PYGZus{}t} \PYG{o}{*}\PYG{n}{kb} \PYG{o}{=} \PYG{n}{lv\PYGZus{}kb\PYGZus{}create}\PYG{p}{(}\PYG{n}{lv\PYGZus{}scr\PYGZus{}act}\PYG{p}{(}\PYG{p}{)}\PYG{p}{,} \PYG{n+nb}{NULL}\PYG{p}{)}\PYG{p}{;}
\PYG{n}{lv\PYGZus{}kb\PYGZus{}set\PYGZus{}cursor\PYGZus{}manage}\PYG{p}{(}\PYG{n}{kb}\PYG{p}{,} \PYG{n+nb}{true}\PYG{p}{)}\PYG{p}{;}
\PYG{n}{lv\PYGZus{}kb\PYGZus{}set\PYGZus{}style}\PYG{p}{(}\PYG{n}{kb}\PYG{p}{,} \PYG{n}{LV\PYGZus{}KB\PYGZus{}STYLE\PYGZus{}BG}\PYG{p}{,} \PYG{o}{\PYGZam{}}\PYG{n}{lv\PYGZus{}style\PYGZus{}transp\PYGZus{}tight}\PYG{p}{)}\PYG{p}{;}
\PYG{n}{lv\PYGZus{}kb\PYGZus{}set\PYGZus{}style}\PYG{p}{(}\PYG{n}{kb}\PYG{p}{,} \PYG{n}{LV\PYGZus{}KB\PYGZus{}STYLE\PYGZus{}BTN\PYGZus{}REL}\PYG{p}{,} \PYG{o}{\PYGZam{}}\PYG{n}{rel\PYGZus{}style}\PYG{p}{)}\PYG{p}{;}
\PYG{n}{lv\PYGZus{}kb\PYGZus{}set\PYGZus{}style}\PYG{p}{(}\PYG{n}{kb}\PYG{p}{,} \PYG{n}{LV\PYGZus{}KB\PYGZus{}STYLE\PYGZus{}BTN\PYGZus{}PR}\PYG{p}{,} \PYG{o}{\PYGZam{}}\PYG{n}{pr\PYGZus{}style}\PYG{p}{)}\PYG{p}{;}

\PYG{c+cm}{/*Create a text area. The keyboard will write here*/}
\PYG{n}{lv\PYGZus{}obj\PYGZus{}t} \PYG{o}{*}\PYG{n}{ta} \PYG{o}{=} \PYG{n}{lv\PYGZus{}ta\PYGZus{}create}\PYG{p}{(}\PYG{n}{lv\PYGZus{}scr\PYGZus{}act}\PYG{p}{(}\PYG{p}{)}\PYG{p}{,} \PYG{n+nb}{NULL}\PYG{p}{)}\PYG{p}{;}
\PYG{n}{lv\PYGZus{}obj\PYGZus{}align}\PYG{p}{(}\PYG{n}{ta}\PYG{p}{,} \PYG{n+nb}{NULL}\PYG{p}{,} \PYG{n}{LV\PYGZus{}ALIGN\PYGZus{}IN\PYGZus{}TOP\PYGZus{}MID}\PYG{p}{,} \PYG{l+m+mi}{0}\PYG{p}{,} \PYG{l+m+mi}{10}\PYG{p}{)}\PYG{p}{;}
\PYG{n}{lv\PYGZus{}ta\PYGZus{}set\PYGZus{}text}\PYG{p}{(}\PYG{n}{ta}\PYG{p}{,} \PYG{l+s}{\PYGZdq{}}\PYG{l+s}{\PYGZdq{}}\PYG{p}{)}\PYG{p}{;}

\PYG{c+cm}{/*Assign the text area to the keyboard*/}
\PYG{n}{lv\PYGZus{}kb\PYGZus{}set\PYGZus{}ta}\PYG{p}{(}\PYG{n}{kb}\PYG{p}{,} \PYG{n}{ta}\PYG{p}{)}\PYG{p}{;}
\end{sphinxVerbatim}


\paragraph{MicroPython}
\label{\detokenize{object-types/kb:micropython}}
No examples yet.


\subsubsection{API}
\label{\detokenize{object-types/kb:api}}

\subsection{Label (lv\_label)}
\label{\detokenize{object-types/label:label-lv-label}}\label{\detokenize{object-types/label::doc}}

\subsubsection{Overview}
\label{\detokenize{object-types/label:overview}}
The Labels are the basic objects to \sphinxstylestrong{display text}. There is no limitation in the text size because it’s stored dynamically. You can modify the text in runtime at any time with lv\_label\_set\_text().

You can use \sphinxcode{\sphinxupquote{\textbackslash{}n}} to make line break. For example: \sphinxcode{\sphinxupquote{"line1\textbackslash{}nline2\textbackslash{}n\textbackslash{}nline4"}}

The size of the label object can be automatically expanded to the text size or the text can be manipulated according to several \sphinxstylestrong{long mode policies}:
\begin{itemize}
\item {} 
LV\_LABEL\_LONG\_EXPAND: Expand the object size to the text size (Default)

\item {} 
LV\_LABEL\_LONG\_BREAK: Keep the object width, break (wrap) the too long lines and expand the object height

\item {} 
LV\_LABEL\_LONG\_DOTS: Keep the object size, break the text and write dots in the last line

\item {} 
LV\_LABEL\_LONG\_SCROLL: Expand the object size and scroll the text on the parent (move the label object)

\item {} 
LV\_LABEL\_LONG\_ROLL: Keep the size and roll just the text (not the object)

\end{itemize}

You can specify the long mode with: \sphinxcode{\sphinxupquote{lv\_label\_set\_long\_mode(label, long\_mode)}}

It’s important to note that if you change the \sphinxcode{\sphinxupquote{LONG\_MODE}} the size of the label obejct is already expanded to the text’s size.
So you need to set the label’s size with \sphinxcode{\sphinxupquote{lv\_obj\_set\_size()}} or \sphinxcode{\sphinxupquote{lv\_obj\_set\_width()}} after changing long mode.

Labels are able to show text from a \sphinxstylestrong{static array}. Use: \sphinxcode{\sphinxupquote{lv\_label\_set\_static\_text(label, char\_array)}}. In this case, the text is not stored in the dynamic memory but the given array is used instead. Keep in my the array can’t be a local variable which destroys when the function exits.

You can also use a \sphinxstylestrong{raw character array} as label text. The array doesn’t have to be \sphinxcode{\sphinxupquote{\textbackslash{}0}} terminated. In this case, the text will be saved to the dynamic memory. To set a raw character array use the \sphinxcode{\sphinxupquote{lv\_label\_set\_array\_text(label, char\_array)}} function.

The label’s \sphinxstylestrong{text can be aligned} to the left, right or middle with \sphinxcode{\sphinxupquote{lv\_label\_set\_align(label, LV\_LABEL\_ALIGN\_LEFT/RIGHT/CENTER)}}

You can enable to \sphinxstylestrong{draw a background} for the label with \sphinxcode{\sphinxupquote{lv\_label\_set\_body\_draw(label, draw)}}

In the text, you can use commands to \sphinxstylestrong{re-color parts of the text}. For example: \sphinxcode{\sphinxupquote{"Write a \#ff0000 red\# word"}}. This feature can be enabled individually for each label by \sphinxcode{\sphinxupquote{lv\_label\_set\_recolor()}} function.

The labels can display symbols besides letters. Learn more about symbols \sphinxhref{/Fonts}{here}.

The labels’ \sphinxstylestrong{default style} is \sphinxcode{\sphinxupquote{NULL}} so they inherit the parent’s style.


\subsubsection{Style usage}
\label{\detokenize{object-types/label:style-usage}}\begin{itemize}
\item {} 
Use all properties from \sphinxcode{\sphinxupquote{style.text}}

\item {} 
For background drawing \sphinxcode{\sphinxupquote{style.body properties}} are used

\end{itemize}


\subsubsection{Notes}
\label{\detokenize{object-types/label:notes}}
The label’s \sphinxstylestrong{click enable attribute is disabled} by default. You can enable clicking with \sphinxcode{\sphinxupquote{lv\_obj\_set\_click(label, true)}}


\subsubsection{Example}
\label{\detokenize{object-types/label:example}}
\sphinxincludegraphics{{label-lv_label}.png}Example of Label in LittlevGL Graphics Library 

\begin{sphinxVerbatim}[commandchars=\\\{\}]
\PYG{c+cm}{/*Create label on the screen. By default it will inherit the style of the screen*/}
\PYG{n}{lv\PYGZus{}obj\PYGZus{}t} \PYG{o}{*} \PYG{n}{title} \PYG{o}{=} \PYG{n}{lv\PYGZus{}label\PYGZus{}create}\PYG{p}{(}\PYG{n}{lv\PYGZus{}scr\PYGZus{}act}\PYG{p}{(}\PYG{p}{)}\PYG{p}{,} \PYG{n+nb}{NULL}\PYG{p}{)}\PYG{p}{;}
\PYG{n}{lv\PYGZus{}label\PYGZus{}set\PYGZus{}text}\PYG{p}{(}\PYG{n}{title}\PYG{p}{,} \PYG{l+s}{\PYGZdq{}}\PYG{l+s}{Title Label}\PYG{l+s}{\PYGZdq{}}\PYG{p}{)}\PYG{p}{;}
\PYG{n}{lv\PYGZus{}obj\PYGZus{}align}\PYG{p}{(}\PYG{n}{title}\PYG{p}{,} \PYG{n+nb}{NULL}\PYG{p}{,} \PYG{n}{LV\PYGZus{}ALIGN\PYGZus{}IN\PYGZus{}TOP\PYGZus{}MID}\PYG{p}{,} \PYG{l+m+mi}{0}\PYG{p}{,} \PYG{l+m+mi}{20}\PYG{p}{)}\PYG{p}{;}  \PYG{c+cm}{/*Align to the top*/}

\PYG{c+cm}{/*Create anew style*/}
\PYG{k}{static} \PYG{n}{lv\PYGZus{}style\PYGZus{}t} \PYG{n}{style\PYGZus{}txt}\PYG{p}{;}
\PYG{n}{lv\PYGZus{}style\PYGZus{}copy}\PYG{p}{(}\PYG{o}{\PYGZam{}}\PYG{n}{style\PYGZus{}txt}\PYG{p}{,} \PYG{o}{\PYGZam{}}\PYG{n}{lv\PYGZus{}style\PYGZus{}plain}\PYG{p}{)}\PYG{p}{;}
\PYG{n}{style\PYGZus{}txt}\PYG{p}{.}\PYG{n}{text}\PYG{p}{.}\PYG{n}{font} \PYG{o}{=} \PYG{o}{\PYGZam{}}\PYG{n}{lv\PYGZus{}font\PYGZus{}dejavu\PYGZus{}40}\PYG{p}{;}
\PYG{n}{style\PYGZus{}txt}\PYG{p}{.}\PYG{n}{text}\PYG{p}{.}\PYG{n}{letter\PYGZus{}space} \PYG{o}{=} \PYG{l+m+mi}{2}\PYG{p}{;}
\PYG{n}{style\PYGZus{}txt}\PYG{p}{.}\PYG{n}{text}\PYG{p}{.}\PYG{n}{line\PYGZus{}space} \PYG{o}{=} \PYG{l+m+mi}{1}\PYG{p}{;}
\PYG{n}{style\PYGZus{}txt}\PYG{p}{.}\PYG{n}{text}\PYG{p}{.}\PYG{n}{color} \PYG{o}{=} \PYG{n}{LV\PYGZus{}COLOR\PYGZus{}HEX}\PYG{p}{(}\PYG{l+m+mh}{0x606060}\PYG{p}{)}\PYG{p}{;}

\PYG{c+cm}{/*Create a new label*/}
\PYG{n}{lv\PYGZus{}obj\PYGZus{}t} \PYG{o}{*} \PYG{n}{txt} \PYG{o}{=} \PYG{n}{lv\PYGZus{}label\PYGZus{}create}\PYG{p}{(}\PYG{n}{lv\PYGZus{}scr\PYGZus{}act}\PYG{p}{(}\PYG{p}{)}\PYG{p}{,} \PYG{n+nb}{NULL}\PYG{p}{)}\PYG{p}{;}
\PYG{n}{lv\PYGZus{}obj\PYGZus{}set\PYGZus{}style}\PYG{p}{(}\PYG{n}{txt}\PYG{p}{,} \PYG{o}{\PYGZam{}}\PYG{n}{style\PYGZus{}txt}\PYG{p}{)}\PYG{p}{;}                    \PYG{c+cm}{/*Set the created style*/}
\PYG{n}{lv\PYGZus{}label\PYGZus{}set\PYGZus{}long\PYGZus{}mode}\PYG{p}{(}\PYG{n}{txt}\PYG{p}{,} \PYG{n}{LV\PYGZus{}LABEL\PYGZus{}LONG\PYGZus{}BREAK}\PYG{p}{)}\PYG{p}{;}     \PYG{c+cm}{/*Break the long lines*/}
\PYG{n}{lv\PYGZus{}label\PYGZus{}set\PYGZus{}recolor}\PYG{p}{(}\PYG{n}{txt}\PYG{p}{,} \PYG{n+nb}{true}\PYG{p}{)}\PYG{p}{;}                      \PYG{c+cm}{/*Enable re\PYGZhy{}coloring by commands in the text*/}
\PYG{n}{lv\PYGZus{}label\PYGZus{}set\PYGZus{}align}\PYG{p}{(}\PYG{n}{txt}\PYG{p}{,} \PYG{n}{LV\PYGZus{}LABEL\PYGZus{}ALIGN\PYGZus{}CENTER}\PYG{p}{)}\PYG{p}{;}       \PYG{c+cm}{/*Center aligned lines*/}
\PYG{n}{lv\PYGZus{}label\PYGZus{}set\PYGZus{}text}\PYG{p}{(}\PYG{n}{txt}\PYG{p}{,} \PYG{l+s}{\PYGZdq{}}\PYG{l+s}{Align lines to the middle}\PYG{l+s+se}{\PYGZbs{}n}\PYG{l+s+se}{\PYGZbs{}n}\PYG{l+s}{\PYGZdq{}}
                       \PYG{l+s}{\PYGZdq{}}\PYG{l+s}{\PYGZsh{}000080 Re\PYGZhy{}color\PYGZsh{} \PYGZsh{}0000ff words of\PYGZsh{} \PYGZsh{}6666ff the text\PYGZsh{}}\PYG{l+s+se}{\PYGZbs{}n}\PYG{l+s+se}{\PYGZbs{}n}\PYG{l+s}{\PYGZdq{}}
                       \PYG{l+s}{\PYGZdq{}}\PYG{l+s}{If a line become too long it can be automatically broken into multiple lines}\PYG{l+s}{\PYGZdq{}}\PYG{p}{)}\PYG{p}{;}
\PYG{n}{lv\PYGZus{}obj\PYGZus{}set\PYGZus{}width}\PYG{p}{(}\PYG{n}{txt}\PYG{p}{,} \PYG{l+m+mi}{300}\PYG{p}{)}\PYG{p}{;}                           \PYG{c+cm}{/*Set a width*/}
\PYG{n}{lv\PYGZus{}obj\PYGZus{}align}\PYG{p}{(}\PYG{n}{txt}\PYG{p}{,} \PYG{n+nb}{NULL}\PYG{p}{,} \PYG{n}{LV\PYGZus{}ALIGN\PYGZus{}CENTER}\PYG{p}{,} \PYG{l+m+mi}{0}\PYG{p}{,} \PYG{l+m+mi}{20}\PYG{p}{)}\PYG{p}{;}      \PYG{c+cm}{/*Align to center*/}
\end{sphinxVerbatim}


\subsection{LED (lv\_led)}
\label{\detokenize{object-types/led:led-lv-led}}\label{\detokenize{object-types/led::doc}}

\subsubsection{Overview}
\label{\detokenize{object-types/led:overview}}
The LEDs are rectangle-like (or circle) object. You can set their \sphinxstylestrong{brightness} with \sphinxcode{\sphinxupquote{lv\_led\_set\_bright(led, bright)}}. The brightness should be between 0 (darkest) and 255 (lightest).

Use \sphinxcode{\sphinxupquote{lv\_led\_on(led)}} and \sphinxcode{\sphinxupquote{lv\_led\_off(led)}} to set the brightness to a predefined  ON or OFF value.  The \sphinxcode{\sphinxupquote{lv\_led\_toggle(led)}} toggles between the ON and OFF state.


\subsubsection{Style usage}
\label{\detokenize{object-types/led:style-usage}}
The LED uses one style which can be set by \sphinxcode{\sphinxupquote{lv\_led\_set\_style(led, \&style)}}. To determine the appearance the \sphinxstylestrong{style.body} properties are used. The colors are darkened and shadow width is reduced at a lower brightness and gains its original value at brightness 255 to show a lighting effect.  The default style is: \sphinxcode{\sphinxupquote{lv\_style\_pretty\_color}}.


\subsubsection{Notes}
\label{\detokenize{object-types/led:notes}}\begin{itemize}
\item {} 
Typically the default style is not suitable therefore you have to create you own style. See the Examples.

\end{itemize}


\subsubsection{Example}
\label{\detokenize{object-types/led:example}}
\sphinxincludegraphics{{led-lv_led}.png}LED image

\begin{sphinxVerbatim}[commandchars=\\\{\}]
\PYG{c+cm}{/*Create a style for the LED*/}
\PYG{k}{static} \PYG{n}{lv\PYGZus{}style\PYGZus{}t} \PYG{n}{style\PYGZus{}led}\PYG{p}{;}
\PYG{n}{lv\PYGZus{}style\PYGZus{}copy}\PYG{p}{(}\PYG{o}{\PYGZam{}}\PYG{n}{style\PYGZus{}led}\PYG{p}{,} \PYG{o}{\PYGZam{}}\PYG{n}{lv\PYGZus{}style\PYGZus{}pretty\PYGZus{}color}\PYG{p}{)}\PYG{p}{;}
\PYG{n}{style\PYGZus{}led}\PYG{p}{.}\PYG{n}{body}\PYG{p}{.}\PYG{n}{radius} \PYG{o}{=} \PYG{n}{LV\PYGZus{}RADIUS\PYGZus{}CIRCLE}\PYG{p}{;}
\PYG{n}{style\PYGZus{}led}\PYG{p}{.}\PYG{n}{body}\PYG{p}{.}\PYG{n}{main\PYGZus{}color} \PYG{o}{=} \PYG{n}{LV\PYGZus{}COLOR\PYGZus{}MAKE}\PYG{p}{(}\PYG{l+m+mh}{0xb5}\PYG{p}{,} \PYG{l+m+mh}{0x0f}\PYG{p}{,} \PYG{l+m+mh}{0x04}\PYG{p}{)}\PYG{p}{;}
\PYG{n}{style\PYGZus{}led}\PYG{p}{.}\PYG{n}{body}\PYG{p}{.}\PYG{n}{grad\PYGZus{}color} \PYG{o}{=} \PYG{n}{LV\PYGZus{}COLOR\PYGZus{}MAKE}\PYG{p}{(}\PYG{l+m+mh}{0x50}\PYG{p}{,} \PYG{l+m+mh}{0x07}\PYG{p}{,} \PYG{l+m+mh}{0x02}\PYG{p}{)}\PYG{p}{;}
\PYG{n}{style\PYGZus{}led}\PYG{p}{.}\PYG{n}{body}\PYG{p}{.}\PYG{n}{border}\PYG{p}{.}\PYG{n}{color} \PYG{o}{=} \PYG{n}{LV\PYGZus{}COLOR\PYGZus{}MAKE}\PYG{p}{(}\PYG{l+m+mh}{0xfa}\PYG{p}{,} \PYG{l+m+mh}{0x0f}\PYG{p}{,} \PYG{l+m+mh}{0x00}\PYG{p}{)}\PYG{p}{;}
\PYG{n}{style\PYGZus{}led}\PYG{p}{.}\PYG{n}{body}\PYG{p}{.}\PYG{n}{border}\PYG{p}{.}\PYG{n}{width} \PYG{o}{=} \PYG{l+m+mi}{3}\PYG{p}{;}
\PYG{n}{style\PYGZus{}led}\PYG{p}{.}\PYG{n}{body}\PYG{p}{.}\PYG{n}{border}\PYG{p}{.}\PYG{n}{opa} \PYG{o}{=} \PYG{n}{LV\PYGZus{}OPA\PYGZus{}30}\PYG{p}{;}
\PYG{n}{style\PYGZus{}led}\PYG{p}{.}\PYG{n}{body}\PYG{p}{.}\PYG{n}{shadow}\PYG{p}{.}\PYG{n}{color} \PYG{o}{=} \PYG{n}{LV\PYGZus{}COLOR\PYGZus{}MAKE}\PYG{p}{(}\PYG{l+m+mh}{0xb5}\PYG{p}{,} \PYG{l+m+mh}{0x0f}\PYG{p}{,} \PYG{l+m+mh}{0x04}\PYG{p}{)}\PYG{p}{;}
\PYG{n}{style\PYGZus{}led}\PYG{p}{.}\PYG{n}{body}\PYG{p}{.}\PYG{n}{shadow}\PYG{p}{.}\PYG{n}{width} \PYG{o}{=} \PYG{l+m+mi}{10}\PYG{p}{;}


\PYG{c+cm}{/*Create a LED and switch it ON*/}
\PYG{n}{lv\PYGZus{}obj\PYGZus{}t} \PYG{o}{*} \PYG{n}{led1}  \PYG{o}{=} \PYG{n}{lv\PYGZus{}led\PYGZus{}create}\PYG{p}{(}\PYG{n}{lv\PYGZus{}scr\PYGZus{}act}\PYG{p}{(}\PYG{p}{)}\PYG{p}{,} \PYG{n+nb}{NULL}\PYG{p}{)}\PYG{p}{;}
\PYG{n}{lv\PYGZus{}obj\PYGZus{}set\PYGZus{}style}\PYG{p}{(}\PYG{n}{led1}\PYG{p}{,} \PYG{o}{\PYGZam{}}\PYG{n}{style\PYGZus{}led}\PYG{p}{)}\PYG{p}{;}
\PYG{n}{lv\PYGZus{}obj\PYGZus{}align}\PYG{p}{(}\PYG{n}{led1}\PYG{p}{,} \PYG{n+nb}{NULL}\PYG{p}{,} \PYG{n}{LV\PYGZus{}ALIGN\PYGZus{}IN\PYGZus{}TOP\PYGZus{}MID}\PYG{p}{,} \PYG{l+m+mi}{40}\PYG{p}{,} \PYG{l+m+mi}{40}\PYG{p}{)}\PYG{p}{;}
\PYG{n}{lv\PYGZus{}led\PYGZus{}on}\PYG{p}{(}\PYG{n}{led1}\PYG{p}{)}\PYG{p}{;}

\PYG{c+cm}{/*Copy the previous LED and set a brightness*/}
\PYG{n}{lv\PYGZus{}obj\PYGZus{}t} \PYG{o}{*} \PYG{n}{led2}  \PYG{o}{=} \PYG{n}{lv\PYGZus{}led\PYGZus{}create}\PYG{p}{(}\PYG{n}{lv\PYGZus{}scr\PYGZus{}act}\PYG{p}{(}\PYG{p}{)}\PYG{p}{,} \PYG{n}{led1}\PYG{p}{)}\PYG{p}{;}
\PYG{n}{lv\PYGZus{}obj\PYGZus{}align}\PYG{p}{(}\PYG{n}{led2}\PYG{p}{,} \PYG{n}{led1}\PYG{p}{,} \PYG{n}{LV\PYGZus{}ALIGN\PYGZus{}OUT\PYGZus{}BOTTOM\PYGZus{}MID}\PYG{p}{,} \PYG{l+m+mi}{0}\PYG{p}{,} \PYG{l+m+mi}{40}\PYG{p}{)}\PYG{p}{;}
\PYG{n}{lv\PYGZus{}led\PYGZus{}set\PYGZus{}bright}\PYG{p}{(}\PYG{n}{led2}\PYG{p}{,} \PYG{l+m+mi}{190}\PYG{p}{)}\PYG{p}{;}

\PYG{c+cm}{/*Copy the previous LED and switch it OFF*/}
\PYG{n}{lv\PYGZus{}obj\PYGZus{}t} \PYG{o}{*} \PYG{n}{led3}  \PYG{o}{=} \PYG{n}{lv\PYGZus{}led\PYGZus{}create}\PYG{p}{(}\PYG{n}{lv\PYGZus{}scr\PYGZus{}act}\PYG{p}{(}\PYG{p}{)}\PYG{p}{,} \PYG{n}{led1}\PYG{p}{)}\PYG{p}{;}
\PYG{n}{lv\PYGZus{}obj\PYGZus{}align}\PYG{p}{(}\PYG{n}{led3}\PYG{p}{,} \PYG{n}{led2}\PYG{p}{,} \PYG{n}{LV\PYGZus{}ALIGN\PYGZus{}OUT\PYGZus{}BOTTOM\PYGZus{}MID}\PYG{p}{,} \PYG{l+m+mi}{0}\PYG{p}{,} \PYG{l+m+mi}{40}\PYG{p}{)}\PYG{p}{;}
\PYG{n}{lv\PYGZus{}led\PYGZus{}off}\PYG{p}{(}\PYG{n}{led3}\PYG{p}{)}\PYG{p}{;}

\PYG{c+cm}{/*Create 3 labels next to the LEDs*/}
\PYG{n}{lv\PYGZus{}obj\PYGZus{}t} \PYG{o}{*} \PYG{n}{label} \PYG{o}{=} \PYG{n}{lv\PYGZus{}label\PYGZus{}create}\PYG{p}{(}\PYG{n}{lv\PYGZus{}scr\PYGZus{}act}\PYG{p}{(}\PYG{p}{)}\PYG{p}{,} \PYG{n+nb}{NULL}\PYG{p}{)}\PYG{p}{;}
\PYG{n}{lv\PYGZus{}label\PYGZus{}set\PYGZus{}text}\PYG{p}{(}\PYG{n}{label}\PYG{p}{,} \PYG{l+s}{\PYGZdq{}}\PYG{l+s}{LED On}\PYG{l+s}{\PYGZdq{}}\PYG{p}{)}\PYG{p}{;}
\PYG{n}{lv\PYGZus{}obj\PYGZus{}align}\PYG{p}{(}\PYG{n}{label}\PYG{p}{,} \PYG{n}{led1}\PYG{p}{,} \PYG{n}{LV\PYGZus{}ALIGN\PYGZus{}OUT\PYGZus{}LEFT\PYGZus{}MID}\PYG{p}{,} \PYG{o}{\PYGZhy{}}\PYG{l+m+mi}{40}\PYG{p}{,} \PYG{l+m+mi}{0}\PYG{p}{)}\PYG{p}{;}

\PYG{n}{label} \PYG{o}{=} \PYG{n}{lv\PYGZus{}label\PYGZus{}create}\PYG{p}{(}\PYG{n}{lv\PYGZus{}scr\PYGZus{}act}\PYG{p}{(}\PYG{p}{)}\PYG{p}{,} \PYG{n+nb}{NULL}\PYG{p}{)}\PYG{p}{;}
\PYG{n}{lv\PYGZus{}label\PYGZus{}set\PYGZus{}text}\PYG{p}{(}\PYG{n}{label}\PYG{p}{,} \PYG{l+s}{\PYGZdq{}}\PYG{l+s}{LED half}\PYG{l+s}{\PYGZdq{}}\PYG{p}{)}\PYG{p}{;}
\PYG{n}{lv\PYGZus{}obj\PYGZus{}align}\PYG{p}{(}\PYG{n}{label}\PYG{p}{,} \PYG{n}{led2}\PYG{p}{,} \PYG{n}{LV\PYGZus{}ALIGN\PYGZus{}OUT\PYGZus{}LEFT\PYGZus{}MID}\PYG{p}{,} \PYG{o}{\PYGZhy{}}\PYG{l+m+mi}{40}\PYG{p}{,} \PYG{l+m+mi}{0}\PYG{p}{)}\PYG{p}{;}

\PYG{n}{label} \PYG{o}{=} \PYG{n}{lv\PYGZus{}label\PYGZus{}create}\PYG{p}{(}\PYG{n}{lv\PYGZus{}scr\PYGZus{}act}\PYG{p}{(}\PYG{p}{)}\PYG{p}{,} \PYG{n+nb}{NULL}\PYG{p}{)}\PYG{p}{;}
\PYG{n}{lv\PYGZus{}label\PYGZus{}set\PYGZus{}text}\PYG{p}{(}\PYG{n}{label}\PYG{p}{,} \PYG{l+s}{\PYGZdq{}}\PYG{l+s}{LED Off}\PYG{l+s}{\PYGZdq{}}\PYG{p}{)}\PYG{p}{;}
\PYG{n}{lv\PYGZus{}obj\PYGZus{}align}\PYG{p}{(}\PYG{n}{label}\PYG{p}{,} \PYG{n}{led3}\PYG{p}{,} \PYG{n}{LV\PYGZus{}ALIGN\PYGZus{}OUT\PYGZus{}LEFT\PYGZus{}MID}\PYG{p}{,} \PYG{o}{\PYGZhy{}}\PYG{l+m+mi}{40}\PYG{p}{,} \PYG{l+m+mi}{0}\PYG{p}{)}\PYG{p}{;}
\end{sphinxVerbatim}


\subsection{Line (lv\_line)}
\label{\detokenize{object-types/line:line-lv-line}}\label{\detokenize{object-types/line::doc}}

\subsubsection{Overview}
\label{\detokenize{object-types/line:overview}}
The line object is capable of \sphinxstylestrong{drawing straight lines} between a set of points. The points has to be stored in an \sphinxcode{\sphinxupquote{lv\_point\_t}} array and passed to the object by the \sphinxcode{\sphinxupquote{lv\_line\_set\_points(lines, point\_array, point\_num)}} function.

It is possible to \sphinxstylestrong{automatically set the size} of the line object according to its points. You can enable it with the \sphinxcode{\sphinxupquote{lv\_line\_set\_auto\_size(line, true)}} function. If enabled then when the points are set then the object width and height will be changed according to the max. x and max. y coordinates among the points. The \sphinxstyleemphasis{auto size} is enabled by default.

Basically the \sphinxstyleemphasis{y == 0} point is in the top of the object but you can \sphinxstylestrong{invert the y coordinates} with \sphinxcode{\sphinxupquote{lv\_line\_set\_y\_invert(line, true)}}. After it the y coordinates will be subtracted from object’s height.


\subsubsection{Style usage}
\label{\detokenize{object-types/line:style-usage}}\begin{itemize}
\item {} 
\sphinxstylestrong{style.line} properties are used

\end{itemize}


\subsubsection{Notes}
\label{\detokenize{object-types/line:notes}}

\subsubsection{Example}
\label{\detokenize{object-types/line:example}}
\sphinxincludegraphics{{line-lv_line}.png}Example of Line in LittlevGL Graphics Library 

\begin{sphinxVerbatim}[commandchars=\\\{\}]
\PYG{c+cm}{/*Create an array for the points of the line*/}
\PYG{k}{static} \PYG{n}{lv\PYGZus{}point\PYGZus{}t} \PYG{n}{line\PYGZus{}points}\PYG{p}{[}\PYG{p}{]} \PYG{o}{=} \PYG{p}{\PYGZob{}} \PYG{p}{\PYGZob{}}\PYG{l+m+mi}{5}\PYG{p}{,} \PYG{l+m+mi}{5}\PYG{p}{\PYGZcb{}}\PYG{p}{,} \PYG{p}{\PYGZob{}}\PYG{l+m+mi}{70}\PYG{p}{,} \PYG{l+m+mi}{70}\PYG{p}{\PYGZcb{}}\PYG{p}{,} \PYG{p}{\PYGZob{}}\PYG{l+m+mi}{120}\PYG{p}{,} \PYG{l+m+mi}{10}\PYG{p}{\PYGZcb{}}\PYG{p}{,} \PYG{p}{\PYGZob{}}\PYG{l+m+mi}{180}\PYG{p}{,} \PYG{l+m+mi}{60}\PYG{p}{\PYGZcb{}}\PYG{p}{,} \PYG{p}{\PYGZob{}}\PYG{l+m+mi}{240}\PYG{p}{,} \PYG{l+m+mi}{10}\PYG{p}{\PYGZcb{}} \PYG{p}{\PYGZcb{}}\PYG{p}{;}

\PYG{c+cm}{/*Create line with default style*/}
\PYG{n}{lv\PYGZus{}obj\PYGZus{}t} \PYG{o}{*} \PYG{n}{line1}\PYG{p}{;}
\PYG{n}{line1} \PYG{o}{=} \PYG{n}{lv\PYGZus{}line\PYGZus{}create}\PYG{p}{(}\PYG{n}{lv\PYGZus{}scr\PYGZus{}act}\PYG{p}{(}\PYG{p}{)}\PYG{p}{,} \PYG{n+nb}{NULL}\PYG{p}{)}\PYG{p}{;}
\PYG{n}{lv\PYGZus{}line\PYGZus{}set\PYGZus{}points}\PYG{p}{(}\PYG{n}{line1}\PYG{p}{,} \PYG{n}{line\PYGZus{}points}\PYG{p}{,} \PYG{l+m+mi}{5}\PYG{p}{)}\PYG{p}{;}     \PYG{c+cm}{/*Set the points*/}
\PYG{n}{lv\PYGZus{}obj\PYGZus{}align}\PYG{p}{(}\PYG{n}{line1}\PYG{p}{,} \PYG{n+nb}{NULL}\PYG{p}{,} \PYG{n}{LV\PYGZus{}ALIGN\PYGZus{}IN\PYGZus{}TOP\PYGZus{}MID}\PYG{p}{,} \PYG{l+m+mi}{0}\PYG{p}{,} \PYG{l+m+mi}{20}\PYG{p}{)}\PYG{p}{;}

\PYG{c+cm}{/*Create new style (thin light blue)*/}
\PYG{k}{static} \PYG{n}{lv\PYGZus{}style\PYGZus{}t} \PYG{n}{style\PYGZus{}line2}\PYG{p}{;}
\PYG{n}{lv\PYGZus{}style\PYGZus{}copy}\PYG{p}{(}\PYG{o}{\PYGZam{}}\PYG{n}{style\PYGZus{}line2}\PYG{p}{,} \PYG{o}{\PYGZam{}}\PYG{n}{lv\PYGZus{}style\PYGZus{}plain}\PYG{p}{)}\PYG{p}{;}
\PYG{n}{style\PYGZus{}line2}\PYG{p}{.}\PYG{n}{line}\PYG{p}{.}\PYG{n}{color} \PYG{o}{=} \PYG{n}{LV\PYGZus{}COLOR\PYGZus{}MAKE}\PYG{p}{(}\PYG{l+m+mh}{0x2e}\PYG{p}{,} \PYG{l+m+mh}{0x96}\PYG{p}{,} \PYG{l+m+mh}{0xff}\PYG{p}{)}\PYG{p}{;}
\PYG{n}{style\PYGZus{}line2}\PYG{p}{.}\PYG{n}{line}\PYG{p}{.}\PYG{n}{width} \PYG{o}{=} \PYG{l+m+mi}{2}\PYG{p}{;}

\PYG{c+cm}{/*Copy the previous line and apply the new style*/}
\PYG{n}{lv\PYGZus{}obj\PYGZus{}t} \PYG{o}{*} \PYG{n}{line2} \PYG{o}{=} \PYG{n}{lv\PYGZus{}line\PYGZus{}create}\PYG{p}{(}\PYG{n}{lv\PYGZus{}scr\PYGZus{}act}\PYG{p}{(}\PYG{p}{)}\PYG{p}{,} \PYG{n}{line1}\PYG{p}{)}\PYG{p}{;}
\PYG{n}{lv\PYGZus{}line\PYGZus{}set\PYGZus{}style}\PYG{p}{(}\PYG{n}{line2}\PYG{p}{,} \PYG{o}{\PYGZam{}}\PYG{n}{style\PYGZus{}line2}\PYG{p}{)}\PYG{p}{;}
\PYG{n}{lv\PYGZus{}obj\PYGZus{}align}\PYG{p}{(}\PYG{n}{line2}\PYG{p}{,} \PYG{n}{line1}\PYG{p}{,} \PYG{n}{LV\PYGZus{}ALIGN\PYGZus{}OUT\PYGZus{}BOTTOM\PYGZus{}MID}\PYG{p}{,} \PYG{l+m+mi}{0}\PYG{p}{,} \PYG{o}{\PYGZhy{}}\PYG{l+m+mi}{20}\PYG{p}{)}\PYG{p}{;}

\PYG{c+cm}{/*Create new style (thick dark blue)*/}
\PYG{k}{static} \PYG{n}{lv\PYGZus{}style\PYGZus{}t} \PYG{n}{style\PYGZus{}line3}\PYG{p}{;}
\PYG{n}{lv\PYGZus{}style\PYGZus{}copy}\PYG{p}{(}\PYG{o}{\PYGZam{}}\PYG{n}{style\PYGZus{}line3}\PYG{p}{,} \PYG{o}{\PYGZam{}}\PYG{n}{lv\PYGZus{}style\PYGZus{}plain}\PYG{p}{)}\PYG{p}{;}
\PYG{n}{style\PYGZus{}line3}\PYG{p}{.}\PYG{n}{line}\PYG{p}{.}\PYG{n}{color} \PYG{o}{=} \PYG{n}{LV\PYGZus{}COLOR\PYGZus{}MAKE}\PYG{p}{(}\PYG{l+m+mh}{0x00}\PYG{p}{,} \PYG{l+m+mh}{0x3b}\PYG{p}{,} \PYG{l+m+mh}{0x75}\PYG{p}{)}\PYG{p}{;}
\PYG{n}{style\PYGZus{}line3}\PYG{p}{.}\PYG{n}{line}\PYG{p}{.}\PYG{n}{width} \PYG{o}{=} \PYG{l+m+mi}{5}\PYG{p}{;}

\PYG{c+cm}{/*Copy the previous line and apply the new style*/}
\PYG{n}{lv\PYGZus{}obj\PYGZus{}t} \PYG{o}{*} \PYG{n}{line3} \PYG{o}{=} \PYG{n}{lv\PYGZus{}line\PYGZus{}create}\PYG{p}{(}\PYG{n}{lv\PYGZus{}scr\PYGZus{}act}\PYG{p}{(}\PYG{p}{)}\PYG{p}{,} \PYG{n}{line1}\PYG{p}{)}\PYG{p}{;}
\PYG{n}{lv\PYGZus{}line\PYGZus{}set\PYGZus{}style}\PYG{p}{(}\PYG{n}{line3}\PYG{p}{,} \PYG{o}{\PYGZam{}}\PYG{n}{style\PYGZus{}line3}\PYG{p}{)}\PYG{p}{;}
\PYG{n}{lv\PYGZus{}obj\PYGZus{}align}\PYG{p}{(}\PYG{n}{line3}\PYG{p}{,} \PYG{n}{line2}\PYG{p}{,} \PYG{n}{LV\PYGZus{}ALIGN\PYGZus{}OUT\PYGZus{}BOTTOM\PYGZus{}MID}\PYG{p}{,} \PYG{l+m+mi}{0}\PYG{p}{,} \PYG{o}{\PYGZhy{}}\PYG{l+m+mi}{20}\PYG{p}{)}\PYG{p}{;}
\end{sphinxVerbatim}


\subsection{List (lv\_list)}
\label{\detokenize{object-types/list:list-lv-list}}\label{\detokenize{object-types/list::doc}}

\subsubsection{Overview}
\label{\detokenize{object-types/list:overview}}
The Lists are built from a background \sphinxstylestrong{Page} and \sphinxstylestrong{Buttons} on it. The Buttons contain an optional icon-like Image (which can be a symbol too) and a Label. When the list become long enough it can be scrolled. The \sphinxstylestrong{width of the buttons} is set to maximum according to the object width. The \sphinxstylestrong{height} of the buttons are adjusted automatically according to the content (content height + style.body.padding.ver).

You can \sphinxstylestrong{add new list element} with \sphinxcode{\sphinxupquote{lv\_list\_add(list, "U:/img", "Text", rel\_action)}} or with symbol icon \sphinxcode{\sphinxupquote{lv\_list\_add(list, SYMBOL\_EDIT, "Edit text")}}. If you do no want to add image use \sphinxcode{\sphinxupquote{NULL}} as file name. The function returns with a pointer to the created button to allow further configurations.

You can use \sphinxcode{\sphinxupquote{lv\_list\_get\_btn\_label(list\_btn)}} and \sphinxcode{\sphinxupquote{lv\_list\_get\_btn\_img(list\_btn)}} to \sphinxstylestrong{get the label and the image} of a list button.

In the release action of a button you can get the the \sphinxstylestrong{button’s text} with \sphinxcode{\sphinxupquote{lv\_list\_get\_btn\_text(button)}}. It helps to identify the released list element.

To \sphinxstylestrong{delete a list element} just use \sphinxcode{\sphinxupquote{lv\_obj\_del()}} on the return value of \sphinxcode{\sphinxupquote{lv\_list\_add()}}.

You can \sphinxstylestrong{navigate manually} in the list with \sphinxcode{\sphinxupquote{lv\_list\_up(list)}} and \sphinxcode{\sphinxupquote{lv\_list\_down(list)}}.

You can focus on a button directly using \sphinxcode{\sphinxupquote{lv\_list\_focus(btn, anim\_en)}}.

The \sphinxstylestrong{animation time} of up/down/focus movements can be set via: \sphinxcode{\sphinxupquote{lv\_list\_set\_anim\_time(list, anim\_time)}}. Zero animation time means not animations.


\subsubsection{Style usage}
\label{\detokenize{object-types/list:style-usage}}
The \sphinxcode{\sphinxupquote{lv\_list\_set\_style(list, LV\_LIST\_STYLE\_..., \&style)}} function sets the style of a list. For details explanation of \sphinxstyleemphasis{BG}, \sphinxstyleemphasis{SCRL} and \sphinxstyleemphasis{SB} see \sphinxhref{/Page}{Page}
\begin{itemize}
\item {} 
\sphinxstylestrong{LV\_LIST\_STYLE\_BG} list background style. Default: \sphinxstyleemphasis{lv\_style\_transp\_fit}

\item {} 
\sphinxstylestrong{LV\_LIST\_STYLE\_SCRL} scrollable parts’s style. Default:\_ lv\_style\_pretty\_

\item {} 
\sphinxstylestrong{LV\_LIST\_STYLE\_SB} scrollbars’ style. Default: \sphinxstyleemphasis{lv\_style\_pretty\_color}

\item {} 
\sphinxstylestrong{LV\_LIST\_STYLE\_BTN\_REL} button released style. Default: \sphinxstyleemphasis{lv\_style\_btn\_rel}

\item {} 
\sphinxstylestrong{LV\_LIST\_STYLE\_BTN\_PR} button pressed style. Default: \sphinxstyleemphasis{lv\_style\_btn\_pr}

\item {} 
\sphinxstylestrong{LV\_LIST\_STYLE\_BTN\_TGL\_REL} button toggled released style. Default: \sphinxstyleemphasis{lv\_style\_btn\_tgl\_rel}

\item {} 
\sphinxstylestrong{LV\_LIST\_STYLE\_BTN\_TGL\_PR} button toggled pressed style. Default: \sphinxstyleemphasis{lv\_style\_btn\_tgl\_pr}

\item {} 
\sphinxstylestrong{LV\_LIST\_STYLE\_BTN\_INA} button inactive style. Default: \sphinxstyleemphasis{lv\_style\_btn\_ina}

\end{itemize}


\subsubsection{Notes}
\label{\detokenize{object-types/list:notes}}\begin{itemize}
\item {} 
You can set a transparent background for the list. In this case if you have only a few list buttons the the list will look shorter but become scrollable when more list elements are added.

\item {} 
The button labels default long mode is \sphinxcode{\sphinxupquote{LV\_LABEL\_LONG\_ROLL}}. You can modify it manually. Use \sphinxcode{\sphinxupquote{lv\_list\_get\_btn\_label()}} to get buttons’s label.

\item {} 
To \sphinxstylestrong{modify the height of the buttons} adjust the \sphinxstyleemphasis{body.padding.ver} field of the corresponding style (LV\_LIST\_STYLE\_BTN\_REL , LV\_LIST\_STYLE\_BTN\_PR etc.)

\end{itemize}


\subsubsection{Example}
\label{\detokenize{object-types/list:example}}
\sphinxincludegraphics{{list-lv_list}.png}List image

\begin{sphinxVerbatim}[commandchars=\\\{\}]
\PYG{c+cm}{/*Will be called on click of a button of a list*/}
\PYG{k}{static} \PYG{n}{lv\PYGZus{}res\PYGZus{}t} \PYG{n+nf}{list\PYGZus{}release\PYGZus{}action}\PYG{p}{(}\PYG{n}{lv\PYGZus{}obj\PYGZus{}t} \PYG{o}{*} \PYG{n}{list\PYGZus{}btn}\PYG{p}{)}
\PYG{p}{\PYGZob{}}
    \PYG{n}{printf}\PYG{p}{(}\PYG{l+s}{\PYGZdq{}}\PYG{l+s}{List element click:\PYGZpc{}s}\PYG{l+s+se}{\PYGZbs{}n}\PYG{l+s}{\PYGZdq{}}\PYG{p}{,} \PYG{n}{lv\PYGZus{}list\PYGZus{}get\PYGZus{}btn\PYGZus{}text}\PYG{p}{(}\PYG{n}{list\PYGZus{}btn}\PYG{p}{)}\PYG{p}{)}\PYG{p}{;}

    \PYG{k}{return} \PYG{n}{LV\PYGZus{}RES\PYGZus{}OK}\PYG{p}{;} \PYG{c+cm}{/*Return OK because the list is not deleted*/}
\PYG{p}{\PYGZcb{}}

\PYG{p}{.}
\PYG{p}{.}
\PYG{p}{.}

\PYG{c+cm}{/************************}
\PYG{c+cm}{ * Create a default list}
\PYG{c+cm}{ ************************/}

\PYG{c+cm}{/*Crate the list*/}
\PYG{n}{lv\PYGZus{}obj\PYGZus{}t} \PYG{o}{*} \PYG{n}{list1} \PYG{o}{=} \PYG{n}{lv\PYGZus{}list\PYGZus{}create}\PYG{p}{(}\PYG{n}{lv\PYGZus{}scr\PYGZus{}act}\PYG{p}{(}\PYG{p}{)}\PYG{p}{,} \PYG{n+nb}{NULL}\PYG{p}{)}\PYG{p}{;}
\PYG{n}{lv\PYGZus{}obj\PYGZus{}set\PYGZus{}size}\PYG{p}{(}\PYG{n}{list1}\PYG{p}{,} \PYG{l+m+mi}{130}\PYG{p}{,} \PYG{l+m+mi}{170}\PYG{p}{)}\PYG{p}{;}
\PYG{n}{lv\PYGZus{}obj\PYGZus{}align}\PYG{p}{(}\PYG{n}{list1}\PYG{p}{,} \PYG{n+nb}{NULL}\PYG{p}{,} \PYG{n}{LV\PYGZus{}ALIGN\PYGZus{}IN\PYGZus{}TOP\PYGZus{}LEFT}\PYG{p}{,} \PYG{l+m+mi}{20}\PYG{p}{,} \PYG{l+m+mi}{40}\PYG{p}{)}\PYG{p}{;}

\PYG{c+cm}{/*Add list elements*/}
\PYG{n}{lv\PYGZus{}list\PYGZus{}add}\PYG{p}{(}\PYG{n}{list1}\PYG{p}{,} \PYG{n}{SYMBOL\PYGZus{}FILE}\PYG{p}{,} \PYG{l+s}{\PYGZdq{}}\PYG{l+s}{New}\PYG{l+s}{\PYGZdq{}}\PYG{p}{,} \PYG{n}{list\PYGZus{}release\PYGZus{}action}\PYG{p}{)}\PYG{p}{;}
\PYG{n}{lv\PYGZus{}list\PYGZus{}add}\PYG{p}{(}\PYG{n}{list1}\PYG{p}{,} \PYG{n}{SYMBOL\PYGZus{}DIRECTORY}\PYG{p}{,} \PYG{l+s}{\PYGZdq{}}\PYG{l+s}{Open}\PYG{l+s}{\PYGZdq{}}\PYG{p}{,} \PYG{n}{list\PYGZus{}release\PYGZus{}action}\PYG{p}{)}\PYG{p}{;}
\PYG{n}{lv\PYGZus{}list\PYGZus{}add}\PYG{p}{(}\PYG{n}{list1}\PYG{p}{,} \PYG{n}{SYMBOL\PYGZus{}CLOSE}\PYG{p}{,} \PYG{l+s}{\PYGZdq{}}\PYG{l+s}{Delete}\PYG{l+s}{\PYGZdq{}}\PYG{p}{,} \PYG{n}{list\PYGZus{}release\PYGZus{}action}\PYG{p}{)}\PYG{p}{;}
\PYG{n}{lv\PYGZus{}list\PYGZus{}add}\PYG{p}{(}\PYG{n}{list1}\PYG{p}{,} \PYG{n}{SYMBOL\PYGZus{}EDIT}\PYG{p}{,} \PYG{l+s}{\PYGZdq{}}\PYG{l+s}{Edit}\PYG{l+s}{\PYGZdq{}}\PYG{p}{,} \PYG{n}{list\PYGZus{}release\PYGZus{}action}\PYG{p}{)}\PYG{p}{;}
\PYG{n}{lv\PYGZus{}list\PYGZus{}add}\PYG{p}{(}\PYG{n}{list1}\PYG{p}{,} \PYG{n}{SYMBOL\PYGZus{}SAVE}\PYG{p}{,} \PYG{l+s}{\PYGZdq{}}\PYG{l+s}{Save}\PYG{l+s}{\PYGZdq{}}\PYG{p}{,} \PYG{n}{list\PYGZus{}release\PYGZus{}action}\PYG{p}{)}\PYG{p}{;}

\PYG{c+cm}{/*Create a label above the list*/}
\PYG{n}{lv\PYGZus{}obj\PYGZus{}t} \PYG{o}{*} \PYG{n}{label}\PYG{p}{;}
\PYG{n}{label} \PYG{o}{=} \PYG{n}{lv\PYGZus{}label\PYGZus{}create}\PYG{p}{(}\PYG{n}{lv\PYGZus{}scr\PYGZus{}act}\PYG{p}{(}\PYG{p}{)}\PYG{p}{,} \PYG{n+nb}{NULL}\PYG{p}{)}\PYG{p}{;}
\PYG{n}{lv\PYGZus{}label\PYGZus{}set\PYGZus{}text}\PYG{p}{(}\PYG{n}{label}\PYG{p}{,} \PYG{l+s}{\PYGZdq{}}\PYG{l+s}{Default}\PYG{l+s}{\PYGZdq{}}\PYG{p}{)}\PYG{p}{;}
\PYG{n}{lv\PYGZus{}obj\PYGZus{}align}\PYG{p}{(}\PYG{n}{label}\PYG{p}{,} \PYG{n}{list1}\PYG{p}{,} \PYG{n}{LV\PYGZus{}ALIGN\PYGZus{}OUT\PYGZus{}TOP\PYGZus{}MID}\PYG{p}{,} \PYG{l+m+mi}{0}\PYG{p}{,} \PYG{o}{\PYGZhy{}}\PYG{l+m+mi}{10}\PYG{p}{)}\PYG{p}{;}

\PYG{c+cm}{/*********************}
\PYG{c+cm}{ * Create new styles}
\PYG{c+cm}{ ********************/}
\PYG{c+cm}{/*Create a scroll bar style*/}
\PYG{k}{static} \PYG{n}{lv\PYGZus{}style\PYGZus{}t} \PYG{n}{style\PYGZus{}sb}\PYG{p}{;}
\PYG{n}{lv\PYGZus{}style\PYGZus{}copy}\PYG{p}{(}\PYG{o}{\PYGZam{}}\PYG{n}{style\PYGZus{}sb}\PYG{p}{,} \PYG{o}{\PYGZam{}}\PYG{n}{lv\PYGZus{}style\PYGZus{}plain}\PYG{p}{)}\PYG{p}{;}
\PYG{n}{style\PYGZus{}sb}\PYG{p}{.}\PYG{n}{body}\PYG{p}{.}\PYG{n}{main\PYGZus{}color} \PYG{o}{=} \PYG{n}{LV\PYGZus{}COLOR\PYGZus{}BLACK}\PYG{p}{;}
\PYG{n}{style\PYGZus{}sb}\PYG{p}{.}\PYG{n}{body}\PYG{p}{.}\PYG{n}{grad\PYGZus{}color} \PYG{o}{=} \PYG{n}{LV\PYGZus{}COLOR\PYGZus{}BLACK}\PYG{p}{;}
\PYG{n}{style\PYGZus{}sb}\PYG{p}{.}\PYG{n}{body}\PYG{p}{.}\PYG{n}{border}\PYG{p}{.}\PYG{n}{color} \PYG{o}{=} \PYG{n}{LV\PYGZus{}COLOR\PYGZus{}WHITE}\PYG{p}{;}
\PYG{n}{style\PYGZus{}sb}\PYG{p}{.}\PYG{n}{body}\PYG{p}{.}\PYG{n}{border}\PYG{p}{.}\PYG{n}{width} \PYG{o}{=} \PYG{l+m+mi}{1}\PYG{p}{;}
\PYG{n}{style\PYGZus{}sb}\PYG{p}{.}\PYG{n}{body}\PYG{p}{.}\PYG{n}{border}\PYG{p}{.}\PYG{n}{opa} \PYG{o}{=} \PYG{n}{LV\PYGZus{}OPA\PYGZus{}70}\PYG{p}{;}
\PYG{n}{style\PYGZus{}sb}\PYG{p}{.}\PYG{n}{body}\PYG{p}{.}\PYG{n}{radius} \PYG{o}{=} \PYG{n}{LV\PYGZus{}RADIUS\PYGZus{}CIRCLE}\PYG{p}{;}
\PYG{n}{style\PYGZus{}sb}\PYG{p}{.}\PYG{n}{body}\PYG{p}{.}\PYG{n}{opa} \PYG{o}{=} \PYG{n}{LV\PYGZus{}OPA\PYGZus{}60}\PYG{p}{;}

\PYG{c+cm}{/*Create styles for the buttons*/}
\PYG{k}{static} \PYG{n}{lv\PYGZus{}style\PYGZus{}t} \PYG{n}{style\PYGZus{}btn\PYGZus{}rel}\PYG{p}{;}
\PYG{k}{static} \PYG{n}{lv\PYGZus{}style\PYGZus{}t} \PYG{n}{style\PYGZus{}btn\PYGZus{}pr}\PYG{p}{;}
\PYG{n}{lv\PYGZus{}style\PYGZus{}copy}\PYG{p}{(}\PYG{o}{\PYGZam{}}\PYG{n}{style\PYGZus{}btn\PYGZus{}rel}\PYG{p}{,} \PYG{o}{\PYGZam{}}\PYG{n}{lv\PYGZus{}style\PYGZus{}btn\PYGZus{}rel}\PYG{p}{)}\PYG{p}{;}
\PYG{n}{style\PYGZus{}btn\PYGZus{}rel}\PYG{p}{.}\PYG{n}{body}\PYG{p}{.}\PYG{n}{main\PYGZus{}color} \PYG{o}{=} \PYG{n}{LV\PYGZus{}COLOR\PYGZus{}MAKE}\PYG{p}{(}\PYG{l+m+mh}{0x30}\PYG{p}{,} \PYG{l+m+mh}{0x30}\PYG{p}{,} \PYG{l+m+mh}{0x30}\PYG{p}{)}\PYG{p}{;}
\PYG{n}{style\PYGZus{}btn\PYGZus{}rel}\PYG{p}{.}\PYG{n}{body}\PYG{p}{.}\PYG{n}{grad\PYGZus{}color} \PYG{o}{=} \PYG{n}{LV\PYGZus{}COLOR\PYGZus{}BLACK}\PYG{p}{;}
\PYG{n}{style\PYGZus{}btn\PYGZus{}rel}\PYG{p}{.}\PYG{n}{body}\PYG{p}{.}\PYG{n}{border}\PYG{p}{.}\PYG{n}{color} \PYG{o}{=} \PYG{n}{LV\PYGZus{}COLOR\PYGZus{}SILVER}\PYG{p}{;}
\PYG{n}{style\PYGZus{}btn\PYGZus{}rel}\PYG{p}{.}\PYG{n}{body}\PYG{p}{.}\PYG{n}{border}\PYG{p}{.}\PYG{n}{width} \PYG{o}{=} \PYG{l+m+mi}{1}\PYG{p}{;}
\PYG{n}{style\PYGZus{}btn\PYGZus{}rel}\PYG{p}{.}\PYG{n}{body}\PYG{p}{.}\PYG{n}{border}\PYG{p}{.}\PYG{n}{opa} \PYG{o}{=} \PYG{n}{LV\PYGZus{}OPA\PYGZus{}50}\PYG{p}{;}
\PYG{n}{style\PYGZus{}btn\PYGZus{}rel}\PYG{p}{.}\PYG{n}{body}\PYG{p}{.}\PYG{n}{radius} \PYG{o}{=} \PYG{l+m+mi}{0}\PYG{p}{;}

\PYG{n}{lv\PYGZus{}style\PYGZus{}copy}\PYG{p}{(}\PYG{o}{\PYGZam{}}\PYG{n}{style\PYGZus{}btn\PYGZus{}pr}\PYG{p}{,} \PYG{o}{\PYGZam{}}\PYG{n}{style\PYGZus{}btn\PYGZus{}rel}\PYG{p}{)}\PYG{p}{;}
\PYG{n}{style\PYGZus{}btn\PYGZus{}pr}\PYG{p}{.}\PYG{n}{body}\PYG{p}{.}\PYG{n}{main\PYGZus{}color} \PYG{o}{=} \PYG{n}{LV\PYGZus{}COLOR\PYGZus{}MAKE}\PYG{p}{(}\PYG{l+m+mh}{0x55}\PYG{p}{,} \PYG{l+m+mh}{0x96}\PYG{p}{,} \PYG{l+m+mh}{0xd8}\PYG{p}{)}\PYG{p}{;}
\PYG{n}{style\PYGZus{}btn\PYGZus{}pr}\PYG{p}{.}\PYG{n}{body}\PYG{p}{.}\PYG{n}{grad\PYGZus{}color} \PYG{o}{=} \PYG{n}{LV\PYGZus{}COLOR\PYGZus{}MAKE}\PYG{p}{(}\PYG{l+m+mh}{0x37}\PYG{p}{,} \PYG{l+m+mh}{0x62}\PYG{p}{,} \PYG{l+m+mh}{0x90}\PYG{p}{)}\PYG{p}{;}
\PYG{n}{style\PYGZus{}btn\PYGZus{}pr}\PYG{p}{.}\PYG{n}{text}\PYG{p}{.}\PYG{n}{color} \PYG{o}{=} \PYG{n}{LV\PYGZus{}COLOR\PYGZus{}MAKE}\PYG{p}{(}\PYG{l+m+mh}{0xbb}\PYG{p}{,} \PYG{l+m+mh}{0xd5}\PYG{p}{,} \PYG{l+m+mh}{0xf1}\PYG{p}{)}\PYG{p}{;}

\PYG{c+cm}{/**************************************}
\PYG{c+cm}{ * Create a list with modified styles}
\PYG{c+cm}{ **************************************/}

\PYG{c+cm}{/*Copy the previous list*/}
\PYG{n}{lv\PYGZus{}obj\PYGZus{}t} \PYG{o}{*} \PYG{n}{list2} \PYG{o}{=} \PYG{n}{lv\PYGZus{}list\PYGZus{}create}\PYG{p}{(}\PYG{n}{lv\PYGZus{}scr\PYGZus{}act}\PYG{p}{(}\PYG{p}{)}\PYG{p}{,}\PYG{n}{list1}\PYG{p}{)}\PYG{p}{;}
\PYG{n}{lv\PYGZus{}obj\PYGZus{}align}\PYG{p}{(}\PYG{n}{list2}\PYG{p}{,} \PYG{n+nb}{NULL}\PYG{p}{,} \PYG{n}{LV\PYGZus{}ALIGN\PYGZus{}IN\PYGZus{}TOP\PYGZus{}RIGHT}\PYG{p}{,} \PYG{o}{\PYGZhy{}}\PYG{l+m+mi}{20}\PYG{p}{,} \PYG{l+m+mi}{40}\PYG{p}{)}\PYG{p}{;}
\PYG{n}{lv\PYGZus{}list\PYGZus{}set\PYGZus{}sb\PYGZus{}mode}\PYG{p}{(}\PYG{n}{list2}\PYG{p}{,} \PYG{n}{LV\PYGZus{}SB\PYGZus{}MODE\PYGZus{}AUTO}\PYG{p}{)}\PYG{p}{;}
\PYG{n}{lv\PYGZus{}list\PYGZus{}set\PYGZus{}style}\PYG{p}{(}\PYG{n}{list2}\PYG{p}{,} \PYG{n}{LV\PYGZus{}LIST\PYGZus{}STYLE\PYGZus{}BG}\PYG{p}{,} \PYG{o}{\PYGZam{}}\PYG{n}{lv\PYGZus{}style\PYGZus{}transp\PYGZus{}tight}\PYG{p}{)}\PYG{p}{;}
\PYG{n}{lv\PYGZus{}list\PYGZus{}set\PYGZus{}style}\PYG{p}{(}\PYG{n}{list2}\PYG{p}{,} \PYG{n}{LV\PYGZus{}LIST\PYGZus{}STYLE\PYGZus{}SCRL}\PYG{p}{,} \PYG{o}{\PYGZam{}}\PYG{n}{lv\PYGZus{}style\PYGZus{}transp\PYGZus{}tight}\PYG{p}{)}\PYG{p}{;}
\PYG{n}{lv\PYGZus{}list\PYGZus{}set\PYGZus{}style}\PYG{p}{(}\PYG{n}{list2}\PYG{p}{,} \PYG{n}{LV\PYGZus{}LIST\PYGZus{}STYLE\PYGZus{}BTN\PYGZus{}REL}\PYG{p}{,} \PYG{o}{\PYGZam{}}\PYG{n}{style\PYGZus{}btn\PYGZus{}rel}\PYG{p}{)}\PYG{p}{;} \PYG{c+cm}{/*Set the new button styles*/}
\PYG{n}{lv\PYGZus{}list\PYGZus{}set\PYGZus{}style}\PYG{p}{(}\PYG{n}{list2}\PYG{p}{,} \PYG{n}{LV\PYGZus{}LIST\PYGZus{}STYLE\PYGZus{}BTN\PYGZus{}PR}\PYG{p}{,} \PYG{o}{\PYGZam{}}\PYG{n}{style\PYGZus{}btn\PYGZus{}pr}\PYG{p}{)}\PYG{p}{;}

\PYG{c+cm}{/*Create a label above the list*/}
\PYG{n}{label} \PYG{o}{=} \PYG{n}{lv\PYGZus{}label\PYGZus{}create}\PYG{p}{(}\PYG{n}{lv\PYGZus{}scr\PYGZus{}act}\PYG{p}{(}\PYG{p}{)}\PYG{p}{,} \PYG{n}{label}\PYG{p}{)}\PYG{p}{;}       \PYG{c+cm}{/*Copy the previous label*/}
\PYG{n}{lv\PYGZus{}label\PYGZus{}set\PYGZus{}text}\PYG{p}{(}\PYG{n}{label}\PYG{p}{,} \PYG{l+s}{\PYGZdq{}}\PYG{l+s}{Modified}\PYG{l+s}{\PYGZdq{}}\PYG{p}{)}\PYG{p}{;}
\PYG{n}{lv\PYGZus{}obj\PYGZus{}align}\PYG{p}{(}\PYG{n}{label}\PYG{p}{,} \PYG{n}{list2}\PYG{p}{,} \PYG{n}{LV\PYGZus{}ALIGN\PYGZus{}OUT\PYGZus{}TOP\PYGZus{}MID}\PYG{p}{,} \PYG{l+m+mi}{0}\PYG{p}{,} \PYG{o}{\PYGZhy{}}\PYG{l+m+mi}{10}\PYG{p}{)}\PYG{p}{;}
\end{sphinxVerbatim}


\subsection{Line meter (lv\_lmeter)}
\label{\detokenize{object-types/lmeter:line-meter-lv-lmeter}}\label{\detokenize{object-types/lmeter::doc}}

\subsubsection{Overview}
\label{\detokenize{object-types/lmeter:overview}}
The Line Meter object consists of some \sphinxstylestrong{radial lines} which draw a scale. When setting a new value with \sphinxcode{\sphinxupquote{lv\_lmeter\_set\_value(lmeter, new\_value)}} the proportional part of the scale will be recolored.

The \sphinxcode{\sphinxupquote{lv\_lmeter\_set\_range(lmeter, min, max)}} function sets the \sphinxstylestrong{range} of the line meter.

You can set the \sphinxstylestrong{angle} of the scale and the \sphinxstylestrong{number of the lines} by: \sphinxcode{\sphinxupquote{lv\_lmeter\_set\_scale(lmeter, angle, line\_num)}}. The default angle is 240 and the default line number is 31.


\subsubsection{Styles}
\label{\detokenize{object-types/lmeter:styles}}
The line meter uses one style which can be set by \sphinxcode{\sphinxupquote{lv\_lmeter\_set\_style(lmeter, LV\_LMETER\_STYLE\_MAIN, \&style)}}. The line meter’s properties are derived from the following style attributes:
\begin{itemize}
\item {} 
\sphinxstylestrong{line.color} “inactive line’s” color which are greater then the current value

\item {} 
\sphinxstylestrong{body.main\_color} “active line’s” color at the beginning of the scale

\item {} 
\sphinxstylestrong{body.grad\_color} “active line’s” color at the end of the scale (gradient with main color)

\item {} 
\sphinxstylestrong{body.padding.hor} line length

\item {} 
\sphinxstylestrong{line.width} line width

\end{itemize}

The default style is \sphinxcode{\sphinxupquote{lv\_style\_pretty\_color}}.


\subsubsection{Events}
\label{\detokenize{object-types/lmeter:events}}
Only the \sphinxhref{/overview/events.html\#generic-events}{Genreric events} are sent by the object type.

Learn more about {\hyperref[\detokenize{overview/events::doc}]{\sphinxcrossref{\DUrole{doc,doc,doc}{Events}}}}.


\subsubsection{Keys}
\label{\detokenize{object-types/lmeter:keys}}
No \sphinxstyleemphasis{Keys} are processed by the object type.

Learn more about {\hyperref[\detokenize{overview/indev::doc}]{\sphinxcrossref{\DUrole{doc,doc,doc}{Keys}}}}.


\subsubsection{Example}
\label{\detokenize{object-types/lmeter:example}}
\sphinxincludegraphics{{line-meter-lv_lmeter}.png}Line meter image

\begin{sphinxVerbatim}[commandchars=\\\{\}]
\PYG{c+cm}{/*******************************}
\PYG{c+cm}{ * Create 3 similar line meter}
\PYG{c+cm}{ *******************************/}

\PYG{c+cm}{/*Create a simple style with ticker line width*/}
\PYG{k}{static} \PYG{n}{lv\PYGZus{}style\PYGZus{}t} \PYG{n}{style\PYGZus{}lmeter1}\PYG{p}{;}
\PYG{n}{lv\PYGZus{}style\PYGZus{}copy}\PYG{p}{(}\PYG{o}{\PYGZam{}}\PYG{n}{style\PYGZus{}lmeter1}\PYG{p}{,} \PYG{o}{\PYGZam{}}\PYG{n}{lv\PYGZus{}style\PYGZus{}pretty\PYGZus{}color}\PYG{p}{)}\PYG{p}{;}
\PYG{n}{style\PYGZus{}lmeter1}\PYG{p}{.}\PYG{n}{line}\PYG{p}{.}\PYG{n}{width} \PYG{o}{=} \PYG{l+m+mi}{2}\PYG{p}{;}
\PYG{n}{style\PYGZus{}lmeter1}\PYG{p}{.}\PYG{n}{line}\PYG{p}{.}\PYG{n}{color} \PYG{o}{=} \PYG{n}{LV\PYGZus{}COLOR\PYGZus{}SILVER}\PYG{p}{;}
\PYG{n}{style\PYGZus{}lmeter1}\PYG{p}{.}\PYG{n}{body}\PYG{p}{.}\PYG{n}{main\PYGZus{}color} \PYG{o}{=} \PYG{n}{LV\PYGZus{}COLOR\PYGZus{}HEX}\PYG{p}{(}\PYG{l+m+mh}{0x91bfed}\PYG{p}{)}\PYG{p}{;}         \PYG{c+cm}{/*Light blue*/}
\PYG{n}{style\PYGZus{}lmeter1}\PYG{p}{.}\PYG{n}{body}\PYG{p}{.}\PYG{n}{grad\PYGZus{}color} \PYG{o}{=} \PYG{n}{LV\PYGZus{}COLOR\PYGZus{}HEX}\PYG{p}{(}\PYG{l+m+mh}{0x04386c}\PYG{p}{)}\PYG{p}{;}         \PYG{c+cm}{/*Dark blue*/}

\PYG{c+cm}{/*Create the first line meter */}
\PYG{n}{lv\PYGZus{}obj\PYGZus{}t} \PYG{o}{*} \PYG{n}{lmeter}\PYG{p}{;}
\PYG{n}{lmeter} \PYG{o}{=} \PYG{n}{lv\PYGZus{}lmeter\PYGZus{}create}\PYG{p}{(}\PYG{n}{lv\PYGZus{}scr\PYGZus{}act}\PYG{p}{(}\PYG{p}{)}\PYG{p}{,} \PYG{n+nb}{NULL}\PYG{p}{)}\PYG{p}{;}
\PYG{n}{lv\PYGZus{}lmeter\PYGZus{}set\PYGZus{}range}\PYG{p}{(}\PYG{n}{lmeter}\PYG{p}{,} \PYG{l+m+mi}{0}\PYG{p}{,} \PYG{l+m+mi}{100}\PYG{p}{)}\PYG{p}{;}                   \PYG{c+cm}{/*Set the range*/}
\PYG{n}{lv\PYGZus{}lmeter\PYGZus{}set\PYGZus{}value}\PYG{p}{(}\PYG{n}{lmeter}\PYG{p}{,} \PYG{l+m+mi}{30}\PYG{p}{)}\PYG{p}{;}                       \PYG{c+cm}{/*Set the current value*/}
\PYG{n}{lv\PYGZus{}lmeter\PYGZus{}set\PYGZus{}style}\PYG{p}{(}\PYG{n}{lmeter}\PYG{p}{,} \PYG{o}{\PYGZam{}}\PYG{n}{style\PYGZus{}lmeter1}\PYG{p}{)}\PYG{p}{;}           \PYG{c+cm}{/*Apply the new style*/}
\PYG{n}{lv\PYGZus{}obj\PYGZus{}set\PYGZus{}size}\PYG{p}{(}\PYG{n}{lmeter}\PYG{p}{,} \PYG{l+m+mi}{80}\PYG{p}{,} \PYG{l+m+mi}{80}\PYG{p}{)}\PYG{p}{;}
\PYG{n}{lv\PYGZus{}obj\PYGZus{}align}\PYG{p}{(}\PYG{n}{lmeter}\PYG{p}{,} \PYG{n+nb}{NULL}\PYG{p}{,} \PYG{n}{LV\PYGZus{}ALIGN\PYGZus{}IN\PYGZus{}BOTTOM\PYGZus{}LEFT}\PYG{p}{,} \PYG{l+m+mi}{20}\PYG{p}{,} \PYG{o}{\PYGZhy{}}\PYG{l+m+mi}{20}\PYG{p}{)}\PYG{p}{;}

\PYG{c+cm}{/*Add a label to show the current value*/}
\PYG{n}{lv\PYGZus{}obj\PYGZus{}t} \PYG{o}{*} \PYG{n}{label}\PYG{p}{;}
\PYG{n}{label} \PYG{o}{=} \PYG{n}{lv\PYGZus{}label\PYGZus{}create}\PYG{p}{(}\PYG{n}{lmeter}\PYG{p}{,} \PYG{n+nb}{NULL}\PYG{p}{)}\PYG{p}{;}
\PYG{n}{lv\PYGZus{}label\PYGZus{}set\PYGZus{}text}\PYG{p}{(}\PYG{n}{label}\PYG{p}{,} \PYG{l+s}{\PYGZdq{}}\PYG{l+s}{30\PYGZpc{}}\PYG{l+s}{\PYGZdq{}}\PYG{p}{)}\PYG{p}{;}
\PYG{n}{lv\PYGZus{}label\PYGZus{}set\PYGZus{}style}\PYG{p}{(}\PYG{n}{label}\PYG{p}{,} \PYG{o}{\PYGZam{}}\PYG{n}{lv\PYGZus{}style\PYGZus{}pretty}\PYG{p}{)}\PYG{p}{;}
\PYG{n}{lv\PYGZus{}obj\PYGZus{}align}\PYG{p}{(}\PYG{n}{label}\PYG{p}{,} \PYG{n+nb}{NULL}\PYG{p}{,} \PYG{n}{LV\PYGZus{}ALIGN\PYGZus{}CENTER}\PYG{p}{,} \PYG{l+m+mi}{0}\PYG{p}{,} \PYG{l+m+mi}{0}\PYG{p}{)}\PYG{p}{;}

\PYG{c+cm}{/*Create the second line meter and label*/}
\PYG{n}{lmeter} \PYG{o}{=} \PYG{n}{lv\PYGZus{}lmeter\PYGZus{}create}\PYG{p}{(}\PYG{n}{lv\PYGZus{}scr\PYGZus{}act}\PYG{p}{(}\PYG{p}{)}\PYG{p}{,} \PYG{n}{lmeter}\PYG{p}{)}\PYG{p}{;}
\PYG{n}{lv\PYGZus{}lmeter\PYGZus{}set\PYGZus{}value}\PYG{p}{(}\PYG{n}{lmeter}\PYG{p}{,} \PYG{l+m+mi}{60}\PYG{p}{)}\PYG{p}{;}
\PYG{n}{lv\PYGZus{}obj\PYGZus{}align}\PYG{p}{(}\PYG{n}{lmeter}\PYG{p}{,} \PYG{n+nb}{NULL}\PYG{p}{,} \PYG{n}{LV\PYGZus{}ALIGN\PYGZus{}IN\PYGZus{}BOTTOM\PYGZus{}MID}\PYG{p}{,} \PYG{l+m+mi}{0}\PYG{p}{,} \PYG{o}{\PYGZhy{}}\PYG{l+m+mi}{20}\PYG{p}{)}\PYG{p}{;}

\PYG{n}{label} \PYG{o}{=} \PYG{n}{lv\PYGZus{}label\PYGZus{}create}\PYG{p}{(}\PYG{n}{lmeter}\PYG{p}{,} \PYG{n}{label}\PYG{p}{)}\PYG{p}{;}
\PYG{n}{lv\PYGZus{}label\PYGZus{}set\PYGZus{}text}\PYG{p}{(}\PYG{n}{label}\PYG{p}{,} \PYG{l+s}{\PYGZdq{}}\PYG{l+s}{60\PYGZpc{}}\PYG{l+s}{\PYGZdq{}}\PYG{p}{)}\PYG{p}{;}
\PYG{n}{lv\PYGZus{}obj\PYGZus{}align}\PYG{p}{(}\PYG{n}{label}\PYG{p}{,} \PYG{n+nb}{NULL}\PYG{p}{,} \PYG{n}{LV\PYGZus{}ALIGN\PYGZus{}CENTER}\PYG{p}{,} \PYG{l+m+mi}{0}\PYG{p}{,} \PYG{l+m+mi}{0}\PYG{p}{)}\PYG{p}{;}

\PYG{c+cm}{/*Create the third line meter and label*/}
\PYG{n}{lmeter} \PYG{o}{=} \PYG{n}{lv\PYGZus{}lmeter\PYGZus{}create}\PYG{p}{(}\PYG{n}{lv\PYGZus{}scr\PYGZus{}act}\PYG{p}{(}\PYG{p}{)}\PYG{p}{,} \PYG{n}{lmeter}\PYG{p}{)}\PYG{p}{;}
\PYG{n}{lv\PYGZus{}lmeter\PYGZus{}set\PYGZus{}value}\PYG{p}{(}\PYG{n}{lmeter}\PYG{p}{,} \PYG{l+m+mi}{90}\PYG{p}{)}\PYG{p}{;}
\PYG{n}{lv\PYGZus{}obj\PYGZus{}align}\PYG{p}{(}\PYG{n}{lmeter}\PYG{p}{,} \PYG{n+nb}{NULL}\PYG{p}{,} \PYG{n}{LV\PYGZus{}ALIGN\PYGZus{}IN\PYGZus{}BOTTOM\PYGZus{}RIGHT}\PYG{p}{,} \PYG{o}{\PYGZhy{}}\PYG{l+m+mi}{20}\PYG{p}{,} \PYG{o}{\PYGZhy{}}\PYG{l+m+mi}{20}\PYG{p}{)}\PYG{p}{;}

\PYG{n}{label} \PYG{o}{=} \PYG{n}{lv\PYGZus{}label\PYGZus{}create}\PYG{p}{(}\PYG{n}{lmeter}\PYG{p}{,} \PYG{n}{label}\PYG{p}{)}\PYG{p}{;}
\PYG{n}{lv\PYGZus{}label\PYGZus{}set\PYGZus{}text}\PYG{p}{(}\PYG{n}{label}\PYG{p}{,} \PYG{l+s}{\PYGZdq{}}\PYG{l+s}{90\PYGZpc{}}\PYG{l+s}{\PYGZdq{}}\PYG{p}{)}\PYG{p}{;}
\PYG{n}{lv\PYGZus{}obj\PYGZus{}align}\PYG{p}{(}\PYG{n}{label}\PYG{p}{,} \PYG{n+nb}{NULL}\PYG{p}{,} \PYG{n}{LV\PYGZus{}ALIGN\PYGZus{}CENTER}\PYG{p}{,} \PYG{l+m+mi}{0}\PYG{p}{,} \PYG{l+m+mi}{0}\PYG{p}{)}\PYG{p}{;}

\PYG{c+cm}{/*********************************}
\PYG{c+cm}{ * Create a greater line meter}
\PYG{c+cm}{ *********************************/}

\PYG{c+cm}{/*Create a new style*/}
\PYG{k}{static} \PYG{n}{lv\PYGZus{}style\PYGZus{}t} \PYG{n}{style\PYGZus{}lmeter2}\PYG{p}{;}
\PYG{n}{lv\PYGZus{}style\PYGZus{}copy}\PYG{p}{(}\PYG{o}{\PYGZam{}}\PYG{n}{style\PYGZus{}lmeter2}\PYG{p}{,} \PYG{o}{\PYGZam{}}\PYG{n}{lv\PYGZus{}style\PYGZus{}pretty\PYGZus{}color}\PYG{p}{)}\PYG{p}{;}
\PYG{n}{style\PYGZus{}lmeter2}\PYG{p}{.}\PYG{n}{line}\PYG{p}{.}\PYG{n}{width} \PYG{o}{=} \PYG{l+m+mi}{2}\PYG{p}{;}
\PYG{n}{style\PYGZus{}lmeter2}\PYG{p}{.}\PYG{n}{line}\PYG{p}{.}\PYG{n}{color} \PYG{o}{=} \PYG{n}{LV\PYGZus{}COLOR\PYGZus{}SILVER}\PYG{p}{;}
\PYG{n}{style\PYGZus{}lmeter2}\PYG{p}{.}\PYG{n}{body}\PYG{p}{.}\PYG{n}{padding}\PYG{p}{.}\PYG{n}{hor} \PYG{o}{=} \PYG{l+m+mi}{16}\PYG{p}{;}            \PYG{c+cm}{/*Line length*/}
\PYG{n}{style\PYGZus{}lmeter2}\PYG{p}{.}\PYG{n}{body}\PYG{p}{.}\PYG{n}{main\PYGZus{}color} \PYG{o}{=} \PYG{n}{LV\PYGZus{}COLOR\PYGZus{}LIME}\PYG{p}{;}
\PYG{n}{style\PYGZus{}lmeter2}\PYG{p}{.}\PYG{n}{body}\PYG{p}{.}\PYG{n}{grad\PYGZus{}color} \PYG{o}{=} \PYG{n}{LV\PYGZus{}COLOR\PYGZus{}ORANGE}\PYG{p}{;}

\PYG{c+cm}{/*Create the line meter*/}
\PYG{n}{lmeter} \PYG{o}{=} \PYG{n}{lv\PYGZus{}lmeter\PYGZus{}create}\PYG{p}{(}\PYG{n}{lv\PYGZus{}scr\PYGZus{}act}\PYG{p}{(}\PYG{p}{)}\PYG{p}{,} \PYG{n}{lmeter}\PYG{p}{)}\PYG{p}{;}
\PYG{n}{lv\PYGZus{}obj\PYGZus{}set\PYGZus{}style}\PYG{p}{(}\PYG{n}{lmeter}\PYG{p}{,} \PYG{o}{\PYGZam{}}\PYG{n}{style\PYGZus{}lmeter2}\PYG{p}{)}\PYG{p}{;}
\PYG{n}{lv\PYGZus{}obj\PYGZus{}set\PYGZus{}size}\PYG{p}{(}\PYG{n}{lmeter}\PYG{p}{,} \PYG{l+m+mi}{120}\PYG{p}{,} \PYG{l+m+mi}{120}\PYG{p}{)}\PYG{p}{;}
\PYG{n}{lv\PYGZus{}obj\PYGZus{}align}\PYG{p}{(}\PYG{n}{lmeter}\PYG{p}{,} \PYG{n+nb}{NULL}\PYG{p}{,} \PYG{n}{LV\PYGZus{}ALIGN\PYGZus{}IN\PYGZus{}TOP\PYGZus{}MID}\PYG{p}{,} \PYG{l+m+mi}{0}\PYG{p}{,} \PYG{l+m+mi}{20}\PYG{p}{)}\PYG{p}{;}
\PYG{n}{lv\PYGZus{}lmeter\PYGZus{}set\PYGZus{}scale}\PYG{p}{(}\PYG{n}{lmeter}\PYG{p}{,} \PYG{l+m+mi}{240}\PYG{p}{,} \PYG{l+m+mi}{31}\PYG{p}{)}\PYG{p}{;}
\PYG{n}{lv\PYGZus{}lmeter\PYGZus{}set\PYGZus{}value}\PYG{p}{(}\PYG{n}{lmeter}\PYG{p}{,} \PYG{l+m+mi}{90}\PYG{p}{)}\PYG{p}{;}

\PYG{c+cm}{/*Create a label style with greater font*/}
\PYG{k}{static} \PYG{n}{lv\PYGZus{}style\PYGZus{}t} \PYG{n}{style\PYGZus{}label}\PYG{p}{;}
\PYG{n}{lv\PYGZus{}style\PYGZus{}copy}\PYG{p}{(}\PYG{o}{\PYGZam{}}\PYG{n}{style\PYGZus{}label}\PYG{p}{,} \PYG{o}{\PYGZam{}}\PYG{n}{lv\PYGZus{}style\PYGZus{}pretty}\PYG{p}{)}\PYG{p}{;}
\PYG{n}{style\PYGZus{}label}\PYG{p}{.}\PYG{n}{text}\PYG{p}{.}\PYG{n}{font} \PYG{o}{=} \PYG{o}{\PYGZam{}}\PYG{n}{lv\PYGZus{}font\PYGZus{}dejavu\PYGZus{}60}\PYG{p}{;}
\PYG{n}{style\PYGZus{}label}\PYG{p}{.}\PYG{n}{text}\PYG{p}{.}\PYG{n}{color} \PYG{o}{=} \PYG{n}{LV\PYGZus{}COLOR\PYGZus{}GRAY}\PYG{p}{;}

\PYG{c+cm}{/*Add a label to show the current value*/}
\PYG{n}{label} \PYG{o}{=} \PYG{n}{lv\PYGZus{}label\PYGZus{}create}\PYG{p}{(}\PYG{n}{lmeter}\PYG{p}{,} \PYG{n}{label}\PYG{p}{)}\PYG{p}{;}
\PYG{n}{lv\PYGZus{}label\PYGZus{}set\PYGZus{}text}\PYG{p}{(}\PYG{n}{label}\PYG{p}{,} \PYG{l+s}{\PYGZdq{}}\PYG{l+s}{90\PYGZpc{}}\PYG{l+s}{\PYGZdq{}}\PYG{p}{)}\PYG{p}{;}
\PYG{n}{lv\PYGZus{}obj\PYGZus{}set\PYGZus{}style}\PYG{p}{(}\PYG{n}{label}\PYG{p}{,} \PYG{o}{\PYGZam{}}\PYG{n}{style\PYGZus{}label}\PYG{p}{)}\PYG{p}{;}
\PYG{n}{lv\PYGZus{}obj\PYGZus{}align}\PYG{p}{(}\PYG{n}{label}\PYG{p}{,} \PYG{n+nb}{NULL}\PYG{p}{,} \PYG{n}{LV\PYGZus{}ALIGN\PYGZus{}CENTER}\PYG{p}{,} \PYG{l+m+mi}{0}\PYG{p}{,} \PYG{l+m+mi}{0}\PYG{p}{)}\PYG{p}{;}
\end{sphinxVerbatim}


\subsection{Message box (lv\_mbox)}
\label{\detokenize{object-types/mbox:message-box-lv-mbox}}\label{\detokenize{object-types/mbox::doc}}

\subsection{Page (lv\_page)}
\label{\detokenize{object-types/page:page-lv-page}}\label{\detokenize{object-types/page::doc}}

\subsection{Preload (lv\_preload)}
\label{\detokenize{object-types/preload:preload-lv-preload}}\label{\detokenize{object-types/preload::doc}}

\subsection{Roller (lv\_roller)}
\label{\detokenize{object-types/roller:roller-lv-roller}}\label{\detokenize{object-types/roller::doc}}

\subsection{Slider (lv\_slider)}
\label{\detokenize{object-types/slider:slider-lv-slider}}\label{\detokenize{object-types/slider::doc}}

\subsubsection{Overview}
\label{\detokenize{object-types/slider:overview}}
The Slider object looks like a {\hyperref[\detokenize{object-types/bar::doc}]{\sphinxcrossref{\DUrole{doc,doc,doc}{Bar}}}} supplemented \sphinxstylestrong{with a knob}. The knob can be \sphinxstylestrong{dragged to set a value}. The Slider also can be vertical or horizontal.

\sphinxcode{\sphinxupquote{lv\_slider\_set\_anim\_time(slider, anim\_time)}} sets the animation time in milliseconds.

To set an \sphinxstylestrong{initial value} use \sphinxcode{\sphinxupquote{lv\_slider\_set\_value(slider, new\_value, LV\_ANIM\_ON/OFF)}}.

To specify the \sphinxstylestrong{range} (min, max values) the \sphinxcode{\sphinxupquote{lv\_slider\_set\_range(slider, min , max)}} can be used.

The \sphinxstylestrong{knob can be placed} two ways:
\begin{itemize}
\item {} 
inside the background

\item {} 
on the edges on min/max values

\end{itemize}

Use the \sphinxcode{\sphinxupquote{lv\_slider\_set\_knob\_in(slider, true/false)}} to choose between the modes. (\sphinxstyleemphasis{knob\_in = false} is the default)


\subsubsection{Styles}
\label{\detokenize{object-types/slider:styles}}
You can modify the slider’s styles with \sphinxcode{\sphinxupquote{lv\_slider\_set\_style(slider, LV\_SLIDER\_STYLE\_..., \&style)}}.
\begin{itemize}
\item {} 
\sphinxstylestrong{LV\_SLIDER\_STYLE\_BG} Style of the background. All \sphinxcode{\sphinxupquote{style.body}} properties are used. The \sphinxcode{\sphinxupquote{padding}} values make the knob larger then background. (negative value makes is larger)

\item {} 
\sphinxstylestrong{LV\_SLIDER\_STYLE\_INDIC} Style of the indicator. All \sphinxcode{\sphinxupquote{style.body}} properties are used. The \sphinxcode{\sphinxupquote{padding}} values make the indicator smaller then the background.

\item {} 
\sphinxstylestrong{LV\_SLIDER\_STYLE\_KNOB} Style of the knob.  All \sphinxcode{\sphinxupquote{style.body}} properties are used except \sphinxcode{\sphinxupquote{padding}}.

\end{itemize}


\subsubsection{Events}
\label{\detokenize{object-types/slider:events}}\begin{itemize}
\item {} 
\sphinxstylestrong{LV\_EVENT\_VALUE\_CHANGED} Sent while slider is being dragged or changed with keys.

\end{itemize}


\subsubsection{Keys}
\label{\detokenize{object-types/slider:keys}}\begin{itemize}
\item {} 
\sphinxstylestrong{LV\_KEY\_UP}, \sphinxstylestrong{LV\_KEY\_RIGHT} Increment the slider’s value by 1

\item {} 
\sphinxstylestrong{LV\_KEY\_DOWN}, \sphinxstylestrong{LV\_KEY\_LEFT} Decrement the slider’s value by 1

\end{itemize}


\subsubsection{Example}
\label{\detokenize{object-types/slider:example}}
\sphinxincludegraphics{{slider-lv_slider}.png}Slider image

\begin{sphinxVerbatim}[commandchars=\\\{\}]
\PYG{c+cm}{/*Called when a new value id set on the slider*/}
\PYG{k}{static} \PYG{n}{lv\PYGZus{}res\PYGZus{}t} \PYG{n+nf}{slider\PYGZus{}action}\PYG{p}{(}\PYG{n}{lv\PYGZus{}obj\PYGZus{}t} \PYG{o}{*} \PYG{n}{slider}\PYG{p}{)}
\PYG{p}{\PYGZob{}}
    \PYG{n}{printf}\PYG{p}{(}\PYG{l+s}{\PYGZdq{}}\PYG{l+s}{New slider value: \PYGZpc{}d}\PYG{l+s+se}{\PYGZbs{}n}\PYG{l+s}{\PYGZdq{}}\PYG{p}{,} \PYG{n}{lv\PYGZus{}slider\PYGZus{}get\PYGZus{}value}\PYG{p}{(}\PYG{n}{slider}\PYG{p}{)}\PYG{p}{)}\PYG{p}{;}

    \PYG{k}{return} \PYG{n}{LV\PYGZus{}RES\PYGZus{}OK}\PYG{p}{;}
\PYG{p}{\PYGZcb{}}

\PYG{p}{.}
\PYG{p}{.}
\PYG{p}{.}

\PYG{c+cm}{/*Create a default slider*/}
\PYG{n}{lv\PYGZus{}obj\PYGZus{}t} \PYG{o}{*} \PYG{n}{slider1} \PYG{o}{=} \PYG{n}{lv\PYGZus{}slider\PYGZus{}create}\PYG{p}{(}\PYG{n}{lv\PYGZus{}scr\PYGZus{}act}\PYG{p}{(}\PYG{p}{)}\PYG{p}{,} \PYG{n+nb}{NULL}\PYG{p}{)}\PYG{p}{;}
\PYG{n}{lv\PYGZus{}obj\PYGZus{}set\PYGZus{}size}\PYG{p}{(}\PYG{n}{slider1}\PYG{p}{,} \PYG{l+m+mi}{160}\PYG{p}{,} \PYG{l+m+mi}{30}\PYG{p}{)}\PYG{p}{;}
\PYG{n}{lv\PYGZus{}obj\PYGZus{}align}\PYG{p}{(}\PYG{n}{slider1}\PYG{p}{,} \PYG{n+nb}{NULL}\PYG{p}{,} \PYG{n}{LV\PYGZus{}ALIGN\PYGZus{}IN\PYGZus{}TOP\PYGZus{}RIGHT}\PYG{p}{,} \PYG{o}{\PYGZhy{}}\PYG{l+m+mi}{30}\PYG{p}{,} \PYG{l+m+mi}{30}\PYG{p}{)}\PYG{p}{;}
\PYG{n}{lv\PYGZus{}slider\PYGZus{}set\PYGZus{}action}\PYG{p}{(}\PYG{n}{slider1}\PYG{p}{,} \PYG{n}{slider\PYGZus{}action}\PYG{p}{)}\PYG{p}{;}
\PYG{n}{lv\PYGZus{}bar\PYGZus{}set\PYGZus{}value}\PYG{p}{(}\PYG{n}{slider1}\PYG{p}{,} \PYG{l+m+mi}{70}\PYG{p}{)}\PYG{p}{;}


\PYG{c+cm}{/*Create a label right to the slider*/}
\PYG{n}{lv\PYGZus{}obj\PYGZus{}t} \PYG{o}{*} \PYG{n}{slider1\PYGZus{}label} \PYG{o}{=} \PYG{n}{lv\PYGZus{}label\PYGZus{}create}\PYG{p}{(}\PYG{n}{lv\PYGZus{}scr\PYGZus{}act}\PYG{p}{(}\PYG{p}{)}\PYG{p}{,} \PYG{n+nb}{NULL}\PYG{p}{)}\PYG{p}{;}
\PYG{n}{lv\PYGZus{}label\PYGZus{}set\PYGZus{}text}\PYG{p}{(}\PYG{n}{slider1\PYGZus{}label}\PYG{p}{,} \PYG{l+s}{\PYGZdq{}}\PYG{l+s}{Default}\PYG{l+s}{\PYGZdq{}}\PYG{p}{)}\PYG{p}{;}
\PYG{n}{lv\PYGZus{}obj\PYGZus{}align}\PYG{p}{(}\PYG{n}{slider1\PYGZus{}label}\PYG{p}{,} \PYG{n}{slider1}\PYG{p}{,} \PYG{n}{LV\PYGZus{}ALIGN\PYGZus{}OUT\PYGZus{}LEFT\PYGZus{}MID}\PYG{p}{,} \PYG{o}{\PYGZhy{}}\PYG{l+m+mi}{20}\PYG{p}{,} \PYG{l+m+mi}{0}\PYG{p}{)}\PYG{p}{;}

\PYG{c+cm}{/*Create a bar, an indicator and a knob style*/}
\PYG{k}{static} \PYG{n}{lv\PYGZus{}style\PYGZus{}t} \PYG{n}{style\PYGZus{}bg}\PYG{p}{;}
\PYG{k}{static} \PYG{n}{lv\PYGZus{}style\PYGZus{}t} \PYG{n}{style\PYGZus{}indic}\PYG{p}{;}
\PYG{k}{static} \PYG{n}{lv\PYGZus{}style\PYGZus{}t} \PYG{n}{style\PYGZus{}knob}\PYG{p}{;}

\PYG{n}{lv\PYGZus{}style\PYGZus{}copy}\PYG{p}{(}\PYG{o}{\PYGZam{}}\PYG{n}{style\PYGZus{}bg}\PYG{p}{,} \PYG{o}{\PYGZam{}}\PYG{n}{lv\PYGZus{}style\PYGZus{}pretty}\PYG{p}{)}\PYG{p}{;}
\PYG{n}{style\PYGZus{}bg}\PYG{p}{.}\PYG{n}{body}\PYG{p}{.}\PYG{n}{main\PYGZus{}color} \PYG{o}{=}  \PYG{n}{LV\PYGZus{}COLOR\PYGZus{}BLACK}\PYG{p}{;}
\PYG{n}{style\PYGZus{}bg}\PYG{p}{.}\PYG{n}{body}\PYG{p}{.}\PYG{n}{grad\PYGZus{}color} \PYG{o}{=}  \PYG{n}{LV\PYGZus{}COLOR\PYGZus{}GRAY}\PYG{p}{;}
\PYG{n}{style\PYGZus{}bg}\PYG{p}{.}\PYG{n}{body}\PYG{p}{.}\PYG{n}{radius} \PYG{o}{=} \PYG{n}{LV\PYGZus{}RADIUS\PYGZus{}CIRCLE}\PYG{p}{;}
\PYG{n}{style\PYGZus{}bg}\PYG{p}{.}\PYG{n}{body}\PYG{p}{.}\PYG{n}{border}\PYG{p}{.}\PYG{n}{color} \PYG{o}{=} \PYG{n}{LV\PYGZus{}COLOR\PYGZus{}WHITE}\PYG{p}{;}

\PYG{n}{lv\PYGZus{}style\PYGZus{}copy}\PYG{p}{(}\PYG{o}{\PYGZam{}}\PYG{n}{style\PYGZus{}indic}\PYG{p}{,} \PYG{o}{\PYGZam{}}\PYG{n}{lv\PYGZus{}style\PYGZus{}pretty}\PYG{p}{)}\PYG{p}{;}
\PYG{n}{style\PYGZus{}indic}\PYG{p}{.}\PYG{n}{body}\PYG{p}{.}\PYG{n}{grad\PYGZus{}color} \PYG{o}{=}  \PYG{n}{LV\PYGZus{}COLOR\PYGZus{}GREEN}\PYG{p}{;}
\PYG{n}{style\PYGZus{}indic}\PYG{p}{.}\PYG{n}{body}\PYG{p}{.}\PYG{n}{main\PYGZus{}color} \PYG{o}{=}  \PYG{n}{LV\PYGZus{}COLOR\PYGZus{}LIME}\PYG{p}{;}
\PYG{n}{style\PYGZus{}indic}\PYG{p}{.}\PYG{n}{body}\PYG{p}{.}\PYG{n}{radius} \PYG{o}{=} \PYG{n}{LV\PYGZus{}RADIUS\PYGZus{}CIRCLE}\PYG{p}{;}
\PYG{n}{style\PYGZus{}indic}\PYG{p}{.}\PYG{n}{body}\PYG{p}{.}\PYG{n}{shadow}\PYG{p}{.}\PYG{n}{width} \PYG{o}{=} \PYG{l+m+mi}{10}\PYG{p}{;}
\PYG{n}{style\PYGZus{}indic}\PYG{p}{.}\PYG{n}{body}\PYG{p}{.}\PYG{n}{shadow}\PYG{p}{.}\PYG{n}{color} \PYG{o}{=} \PYG{n}{LV\PYGZus{}COLOR\PYGZus{}LIME}\PYG{p}{;}
\PYG{n}{style\PYGZus{}indic}\PYG{p}{.}\PYG{n}{body}\PYG{p}{.}\PYG{n}{padding}\PYG{p}{.}\PYG{n}{hor} \PYG{o}{=} \PYG{l+m+mi}{3}\PYG{p}{;}
\PYG{n}{style\PYGZus{}indic}\PYG{p}{.}\PYG{n}{body}\PYG{p}{.}\PYG{n}{padding}\PYG{p}{.}\PYG{n}{ver} \PYG{o}{=} \PYG{l+m+mi}{3}\PYG{p}{;}

\PYG{n}{lv\PYGZus{}style\PYGZus{}copy}\PYG{p}{(}\PYG{o}{\PYGZam{}}\PYG{n}{style\PYGZus{}knob}\PYG{p}{,} \PYG{o}{\PYGZam{}}\PYG{n}{lv\PYGZus{}style\PYGZus{}pretty}\PYG{p}{)}\PYG{p}{;}
\PYG{n}{style\PYGZus{}knob}\PYG{p}{.}\PYG{n}{body}\PYG{p}{.}\PYG{n}{radius} \PYG{o}{=} \PYG{n}{LV\PYGZus{}RADIUS\PYGZus{}CIRCLE}\PYG{p}{;}
\PYG{n}{style\PYGZus{}knob}\PYG{p}{.}\PYG{n}{body}\PYG{p}{.}\PYG{n}{opa} \PYG{o}{=} \PYG{n}{LV\PYGZus{}OPA\PYGZus{}70}\PYG{p}{;}
\PYG{n}{style\PYGZus{}knob}\PYG{p}{.}\PYG{n}{body}\PYG{p}{.}\PYG{n}{padding}\PYG{p}{.}\PYG{n}{ver} \PYG{o}{=} \PYG{l+m+mi}{10} \PYG{p}{;}

\PYG{c+cm}{/*Create a second slider*/}
\PYG{n}{lv\PYGZus{}obj\PYGZus{}t} \PYG{o}{*} \PYG{n}{slider2} \PYG{o}{=} \PYG{n}{lv\PYGZus{}slider\PYGZus{}create}\PYG{p}{(}\PYG{n}{lv\PYGZus{}scr\PYGZus{}act}\PYG{p}{(}\PYG{p}{)}\PYG{p}{,} \PYG{n}{slider1}\PYG{p}{)}\PYG{p}{;}
\PYG{n}{lv\PYGZus{}slider\PYGZus{}set\PYGZus{}style}\PYG{p}{(}\PYG{n}{slider2}\PYG{p}{,} \PYG{n}{LV\PYGZus{}SLIDER\PYGZus{}STYLE\PYGZus{}BG}\PYG{p}{,} \PYG{o}{\PYGZam{}}\PYG{n}{style\PYGZus{}bg}\PYG{p}{)}\PYG{p}{;}
\PYG{n}{lv\PYGZus{}slider\PYGZus{}set\PYGZus{}style}\PYG{p}{(}\PYG{n}{slider2}\PYG{p}{,} \PYG{n}{LV\PYGZus{}SLIDER\PYGZus{}STYLE\PYGZus{}INDIC}\PYG{p}{,}\PYG{o}{\PYGZam{}}\PYG{n}{style\PYGZus{}indic}\PYG{p}{)}\PYG{p}{;}
\PYG{n}{lv\PYGZus{}slider\PYGZus{}set\PYGZus{}style}\PYG{p}{(}\PYG{n}{slider2}\PYG{p}{,} \PYG{n}{LV\PYGZus{}SLIDER\PYGZus{}STYLE\PYGZus{}KNOB}\PYG{p}{,} \PYG{o}{\PYGZam{}}\PYG{n}{style\PYGZus{}knob}\PYG{p}{)}\PYG{p}{;}
\PYG{n}{lv\PYGZus{}obj\PYGZus{}align}\PYG{p}{(}\PYG{n}{slider2}\PYG{p}{,} \PYG{n}{slider1}\PYG{p}{,} \PYG{n}{LV\PYGZus{}ALIGN\PYGZus{}OUT\PYGZus{}BOTTOM\PYGZus{}MID}\PYG{p}{,} \PYG{l+m+mi}{0}\PYG{p}{,} \PYG{l+m+mi}{30}\PYG{p}{)}\PYG{p}{;} \PYG{c+cm}{/*Align below \PYGZsq{}bar1\PYGZsq{}*/}

\PYG{c+cm}{/*Create a second label*/}
\PYG{n}{lv\PYGZus{}obj\PYGZus{}t} \PYG{o}{*} \PYG{n}{slider2\PYGZus{}label} \PYG{o}{=} \PYG{n}{lv\PYGZus{}label\PYGZus{}create}\PYG{p}{(}\PYG{n}{lv\PYGZus{}scr\PYGZus{}act}\PYG{p}{(}\PYG{p}{)}\PYG{p}{,} \PYG{n}{slider1\PYGZus{}label}\PYG{p}{)}\PYG{p}{;}
\PYG{n}{lv\PYGZus{}label\PYGZus{}set\PYGZus{}text}\PYG{p}{(}\PYG{n}{slider2\PYGZus{}label}\PYG{p}{,} \PYG{l+s}{\PYGZdq{}}\PYG{l+s}{Modified}\PYG{l+s}{\PYGZdq{}}\PYG{p}{)}\PYG{p}{;}
\PYG{n}{lv\PYGZus{}obj\PYGZus{}align}\PYG{p}{(}\PYG{n}{slider2\PYGZus{}label}\PYG{p}{,} \PYG{n}{slider2}\PYG{p}{,} \PYG{n}{LV\PYGZus{}ALIGN\PYGZus{}OUT\PYGZus{}LEFT\PYGZus{}MID}\PYG{p}{,} \PYG{o}{\PYGZhy{}}\PYG{l+m+mi}{30}\PYG{p}{,} \PYG{l+m+mi}{0}\PYG{p}{)}\PYG{p}{;}
\end{sphinxVerbatim}


\subsection{Spinbox (lv\_spinbox)}
\label{\detokenize{object-types/spinbox:spinbox-lv-spinbox}}\label{\detokenize{object-types/spinbox::doc}}

\subsection{Switch (lv\_sw)}
\label{\detokenize{object-types/sw:switch-lv-sw}}\label{\detokenize{object-types/sw::doc}}

\subsubsection{Overview}
\label{\detokenize{object-types/sw:overview}}
The Switch can be used to \sphinxstylestrong{turn on/off}  something. The look like a little slider. The state of the switch can be changed by:
\begin{itemize}
\item {} 
Clicking on it

\item {} 
Sliding it

\item {} 
Using \sphinxcode{\sphinxupquote{lv\_sw\_on(sw)}} and \sphinxcode{\sphinxupquote{lv\_sw\_off(sw)}} functions

\end{itemize}

A \sphinxstylestrong{callback function} can be assigned to call when the user uses the switch: \sphinxcode{\sphinxupquote{lv\_sw\_set\_action(sw, my\_action)}}.

\sphinxstylestrong{New in v5.3}: Switches can be animated by calling \sphinxcode{\sphinxupquote{lv\_sw\_set\_anim\_time(sw, anim\_ms)}}.


\subsubsection{Style usage}
\label{\detokenize{object-types/sw:style-usage}}
You can modify the Switch’s styles with \sphinxcode{\sphinxupquote{lv\_sw\_set\_style(sw, LV\_SW\_STYLE\_..., \&style)}}.
\begin{itemize}
\item {} 
\sphinxstylestrong{LV\_SW\_STYLE\_BG} Style of the background. All \sphinxstyleemphasis{style.body} properties are used. The \sphinxstyleemphasis{padding} values make the Switch smaller then the knob. (negative value makes is larger)

\item {} 
\sphinxstylestrong{LV\_SW\_STYLE\_INDIC} Style of the indicator. All \sphinxstyleemphasis{style.body} properties are used. The \sphinxstyleemphasis{padding} values make the indicator smaller then the background.

\item {} 
\sphinxstylestrong{LV\_SW\_STYLE\_KNOB\_OFF} Style of the knob when the switch is off.  The \sphinxstyleemphasis{style.body} properties are used except padding.

\item {} 
\sphinxstylestrong{LV\_SW\_STYLE\_KNOB\_ON} Style of the knob when the switch is on.  The \sphinxstyleemphasis{style.body} properties are used except padding.

\end{itemize}


\subsubsection{Notes}
\label{\detokenize{object-types/sw:notes}}\begin{itemize}
\item {} 
The Knob is not a real object it is only drawn above the Bar

\end{itemize}


\subsubsection{Example}
\label{\detokenize{object-types/sw:example}}
\sphinxincludegraphics{{switch-lv_sw}.png}Switch image

\begin{sphinxVerbatim}[commandchars=\\\{\}]
\PYG{c+cm}{/*Create styles for the switch*/}
\PYG{k}{static} \PYG{n}{lv\PYGZus{}style\PYGZus{}t} \PYG{n}{bg\PYGZus{}style}\PYG{p}{;}
\PYG{k}{static} \PYG{n}{lv\PYGZus{}style\PYGZus{}t} \PYG{n}{indic\PYGZus{}style}\PYG{p}{;}
\PYG{k}{static} \PYG{n}{lv\PYGZus{}style\PYGZus{}t} \PYG{n}{knob\PYGZus{}on\PYGZus{}style}\PYG{p}{;}
\PYG{k}{static} \PYG{n}{lv\PYGZus{}style\PYGZus{}t} \PYG{n}{knob\PYGZus{}off\PYGZus{}style}\PYG{p}{;}
\PYG{n}{lv\PYGZus{}style\PYGZus{}copy}\PYG{p}{(}\PYG{o}{\PYGZam{}}\PYG{n}{bg\PYGZus{}style}\PYG{p}{,} \PYG{o}{\PYGZam{}}\PYG{n}{lv\PYGZus{}style\PYGZus{}pretty}\PYG{p}{)}\PYG{p}{;}
\PYG{n}{bg\PYGZus{}style}\PYG{p}{.}\PYG{n}{body}\PYG{p}{.}\PYG{n}{radius} \PYG{o}{=} \PYG{n}{LV\PYGZus{}RADIUS\PYGZus{}CIRCLE}\PYG{p}{;}

\PYG{n}{lv\PYGZus{}style\PYGZus{}copy}\PYG{p}{(}\PYG{o}{\PYGZam{}}\PYG{n}{indic\PYGZus{}style}\PYG{p}{,} \PYG{o}{\PYGZam{}}\PYG{n}{lv\PYGZus{}style\PYGZus{}pretty\PYGZus{}color}\PYG{p}{)}\PYG{p}{;}
\PYG{n}{indic\PYGZus{}style}\PYG{p}{.}\PYG{n}{body}\PYG{p}{.}\PYG{n}{radius} \PYG{o}{=} \PYG{n}{LV\PYGZus{}RADIUS\PYGZus{}CIRCLE}\PYG{p}{;}
\PYG{n}{indic\PYGZus{}style}\PYG{p}{.}\PYG{n}{body}\PYG{p}{.}\PYG{n}{main\PYGZus{}color} \PYG{o}{=} \PYG{n}{LV\PYGZus{}COLOR\PYGZus{}HEX}\PYG{p}{(}\PYG{l+m+mh}{0x9fc8ef}\PYG{p}{)}\PYG{p}{;}
\PYG{n}{indic\PYGZus{}style}\PYG{p}{.}\PYG{n}{body}\PYG{p}{.}\PYG{n}{grad\PYGZus{}color} \PYG{o}{=} \PYG{n}{LV\PYGZus{}COLOR\PYGZus{}HEX}\PYG{p}{(}\PYG{l+m+mh}{0x9fc8ef}\PYG{p}{)}\PYG{p}{;}
\PYG{n}{indic\PYGZus{}style}\PYG{p}{.}\PYG{n}{body}\PYG{p}{.}\PYG{n}{padding}\PYG{p}{.}\PYG{n}{hor} \PYG{o}{=} \PYG{l+m+mi}{0}\PYG{p}{;}
\PYG{n}{indic\PYGZus{}style}\PYG{p}{.}\PYG{n}{body}\PYG{p}{.}\PYG{n}{padding}\PYG{p}{.}\PYG{n}{ver} \PYG{o}{=} \PYG{l+m+mi}{0}\PYG{p}{;}

\PYG{n}{lv\PYGZus{}style\PYGZus{}copy}\PYG{p}{(}\PYG{o}{\PYGZam{}}\PYG{n}{knob\PYGZus{}off\PYGZus{}style}\PYG{p}{,} \PYG{o}{\PYGZam{}}\PYG{n}{lv\PYGZus{}style\PYGZus{}pretty}\PYG{p}{)}\PYG{p}{;}
\PYG{n}{knob\PYGZus{}off\PYGZus{}style}\PYG{p}{.}\PYG{n}{body}\PYG{p}{.}\PYG{n}{radius} \PYG{o}{=} \PYG{n}{LV\PYGZus{}RADIUS\PYGZus{}CIRCLE}\PYG{p}{;}
\PYG{n}{knob\PYGZus{}off\PYGZus{}style}\PYG{p}{.}\PYG{n}{body}\PYG{p}{.}\PYG{n}{shadow}\PYG{p}{.}\PYG{n}{width} \PYG{o}{=} \PYG{l+m+mi}{4}\PYG{p}{;}
\PYG{n}{knob\PYGZus{}off\PYGZus{}style}\PYG{p}{.}\PYG{n}{body}\PYG{p}{.}\PYG{n}{shadow}\PYG{p}{.}\PYG{n}{type} \PYG{o}{=} \PYG{n}{LV\PYGZus{}SHADOW\PYGZus{}BOTTOM}\PYG{p}{;}

\PYG{n}{lv\PYGZus{}style\PYGZus{}copy}\PYG{p}{(}\PYG{o}{\PYGZam{}}\PYG{n}{knob\PYGZus{}on\PYGZus{}style}\PYG{p}{,} \PYG{o}{\PYGZam{}}\PYG{n}{lv\PYGZus{}style\PYGZus{}pretty\PYGZus{}color}\PYG{p}{)}\PYG{p}{;}
\PYG{n}{knob\PYGZus{}on\PYGZus{}style}\PYG{p}{.}\PYG{n}{body}\PYG{p}{.}\PYG{n}{radius} \PYG{o}{=} \PYG{n}{LV\PYGZus{}RADIUS\PYGZus{}CIRCLE}\PYG{p}{;}
\PYG{n}{knob\PYGZus{}on\PYGZus{}style}\PYG{p}{.}\PYG{n}{body}\PYG{p}{.}\PYG{n}{shadow}\PYG{p}{.}\PYG{n}{width} \PYG{o}{=} \PYG{l+m+mi}{4}\PYG{p}{;}
\PYG{n}{knob\PYGZus{}on\PYGZus{}style}\PYG{p}{.}\PYG{n}{body}\PYG{p}{.}\PYG{n}{shadow}\PYG{p}{.}\PYG{n}{type} \PYG{o}{=} \PYG{n}{LV\PYGZus{}SHADOW\PYGZus{}BOTTOM}\PYG{p}{;}

\PYG{c+cm}{/*Create a switch and apply the styles*/}
\PYG{n}{lv\PYGZus{}obj\PYGZus{}t} \PYG{o}{*}\PYG{n}{sw1} \PYG{o}{=} \PYG{n}{lv\PYGZus{}sw\PYGZus{}create}\PYG{p}{(}\PYG{n}{lv\PYGZus{}scr\PYGZus{}act}\PYG{p}{(}\PYG{p}{)}\PYG{p}{,} \PYG{n+nb}{NULL}\PYG{p}{)}\PYG{p}{;}
\PYG{n}{lv\PYGZus{}sw\PYGZus{}set\PYGZus{}style}\PYG{p}{(}\PYG{n}{sw1}\PYG{p}{,} \PYG{n}{LV\PYGZus{}SW\PYGZus{}STYLE\PYGZus{}BG}\PYG{p}{,} \PYG{o}{\PYGZam{}}\PYG{n}{bg\PYGZus{}style}\PYG{p}{)}\PYG{p}{;}
\PYG{n}{lv\PYGZus{}sw\PYGZus{}set\PYGZus{}style}\PYG{p}{(}\PYG{n}{sw1}\PYG{p}{,} \PYG{n}{LV\PYGZus{}SW\PYGZus{}STYLE\PYGZus{}INDIC}\PYG{p}{,} \PYG{o}{\PYGZam{}}\PYG{n}{indic\PYGZus{}style}\PYG{p}{)}\PYG{p}{;}
\PYG{n}{lv\PYGZus{}sw\PYGZus{}set\PYGZus{}style}\PYG{p}{(}\PYG{n}{sw1}\PYG{p}{,} \PYG{n}{LV\PYGZus{}SW\PYGZus{}STYLE\PYGZus{}KNOB\PYGZus{}ON}\PYG{p}{,} \PYG{o}{\PYGZam{}}\PYG{n}{knob\PYGZus{}on\PYGZus{}style}\PYG{p}{)}\PYG{p}{;}
\PYG{n}{lv\PYGZus{}sw\PYGZus{}set\PYGZus{}style}\PYG{p}{(}\PYG{n}{sw1}\PYG{p}{,} \PYG{n}{LV\PYGZus{}SW\PYGZus{}STYLE\PYGZus{}KNOB\PYGZus{}OFF}\PYG{p}{,} \PYG{o}{\PYGZam{}}\PYG{n}{knob\PYGZus{}off\PYGZus{}style}\PYG{p}{)}\PYG{p}{;}
\PYG{n}{lv\PYGZus{}obj\PYGZus{}align}\PYG{p}{(}\PYG{n}{sw1}\PYG{p}{,} \PYG{n+nb}{NULL}\PYG{p}{,} \PYG{n}{LV\PYGZus{}ALIGN\PYGZus{}CENTER}\PYG{p}{,} \PYG{l+m+mi}{0}\PYG{p}{,} \PYG{o}{\PYGZhy{}}\PYG{l+m+mi}{50}\PYG{p}{)}\PYG{p}{;}

\PYG{c+cm}{/*Copy the first switch and turn it ON*/}
\PYG{n}{lv\PYGZus{}obj\PYGZus{}t} \PYG{o}{*}\PYG{n}{sw2} \PYG{o}{=} \PYG{n}{lv\PYGZus{}sw\PYGZus{}create}\PYG{p}{(}\PYG{n}{lv\PYGZus{}scr\PYGZus{}act}\PYG{p}{(}\PYG{p}{)}\PYG{p}{,} \PYG{n}{sw1}\PYG{p}{)}\PYG{p}{;}
\PYG{n}{lv\PYGZus{}sw\PYGZus{}on}\PYG{p}{(}\PYG{n}{sw2}\PYG{p}{)}\PYG{p}{;}
\PYG{n}{lv\PYGZus{}obj\PYGZus{}align}\PYG{p}{(}\PYG{n}{sw2}\PYG{p}{,} \PYG{n+nb}{NULL}\PYG{p}{,} \PYG{n}{LV\PYGZus{}ALIGN\PYGZus{}CENTER}\PYG{p}{,} \PYG{l+m+mi}{0}\PYG{p}{,} \PYG{l+m+mi}{50}\PYG{p}{)}\PYG{p}{;}
\end{sphinxVerbatim}


\subsection{Table (lv\_table)}
\label{\detokenize{object-types/table:table-lv-table}}\label{\detokenize{object-types/table::doc}}

\subsection{Tab view (lv\_tabview)}
\label{\detokenize{object-types/tabview:tab-view-lv-tabview}}\label{\detokenize{object-types/tabview::doc}}

\subsection{Text area (lv\_ta)}
\label{\detokenize{object-types/ta:text-area-lv-ta}}\label{\detokenize{object-types/ta::doc}}

\subsection{Tile view (lv\_tileview)}
\label{\detokenize{object-types/tileview:tile-view-lv-tileview}}\label{\detokenize{object-types/tileview::doc}}

\subsection{Window (lv\_win)}
\label{\detokenize{object-types/win:window-lv-win}}\label{\detokenize{object-types/win::doc}}


\renewcommand{\indexname}{Index}
\printindex
\end{document}